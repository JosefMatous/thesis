\vspace*{-0.25em}
\section{Outline and Contributions}
\vspace*{-0.5em}

\subsubsection{Chapter~\ref{chap:collision_avoidance}: \nameref{chap:collision_avoidance}}

This chapter differs from the rest of the thesis in the considered vehicle model.
In this chapter, we consider overactuated vehicles, \emph{i.e.,} vehicles with more actuators than \acrfullpl{dof}.
As mentioned in Section~\ref{sec:introduction_literature}, reactive collision avoidance should be included at the lowest-possible control level.
Overactuated vehicles often use control allocation in their lowest-level controller \cite{johansen_control_2013}.
Most control allocation methods are based on numerical optimization \cite{oppenheimer_control_2006,harkegard_dynamic_2004,johansen_constrained_2004} which makes them ideal for augmenting with \acrfullpl{cbf}.

The main contribution of Chapter~\ref{chap:collision_avoidance} is a reactive \acrfull{colav} algorithm that is included at the lowest level in the control pipeline, \emph{i.e.} in the control allocation, to ensure the safety of the vehicle regarding collision avoidance.
Since it is included at the lowest-possible control level, it also ensures the ``baseline'' safety of any other higher level (long term/deliberate) planners of the vehicle guidance, navigation and control system.
The algorithm can easily be implemented on vehicles that apply a numerical optimization-based method to control allocation.
Moreover, the algorithm does not rely on any communication between the vehicles; the only required information is the position and velocity of other vehicles.
The chapter extends the results in \cite{thyri_reactive_2020}, which only considers \acrfullpl{asv} and simple encounters between one \gls{asv} and a vessel moving at a constant course and speed, making the method applicable to a wider range of vehicles and scenarios with multiple autonomous vehicles.

%\subsection{Part~\ref{part:NSB}: \nameref{part:NSB}}

%This section investigates the use of the \acrlong{nsb} algorithm for solving the formation path-following problem on a fleet of underactuated \glspl{auv}.

\subsubsection{Chapter~\ref{chap:5dof_nsb}: \nameref{chap:5dof_nsb}}

As mentioned in Section~\ref{sec:introduction_literature}, the \gls{nsb} algorithm has been applied to kinematic vehicles (\emph{i.e.,} vehicles with single-integrator dynamics), as well as \glspl{asv} and \glspl{auv} moving in the horizontal plane.
Chapter~\ref{chap:5dof_nsb} applies the \gls{nsb} algorithm to \glspl{auv} moving in three dimensions.

Specifically, the chapter extends the results of \cite{eek_formation_2021}, where an \gls{nsb} algorithm is used to guide two \glspl{asv}, by proposing an algorithm that works with an arbitrary number of \glspl{auv} with five \glspl{dof} moving in 3D.
We solve the formation path-following problem by defining three tasks: collision avoidance, formation keeping, and path following.
The tasks are combined using the \gls{nsb} algorithm to achieve the desired behavior.
Similarly to \cite{eek_formation_2021}, we solve the path-following task using \gls{los} guidance.
Using the cascaded systems theory results of \cite{pettersen_lyapunov_2017}, we prove that the closed-loop system consisting of a 3D \gls{los} guidance law, combined with surge, pitch, and yaw autopilots based on \cite{moe_LOS_2016}, is \glspl{usges} and \glspl{ugas}.
The theoretical results are verified through numerical simulations.

\subsubsection{Chapter~\ref{chap:NSB_R}: \nameref{chap:NSB_R}}

This chapter further extends the \gls{nsb} algorithm proposed in Chapter~\ref{chap:5dof_nsb}.
The algorithm in Chapter~\ref{chap:5dof_nsb} uses a 5\gls{dof} \gls{auv} model, considers only inter-vehicle collision avoidance, and proves only the stability of the path-following task.
Furthermore, the orientation of the 5\gls{dof} model was expressed using Euler angles, which causes singularities for a pitch angle of $\pm90$ degrees.

This work applies the \gls{nsb} algorithm to a full 6\gls{dof} model, uses rotation matrices to describe the attitude of the vehicles to avoid singularities, modifies and extends the tasks, and proves the stability of the combined path-following and formation-keeping tasks.
We also add a scheme for obstacle avoidance and a scheme that keeps the vehicles within a given range of depths.
As opposed to the previous work, we do not limit the analysis to a specific low-level attitude controller.
Consequently, the new algorithm can be integrated into existing on-board controllers.
Assuming that the existing low-level controller allows exponential tracking, we use results from cascaded systems theory~\cite{pettersen_lyapunov_2017} to prove that the closed-loop system composed by the \gls{nsb} algorithm and the low-level controller is \acrfullpl{usges}.
We verify the results in numerical simulations.

\subsubsection{Chapter~\ref{chap:distr_NSB}: \nameref{chap:distr_NSB}}

The algorithms presented in Chapters~\ref{chap:5dof_nsb} and \ref{chap:NSB_R} are centralized, making them difficult to use in real-life applications.
As mentioned in Section~\ref{sec:introduction_literature}, there are distributed \gls{nsb} algorithms \cite{ahmad_multirobot_2014,tan_coordinated_2022}.
However, these algorithms work by dividing the fleet into smaller, fully connected subgroups, or by using leading vehicles.
In both approaches, there is still a requirement for fast and reliable communications within the subgroups and between the leaders.
Furthermore, the leader-follower scheme is vulnerable to failures of the leading vehicles.

Chapter~\ref{chap:distr_NSB} presents an approach that is fully distributed, so that the fleet does not need to decompose into subgroups nor requires leading vehicles.
To do so, the proposed algorithm combines the centralized schemes presented in Chapters~\ref{chap:5dof_nsb} and \ref{chap:NSB_R} with a consensus algorithm.
Specifically, we first propose a continuous-time consensus algorithm and prove its stability using Lyapunov analysis.
Then, we present a modified discrete-time version of the algorithm based on event-triggered control.
The effectiveness of both the continuous- and discrete-time algorithms is demonstrated in numerical simulation.
Furthermore, the discrete-time version is also tested in field experiments.

%\subsection{Part~\ref{part:hand_position}: \nameref{part:hand_position}}

%In this part, we define the hand position concept for 6\gls{dof} underactuated \glspl{auv}, and present some applications.

\subsubsection{Chapter~\ref{chap:handpos_definition}: \nameref{chap:handpos_definition}}

This thesis studies slender torpedo-shaped \acrfullpl{auv} with a propeller that provides forward (surge) thrust, and fins that provide torque.
The configuration of actuators means that \glspl{auv} are \emph{underactuated}, as we cannot directly control the lateral (sway and heave) velocities.
Most control algorithms use the so-called \emph{neutral point} of the vehicle as the output of the system (see \figref{fig:handpos_def_motivation}a).
The neutral point is a location on the $x$-axis (the stern-bow axis) of the vehicle such that, if chosen as the origin of the vehicle's body-fixed coordinate frame, the lateral motion of the vehicle is not affected by its control inputs.
Due to the underactuated nature of the \gls{auv}, controlling the neutral point requires specialized algorithms, such as \acrlong{los} guidance \cite{caharija_path-following-ILOS_2016,xiang_path-following-robust_2017,miao_path-following-curvilinear_2017,borhaug_straight_2007}.
In this chapter, we propose to use the \emph{hand position} concept to control the \gls{auv}.
The hand position is a point located a given distance in front of the neutral point along the vehicle's $x$-axis (see \figref{fig:handpos_def_motivation}b for an illustration).
The concept was first introduced in \cite{pomet_hand-position_1992} to stabilize nonholonomic vehicles with unicycle dynamics.
Later, it was applied to control formations of unicycles \cite{lawton_hand-position-formation_2003}.
The concept was then extended to marine vehicles moving in the horizontal plane \cite{paliotta_trajectory_2019}, and two- and three-dimensional Euler-Lagrange-like systems \cite{cai_hand-position-rigidity-planar_2015,li_hand-position-rigidity-3d_2021}.

\begin{figure}[b]
    \centering
    \def\svgwidth{0.65\textwidth}
    \import{figures/handpos_def}{motivation.pdf_tex}
    \caption{Illustration of (a) the traditional approach where the aim is to control the neutral point of the vehicle, and (b) the proposed hand position-based approach. The dashed line represents the body-fixed $x$-axis.}
    \label{fig:handpos_def_motivation}
\end{figure}

There are two main advantages to using the hand position concept.
The first advantage stems from the applications of AUVs.
The aim of many scientific missions is to scan a given area using a sensor attached to the AUV.
Since the position of the sensor typically does not coincide with the neutral point, there may be a significant offset between the sensor and the desired trajectory, caused by the sea loads (see \figref{fig:handpos_def_motivation}a).
In some cases, the hand position can be chosen such that it coincides with the position of the sensor, allowing to scan the area more accurately.
The second advantage is that if we choose the hand position as the output of our system, it is possible to transform the nonlinear underactuated vehicle model to a double integrator, using output feedback linearization.
This allows us to apply advanced control strategies, \emph{e.g.,} various consensus algorithms \cite{cai_hand-position-rigidity-planar_2015,li_hand-position-rigidity-3d_2021,lawton_hand-position-formation_2003,restrepo_formation_2022} that cannot be directly used on nonholonomic or underactuated vehicles.

Note that the three-dimensional Euler-Lagrange-like system used in \cite{li_hand-position-rigidity-3d_2021} does not accurately represent AUVs, since it does not consider the Coriolis and centripetal effects or the restoring forces (gravity and buoyancy).
Furthermore, the model in \cite{li_hand-position-rigidity-3d_2021} has five \acrfullpl{dof}: three position coordinates, pitch angle, and yaw angle.
The use of Euler angles inherently introduces singularities into the system.

The goal of this chapter is thus to further extend the hand position concept to AUVs moving in three dimensions.
We employ a more realistic AUV model than in \cite{li_hand-position-rigidity-3d_2021}. We model the full 6DOF motion and use rotation matrices to describe the orientation of the vehicle, thus avoiding singularities.
Using Lyapunov analysis, we derive the sufficient conditions for boundedness of the internal states, \emph{i.e.,} the orientation and the angular velocities, for a generic hand position-based controller.

\subsubsection{Chapter~\ref{chap:handpos_trajectory}: \nameref{chap:handpos_trajectory}}

In this chapter, we use the hand position concept to solve the trajectory-tracking and path-following problems.
Specifically, we show that these two problems can be solved using the hand position transformation and a simple PID controller.
We analyze the closed-loop behavior of the system and prove that the proposed controllers exponentially track the desired trajectory or path, while the angular velocities of the vehicle remain bounded.
Moreover, we prove that in the special case when the desired trajectory or path is a straight line, the whole closed-loop system is exponentially stable.
The theoretical results are verified both in numerical simulations and experiments.

\subsubsection{Chapter~\ref{chap:handpos_MPC}: \nameref{chap:handpos_MPC}}

In this chapter, we employ the spline-based \gls{mpc} presented in \cite{saska_2016_predictive,van_parys_2017_DMPC} to solve the formation path-following problem.
The proposed scheme is applicable to a wide range of vehicles.
The only restriction is that the model of the vehicle must be differentially flat.
The spline-based \gls{mpc} scheme can thus be seen as a trade-off between lower computational requirements and more restrictive assumptions on the model.

We present the results of two numerical case studies.
The first case study considers a fleet of \glspl{auv}.
To make the \gls{auv} model differentially flat, we employ the hand position transformation.
The second case study considers a group of differential drive robots modeled as first-order unicycles.

\subsubsection{Chapter~\ref{chap:handpos_tracking}: \nameref{chap:handpos_tracking}}

This chapter investigates the tracking-in-formation problem for a fleet of \glspl{auv}.
This problem is similar to the formation path-following problem, except that the fleet should track a desired trajectory instead of following a desired path.
We assume that the \glspl{auv} communicate over a directed topology and are subject to the \gls{colav} and connectivity constraints discussed in Section~\ref{sec:introduction_literature}.

Under the control designs proposed in the literature, in many instances, in order to guarantee the satisfaction of the inter-agent constraints, the vehicles are forced to move backwards, oftentimes during a prolonged period of time and at relatively high speeds (for backwards motion of a marine vehicle). 
However, although marine vehicles are able to move backwards, they are not well-suited to do so due to their shapes and their propulsion system. 
This issue, however, has not been addressed in the literature of multi-\gls{auv} systems.

In Chapter~\ref{chap:handpos_tracking}, we propose a distributed control law that solves the tracking-in-formation problem for multiple marine vehicles interacting over a directed communication graph and that guarantees, simultaneously, connectivity preservation and inter-agent collision avoidance.
Moreover, we address the issue of backwards motion by imposing a non-negativity constraint on the surge velocity of the vehicles.
More precisely, on one hand we encode via \glspl{blf} the proximity and safety constraints as \emph{hard constraints} that need to be always satisfied.
On the other hand, we encode the non-negativity of the surge velocity as a \emph{soft constraint}, so that it is imposed on the vehicles as long as it does not interfere with the hard constraints, in which case it is dynamically relaxed.
The proposed controller is based on the hand-position-based input-output feedback linearization method presented in Chapter~\ref{chap:handpos_definition} and on the so-called \emph{edge-agreement} representation of multi-agent systems \cite{mesbahi_graph_2010}, in which the relative states of the connected agents are used instead of the absolute ones, making it well adapted to practical applications where, usually, only relative measurements are available.
With regards to the stability analysis, differing from most of the existing works in the literature, where only non-uniform convergence to the formation and to the target vehicle is guaranteed, we establish \emph{almost-everywhere} uniform asymptotic stability of the tracking-in-formation objective and we show that the output error dynamics converge to the origin exponentially fast, while satisfying the constraints.

\subsubsection{Chapter~\ref{chap:handpos_NSB}: \nameref{chap:handpos_NSB}}

This chapter presents an extended \acrfull{nsb} algorithm for vehicles with second-order dynamics.
The \gls{nsb} algorithm, as presented in the existing literature, is developed for kinematic single-integrator systems \cite{arrichiello_formation_2006,matous_singularity_2023,eek_formation_2021}.
Although existing \gls{nsb} methods are developed for first-order systems, \gls{auv} dynamics are inherently second-order. Therefore, any first-order solution is necessarily perturbed by the dynamics of the maneuvering controller.
In Chapter~\ref{chap:handpos_NSB}, we extend the \gls{nsb} method to vehicles with double integrator dynamics and propose an algorithm that uses the \acrlong{soclik} equation to control the task variables through desired acceleration. The procedure is inspired by robotic manipulators, where second-order methods are more common, due to the inherent second-order dynamics of mechanical systems \cite{siciliano_differential_2009, chiaverini_kinematically_2008}. Compared to the existing methods, our formulation handles the second-order dynamics directly in the task space as interpretable mass-spring-damper systems.

We apply the proposed \gls{nsb} method to a fleet of \glspl{auv}.
To make the proposed method applicable to \glspl{auv}, we use the hand position transformation.
Subsequently, through the design of specific path-following, formation-keeping, and collision-avoidance tasks, we can control the fleet to follow a given path in a formation while avoiding collisions both within the fleet and with external obstacles. 
We prove the stability of the control scheme using Lyapunov analysis and verify its effectiveness in simulations. 
Because our reformulated \gls{nsb} method works directly with the second-order system, there is no need to transform desired velocities or accelerations into surge and orientation references, as has been done in previous works.
This reduces one level of complexity in the controller design.
