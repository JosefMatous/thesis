\chapter{Control of AUVs Under Hard and Soft Constraints}
\label{chap:handpos_tracking}

This chapter investigates the tracking-in-formation problem for a group of underactuated autonomous marine vehicles interconnected over a directed topology.
The agents are subject to \emph{hard} inter-agent constraints, \emph{i.e.,} connectivity maintenance and collision avoidance, and \emph{soft} constraints, specifically on the non-negativity of the surge velocity, as well as to constant disturbances in the form of unknown ocean currents.
The control approach is based on two concepts: the 3D hand position output linearization presented in Chapter~\ref{chap:handpos_definition}, and the edge-based framework for multi-agent consensus under constraints.
We establish \emph{almost-everywhere} uniform asymptotic stability of the output dynamics with guaranteed respect of the constraints.
Numerical and high-fidelity simulations are provided to illustrate the effectiveness of our approach.
The contents of this chapter are based on \cite{restrepo_formation_2022,restrepo_tracking_2023}.

The chapter is organized as follows.
In Section \ref{sec:preliminaries} we present the model of the multi-agent system and the problem formulation. For clarity of exposition, in Section~\ref{sec:ctrl} we present the control design when considering only the hard constraints, followed by the stability analysis. Then, in Section~\ref{sec:soft} we present the control design adding the soft constraints. Finally, the results of numerical simulations are presented in Section~\ref{sec:simulations}.

\vspace*{-1em}

\section{Model and Problem Formulation}\label{sec:preliminaries}

\vspace*{-0.5em}

\subsection{Model of the Marine Vehicle}

We consider underactuated \glspl{auv} with six \glspl{dof}, and apply the hand position transformation from Chapter~\ref{chap:handpos_definition}.
Recalling \eqref{eq:handpos_def_hand_position_dynamics}, the dynamics of the transformed system are

\vspace*{-1.7em}

\begin{subequations}
    \begin{align}
        \dot{\mat{x}}_1 &= \mat{x}_2 + \ocean, \label{eq:handpos_tracking_x_1_dot_transformed}\\
        \dot{\mat{x}}_2 &= \bs{\mu}, \label{eq:handpos_tracking_x_2_dot_transformed}\\
        \dot{\mat{R}} &= \mat{R}\mat{S}(\bs{\omega}), \label{eq:handpos_tracking_R_dot_CL} \\
        \dot{\bs{\omega}} &= \Bar{\handl} \times \left(\mat{R}\T\bs{\mu} + \mathcal{D}_{\bvel}(\vel) + \mathcal{C}_{\bvel}(\vel) - \bs{\omega} \times \mat{R}\T \mat{x}_2\right) \label{eq:handpos_tracking_omega_dot_CL} \\
            & \quad - \left(\Bar{\handl}\handl\T\right) \left(\mathcal{D}_{\bs{\omega}}(\vel) + \mat{M}_{22}^{\prime}\left(Wz_{gb} \mat{e}_3 \times \mat{R}\T \mat{e}_3\right)\right). \nonumber
    \end{align} \label{eq:handpos_tracking_hand_position_dynamics}
\end{subequations}

\subsection{Problem Statement}

We consider a multi-agent system composed of $N$ marine vehicles modeled by \eqref{eq:handpos_tracking_hand_position_dynamics}.
The states of the vehicles are denoted by double subscripts (\emph{e.g.,} the hand position of vehicle $i$ is $\mat{x}_{1, i}$).
The interaction between the \glspl{auv} is given by a directed graph $\mathcal{G}=(\mathcal{V},\mathcal{E})$ which is either a spanning tree or a cycle.
Moreover, we consider that the multi-agent system is subject to inter-agent output constraints.
For one part, these constraints may come from embedded relative-measurements devices, which are reliable only if used within a limited range. Hence, the vehicles must remain within a limited distance from their neighbors in order to maintain the connectivity of the graph.
Furthermore, to ensure the safety of the system, the agents must avoid collisions among themselves, that is, they must always guarantee a minimal distance with respect to their neighbors.
These connectivity and collision-avoidance constraints may be defined as a set of restrictions on $\mat{x}_1$. Also, such constraints may be considered as \emph{hard} constraints since they are fundamental for ensuring the safety of the system and for reaching the control goal.

More precisely, define the relative output 
\begin{equation}\label{eq:z1k-def}
    \mat{z}_{1, k} = \mat{x}_{1, i} - \mat{x}_{1, j}, \quad \forall k\leq M, \quad e_k=(i,j)\in\mathcal{E},
\end{equation}
where $M$ is the cardinality of $\mathcal{E}$.
For each $k\leq M$, let $\delta_k$ and ${\Delta_k}$ be, the minimal and maximal distances between agents $i$ and $j$ so that collisions are avoided and that the communication through edge $e_k$ is reliable, respectively.
Then, the set of inter-agent output constraints is defined as
\begin{equation}
\label{eq:setDk} \mathcal D_k = \big\{\mat{z}_{1k}\in\mathbb{R}^{3}:\delta_k<\norm{\mat{z}_{1k}}<{\Delta_k}\big\},\quad \forall\, k\leq M.
\end{equation}

Coming back to each individual agent, it is important to note that, in practice, marine vehicles are not optimized for moving backwards. However, backwards motion is not prevented from the dynamical model \eqref{eq:handpos_tracking_hand_position_dynamics}.
Therefore, in order to let the vehicles evolve in an optimal way, besides the inter-agent connectivity and collision-avoidance constraints, we could formulate the additional constraints
\begin{equation}
\label{eq:posvel} u_{r, i}(t)> 0,\quad \forall\, i\leq N, \quad \forall t\geq 0.
\end{equation}
However, in some cases, the constraints \eqref{eq:posvel} could conflict with the constraints defined by the set \eqref{eq:setDk}. 
Indeed, there might exist situations when the only way to avoid a collision or avoid losing connectivity is to move backwards. 
Moreover, although not optimized to, marine vehicles \emph{can} move backwards.
The latter fact motivates us to reformulate the constraints \eqref{eq:posvel} as \emph{soft} constraints, that is, to impose a positive surge velocity as long as this does not interfere with the satisfaction of the hard constraints \eqref{eq:setDk}.
We formulate these soft constraints as follows:
\begin{equation}
\label{eq:soft} u_{r, i}(t) + \rho_i(t) > 0,\quad \forall\, i\leq N, \quad \forall t\geq 0,
\end{equation}
where $\rho_i:\mathbb{R}_{\geq 0}\to\mathbb{R}_{\geq 0}$ will be defined later, such that $\rho_i(t)\approxeq0$ when there are no conflicts with the hard constraints and $\rho_i(t)>0$ otherwise, allowing $u_{r, i}(t)$ to become negative. Akin to \eqref{eq:setDk}, we may define the set of soft constraints as
\begin{equation}
\label{eq:setC} \mathcal{C}_i = \big\{u_{r, i}\in\mathbb{R}:u_{r, i}>-\rho_i(t)\big\},\quad \forall\, i\leq N.
\end{equation}

Now, let $\mat{x}_{1, o}$, $\mat{x}_{2, o}$, and $\bs{\mu}_o$ define the position, velocity, and acceleration of a virtual target, and let its dynamics be modeled as a second-order integrator
\begin{equation}\label{360}
\dot{\mat{x}}_{1, o} = \mat{x}_{2, o},\quad
\dot{\mat{x}}_{2, o} = \bs{\mu}_o(t).
\end{equation}
Moreover, assume the following.
\begin{asm}\label{hyp:target}
	For all $t$, there exist positive constants $\underline{x}_{2, o}$, $\overline{x}_{2, o}$, and $\overline{\mu}_o$ such that $\norm{\ocean} < \underline{x}_{2, o}\leq \norm{\mat{x}_{2, o}(t)} \leq \overline{x}_{2, o}$, and $\norm{\bs{\mu}_o(t)}\leq \overline{\mu}_o$.
\end{asm}

Then, the control goal is for the $N$ marine vehicles to achieve the desired formation and track the target modeled by \eqref{360}, all while guaranteeing that the hard constraints given by the set $\mathcal{D}_k$ in \eqref{eq:setDk} and the soft constraints \eqref{eq:soft} are respected.

For the tracking part of the problem, we consider the case that only one agent, labeled $i=1$ without loss of generality, has access to the state of the target, and knows the upper bound $\overline{\mu}_o$ on the target's acceleration.
Let $\mat{z}_{1, o}^d \in \mathbb{R}^3$ be the desired displacement with respect to the target. Then, we define the tracking error states as
\begin{equation}\label{367}
\tilde{\mat{z}}_{1, o} = \mat{x}_{1, 1} - \mat{x}_{1, o} - \mat{z}_{1, o}^d, \quad
\mat{z}_{2, o}= \mat{x}_{2, 1} - \mat{x}_{2, o}.
\end{equation}

To address the formation part, we rely on the edge-agreement framework \cite{mesbahi_graph_2010} where instead of considering the states of each individual agent (the nodes of the graph), we consider the variables $\mat{z}_{1, k}$ defined in \eqref{eq:z1k-def} which correspond to the edges in the graph.
Hence, let us denote by $\mat{E} \in \mathbb{R}^{N\times M}$ the incidence matrix of graph $\mathcal{G}$, where its $(i,k)$th entry is defined as follows: $\left[\mat{E}\right]_{ik} = -1$ if $i$ is the terminal node of edge $e_k$, $\left[\mat{E}\right]_{ik} = 1$ if $i$ is the initial node of edge $e_k$, and $\left[\mat{E}\right]_{ik} = 0$ otherwise.
Let $\mat{x}_1\T = \left[\mat{x}_{1, 1}\T\,\cdots\,\mat{x}_{1, N}\T\right]$ be the collection of the hand-position coordinates of all the agents of the system, and let $\mat{z}_1\T = [\mat{z}_{1, 1}\T\cdots \mat{z}_{1, M}\T]\T$ be the collection of all the relative positions between the pairs of neighboring agents.
Then, we can express the edge states in the following compact form
\begin{equation}\begin{aligned}\label{eq:z1-def}
\mat{z}_1= [\mat{E}\T \otimes \mat{I}_3] \mat{x}_1.
\end{aligned}\end{equation}
The formation error, in turn, is defined as
\begin{equation}\begin{aligned}\label{eq:tildez1-def}
\tilde{\mat{z}}_1=[\mat{E}\T \otimes \mat{I}_3] \mat{x}_1 - \mat{z}_1^d, \qquad
\mat{z}_1^{d, {\rm T}} = \left[\mat{z}_{1, 1}^{d{\rm T}}\cdots \mat{z}_{1, M}^{d{\rm T}}\right]
\end{aligned}\end{equation}
where $\mat{z}_{1, k}^d\in\mathbb{R}^{3}$ denotes the desired relative position between a pair of neighboring agents over edge $e_k$.
In the same way, let $\mat{x}_2\T=\left[\mat{x}_{2, 1}\T\,\cdots\,\mat{x}_{2, N}\T\right]$ be the collection of the hand-position velocities.
Then, the relative hand-position velocities in the edge coordinates, $\mat{z}_2$, are given by
\begin{equation}\begin{aligned}\label{eq:z2-def}
\mat{z}_2=[\mat{E}\T \otimes \mat{I}_3] \mat{x}_2.
\end{aligned}\end{equation}
Let us also define the collection of the control inputs $\bs{\mu}\T=\left[\bs{\mu}_1\T\,\cdots\,\bs{\mu}_N\T\right]$.
Then, mathematically, the tracking-in-formation problem translates into designing a distributed controller $\bs{\mu}$ such that 
\begin{subequations}\label{n194}\begin{align}
	\lim\limits_{t \rightarrow \infty} \tilde{\mat{z}}_{1, o}(t) &= \mat{0} & \lim\limits_{t \rightarrow \infty} \mat{z}_{2, o}(t) &= \mat{0} \label{n194-a}\\
	\lim\limits_{t \rightarrow \infty} \tilde{\mat{z}}_1(t) &= \mat{0} & \lim\limits_{t \rightarrow \infty} \mat{z}_2(t) &= \mat{0}. \label{n194-b}
\end{align}\end{subequations}

Now, as observed in \cite{restrepo_edge-directed_2020}, using an appropriate labeling of the edges, the incidence matrix can be expressed as $\mat{E} = \left[\mat{E}_t\quad \mat{E}_c\right],$
where $\mat{E}_t\in\mathbb{R}^{N\times(N-1)}$ denotes the full column-rank incidence matrix corresponding to an arbitrary spanning tree $\mathcal G_t \subset \mathcal G$ and $\mat{E}_c \in \mathbb{R}^{N\times(M-N+1)}$ represents the incidence matrix corresponding to the remaining edges in $\mathcal G \setminus \mathcal G_t$.
Similarly, the error edge states may be split as $\mat{z}_\iota=\left[\mat{z}_{\iota, t}\T\;\;\mat{z}_{\iota, c}\T\right]\T$, $\iota=\{1,2\}$, where $\mat{z}_{\iota, t}\in\mathbb{R}^{3(N-1)}$ are the states corresponding to the edges of $\mathcal G_t$ and $\mat{z}_{\iota, c} \in \mathbb{R}^{n(M-N+1)}$ denote the states of the remaining edges, $\mathcal G \setminus \mathcal G_t$.
Note that the spanning tree and the error variables associated with its edges are sufficient to describe the errors of the multi-agent system.
Indeed, defining the following transformation matrix
\begin{equation}\label{n205}
\mathcal{R} = \left[\mat{I}_{N-1}\quad \mat{T}\right], \quad \mat{T}=\left(\mat{E}_t\T \mat{E}_t\right)^{-1}\mat{E}_t\T \mat{E}_c,
\end{equation}
we get the following identities
\begin{equation}\label{n209}
\mat{E} = \mat{E}_t \mathcal{R}, \quad \mat{z}_{\iota} = \big[\mathcal{R}\T\otimes \mat{I}_3\big] \tilde{\mat{z}}_{\iota, t}, \;\; \iota\in\{1,2\}.
\end{equation}
We have thus shown that is is possible to obtain the incidence matrix and the error variables corresponding to the original graph from its spanning tree.

%Then, akin to \eqref{n209}, after \eqref{eq:tildez1-def}, \eqref{eq:z2-def}, \eqref{n209}, and \eqref{n203}, denoting $z_{1t}^d\in\mathbb{R}^{n(N-1)}$ as the vector of desired relative displacements corresponding to $\mathcal G_t$, we have
%\begin{equation}
%\label{n212} \tilde z_1 = \big[R\T\otimes \mat{I}_3\big] \tilde z_{1t}, \quad z_2 = \big[R\T\otimes \mat{I}_3\big] z_{2t}.
%\end{equation}
Then, using \eqref{n209} and denoting $\mat{z}_{1, t}^d\in\mathbb{R}^{n(N-1)}$ as the vector of desired relative displacements corresponding to $\mathcal G_t$, a reduced-order model for the external dynamics in terms of the edges of a spanning tree is given by
\begin{subequations}\allowdisplaybreaks\label{n216}\begin{align}
	\dot{\mat{z}}_{1, o}&=\mat{z}_{2, o}+\ocean\label{n217-t}\\
	\dot{\tilde{\mat{z}}}_{1, t} &=\mat{z}_{2, t}\label{n217}\\
	\dot{\mat{z}}_{2, o}&=\bs{\mu}_1-\bs{\mu}_o(t)\label{n218-t}\\
	\dot{\mat{z}}_{2, t}&=\big[\mat{E}_t\T\otimes \mat{I}_3\big]\bs{\mu}.\label{n218}
	\end{align}\end{subequations}
In these coordinates, the control objective as defined in \eqref{n194} is achieved if the origin of system \eqref{n216} is asymptotically stabilized.
More precisely, we consider the following problem. 

{\it Tracking-in-formation under hard and soft constraints:} Consider a system of $N$ autonomous marine vehicles with dynamics given by \eqref{eq:handpos_tracking_hand_position_dynamics}, interacting over a directed graph which is either a spanning tree or a cycle.
Assume, in addition, that the agents are subject to the hard inter-agent output constraints given by the set defined in \eqref{eq:setDk} and the soft constraints given by \eqref{eq:setC}.
Under these conditions, find distributed controllers $\bs{\mu}_i$, $i\leq N$, that asymptotically stabilize the origin of \eqref{n216} and render the sets \eqref{eq:setDk} and \eqref{eq:setC} forward invariant, \emph{i.e.}, $\mat{z}_{1, k}(t_0)\in\mathcal{D}_k$ ($u_{r, i}(t_0)\in\mathcal{C}_i$) implies that $\mat{z}_{1, k}(t)\in \mathcal{D}_k$ ($u_{r, i}(t)\in\mathcal{C}_i$), $\forall k\leq M$ ($\forall i\leq N$) and $\forall t\geq t_0$.

\section{Control Design for Tracking under Proximity and Safety Constraints}
\label{sec:ctrl}

For clarity of exposition, in this section we begin by presenting the control design considering \emph{only} the hard constraints, i.e. connectivity maintenance and collision avoidance. The inclusion of the soft constraints is addressed in Section~\ref{sec:soft}.

We will show how the tracking-in-formation problem, with the previously formulated inter-agent constraints, can be solved following a backstepping approach which is well adapted to the normal form of the external dynamics \eqref{n216}. We start by defining a virtual control law for \eqref{n217-t}-\eqref{n217} with $\mat{z}_{2, o}$ and $\mat{z}_{2, t}$ as inputs. 
In order to account for the output constraints, a good choice of control design for the virtual inputs consists in using the gradient of a \acrfull{blf} \cite{tee_barrier_2009}.

\subsection{On Barrier Lyapunov Functions}

\Glspl{blf} are reminiscent of Lyapunov functions in that they are positive definite, but their domain is restricted by design to open subsets of the Euclidean space and they grow unbounded as their argument approaches the boundary of their domain.
We define them as follows, \emph{c.f.,} \cite{tee_barrier_2009}.
\begin{dfn}[BLF]\label{def:TAC-barrier-function}
	Consider the system $\dot{x} = f(x)$ and let $\mathcal M \subset \mathbb{R}^n$ be an open set containing the origin.
	A \gls{blf} is a positive definite function $V : \mathcal M \mapsto \mathbb{R}_{\geq 0}$ that satisfies
	\[%% \dot V(x)=
	\nabla V(x)\T f(x) = \frac{\partial V(x)}{\partial x}\T f(x)\leq 0,\]	
	and $V(x) \rightarrow \infty$ and $\norm{\nabla V(x)} \rightarrow \infty$ as $x \rightarrow \partial \mathcal M$. 
\end{dfn}

Akin to \eqref{eq:setDk}, the inter-agent constraints in terms of the formation error are given, for all $k\leq M$, by the set
\begin{equation}\label{n367}
\tilde{\mathcal{D}}_k=\{\tilde{\mat{z}}_{1, k}\in\mathbb{R}^{3}\,|\,\delta_k<\norm{\tilde{\mat{z}}_{1, k}+\mat{z}_{1, k}^d}<\Delta_k\}.
\end{equation}
Then, for each $k\leq M$, we define a candidate \gls{blf} $W_k:\mathcal{\tilde D}_k\mapsto\mathbb{R}_{\geq 0}$, of the form
\begin{equation}\label{317}
W_k(\tilde{\mat{z}}_{1, k}) = \frac{1}{2}\left[\norm{\tilde{\mat{z}}_{1, k}}^2+B_k(\tilde{\mat{z}}_{1, k}+\mat{z}_{1, k}^d)\right],
\end{equation}
where 
\begin{equation*}\label{520}\begin{aligned}
B_k(\mat{z}_{1, k}) =&\; 
\kappa_{1, k}\left[\ln\left(\frac{\Delta_k^2}{\Delta_k^2-\norm{\mat{z}_{1, k}}^2}\right)-\ln\left(\frac{\Delta_k^2}{\Delta_k^2-\norm{\mat{z}_{1, k}^d}^2}\right)\right]
\\&\hspace{-2mm}+\kappa_{2, k}\left[\ln\left(\frac{\norm{\mat{z}_{1, k}}^2}{\norm{\mat{z}_{1, k}}^2-\delta_k^2}\right)-\ln\left(\frac{\norm{\mat{z}_{1, k}^d}^2}{\norm{\mat{z}_{1, k}^d}^2-\delta_k^2}\right)\right], 
\end{aligned}\end{equation*}
\begin{equation*}\label{379}
\kappa_{1, k} = \frac{\delta_k^2}{\norm{\mat{z}_{1, k}^d}^2(\norm{\mat{z}_{1, k}^d}^2-\delta_k^2)},\quad \kappa_{2, k}=\frac{1}{\Delta_k^2-\norm{\mat{z}_{1, k}^d}^2}.
\end{equation*}

Note that $B_k$ is a non-negative function that satisfies: ${B}_k(\mat{z}_{1, k}^d)\! =\! 0$, $\nabla\!{B}_k(\mat{z}_{1, k}^d)\! =\! \mat{0}$, and ${B}_k(\tilde{\mat{z}}_{1, k}+\mat{z}_{1, k}^d)\rightarrow\infty$ as either $\norm{\tilde{\mat{z}}_{1, k}+\mat{z}_{1, k}^d}\rightarrow\Delta_k$ or $\norm{\tilde{\mat{z}}_{1, k}+\mat{z}_{1, k}^d}\rightarrow\delta_k$.
Therefore, the candidate \gls{blf} \eqref{317} satisfies: $W_k(\tilde{\mat{z}}_{1, k})\rightarrow\infty$ as either $\norm{\tilde{\mat{z}}_{1, k}+\mat{z}_{1, k}^d}\rightarrow\Delta_k$ or $\norm{\tilde{\mat{z}}_{1, k}+\mat{z}_{1, k}^d}\rightarrow\delta_k$.

\begin{rmk}\label{rmk:V:equilibria}
	The function in \eqref{317} is reminiscent of scalar potential functions in constrained environments. Hence, the appearance of multiple critical points is inevitable \cite{rimon_exact_1992}.
	Indeed, the gradient of the \gls{blf} \eqref{317}, $\nabla W_k(\tilde{\mat{z}}_{1, k})$, vanishes at the origin and at an isolated saddle point separated from the origin. 
	Therefore, when using the gradient of \eqref{317}, the closed-loop system has multiple equilibria.
	We address such technicalities using tools for \textit{multi-stable} systems \cite{forni_cascade_2016,monzon2006local}.
\end{rmk}

Now, we define a \gls{blf} for the multi-agent system as
\begin{equation}\label{512}
W(\tilde{\mat{z}}_{1})=\sum_{k\leq M} \varrho_k W_k(\tilde{\mat{z}}_{1, k}), \quad \varrho_k>0,
\end{equation}
and, in light of Remark~\ref{rmk:V:equilibria}, let $\tilde{\mat{z}}_1^*\in\mathbb{R}^{nM}$ denote the vector containing the saddle points of the \gls{blf} for each edge \eqref{317}.
Moreover, let us define the disjoint set
\begin{equation}\label{517}
\mathcal{W}=\{0\}\cup\{\tilde{\mat{z}}_1^*\},
\end{equation}
which corresponds to the critical points of $W$ in \eqref{317}.
Then, $W$ satisfies
\begin{equation}\label{eq:BLF-bounds}
\frac{a_1}{2}\dst{\tilde{\mat{z}}_{1}}{\mathcal{W}}^2 \leq W(\tilde{\mat{z}}_{1}) \leq a_2\norm{\nabla W(\tilde{\mat{z}}_{1})}^2,
\end{equation}
where $a_{1}, a_2 > 0$ and $\dst{\tilde{\mat{z}}_{1}}{\mathcal{W}} = \min \big\{\norm{\tilde{\mat{z}}_{1}}, \norm{\tilde{\mat{z}}_{1} - \tilde{\mat{z}}_{1}^*}\big\}$.

\subsection{Control Design for Systems over Directed Graphs}

Let us define the so-called in-incidence matrix $\mat{E}_\odot \in \mathbb{R}^{N\times M}$, whose elements are defined as follows: $\left[\mat{E}_\odot\right]_{i, k} = -1$ if $i$ is the terminal node of edge $e_k$ and $\left[\mat{E}_\odot\right]_{ik} = 0$ otherwise. Then, in the edge-agreement framework, the virtual controllers are
\begin{equation}\label{n284}\begin{aligned}[b]
\mat{z}_{2, t}^*&=[\mat{E}_t\T \otimes \mat{I}_3] \mat{x}_2^*\\
\mat{x}_2^*&=-c_1[\mat{E}_\odot\otimes \mat{I}_3] \nabla W(\tilde{\mat{z}}_{1, t})-c_1[\mat{C}\otimes \mat{I}_3]\tilde{\mat{z}}_{1, o} - \oceanhat,
\end{aligned}\end{equation}
where $c_1$ is a positive gain, $\oceanhat$ is a vector of estimates of the ocean current for each agent, and $\mat{C}\T=\left[1\;\;\mat{0}_{N-1}\T\right]$.
To avoid a cumbersome notation we write $\nabla W(\tilde{\mat{z}}_{1, t})$ in place of the more appropriate spelling $\nabla W\left(\left[\mathcal{R}\T\otimes \mat{I}_3\right]\tilde{\mat{z}}_{1, t}\right)$.

% \begin{rmk}
% 	For undirected graphs the virtual control law takes the form
% 	\begin{equation}
% 	\nu_{h}^*=-c_1[E\otimes \mat{I}_3]\nabla W(\tilde z_{1})-c_1[C\otimes \mat{I}_3]\tilde z_{1o}-\hat V,
% 	\end{equation}
% 	where $E$ is the incidence matrix, cf. \cite{restrepo_ECC_2022}.
% \end{rmk}

Defining velocity errors $\tilde{\mat{z}}_{2, t} = \mat{z}_{2, t} - \mat{z}_{2, t}^*$, $\tilde{\mat{z}}_{2, o} = \tilde{\mat{z}}_{2, 1} - \mat{z}_{2, o}$, and using \eqref{n284}, the subsystem \eqref{n217-t}--\eqref{n217} becomes
\begin{equation}\label{553}\begin{aligned}[b]
\left[\begin{matrix}
\dot{\tilde{\mat{z}}}_{1, o}\\\dot{\tilde{\mat{z}}}_{1, t}
\end{matrix}\right]=&-c_1\left[\left[\begin{matrix}
1&\mat{C}\T \mat{E}_t\\\mat{E}_t\T \mat{C}&\mat{E}_t\T \mat{E}_\odot
\end{matrix}\right]\otimes \mat{I}_3\right]\left[\begin{matrix}
{\tilde{\mat{z}}}_{1, o}\\\nabla W({\tilde{\mat{z}}}_{1})
\end{matrix}\right]\\&+\left[\left[\begin{matrix}
\mat{C}\T\\\mat{E}_t\T
\end{matrix}\right]\otimes \mat{I}_3\right]\oceantilde +\left[\begin{matrix}
\tilde{\mat{z}}_{2, o}\\\tilde{\mat{z}}_{2, t}
\end{matrix}\right]
\end{aligned}\end{equation}
where the estimation error $\oceantilde$ is defined as
\begin{equation}\label{548}\begin{aligned}[b]
\oceantilde&=\bar{\mat{V}}_c-\oceanhat,\quad \text{with } \bar{\mathbf{V}}_c = \mat{1}_N\otimes\ocean.
\end{aligned}\end{equation}
With the aim of making $\oceantilde \rightarrow \mat{0}$, we design the adaptation law
\begin{subequations}\label{560}\begin{align}
	\oceanhat &= c_v\left(\mat{x}_1-\bs{\varphi}\right), \quad c_v>0\label{560-a}\\
	\dot{\bs{\varphi}} &= \mat{x}_2 + \oceanhat.\label{560-b}
\end{align}\end{subequations}
Using \eqref{560} and \eqref{eq:handpos_tracking_x_1_dot_transformed}--\eqref{eq:handpos_tracking_x_2_dot_transformed}, the derivative of \eqref{548} becomes
\begin{equation}\label{564}\begin{aligned}[b]
\dot{\tilde{\mat{V}}}_c&= -c_v\left(\mat{x}_2+\bar{\mat{V}}_c-\mat{x}_2-\oceanhat\right)=-c_v\oceantilde.
\end{aligned}\end{equation}

In the error coordinates $\tilde{\mat{z}}_{2, o}$ and $\tilde{\mat{z}}_{2, t}$, we have
\begin{subequations}\label{559}\begin{align}
	\dot{\tilde{\mat{z}}}_{2, o} &= \bs{\mu}_1 - \dot{\mat{x}}_{2, 1}^*\label{559-a}\\
	\dot{\tilde{\mat{z}}}_{2, t} &= \big[\mat{E}_t\T\otimes \mat{I}_3\big]\bs{\mu} - \dot{\mat{z}}_{2, t}^*.\label{559-b}
	\end{align}\end{subequations}
Hence, we design the tracking-in-formation control law as
\begin{equation}\label{n338}\begin{aligned}[b]
\bs{\mu} = &-c_2\big[\mat{E}_\odot \mathcal{R}\T\otimes \mat{I}_3\big]\tilde{\mat{z}}_{2, t}-c_2[\mat{C} \otimes \mat{I}_3]\tilde{\mat{z}}_{2, o} + \dot{\bar{\mat{x}}}_2^*\\&-\gamma\rm{sign}\left(\big[\mat{E}_\odot \mathcal{R}\T\otimes \mat{I}_3\big]\tilde{\mat{z}}_{2, t}+[\mat{C} \otimes \mat{I}_3]\tilde{\mat{z}}_{2, o}\right) + \bs{\mu}^*(t)
\end{aligned}\end{equation}
where $c_2,\gamma>0$, ${\bar{\mat{x}}}_2^*=-c_1[\mat{E}_\odot\!\otimes\! \mat{I}_3]\nabla W(\tilde{\mat{z}}_{1})-c_1[\mat{C}\!\otimes\! \mat{I}_3]\tilde{\mat{z}}_{1, o}$. 
The signal $\bs{\mu}^*(t)\T=\left[\bs{\mu}_1^*(t)\;\cdots\;\bs{\mu}_N^*(t)\right]$, satisfying $\norm{\bs{\mu}^*(t)}\leq\bar\mu^*$ for a constant $\bar\mu^*$, is given by
\begin{equation}\label{417}
\bs{\mu}_i^*(t)=\mat{R}_i\T f_{u, i}^* \mat{e}_1,
\end{equation}
where $f_{u, i}^*$ is an additional bounded control input that will be used to deal with the soft constraints, \emph{c.f.,} Section~\ref{sec:soft}. 

\subsection{Closed-Loop Analysis}

First, let us define $\bs{\varsigma}_1\T = \left[\tilde{\mat{z}}_{1, o}\T\;\;\tilde{\mat{z}}_{1, t}\T\right]$, and $\bs{\varsigma}_2\T=\left[\tilde{\mat{z}}_{2, o}\T\;\;\tilde{\mat{z}}_{2, t}\T\right]$.
We note that the control goal defined in \eqref{n194} is equivalent to $\lim\limits_{t \rightarrow \infty} \bs{\varsigma}_1 = \mat{0}$, $\lim\limits_{t \rightarrow \infty} \bs{\varsigma}_2 = \mat{0}$.
Consequently, $\bs{\varsigma}_1$ and $\bs{\varsigma}_2$ are valid error variables.
From \eqref{553}, \eqref{559}, and \eqref{n338}, the closed-loop dynamics are
\begin{subequations}\label{n560}\begin{align}
	\dot{\bs{\varsigma}}_1 =& -c_1\mathcal{L}_1\bar{\bs{\varsigma}}_1+\bs{\varsigma}_2+\mathcal{T}_1\tilde{\mat{V}}_c, \label{n560-a}\\
	\dot{\bs{\varsigma}}_2 =& -c_2\mathcal{L}_2\bs{\varsigma}_{2\!}+\!c_{v\!}\mathcal{T}_1\!\tilde{\mat{V}}_c\!+\!\mathcal{T}_1\!\left[\bs{\mu}^*\!(t)-\mathbf{1}_{N}\!\otimes\!\bs{\mu}_o(t)\right]\!-\!\gamma\mathcal{T}_1{\rm sign}\!\left(\mathcal{T}_2\T\!\bs{\varsigma}_2\right)\!\mathrlap{,}\label{n560-b}\\
	\dot{\tilde{\mat{V}}}_c =& -c_v\tilde{\mat{V}}_c, \label{n560-c}
\end{align}\end{subequations}	
where
\begin{subequations} \label{570} \begin{align}
	\bar\varsigma_1\T &= \left[\tilde{\mat{z}}_{1, o}\;\;\nabla W(\tilde{\mat{z}}_{1})\T \right], & & \label{570-a} \\
	\mathcal{L}_1&=\left[\left[\begin{matrix}
		1&\mat{C}\T \mat{E}_t\\\mat{E}_t\T \mat{C}&\mat{E}_t\T \mat{E}_\odot
		\end{matrix}\right]\otimes \mat{I}_3\right],& 
		\mathcal{T}_1 &=\left[\left[\begin{matrix}
		\mat{C}\T\\\mat{E}_t\T
		\end{matrix}\right]\!\otimes\! \mat{I}_3\right] \label{570-b} \\
		\mathcal{L}_2&=\left[\left[\begin{matrix}
		1&\mat{C}\T \mat{E}_t\\\mat{E}_t\T \mat{C}&\mat{E}_t\T \mat{E}_\odot \mathcal{R}\T
		\end{matrix}\right]\otimes \mat{I}_3\right],&
		\mathcal{T}_2&=\left[\left[\begin{matrix}
		\mat{C}\T\\\mathcal{R} \mat{E}_\odot\T
		\end{matrix}\right]\!\otimes\! \mat{I}_3\right]. \label{570-c}
\end{align} \end{subequations}

The first part of the main result is stated as follows:
\begin{prop}\label{thm:USV}
	Consider $N$ \glspl{auv}, each described by the model \eqref{eq:handpos_tracking_hand_position_dynamics}, and interconnected over a directed graph which is either a spanning tree or a cycle.
	Then, under Assumption~\ref{hyp:target} and
	\begin{equation}
		\gamma\geq\bar\mu^*+\sqrt{2N}\bar\mu_o, \label{590}
	\end{equation}
	the controller \eqref{n338} renders the constraints set \eqref{eq:setDk} forward invariant and guarantees the achievement of the tracking-in-formation objective \eqref{n194} for almost all initial conditions such that $\mat{z}_{1, k}(t_0)\in\mathcal{D}_k$, for all $k\leq M$.

	Moreover, let us define $\bs{\xi}_{2, d} = \mat{x}_{2, o}$ and $a_x, \Bar{\alpha}_y, \Bar{\alpha}_z$ in accordance with \eqref{eq:handpos_def_a_bounds}.
	The internal dynamics are ultimately bounded for almost all initial conditions if $a_x, \Bar{\alpha}_y, \Bar{\alpha}_z > 0$.
\end{prop}

\begin{proof}
	We begin by analyzing the external dynamics \eqref{n560}.
	First define the function 
	\begin{equation}\label{n587}
	W_1(\bs{\varsigma}_1)=\frac{1}{2}\norm{\tilde{\mat{z}}_{1, o}}^2 + W(\tilde{\mat{z}}_{1, t}),
	\end{equation}
	where, with a slight abuse of notation, $\tilde{\mat{z}}_{1, t} \mapsto W(\tilde z_{1t})$ is defined in \eqref{512}.
	The derivative of $W_1$ along the trajectories of \eqref{n560-a} yields
	\begin{equation}\label{n744}
	\dot W_1(\bs{\varsigma}_1) =-c_1\bar{\bs{\varsigma}}_1\T\mathcal{L}_1\bar{\bs{\varsigma}}_1+\bar{\bs{\varsigma}}_1\T\bs{\varsigma}_2+\bar{\bs{\varsigma}}_1\T\mathcal{T}_1\tilde{\mat{V}}_c.
	\end{equation}
	Note that for any directed graph containing a spanning tree, $-\mat{E}_t\T \mat{E}_\odot$ is Hurwitz (\emph{c.f.,} \cite[Proposition~1]{restrepo_edge-directed_2020}).
	Consequently, from \eqref{570-b}, we can conclude that there exist $c_1', c_v' > 0$ such that
	\begin{equation}\label{n771}
	\dot W_1(\bs{\varsigma}_1) \leq-c_1'\norm{\bar{\bs{\varsigma}}_1}^2+\norm{\bar{\bs{\varsigma}}_1}\norm{\bs{\varsigma}_2}+c_v'\norm{\bar{\bs{\varsigma}}_1}\norm{\oceantilde}.
	\end{equation}
	
	Moreover, for any directed graph containing a spanning tree, it follows that $-\mat{E}_t\T \mat{E}_\odot \mathcal{R}\T$ is Hurwitz (\emph{c.f.,} \cite{mukherjee_robustness_2018}).
	Consequently, we can define the following candidate Lyapunov function 
	\begin{equation}\label{n745}
	W_2(\bs{\varsigma}_2)=\frac{1}{2}\bs{\varsigma}_2\T \mat{P}\bs{\varsigma}_2
	\end{equation}
	where $\mat{P}$ is a positive definite such that for any positive definite $\mat{Q}$, it holds that $\mathcal{L}_2\T \mat{P} + \mat{P}\mathcal{L}_2=\mat{Q}$.
	Then, the derivative of \eqref{n745} along the trajectories of \eqref{n560-b}, is defined by the differential inclusion $\bs{\varsigma}_2\in F_2(t,\bs{\varsigma}_2)$, where
	\begin{equation*}
	F_2(t,\bs{\varsigma}_2)\!=\!\begin{cases}
	\eqref{n560-b},& \text{if } \mathcal{T}_2\T\bs{\varsigma}_2\neq \mat{0},\\
	-c_2\mathcal{L}_2\bs{\varsigma}_2+c_{v\!}\mathcal{T}_1\oceantilde-\gamma\lambda+\!\mathcal{T}_{1\!}\left[\bs{\mu}^*\!(t)-\mathbf{1}_{N}\!\otimes\!\bs{\mu}_o(t)\right],& \text{if } \mathcal{T}_2\T\bs{\varsigma}_2=\mat{0},
	\end{cases}
	\end{equation*}
	and $\lambda\in\left[-1,1\right]$.
	Thus, using \eqref{590} and the fact that $\norm{s}_1 = s\T{\rm sign}(s)$, the derivative of $W_2$ is
	\begin{align}
	\nonumber\dot W_2(\bs{\varsigma}_2) &=-c_2\bs{\varsigma}_2\T \mat{P}\mathcal{L}_2\bs{\varsigma}_2+c_v\bs{\varsigma}_2\T \mat{P}\mathcal{T}_1\oceantilde-\gamma\bs{\varsigma}_2\T \mat{P}\mathcal{T}_1{\rm sign}\left(\mathcal{T}_2\T\bs{\varsigma}_2\right)\\
	\nonumber&\quad +\bs{\varsigma}_2\T \mat{P}\mathcal{T}_1\left[\bs{\mu}^*(t)-\mathbf{1}_{N}\otimes\bs{\mu}_o(t)\right]\\
	&\leq-c_2'\norm{\bs{\varsigma}_2}^2+c_v''\norm{\bs{\varsigma}_2}\norm{\oceantilde},\label{n775}
	\end{align}
	where $c_2'$ and $c_v''$ are positive constants.
	
	Next, let us define $\bs{\varsigma}\T=\left[\bs{\varsigma}_1\T\;\;\bs{\varsigma}_2\T\;\;\oceantilde\T\right]$ and define the candidate Lyapunov function 
	\begin{equation}\label{n780}
	W_\varsigma(\bs{\varsigma})=W_1(\bs{\varsigma}_1)+\kappa_1W_2(\bs{\varsigma}_2)+\frac{\kappa_2}{2}\norm{\oceantilde}^2,
	\end{equation}
	where $\kappa_1$ and $\kappa_2$ are positive constants.
	From \eqref{n771}, \eqref{n775}, and \eqref{n560-c}, we have
	\begin{equation}\label{n784} \begin{aligned}
	\dot W_\varsigma(\bs{\varsigma}) &\leq -c_1'\norm{\bar{\bs{\varsigma}}_1}^2-\kappa_1c_2'\norm{\bs{\varsigma}_2}^2-\kappa_2c_v\norm{\oceantilde}^2+\norm{\bar{\bs{\varsigma}}_1}\norm{\bs{\varsigma}_2} \\
	&\quad +c_v'\norm{\bs{\varsigma}_1}\norm{\oceantilde}+\kappa_1c_v''\norm{\bs{\varsigma}_2}\norm{\oceantilde}.
	\end{aligned} \end{equation}
	Setting $\kappa_1,\,\kappa_2$ large enough, we can find $\bar{c}_1, \bar{c}_2, \bar{c}_3, \bar{c} > 0$ such that
	\begin{equation}\label{n790}
	\dot W_\varsigma(\bs{\varsigma})\leq -\bar c_1\norm{\bar{\bs{\varsigma}}_1}^2-\bar c_2\norm{\bs{\varsigma}_2}^2-\bar c_v\norm{\oceantilde}^2
		\leq-\bar c\, W_\varsigma(\bs{\varsigma}).
	\end{equation}
	Now, let $\mathcal{W}_\varsigma$ be the set of the equilibria of the closed-loop system \eqref{n560}.
	Recalling Remark~\ref{rmk:V:equilibria}, $\mathcal{W}_\varsigma$ is given by
	\begin{equation}
	\mathcal{W}_\varsigma = \{0\}\times\mathcal{W}\times\{0\}\times\{0\}^{2(N-1)}\times\{0\}^{2N}
	\end{equation}
	where $\mathcal{W}$ is defined in \eqref{517}. Then, from \eqref{eq:BLF-bounds} we have
	\begin{equation}\label{n799}\begin{aligned}
	\dot W_\varsigma(\bs{\varsigma})\leq& -\bar c'\dst{\bs{\varsigma}}{\mathcal{W}_\varsigma}^2.
	\end{aligned}\end{equation}
	Thus, the closed-loop system \eqref{n560} is uniformly asymptotically multi-stable at $\mathcal{W}_\varsigma$, \emph{c.f.,}~\cite{forni_cascade_2016}.
	Furthermore, the critical point $\tilde{\mat{z}}_1^*$ of the barrier Lyapunov function is a saddle point.
	After \cite[Proposition~1]{monzon2006local}, it follows that the region of attraction of the unstable equilibrium $\tilde{\mat{z}}_1^*$ has zero Lebesgue measure. Therefore, we conclude that the origin of \eqref{n560} is {\it almost-everywhere} uniformly asymptotically stable in $\mathbb{D}=\mathbb{R}^3\times\mathcal{\tilde D}\times \mathbb{R}^{3M}\times \mathbb{R}^{3N}$, except for a zero-measure set of initial conditions.
	
	In order to establish forward invariance of the set $\mathcal{\tilde D}$ we proceed by contradiction.
	Assume that there exists $T>0$ such that $\tilde{\mat{z}}_{1, k}(t) \in \mathcal{\tilde D}_k$ for all $t \in [t_0,t_0+T)$, but $\tilde{\mat{z}}_{1, k}(t_0+T) \notin \mathcal{\tilde D}_k$ for at least one $k\leq M$.
	In other words, we have $\norm{\mat{z}_{1, k}(t)}\rightarrow \Delta_k$ or $\norm{\mat{z}_{1, k}(t)}\rightarrow \delta_k$ as $t\rightarrow t_0+T$ for at least one ${k\leq M}$.
	From the definition of $\tilde{\mat{z}}_{1, t}\mapsto W(\tilde{\mat{z}}_{1, t})$ in \eqref{512} and $\tilde{\mat{z}}_{1, k}\mapsto W_k(\tilde{\mat{z}}_{1, k})$ in \eqref{317}, this implies that $W_\varsigma(\bs{\varsigma}(t)) \rightarrow\infty$ as $t\rightarrow t_0+T$ which is in contradiction with \eqref{n790}.
	We can therefore conclude that $W_\varsigma(\bs{\varsigma}(t))$ is bounded for all initial conditions such that $\tilde{\mat{z}}_{1}(t_0)\in \mathcal{\tilde D}$, therefore, $W_\varsigma(\bs{\varsigma}(t))\leq W_\varsigma(\bs{\varsigma}(t_0)) <\infty$ for all $\bs{\varsigma}(t_0)\in\mathbb{D}$ and all $t\geq t_0$.
	The respect of the inter-agent constraints follows from the forward invariance of $\mathcal{\tilde D}$.
	
	Since \eqref{n560} is asymptotically stable at the origin with domain of attraction $\mathbb{D}$ it follows that for almost all initial conditions $\bs{\varsigma}(t_0)\in\mathbb{D}$, there exist small positive constants $\underline\epsilon(\bs{\varsigma}(t_0))$ and $\overline\epsilon(\bs{\varsigma}(t_0))$ such that $\tilde{\mat{z}}_{1, k}(t)\in\mathcal{\tilde D}_{\epsilon k}$, where
	\begin{equation} \label{n800}
	\mathcal{\tilde D}_{\epsilon k}\!=\!\left\{\tilde{\mat{z}}_{1, k}\in\mathbb{R}^{3}\,|\,\delta_k+\underline\epsilon\leq\norm{\tilde{\mat{z}}_{1, k}+\mat{z}_{1, k}^d}\leq\Delta_k-\overline\epsilon\right\},\; \forall k\leq M.
	\end{equation}
	Moreover, for any $\tilde{\mat{z}}_{1}\in\mathcal{\tilde D}_{\epsilon}$, with $\mathcal{\tilde D}_{\epsilon}=\bigcap_{k\leq M} \mathcal{\tilde{D}}_{\epsilon k}$, we have that the \gls{blf} $W$ in \eqref{512} satisfies		
	\begin{equation}\label{629}
	\frac{a_1}{2}\dst{\tilde{\mat{z}}_{1. t}}{\mathcal{W}}^2\leq W(\tilde{\mat{z}}_{1, t})\leq \frac{a_2'}{2}\dst{\tilde{\mat{z}}_{1, t}}{\mathcal{W}}^2.
	\end{equation}	
	Therefore, from \eqref{629} and \eqref{n790}, we conclude that for almost all initial conditions $\bs{\varsigma}(t_0)\in\mathbb{D}$, the trajectories $\bs{\varsigma}(t)$ of the external dynamics converge to the origin exponentially.

	The ultimate boundedness of the internal dynamics can then be proven using Lemma~\ref{lemma:handpos_def_ultimate_boundedness}.
	We showed that for almost all initial conditions $\bs{\varsigma}(t_0)\in\mathbb{D}$, there exist small positive constants $\underline\epsilon(\bs{\varsigma}(t_0))$ and $\overline\epsilon(\bs{\varsigma}(t_0))$ such that $\tilde{\mat{z}}_{1, k}(t)\in\mathcal{\tilde D}_{\epsilon k}$, with $\mathcal{\tilde D}_{\epsilon k}$ defined in \eqref{n800}.
	Consequently, for almost all initial conditions $\bs{\varsigma}(t_0)\in\mathbb{D}$, the input $\bs{\mu}$ defined in \eqref{n338} is bounded.
	Therefore, if $a_x, \Bar{\alpha}_y, \Bar{\alpha}_z > 0$, then all assumptions of Lemma~\ref{lemma:handpos_def_ultimate_boundedness} are satisfied, and the internal dynamics are ultimately bounded.
\end{proof}

\section{Control Design for Tracking with Hard and Soft Constraints}
\label{sec:soft}

In this section we build on the results of Section~\ref{sec:ctrl} to include the soft constraints defined in \eqref{eq:soft} that act on the surge velocity of the marine vehicles.
For this purpose we explicitly design the additional control input $f_{u, i}^*$ introduced in Eq. \eqref{417} and we analyze the closed-loop system in terms of the barrier function framework.

%\subsection{Control design for soft-constraints satisfaction}

Consider the surge velocity subsystem \eqref{eq:background_component_form_u_dot} with an additional control input. That is,
\begin{equation}\label{622}
	\dot{u}_{r, i} = F_{u}(\vel_{r, i}) + f_{u, i} + f_{u, i}^*.
\end{equation}
In order to satisfy the soft constraints, we design the additional input $f_{u, i}^*$ as the gradient of a barrier function.
For each agent $i$, define the barrier function 
\begin{equation}\label{628}
	U_i(t,u_{r, i}):=-\ln\left(\frac{u_{r, i} + \rho_i(t)}{u_{r, i} + \rho_i(t) +1}\right).
\end{equation}
Note that if $u_{r, i} + \rho_i(t)>0$, then $U_i(t,u_{r, i})>0$ for all $t\geq0$, and $U_i(t,u_{r, i})\rightarrow\infty$ as $u_{r, i} + \rho_i(t)\rightarrow0$.
Then, we set the additional control input to
\begin{equation}\label{630}
 f_{u, i}^*= -\kappa_u\nabla U_i(t,u_{r, i}) = -\kappa_u\frac{\partial U_i(t,u_{r, i})}{\partial u_{r, i}}, \quad i\leq N,
\end{equation}
with $\kappa_u>0$ and
\begin{equation}\label{631}
\dot{\rho}_i= -\kappa_\rho \rho_i + \frac{1}{2}\left[1-{\rm sign}\left(\sigma-\abs{F_{u}(\vel_{r, i})+f_{u, i}}\right)\right]\abs{F_{u}(\vel_{r, i})+f_{u, i}},
\end{equation}
where $\kappa_\rho,\;\sigma>0$ are design constants.
Initially, we set $\rho_i(t_0)=0$.

\begin{rmk}\label{rmk:rho}
	Note that under \eqref{631} and the initial condition $\rho_i(t_0)=0$, we have that $\rho_i(t)\geq 0$, for all $t\geq t_0$. To see this, note that second term on the right-hand side of \eqref{631} is always positive. Therefore, $\dot{\rho}_i(t)\geq -\kappa_\rho \rho_i(t)$, which means that the set $\mathcal{C}_\rho=\{\rho_i\in\mathbb{R}\, | \,\rho_i\geq 0\}$ is forward invariant.
\end{rmk}

\begin{rmk}
	The definition of \eqref{631} is loosely inspired by the framework developed in \cite{mehdifar2022funnel} to deal with hard and soft constraints in the setting of prescribed-performance control of single-agent systems.
	The signal $\rho_i(t)$ adjusts the soft constraints whenever the hard constraints become conflicting. 
	Note that when the term $\abs{F_{u}+f_{u, i}}$ is less than or equal to a given positive constant $\sigma$, it means that the edges connected to vehicle $i$ are far from the border of the set $\tilde{\mathcal{D}}_k$, since under the barrier-function-based law \eqref{n338} the controller $f_{u, i}$ grows unbounded as $\mat{z}_{1, k}\rightarrow\partial\tilde{\mathcal{D}}_k$ for any $k\leq M$.
	In this case, the second term on the right-hand side of \eqref{631} is equal to zero. Hence, assuming that in an interval $t\in[t_0,t_0+T]$, $\abs{F_{u}+f_{u, i}}\leq \sigma$, then \eqref{eq:soft}, with $\rho(t)=0$, corresponds to a positive-velocity constraint.
	Conversely, when $\abs{F_{u}+f_{u, i}}>\sigma$, the right-hand side of \eqref{631} becomes positive and $\rho_i$ grows. Hence, $u_{r, i}$ may take negative values, \emph{i.e.,} $u_{r, i}>-\rho_i(t)$, avoiding possible conflicts between the constraints. Then, as the vehicles move away from the border of the set $\tilde{\mathcal{D}}_k$, $\abs{F_{u}+f_{u, i}}\leq\sigma$ again and $\rho_i(t)\rightarrow0$ exponentially fast, recovering the non-negativity constraint.
\end{rmk}

Then, the second part of our main result is stated as follows:
\begin{prop}\label{thm:USV-2}
	Consider $N$ \glspl{auv}, each described by the model \eqref{eq:handpos_tracking_hand_position_dynamics}, and interconnected over a directed graph which is either a spanning tree or a cycle.
	Then, under the same assumptions as in Proposition~\ref{thm:USV}, and with initial conditions such that $\mat{z}_{1, k}(t_0)\in\mathcal{D}_k$ for all $k\leq M$ and $u_{r, i}(t_0)\in\mathcal{C}_i$ for all $i\leq N$, the controller \eqref{n338}, with $f_{u, i}^*$ given by \eqref{630}, achieves the tracking-in-formation objective \eqref{n194} almost everywhere and renders the constraints sets \eqref{eq:setDk} and \eqref{eq:setC} forward invariant.
	Moreover, the internal dynamics are ultimately bounded.
\end{prop}

\begin{proof}
	In Proposition~\ref{thm:USV} we established that the controller \eqref{n338} with the additional bounded input $f_{u, i}^*$	renders the hard-constraints set \eqref{eq:setDk} forward invariant and guarantees the achievement of the tracking-in-formation objective \eqref{n194} for almost all initial conditions such that $\mat{z}_{1, k}(t_0)\in\mathcal{D}_k$, for all $k\leq M$.
	Therefore, to prove Proposition~\ref{thm:USV-2}, what remains is to show that $f_{u, i}^*$ given by \eqref{630} is bounded and guarantees the satisfaction of the soft constraints \eqref{eq:posvel}.	
	For that purpose, consider the barrier function \eqref{628}, whose derivative along the trajectories of \eqref{622} in $\mathcal{C}_i$ yields 
	\begin{equation}\label{667}
		\dot{U}_{i}(t,u_{r, i}) = \nabla U_i(t,u_{r, i})\left(-\kappa_u\nabla U_i(t,u_{r, i}) + F_{u}(\vel_{r, i}) + f_{u, i} + \dot{\rho}_i(t)\right).
	\end{equation}
	
	Now, in view of \eqref{631}, we split the analysis into two cases.
	
	\noindent \emph{Case 1} ($\abs{F_{u}+f_{u, i}}\leq \sigma$): in this case \eqref{667} becomes
	\begin{equation}\label{677}\begin{aligned}
	\dot{U}_{i}(t,u_{r, i})\leq&-\kappa_u\abs{\nabla U_i(t,u_{r, i})}^2+\abs{\nabla U_i(t,u_{r, i})}\left[\sigma - \kappa_\rho\rho_i(t)\right]\\
	\leq& -\kappa_u'\abs{\nabla U_i(t,u_{ri})}^2+\lambda_\sigma\sigma^2+\frac{\kappa_\rho\rho_i(t)}{(u_{r, i} + \rho_i(t))(u_{r, i} + \rho_i(t) +1)},
	\end{aligned}\end{equation}
	with $\kappa_u',\;\lambda_\sigma>0$.
	Since $\rho_i(t)$ is non-negative in $\mathcal{C}_i$ for all $t\geq t_0$, \emph{c.f.,} Remark~\ref{rmk:rho}, the last term on the right-hand side of \eqref{677} is bounded by a constant $\lambda_\rho>0$. Therefore, we have 
	\begin{equation}\label{681}
	\dot{U}_{i}(t,u_{r, i})\leq-\kappa_u\abs{\nabla U_i(t,u_{r, i})}^2+\lambda_\sigma\sigma^2+\lambda_\rho.
	\end{equation}
	
	\noindent \emph{Case 2} ($\abs{F_{u}+f_{u, i}}> \sigma$): for $u_{r, i}\in\mathcal{C}_i$, \eqref{667} becomes
	\begin{equation}\label{687}\begin{aligned}
	\dot{U}_{i}(t,u_{r, i})\leq&\abs{\nabla U_i(t,u_{r, i})}\left[\abs{F_{u}+f_{u, i}} - \kappa_\rho\rho_i(t) - \abs{F_{u}+f_{u, i}}\right]\\
	& -\kappa_u\abs{\nabla U_i(t,u_{r, i})}^2 \\
	\leq& -\kappa_u\abs{\nabla U_i(t,u_{r, i})}^2+\lambda_\rho.
	\end{aligned}\end{equation}
	
	From \eqref{681}-\eqref{687} we conclude that for $u_{r, i}\in\mathcal{C}_i$, the function $U_i(t,u_{r, i})$ is bounded along the trajectories. Therefore, akin to the forward invariance of $\mathcal{\tilde D}$ established in the proof of Proposition~\ref{thm:USV}, it is straightforward to show forward invariance of $\mathcal{C}_i$.
	Hence, there exists a constant $\bar{f}^*$ such that $\abs{f_{u, i}(t)}\leq\bar{f}^*$ for all $t\geq t_0$.

	Note that the additional input $f_{u, i}^*$ does not affect the internal dynamics. Hence, the analysis of the internal dynamics presented in the proof of Proposition~\ref{thm:USV} still holds under the action of $f_{u, i}^*$.	
\end{proof}

\section{Simulations}
\label{sec:simulations}

In this section we illustrate the performance of the controller \eqref{n338} through simulations in MATLAB and \gls{dune} \cite{dune}.
The MATLAB simulation enables us to validate the closed-loop behavior under ideal conditions.
The \acrfull{dune} is a software platform designed to run on autonomous underwater vehicles. It also contains a high-fidelity \gls{auv} simulator, allowing us to validate the proposed control algorithm, reproducing as closely as possible a laboratory experiment.

%\textcolor{red}{Could you also add/modify this last sentence on DUNE?}

The simulation case consists in the tracking-in-formation problem for six \glspl{auv} subject to hard (proximity and collision-avoidance) and soft (positive surge velocity) constraints.
We assume that the vehicles are interconnected at the initial time and that the controller must preserve this connectivity.
We further assume that the \glspl{auv} interact over a directed spanning tree illustrated in \figref{fig:topo} and that only \gls{auv} 1 has knowledge of the (relative) state of the target (labeled 0).

\begin{figure}[b]
	\centering
	\begin{subfigure}{0.45\textwidth}
	\centering
	\begin{tikzpicture}[auto, scale=0.85]%, node distance=34mm, thick]
	\small
	\tikzset{mynode/.style={circle,draw,minimum size=14pt,inner sep=0pt,thick},
		myarrow/.style={-, >=latex', shorten >=0pt, line width=0.6pt},
		myarrow_dir/.style={->, >=latex', shorten >=0pt, line width=0.6pt},}
	
	%    \node at (0,-1) {$\bullet$};
	\node[mynode] (n0) at (0.65,0.38) {$0$};
	\node[mynode] (n1) at (0,-0.4) {$1$};
	\node[mynode] (n2) at (1.3,-0.4) {$2$};    
	\node[mynode] (n3) at (-1.1,.38) {$3$};
	\node[mynode] (n4) at (-1.1,-.4) {$4$};
	\node[mynode] (n5) at (2.4,.38) {$5$};
	\node[mynode] (n6) at (2.4,-.4) {$6$};
	
	\path[myarrow_dir] (n1) edge node [yshift=11.5pt, xshift=15pt] {$e_2$} (n3);
	\path[myarrow_dir] (n1) edge node {$e_3$} (n4);
	\path[myarrow_dir] (n1) edge node [yshift=-11.5pt] {$e_1$} (n2);
	\path[myarrow_dir] (n2) edge node {$e_4$} (n5);
	\path[myarrow_dir] (n2) edge node [yshift=-11.5pt, xshift=0pt] {$e_5$} (n6);
	\path[myarrow_dir] (n0) edge node [yshift=5.4pt, xshift=0pt]{$e_0$} (n1);
	
	\end{tikzpicture}
	\vspace{5mm}
	\caption{Interaction topology. \label{fig:topo}}
	\vspace*{-3mm}
	\end{subfigure}
	\hspace*{0.05\textwidth}
	\begin{subfigure}{0.45\textwidth}
		\def\svgwidth{\textwidth}
		\import{figures/handpos_tracking}{formation.pdf_tex}
		\vspace*{-6mm}
		\caption{Desired formation shape. \label{fig:handpos_formation}}
		\vspace*{-3mm}
	\end{subfigure}
	\caption{Illustrations of the simulated example.}
\end{figure}

\subsection{MATLAB Example}
\pgfplotsset{table/search path={figures/handpos_tracking/data}}
\begin{figure}[p]
    \centering
    \begin{subfigure}{0.7\textwidth}
        \centering
        \def\svgwidth{\textwidth}
        \import{figures/handpos_tracking}{sim_trajectory.pdf_tex}
        \caption{The 3D trajectory of the \glspl{auv}. The black line represents the virtual target.}
        \label{fig:handpos_tracking_trajectory}
    \end{subfigure}  
    \begin{subfigure}{0.47\textwidth}
        % This file was created by matlab2tikz.
%
%The latest updates can be retrieved from
%  http://www.mathworks.com/matlabcentral/fileexchange/22022-matlab2tikz-matlab2tikz
%where you can also make suggestions and rate matlab2tikz.
%
\definecolor{mycolor1}{rgb}{0.24220,0.15040,0.66030}%
\definecolor{mycolor2}{rgb}{0.27800,0.35560,0.97770}%
\definecolor{mycolor3}{rgb}{0.15400,0.59020,0.92180}%
\definecolor{mycolor4}{rgb}{0.07040,0.74570,0.72580}%
\definecolor{mycolor5}{rgb}{0.50440,0.79930,0.34800}%
\definecolor{mycolor6}{rgb}{0.98710,0.73475,0.24375}%
%
\begin{tikzpicture}

\begin{axis}[%
width=0.8\textwidth,
height=20mm,
scale only axis,
xmin=0,
xmax=300,
xlabel style={font=\color{white!15!black}, yshift=1.5mm},
xlabel={$t$ [s]},
ymin=0,
ymax=90,
ylabel style={font=\color{white!15!black}, yshift=-1.5mm},
ylabel={$\|\tilde{\mat{z}}_{1, k}\|$ [m]},
axis background/.style={fill=white},
title style={font=\bfseries, yshift=-2mm},
title={Formation tracking errors},
legend style={legend cell align=left, align=left, draw=white!15!black},
legend columns=2,
xmajorgrids,
ymajorgrids
]
\addplot [color=mycolor1, line width=1.0pt]
  table[]{sim_formation_errors-1.tsv};
\addlegendentry{$\text{e}_\text{1}$}

\addplot [color=mycolor2, line width=1.0pt]
  table[]{sim_formation_errors-2.tsv};
\addlegendentry{$\text{e}_\text{2}$}

\addplot [color=mycolor3, line width=1.0pt]
  table[]{sim_formation_errors-3.tsv};
\addlegendentry{$\text{e}_\text{3}$}

\addplot [color=mycolor4, line width=1.0pt]
  table[]{sim_formation_errors-4.tsv};
\addlegendentry{$\text{e}_\text{4}$}

\addplot [color=mycolor5, line width=1.0pt]
  table[]{sim_formation_errors-5.tsv};
\addlegendentry{$\text{e}_\text{5}$}

\addplot [color=mycolor6, line width=1.0pt]
  table[]{sim_formation_errors-6.tsv};
\addlegendentry{$\text{e}_\text{0}$}

\end{axis}
\end{tikzpicture}%

        \caption{Norms of the formation-keeping errors, $\tilde{\mat{z}}_{1, k}$, and the tracking error, $\tilde{\mat{z}}_{1, o}$.}
        \label{fig:handpos_tracking_formation_errors}
    \end{subfigure}
    \begin{subfigure}{0.47\textwidth}
        % This file was created by matlab2tikz.
%
%The latest updates can be retrieved from
%  http://www.mathworks.com/matlabcentral/fileexchange/22022-matlab2tikz-matlab2tikz
%where you can also make suggestions and rate matlab2tikz.
%
\definecolor{mycolor1}{rgb}{0.24220,0.15040,0.66030}%
\definecolor{mycolor2}{rgb}{0.27800,0.35560,0.97770}%
\definecolor{mycolor3}{rgb}{0.15400,0.59020,0.92180}%
\definecolor{mycolor4}{rgb}{0.07040,0.74570,0.72580}%
\definecolor{mycolor5}{rgb}{0.50440,0.79930,0.34800}%
\definecolor{mycolor6}{rgb}{0.98710,0.73475,0.24375}%
%
\begin{tikzpicture}

\begin{axis}[%
width=0.8\textwidth,
height=20mm,
scale only axis,
xmin=0,
xmax=300,
xlabel style={font=\color{white!15!black}, yshift=1.5mm},
xlabel={$t$ [s]},
ymin=-0.1,
ymax=3.4,
ylabel style={font=\color{white!15!black}, yshift=-1.5mm},
ylabel={$\|\mat{z}_{2, k}\|$ [m/s]},
axis background/.style={fill=white},
title style={font=\bfseries, yshift=-2mm},
title={Velocity errors},
legend style={legend cell align=left, align=left, draw=white!15!black},
legend columns=2,
xmajorgrids,
ymajorgrids
]
\addplot [color=mycolor1, line width=1.0pt]
  table[]{sim_velocity_errors-1.tsv};
\addlegendentry{$\text{e}_\text{1}$}

\addplot [color=mycolor2, line width=1.0pt]
  table[]{sim_velocity_errors-2.tsv};
\addlegendentry{$\text{e}_\text{2}$}

\addplot [color=mycolor3, line width=1.0pt]
  table[]{sim_velocity_errors-3.tsv};
\addlegendentry{$\text{e}_\text{3}$}

\addplot [color=mycolor4, line width=1.0pt]
  table[]{sim_velocity_errors-4.tsv};
\addlegendentry{$\text{e}_\text{4}$}

\addplot [color=mycolor5, line width=1.0pt]
  table[]{sim_velocity_errors-5.tsv};
\addlegendentry{$\text{e}_\text{5}$}

\addplot [color=mycolor6, line width=1.0pt]
  table[]{sim_velocity_errors-6.tsv};
\addlegendentry{$\text{e}_\text{0}$}

\end{axis}
\end{tikzpicture}%
        \caption{Norms of the velocity errors $\mat{z}_{2, k}$ and $\mat{z}_{2, o}$.}
        \label{fig:handpos_tracking_velocity_errors}
    \end{subfigure}
    \begin{subfigure}{0.47\textwidth}
        % This file was created by matlab2tikz.
%
%The latest updates can be retrieved from
%  http://www.mathworks.com/matlabcentral/fileexchange/22022-matlab2tikz-matlab2tikz
%where you can also make suggestions and rate matlab2tikz.
%
\definecolor{mycolor1}{rgb}{0.24220,0.15040,0.66030}%
\definecolor{mycolor2}{rgb}{0.27800,0.35560,0.97770}%
\definecolor{mycolor3}{rgb}{0.15400,0.59020,0.92180}%
\definecolor{mycolor4}{rgb}{0.07040,0.74570,0.72580}%
\definecolor{mycolor5}{rgb}{0.50440,0.79930,0.34800}%
%
\begin{tikzpicture}

\begin{axis}[%
width=0.8\textwidth,
height=20mm,
scale only axis,
xmin=0,
xmax=300,
xlabel style={font=\color{white!15!black}, yshift=1.5mm},
xlabel={$t$ [s]},
ymin=0,
ymax=55,
ylabel style={font=\color{white!15!black}, yshift=-1.5mm},
ylabel={$\|\mat{z}_{2, k}\|$ [m]},
axis background/.style={fill=white},
title style={font=\bfseries, yshift=-2mm},
title={Inter-vehicle distances},
legend style={legend cell align=left, align=left, draw=white!15!black, at={(0.98,1)}, anchor=north east},
legend columns=2,
xmajorgrids,
ymajorgrids
]
\addplot [color=mycolor1, line width=1.0pt]
  table[]{sim_distances-1.tsv};
\addlegendentry{$\text{e}_\text{1}$}

\addplot [color=mycolor2, line width=1.0pt]
  table[]{sim_distances-2.tsv};
\addlegendentry{$\text{e}_\text{2}$}

\addplot [color=mycolor3, line width=1.0pt]
  table[]{sim_distances-3.tsv};
\addlegendentry{$\text{e}_\text{3}$}

\addplot [color=mycolor4, line width=1.0pt]
  table[]{sim_distances-4.tsv};
\addlegendentry{$\text{e}_\text{4}$}

\addplot [color=mycolor5, line width=1.0pt]
  table[]{sim_distances-5.tsv};
\addlegendentry{$\text{e}_\text{5}$}

\addplot [color=black, dashed, forget plot, line width=1.25pt]
  table[]{sim_distances-6.tsv};
\addplot [color=black, dashed, forget plot, line width=1.25pt]
  table[]{sim_distances-7.tsv};
\end{axis}
\end{tikzpicture}%
        \caption{Inter-vehicle distances $\norm{\mat{z}_{1, k}}$. The dashed lines represent the limits $\delta_k$ and $\Delta_k$.}
        \label{fig:handpos_tracking_distances}
    \end{subfigure}
    \begin{subfigure}{0.47\textwidth}
        % This file was created by matlab2tikz.
%
%The latest updates can be retrieved from
%  http://www.mathworks.com/matlabcentral/fileexchange/22022-matlab2tikz-matlab2tikz
%where you can also make suggestions and rate matlab2tikz.
%
\definecolor{mycolor1}{rgb}{0.24220,0.15040,0.66030}%
\definecolor{mycolor2}{rgb}{0.27800,0.35560,0.97770}%
\definecolor{mycolor3}{rgb}{0.15400,0.59020,0.92180}%
\definecolor{mycolor4}{rgb}{0.07040,0.74570,0.72580}%
\definecolor{mycolor5}{rgb}{0.50440,0.79930,0.34800}%
\definecolor{mycolor6}{rgb}{0.98710,0.73475,0.24375}%
%
\begin{tikzpicture}

\begin{axis}[%
width=0.8\textwidth,
height=20mm,
scale only axis,
xmin=0,
xmax=300,
xlabel style={font=\color{white!15!black}, yshift=1.5mm},
xlabel={$t$ [s]},
ymin=-1,
ymax=5,
ylabel style={font=\color{white!15!black}, yshift=-1.5mm},
ylabel={$u_{r, i}$ [m/s]},
axis background/.style={fill=white},
title style={font=\bfseries, yshift=-2mm},
title={Surge velocities},
legend style={legend cell align=left, align=left, draw=white!15!black},
legend columns=2,
xmajorgrids,
ymajorgrids
]
\addplot [color=mycolor1, line width=1.0pt]
  table[]{sim_surge-1.tsv};
%\addlegendentry{AUV 1}

\addplot [color=mycolor2, line width=1.0pt]
  table[]{sim_surge-2.tsv};
%\addlegendentry{AUV 2}

\addplot [color=mycolor3, line width=1.0pt]
  table[]{sim_surge-3.tsv};
%\addlegendentry{AUV 3}

\addplot [color=mycolor4, line width=1.0pt]
  table[]{sim_surge-4.tsv};
%\addlegendentry{AUV 4}

\addplot [color=mycolor5, line width=1.0pt]
  table[]{sim_surge-5.tsv};
%\addlegendentry{AUV 5}

\addplot [color=mycolor6, line width=1.0pt]
  table[]{sim_surge-6.tsv};
%\addlegendentry{AUV 6}

\end{axis}
\end{tikzpicture}%
        \caption{Surge velocities of the \glspl{auv}.}
        \label{fig:handpos_tracking_surge}
    \end{subfigure}
    \caption{Results of numerical simulations in MATLAB.}
    \label{fig:handpos_tracking_sim}
\end{figure}


Here we simulate six \acrfullpl{lauv}.
The ocean current velocity is set to $\ocean= \left[0.05\;\;-0.08\;\;-0.03\right]\T$.
The desired formation is illustrated in \figref{fig:handpos_formation}.
The desired relative positions $\mat{z}_{1, k}^d$ are given by
\begin{equation}
	\begin{bmatrix}
		\mat{z}_{1, 1}^d & \cdots & \mat{z}_{1, 5}^d
	\end{bmatrix}
	=
	\begin{bmatrix}
		20 & 10 & 10 & 10 & 10\\ 0 & 15 & -15 & 15 & -15\\ 0 & -5 & -5 & -5 & -5
	\end{bmatrix}
\end{equation}
The trajectory of the virtual target is
\begin{equation}
	\mat{x}_{1, o} = \inlinevector{a \cos(\omega_o t), b \sin(\omega_o t), c \sin(\omega_o t)^2},
\end{equation}
where $a = 60$, $b = 40$, $c = 10$, and $\omega_o = \frac{\pi}{150}$.

The maximal and minimal distance parameters are $\Delta_k=\SI{50}{\meter}$ and $\delta_k=\SI{5}{\meter}$.
The hand position point is chosen at $l=\SI{5}{\meter}$, and the control gains are set to $c_1=0.1$, $c_2=0.5$, $\gamma=0.25$, $c_v=0.2$, $\kappa_u=0.1$, $\kappa_\rho=4$, and $\sigma=0.3$.
Furthermore, in order to avoid discontinuities in the control, the non-smooth ${\rm sign}(s)$ function in \eqref{n338} and \eqref{631} is replaced by the smooth approximation $\tanh\left(c_a\, s\right)$, with $c_a = 10^3$.

\figref{fig:handpos_tracking_sim} presents the results of the simulation scenario.
Specifically, \figref{fig:handpos_tracking_trajectory} shows the 3D trajectories of the \glspl{auv}.
We can see that the vehicles successfully reach the desired formation while following the target, as is also evidenced from the formation and tracking errors in \figref{fig:handpos_tracking_formation_errors} and the velocity errors in \figref{fig:handpos_tracking_velocity_errors}.
Furthermore, note that the hard connectivity and collision-avoidance constraints, shown as dashed black lines in \figref{fig:handpos_tracking_distances}, are always respected.
The soft constraints are satisfied as well, and the surge velocities are kept non-negative, as shown in \figref{fig:handpos_tracking_surge}.
%To see this more clearly, Fig.~\ref{fig:ur} shows the surge velocities for an identical simulation scenario, except for the soft-constraint requirement on the velocities. It is clear by comparing with Fig.~\ref{fig:ur2}, that without the proposed strategy, the surge velocities would take greater negative values.

\subsection{DUNE simulations}
\begin{figure}[p]
    \centering
    \begin{subfigure}{0.7\textwidth}
        \centering
        \def\svgwidth{\textwidth}
        \import{figures/handpos_tracking}{hifi_trajectory.pdf_tex}
        \vspace*{-6mm}
        \caption{The 3D trajectory of the \glspl{auv}. The black line represents the virtual target.}
        \label{fig:handpos_tracking_hifi_trajectory}
    \end{subfigure}  
    \begin{subfigure}{0.47\textwidth}
        \hspace*{-2mm}
        % This file was created by matlab2tikz.
%
%The latest updates can be retrieved from
%  http://www.mathworks.com/matlabcentral/fileexchange/22022-matlab2tikz-matlab2tikz
%where you can also make suggestions and rate matlab2tikz.
%
\definecolor{mycolor1}{rgb}{0.24220,0.15040,0.66030}%
\definecolor{mycolor2}{rgb}{0.27800,0.35560,0.97770}%
\definecolor{mycolor3}{rgb}{0.15400,0.59020,0.92180}%
\definecolor{mycolor4}{rgb}{0.07040,0.74570,0.72580}%
\definecolor{mycolor5}{rgb}{0.50440,0.79930,0.34800}%
\definecolor{mycolor6}{rgb}{0.98710,0.73475,0.24375}%
%
\begin{tikzpicture}

\begin{axis}[%
width=0.8\textwidth,
height=20mm,
scale only axis,
xmin=0,
xmax=300,
xlabel style={font=\color{white!15!black}, yshift=1.5mm},
xlabel={$t$ [s]},
ymin=-1,
ymax=110,
ylabel style={font=\color{white!15!black}, yshift=-1.5mm},
ylabel={$\|\tilde{\mat{z}}_{1, k}\|$ [m]},
axis background/.style={fill=white},
title style={font=\bfseries, yshift=-2mm},
title={Formation tracking errors},
legend style={legend cell align=left, align=left, draw=white!15!black},
legend columns=2,
xmajorgrids,
ymajorgrids
]
\addplot [color=mycolor1, line width=1.0pt]
  table[]{hifi_formation_errors-1.tsv};
\addlegendentry{$\text{e}_\text{1}$}

\addplot [color=mycolor2, line width=1.0pt]
  table[]{hifi_formation_errors-2.tsv};
\addlegendentry{$\text{e}_\text{2}$}

\addplot [color=mycolor3, line width=1.0pt]
  table[]{hifi_formation_errors-3.tsv};
\addlegendentry{$\text{e}_\text{3}$}

\addplot [color=mycolor4, line width=1.0pt]
  table[]{hifi_formation_errors-4.tsv};
\addlegendentry{$\text{e}_\text{4}$}

\addplot [color=mycolor5, line width=1.0pt]
  table[]{hifi_formation_errors-5.tsv};
\addlegendentry{$\text{e}_\text{5}$}

\addplot [color=mycolor6, line width=1.0pt]
  table[]{hifi_formation_errors-6.tsv};
\addlegendentry{$\text{e}_\text{0}$}

\end{axis}
\end{tikzpicture}%

        \vspace*{-6mm}
        \caption{Norms of the formation-keeping errors, $\tilde{\mat{z}}_{1, k}$, and the tracking error, $\tilde{\mat{z}}_{1, o}$.}
        \label{fig:handpos_tracking_hifi_formation_errors}
    \end{subfigure}
    \begin{subfigure}{0.47\textwidth}
        \hspace*{1mm}
        % This file was created by matlab2tikz.
%
%The latest updates can be retrieved from
%  http://www.mathworks.com/matlabcentral/fileexchange/22022-matlab2tikz-matlab2tikz
%where you can also make suggestions and rate matlab2tikz.
%
\definecolor{mycolor1}{rgb}{0.24220,0.15040,0.66030}%
\definecolor{mycolor2}{rgb}{0.27800,0.35560,0.97770}%
\definecolor{mycolor3}{rgb}{0.15400,0.59020,0.92180}%
\definecolor{mycolor4}{rgb}{0.07040,0.74570,0.72580}%
\definecolor{mycolor5}{rgb}{0.50440,0.79930,0.34800}%
\definecolor{mycolor6}{rgb}{0.98710,0.73475,0.24375}%
%
\begin{tikzpicture}

\begin{axis}[%
width=0.8\textwidth,
height=20mm,
scale only axis,
xmin=0,
xmax=300,
xlabel style={font=\color{white!15!black}, yshift=1.5mm},
xlabel={$t$ [s]},
ymin=0,
ymax=3,
ylabel style={font=\color{white!15!black}, yshift=-1.5mm},
ylabel={$\|\mat{z}_{2, k}\|$ [m/s]},
axis background/.style={fill=white},
title style={font=\bfseries, yshift=-2mm},
title={Velocity errors},
legend style={legend cell align=left, align=left, draw=white!15!black, inner sep=1.5pt, at={(1,1)}, anchor=north east},
legend columns=2,
xmajorgrids,
ymajorgrids
]
\addplot [color=mycolor1, line width=1.0pt]
  table[]{hifi_velocity_errors-1.tsv};
\addlegendentry{$\text{e}_\text{1}$}

\addplot [color=mycolor2, line width=1.0pt]
  table[]{hifi_velocity_errors-2.tsv};
\addlegendentry{$\text{e}_\text{2}$}

\addplot [color=mycolor3, line width=1.0pt]
  table[]{hifi_velocity_errors-3.tsv};
\addlegendentry{$\text{e}_\text{3}$}

\addplot [color=mycolor4, line width=1.0pt]
  table[]{hifi_velocity_errors-4.tsv};
\addlegendentry{$\text{e}_\text{4}$}

\addplot [color=mycolor5, line width=1.0pt]
  table[]{hifi_velocity_errors-5.tsv};
\addlegendentry{$\text{e}_\text{5}$}

\addplot [color=mycolor6, line width=1.0pt]
  table[]{hifi_velocity_errors-6.tsv};
\addlegendentry{$\text{e}_\text{0}$}

\end{axis}
\end{tikzpicture}%
        \vspace*{-6mm}
        \caption{Norms of the velocity errors $\mat{z}_{2, k}$ and $\mat{z}_{2, o}$.}
        \label{fig:handpos_tracking_hifi_velocity_errors}
    \end{subfigure}
    \begin{subfigure}{0.47\textwidth}
        \hspace*{-2mm}
        % This file was created by matlab2tikz.
%
%The latest updates can be retrieved from
%  http://www.mathworks.com/matlabcentral/fileexchange/22022-matlab2tikz-matlab2tikz
%where you can also make suggestions and rate matlab2tikz.
%
\definecolor{mycolor1}{rgb}{0.24220,0.15040,0.66030}%
\definecolor{mycolor2}{rgb}{0.27800,0.35560,0.97770}%
\definecolor{mycolor3}{rgb}{0.15400,0.59020,0.92180}%
\definecolor{mycolor4}{rgb}{0.07040,0.74570,0.72580}%
\definecolor{mycolor5}{rgb}{0.50440,0.79930,0.34800}%
%
\begin{tikzpicture}

\begin{axis}[%
width=0.8\textwidth,
height=20mm,
scale only axis,
xmin=0,
xmax=300,
xlabel style={font=\color{white!15!black}, yshift=1.5mm},
xlabel={$t$ [s]},
ymin=0,
ymax=55,
ylabel style={font=\color{white!15!black}, yshift=-1.5mm},
ylabel={$\|\mat{z}_{2, k}\|$ [m]},
axis background/.style={fill=white},
title style={font=\bfseries, yshift=-2mm},
title={Inter-vehicle distances},
legend style={legend cell align=left, align=left, draw=white!15!black, at={(0.98,1)}, anchor=north east},
legend columns=2,
xmajorgrids,
ymajorgrids
]
\addplot [color=mycolor1, line width=1.0pt]
  table[]{hifi_distances-1.tsv};
\addlegendentry{$\text{e}_\text{1}$}

\addplot [color=mycolor2, line width=1.0pt]
  table[]{hifi_distances-2.tsv};
\addlegendentry{$\text{e}_\text{2}$}

\addplot [color=mycolor3, line width=1.0pt]
  table[]{hifi_distances-3.tsv};
\addlegendentry{$\text{e}_\text{3}$}

\addplot [color=mycolor4, line width=1.0pt]
  table[]{hifi_distances-4.tsv};
\addlegendentry{$\text{e}_\text{4}$}

\addplot [color=mycolor5, line width=1.0pt]
  table[]{hifi_distances-5.tsv};
\addlegendentry{$\text{e}_\text{5}$}

\addplot [color=black, dashed, forget plot, line width=1.25pt]
  table[]{hifi_distances-6.tsv};
\addplot [color=black, dashed, forget plot, line width=1.25pt]
  table[]{hifi_distances-7.tsv};
\end{axis}
\end{tikzpicture}%
        \vspace*{-6mm}
        \caption{Inter-vehicle distances $\norm{\mat{z}_{1, k}}$. The dashed lines represent the limits $\delta_k$ and $\Delta_k$.}
        \label{fig:handpos_tracking_hifi_distances}
    \end{subfigure}
    \begin{subfigure}{0.47\textwidth}
        \vspace*{-40mm}
        \hspace*{-2mm}
        % This file was created by matlab2tikz.
%
%The latest updates can be retrieved from
%  http://www.mathworks.com/matlabcentral/fileexchange/22022-matlab2tikz-matlab2tikz
%where you can also make suggestions and rate matlab2tikz.
%
\definecolor{mycolor1}{rgb}{0.24220,0.15040,0.66030}%
\definecolor{mycolor2}{rgb}{0.27800,0.35560,0.97770}%
\definecolor{mycolor3}{rgb}{0.15400,0.59020,0.92180}%
\definecolor{mycolor4}{rgb}{0.07040,0.74570,0.72580}%
\definecolor{mycolor5}{rgb}{0.50440,0.79930,0.34800}%
\definecolor{mycolor6}{rgb}{0.98710,0.73475,0.24375}%
%
\begin{tikzpicture}

\begin{axis}[%
width=0.8\textwidth,
height=20mm,
scale only axis,
xmin=0,
xmax=300,
xlabel style={font=\color{white!15!black}, yshift=1.5mm},
xlabel={$t$ [s]},
ymin=-1,
ymax=2,
ylabel style={font=\color{white!15!black}, yshift=-1.5mm},
ylabel={$u_{r, i}$ [m/s]},
axis background/.style={fill=white},
title style={font=\bfseries, yshift=-2mm},
title={Surge velocities},
legend style={legend cell align=left, align=left, draw=white!15!black},
legend columns=2,
xmajorgrids,
ymajorgrids
]
\addplot [color=mycolor1, line width=1.0pt]
  table[]{hifi_surge-2.tsv};
%\addlegendentry{AUV 1}

\addplot [color=mycolor2, line width=1.0pt]
  table[]{hifi_surge-1.tsv};
%\addlegendentry{AUV 2}

\addplot [color=mycolor3, line width=1.0pt]
  table[]{hifi_surge-3.tsv};
%\addlegendentry{AUV 3}

\addplot [color=mycolor4, line width=1.0pt]
  table[]{hifi_surge-4.tsv};
%\addlegendentry{AUV 4}

\addplot [color=mycolor5, line width=1.0pt]
  table[]{hifi_surge-5.tsv};
%\addlegendentry{AUV 5}

\addplot [color=mycolor6, line width=1.0pt]
  table[]{hifi_surge-6.tsv};
%\addlegendentry{AUV 6}

\end{axis}
\end{tikzpicture}%
        \vspace*{-6mm}
        \caption{Surge velocities of the \glspl{auv}.}
        \label{fig:handpos_tracking_hifi_surge}
    \end{subfigure}
    \caption{Results of numerical simulations in \gls{dune}.}
    \label{fig:handpos_tracking_hifi}
\end{figure}


Here, we simulate a formation of six \glspl{lauv} using \gls{dune}.
The parameters of the simulation are chosen identically to the MATLAB example.

\figref{fig:handpos_tracking_hifi} presents the results of the simulation.
Specifically, \figref{fig:handpos_tracking_hifi_trajectory} shows the 3D trajectories of the \glspl{auv}.
We can see that the vehicles manage to reach the desired formation. However, the transient behavior is different from the one under the ideal conditions of the MATLAB simulation.
One reason behind these differences is that in the DUNE simulation, the torque produced by the fins depends on the speed of the vehicle.
Consequently, the \glspl{auv} cannot turn if their speed is too low.
In addition, the surge thrust of the \glspl{auv} is limited.
Consequently, the \glspl{auv} can only reach a surge velocity of $\SI{1.8}{\meter\per\second}$, as shown in \figref{fig:handpos_tracking_hifi_surge}.
Unlike the MATLAB simulations, the soft constraints cannot always be satisfied, and the surge velocities of \glspl{auv} 2 and 5 briefly drop below zero.
Having a negative surge velocity is necessary to satisfy the hard connectivity and collision-avoidance constraints shown as dashed black lines in \figref{fig:handpos_tracking_hifi_distances}.
The formation-tracking errors and the velocity errors are shown in Figures~\ref{fig:handpos_tracking_hifi_formation_errors} and \ref{fig:handpos_tracking_hifi_velocity_errors}, respectively.
We can see that these errros do not converge to zero but rather to a small area around zero.
These nonzero steady-state errors are caused by two factors: the uncertainty of the navigation system, and the delay in the actuators.
