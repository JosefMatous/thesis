\chapter{Tracking Control of Cooperative Marine Vehicles Under Hard and Soft Constraints}
\label{chap:handpos_tracking}

