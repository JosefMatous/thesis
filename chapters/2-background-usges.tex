\section{Uniform Semiglobal Exponential Stability}
\label{sec:background_USGES}

This section discusses the concept of \gls{usges}.
In some cases, dynamical systems cannot attain global stability due to, for instance, high-order nonlinearities, the choice of the control law, or actuator saturations.
An example of such a system is marine vehicles controlled by \acrlong{los} guidance laws.
In \cite{fossen_uniform_2014}, it has been shown that the structure of proportional \gls{los} guidance laws prevents the system from having global exponential convergence.

\Acrlong{usges} has been studied, \emph{e.g.,} in \cite{loria_cascaded_2005,pettersen_lyapunov_2017}.
In these works, \gls{usges} is defined as follows.

\begin{dfn}[USGES]
    Consider a nonlinear system given by the following set of \glspl{ode}
    \begin{align}
        \dot{\mat{x}} &= f(t, \mat{x}), &
        \mat{x}(0) &= \mat{x}_0,
        \label{eq:background_USGES_ODE}
    \end{align}
    with the origin $\mat{x} = \mat{0}$ being the equilibrium point of the system.

    Let $\mat{x}(t | \mat{x}_0)$ be a solution to \eqref{eq:background_USGES_ODE} that is defined for all $t \geq 0$.
    The origin $\mat{x} = \mat{0}$ is a \glspl{usges} equilibrium point of \eqref{eq:background_USGES_ODE} if for all $\Delta > 0$, there exist positive constants $k_{\Delta}$ and $\lambda_{\Delta}$ such that $\forall \mat{x}_0 \in \ball{\Delta}$
    \begin{align}
        &\norm{\mat{x}(t | \mat{x}_0)} \leq k_{\Delta} \norm{\mat{x}_0} {\rm e}^{-\lambda_{\Delta}t},&
        &\forall t \geq 0.
    \end{align}
\end{dfn}

\begin{rmk}
    The work in \cite{pettersen_lyapunov_2017} studies parametric systems, i.e., systems with \glspl{ode} in the following form
    \begin{equation}
        \dot{\mat{x}} = f(\mat{x}, t, \theta),
    \end{equation}
    where $\theta \in \Theta \subset \mathbb{R}^m$ is a constant parameter.
    However, since this thesis does not consider parametric systems, and since the parameter $\theta$ is assumed constant, we can omit the parametric dependence for the sake of simplicity.
\end{rmk}

\subsection{Lyapunov sufficient conditions for uniform semiglobal exponential stability}
In this section, we restate Theorem 5 and Proposition 9 from \cite{pettersen_lyapunov_2017}.

Theorem 5 introduces sufficient conditions for \acrfull{usges} of nonlinear systems.
\begin{theorem}[Theorem 5. \cite{pettersen_lyapunov_2017}]
    \label{thm:background_USGES}
    Consider the system given in \eqref{eq:background_USGES_ODE}.
    If for any $\Delta > 0$, there exist a continuously differentiable Lyapunov function $V_{\Delta} : \mathbb{R}_{\geq 0} \times \ball{\Delta} \mapsto \mathbb{R}_{\geq 0}$ and positive constants $k_{1_\Delta}$, $k_{2_\Delta}$, $k_{3_\Delta}$, and $a$, such that $\forall \mat{x} \in \ball{\Delta}$, $\forall t \geq 0$
    \begin{subequations}
        \begin{align}
            k_{1_\Delta} \norm{\mat{x}}^a \leq V_{\Delta}(t, \mat{x}) &\leq k_{2_\Delta} \norm{\mat{x}}^a, \\
            \lim_{\Delta \rightarrow \infty} \left(\frac{k_{1_\Delta}}{k_{2_\Delta}}\right)^{1/a} \Delta &= \infty, \\
            \frac{\partial V_{\Delta}}{\partial t} + \frac{\partial V_{\Delta}}{\partial \mat{x}} f(t, \mat{x}) &\leq -k_{3_\Delta} \norm{\mat{x}}^a,
        \end{align}
    \end{subequations}
    then the origin of \eqref{eq:background_USGES_ODE} is \glspl{usges}.
\end{theorem}

Proposition 9 then introduces sufficient conditions for \acrfull{usges} of nonlinear cascaded systems.
\begin{prop}[Proposition 9. \cite{pettersen_lyapunov_2017}]
    \label{prop:background_cascade}
    Consider the following cascaded nonlinear time-varying system
    \begin{subequations}
        \begin{align}
            \dot{\mat{x}}_1 &= f_1(t, \mat{x}_1) + g(t, \mat{x}_1)\mat{x}_2, \\
            \dot{\mat{x}}_2 &= f_2(t, \mat{x}_2),
        \end{align}
        \label{eq:background_USGES_cascade}
    \end{subequations}

    \noindent where $t \in \mathbb{R}_{\geq 0}$, $\mat{x}_1 \in \mathbb{R}^{n_1}$, $\mat{x}_2 \in \mathbb{R}^{n_2}$.
    The functions $f_1$, $f_2$, and $g$ are continuous in $t$ and locally Lipschitz in $\mat{x}_1$ and $\mat{x}_2$.
    Furthermore, $f_1$ is assumed $\mathcal{C}^1$ in $t$ and $\mat{x}_1$, and the origin $\left[\mat{x}_1\T, \mat{x}_2\T\right] = \mat{0}\T$ is an equilibrium point of \eqref{eq:background_USGES_cascade}.

    Let each of the systems
    \begin{align}
        \dot{\mat{x}}_1 &= f_1(t, \mat{x}_1), \label{eq:background_nominal} \\
        \dot{\mat{x}}_2 &= f_2(t, \mat{x}_2), 
    \end{align}
    be \glspl{ugas} and satisfy the conditions of Theorem~\ref{thm:background_USGES}.

    Then, the origin of the cascaded system \eqref{eq:background_USGES_cascade} is \glspl{usges} and \glspl{ugas} if the following two assumptions hold
    \begin{enumerate}
        \item There exist constants $c_1, c_2, \eta > 0$ and a positive definite, radially unbounded Lyapunov function $V : \mathbb{R}_{\geq 0} \times \mathbb{R}^{n_1}$ of \eqref{eq:background_nominal} such that $\dot{V}(t, \mat{x}_1) \leq 0$ and
        \begin{subequations}
            \begin{align}
                \norm{\frac{\partial V}{\partial \mat{x}_1}} \norm{\mat{x}_1} &\leq c_1 V,& \forall&\norm{\mat{x}_1} \geq \eta, \\
                \norm{\frac{\partial V}{\partial \mat{x}_1}} &\leq c_2,& \forall&\norm{\mat{x}_1} \leq \eta.
            \end{align}
        \end{subequations}
        \item There exist two continuous functions $\alpha_1, \alpha_2: \mathbb{R}_{\geq 0} \mapsto \mathbb{R}_{\geq 0}$ such that
        \begin{equation}
            \norm{g(t, \mat{x}_1, \mat{x}_2)} \leq \alpha_1 \left(\norm{\mat{x}_2}\right) + \alpha_2 \left(\norm{\mat{x}_2}\right) \norm{\mat{x}_1}.
            \label{eq:background_USGES_cascade_assumption_2}
        \end{equation}
    \end{enumerate}
\end{prop}
