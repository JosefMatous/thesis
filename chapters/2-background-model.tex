\section{Mathematical Models of Marine Vehicles}
\label{sec:model}

Mathematical models are vital to the design, analysis, and verification of control algorithms.
For complex systems, such as marine vehicles, there typically exist different types of models.
These models often represent a trade-off between simplicity and fidelity.
On one end of the spectrum, there are high-fidelity models.
These models are an accurate representation of the system.
Consequently, high-fidelity models are used in ship-handling simulators, as well as in some simulation environments, \emph{e.g.,} the \gls{dune} \cite{dune}, to train the operators and to verify the effectiveness of control algorithms.
However, due to their complexity, these models are not suitable for designing or analyzing control algorithms.

On the other end of the spectrum, there are control-oriented models.
Compared to high-fidelity models, the structure of control-oriented models is simpler.
This simplicity means that we can use these models to design control algorithms and analyze their closed-loop properties.
The control-oriented models are thus designed to capture the system properties that are the most significant and relevant for the design of the control system.
The inaccuracies of control-oriented models can be compensated by designing robust controllers.

This section presents control-oriented models of \acrfullpl{asv} and \acrfullpl{auv}.
The presented models are similar to those used in \cite{borhaug_straight_2007} and \cite{fredriksen_global_2006}.
The models are based on the matrix-vector models of marine vehicles \cite{fossen_handbook_2011} and some simplifying assumptions.

\subsection{State Variables and Degrees of Freedom}
Marine vehicles are typically modeled as rigid bodies.
A rigid body moving in three-dimensional space has six \glspl{dof}, three for position and three for orientation.

The position of a marine vehicle is commonly expressed in a local \gls{ned} coordinate frame.
Although the \gls{ned} frame is not inertial, it is often used as an approximation of an inertial coordinate frame for short-term and short-distance missions, since the effect of Earth's rotation on the vehicles is negligible.
In general, we will denote the position of the vehicles as $\mat{p} = \inlinevector{x, y, z}$.

The orientation of a vehicle can be expressed using \emph{Euler angles}, $\bs{\Theta} = \inlinevector{\phi, \theta, \psi}$, where $\phi$ is the roll angle, $\theta$ is the pitch angle, and $\psi$ is the yaw angle.
The complete position and orientation vector of the vehicle is then given by $\pose\T = \left[\mat{p}\T, \bs{\Theta}\T\right]$.

Although Euler angles can represent any orientation,
in some cases, this representation is not unique.
For example, the following two sets of Euler angles represent the same attitude
\begin{align}
    \bs{\Theta}_1 &= \inlinevector{\frac{\pi}{2}, \frac{\pi}{2}, 0}, &
    \bs{\Theta}_2 &= \inlinevector{0, \frac{\pi}{2}, -\frac{\pi}{2}}. &
\end{align}
At these attitudes, there exist mathematical singularities called \emph{gimbal locks} \cite{chaturvedi_rigid-body_2011}.
Furthermore, the use of Euler angles in control may lead to a phenomenon called \emph{unwinding} \cite{sanjay_topological_2000}, in which the vehicle performs an unnecessary rotation to reach the desired attitude.

The orientation of a vehicle can also be described using a \emph{rotation matrix}.
Rotation matrices are members of the special orthogonal group $SO(3)$.
Unlike Euler angles, rotation matrices do not suffer from singularities.
For a given set of Euler angles, the corresponding rotation matrix is given by \cite{fossen_handbook_2011}
\begin{equation}
    \mat{R}(\phi, \theta, \psi)
    =
    \begin{bmatrix} 
        {\rm c}_{\psi}\,{\rm c}_{\theta} & {\rm c}_{\psi}\,{\rm s}_{\phi}\,{\rm s}_{\theta}-{\rm c}_{\phi}\,{\rm s}_{\psi} & {\rm s}_{\phi}\,{\rm s}_{\psi}+{\rm c}_{\phi}\,{\rm c}_{\psi}\,{\rm s}_{\theta}\\ {\rm c}_{\theta}\,{\rm s}_{\psi} & {\rm c}_{\phi}\,{\rm c}_{\psi}+{\rm s}_{\phi}\,{\rm s}_{\psi}\,{\rm s}_{\theta} & {\rm c}_{\phi}\,{\rm s}_{\psi}\,{\rm s}_{\theta}-{\rm c}_{\psi}\,{\rm s}_{\phi}\\ -{\rm s}_{\theta} & {\rm c}_{\theta}\,{\rm s}_{\phi} & {\rm c}_{\phi}\,{\rm c}_{\theta} 
    \end{bmatrix},
    \label{eq:background_Rzyx}
\end{equation}
where ${\rm c}_x$ and ${\rm s}_x$ represent the cosine and sine of the corresponding angle.

Next, let us discuss the representation of velocities.
The velocities of the vehicle are expressed in the \emph{body-fixed} frame, a non-inertial coordinate frame attached to the vehicle, with the $x$-axis pointing towards the bow (front) side, the $y$-axis pointing to the starboard (right) side, and the $z$-axis pointing to the bottom side of the vehicle.
The \emph{linear velocities} of the vehicle $\bvel = \inlinevector{u, v, w}$ consist of the surge, sway, and heave velocities.
The \emph{angular velocities} of the vehicle $\bs{\omega} = \inlinevector{p, q, r}$ consist of the roll, pitch, and yaw rates.
The full velocity vector is then given by $\vel\T = \left[\bvel\T, \bs{\omega}\T\right]$.

Throughout the thesis, we denote velocities in the body-fixed frame as $\bvel$, while velocities in the inertial frame are denoted as $\ivel$.
The difference between these two types of velocities is illustrated in \figref{fig:background_body_fixed_inertial}.

\begin{figure}[b]
    \centering
    \begin{subfigure}{0.4\textwidth}
        \centering
        \def\svgwidth{0.9\textwidth}
        \import{figures/background}{auv_body_fixed.pdf_tex}
        \caption{Illustration of body-fixed velocities. The kinematics of the vehicle are $\dot{\mat{p}} = \mat{R} \bvel$.}
    \end{subfigure}
    \hspace*{0.05\textwidth}
    \begin{subfigure}{0.4\textwidth}
        \centering
        \def\svgwidth{0.9\textwidth}
        \import{figures/background}{auv_inertial.pdf_tex}
        \caption{Illustration of inertial velocities. The kinematics of the vehicle are $\dot{\mat{p}} = \ivel$.}
    \end{subfigure}
    \caption{Illustration of the difference between body-fixed and inertial velocities.}
    \label{fig:background_body_fixed_inertial}
\end{figure}

Finally, let us discuss simplified 3\gls{dof} and 5\gls{dof} models.
In the case of \glspl{asv} or \glspl{auv} moving in the horizontal plane, we often assume that the roll and pitch angles are zero, and that the depth is constant.
Consequently, we can disregard the roll, pitch, and heave motion of the vehicle, and derive a simplified 3\gls{dof} model with $\pose = \inlinevector{x, y, \psi}$ and $\vel = \inlinevector{u, v, r}$.
In the case of slender, torpedo-shaped \glspl{auv}, the roll motion is assumed to be small and self-stabilizing by the design of the vehicle.
Consequently, we can disregard the roll motion and derive a simplified 5\gls{dof} model with $\pose = \inlinevector{x, y, z, \theta, \psi}$ and $\vel = \inlinevector{u, v, w, q, r}$.

\subsection{Kinematics}
First, let us discuss the kinematics of the vehicles, starting with the 6\gls{dof} model.
The time-derivative of the position is
\begin{equation}
    \dot{\mat{p}} = \mat{R} \bvel.
    \label{eq:background_p_dot_6DOF}
\end{equation}
The time-derivative of the Euler angles is given by \cite{fossen_handbook_2011}
\begin{align}
    \dot{\bs{\Theta}} &= \mat{T}(\bs{\Theta}) \bs{\omega}, &
    \mat{T}(\bs{\Theta}) &= 
    \begin{bmatrix}
        1 & {\rm s}_{\phi}{\rm t}_{\theta} & {\rm c}_{\phi}{\rm t}_{\theta} \\ 0 & {\rm c}_{\phi} & -{\rm s}_{\phi} \\ 0 & {\rm s}_{\phi}/{\rm c}_{\theta} & {\rm c}_{\phi}/{\rm c}_{\theta}
    \end{bmatrix},
    \label{eq:background_theta_dot_6DOF}
\end{align}
where ${\rm t}_{\theta} = \tan(\theta)$.
Due to the aforementioned singularities, $\dot{\bs{\Theta}}$ is not defined for $\theta = \pm \pi/2$.
The time-derivative of a rotation matrix is given by
\begin{align}
    \dot{\mat{R}} &= \mat{R} \mat{S}(\omega), &
    \mat{S}(\omega) &=
    \begin{bmatrix}
        0 & -r & q\\ r & 0 & -p\\ -q & p & 0
    \end{bmatrix}.
\end{align}

To derive the kinematics of the 5\gls{dof} model, we simply substitute $\phi = 0$ and $p = 0$ into \eqref{eq:background_p_dot_6DOF} and \eqref{eq:background_theta_dot_6DOF}, and get
\begin{subequations}
    \begin{align}
        \dot{\mat{p}} &= \mat{R}(0, \theta, \psi) \bvel, \\
        \dot{\theta} &= q, \\
        \dot{\psi} &= \frac{r}{\cos(\theta)}.
    \end{align}
\end{subequations}

Similarly, we can derive the kinematics of the 3\gls{dof} model by substituting $z = \phi = \theta = w = p = q = 0$ into \eqref{eq:background_p_dot_6DOF} and \eqref{eq:background_theta_dot_6DOF}
\begin{align}
    \dot{\pose} &= \mat{J}(\psi) \vel, &
    \mat{J}(\psi) &=
    \begin{bmatrix}
        {\rm c}_{\psi} & -{\rm s}_{\psi} & 0 \\
        {\rm s}_{\psi} & {\rm c}_{\psi} & 0 \\
        0 & 0 & 1
    \end{bmatrix}.
\end{align}

\subsection{Dynamics}
When modeling the dynamics of marine vehicles, we often need to consider the effect of \emph{sea loads} such as waves, wind, and ocean currents.
Let $\ocean \in \mathbb{R}^3$ be a vector that represents the velocity of the ocean current in the inertial coordinate frame.
Since the dynamics of ocean currents are typically much slower than the dynamics of the vehicle, the ocean current can be considered constant and irrotational.
Let $\bvel_c = \mat{R}\T \ocean$ denote the velocity of the ocean current expressed in the vehicle's body-fixed frame.
Furthermore, let $\bvel_r = \bvel - \bvel_c \triangleq \inlinevector{u_r, v_r, w_r}$ denote the relative surge, sway, and heave velocity of the vehicle, and let $\vel_r\T = \left[\bvel_r\T, \bs{\omega}\T\right]$ denote the full relative velocity vector.
The dynamics of the vehicle can then be expressed using the following matrix-vector model \cite{fossen_handbook_2011}
\begin{equation}
    \mat{M} \dot{\vel}_r + \mat{C}(\vel_r)\vel_r + \mat{D}(\vel_r)\vel_r + \mat{g}(\mat{R}) = \bs{\tau},
    \label{eq:background_nu_dot_relative}
\end{equation}
where $\mat{M}$ is the mass and inertia matrix, including the added mass effects, $\mat{C}(\vel_r)$ is the Coriolis and centripetal matrix, also including the added mass, $\mat{D}(\vel_r)$ is the hydrodynamic damping matrix, $\mat{g}(\mat{R})$ represents the effects of gravity and buoyancy, and $\bs{\tau}$ represents additional forces and torques such as the effects of actuators and external disturbances.

The model in \eqref{eq:background_nu_dot_relative} can also be expressed in terms of absolute velocities
\begin{equation}
    \mat{M} \left(\dot{\vel} - \dot{\vel}_c\right) + \mat{C}(\vel - \vel_c)(\vel - \vel_c) + \mat{D}(\vel - \vel_c)(\vel - \vel_c) + \mat{g}(\mat{R}) = \bs{\tau},
\end{equation}
where $\vel_c\T = \left[\bvel_c\T, \mat{0}\T\right]$.

The inertia matrix $\mat{M}$ is symmetric positive definite, the damping matrix $\mat{D}$ is positive definite, and the Coriolis matrix $\mat{C}$ is parametrized so that it is skew-symmetric.
There exist multiple expressions for the Coriolis matrix, \emph{e.g.,}
\begin{equation}
    \label{eq:background_coriolis}
    \mat{C}(\vel_r)\! =\!
    \begin{bmatrix}
        \mat{O}_{3 \times 3} & \!\!-\mat{S}(\mat{M}_{11}\bvel_r + \mat{M}_{12}\bs{\omega}) \\
        -\mat{S}(\mat{M}_{11}\bvel_r + \mat{M}_{12}\bs{\omega}) & \!\!-\mat{S}(\mat{M}_{21}\bvel_r + \mat{M}_{22}\bs{\omega})
    \end{bmatrix}, \,
    \begin{bmatrix}
        \mat{M}_{11} & \!\!\!\mat{M}_{12} \\ \mat{M}_{21} & \!\!\!\mat{M}_{22}
    \end{bmatrix}\!
    = \!\mat{M}.
\end{equation}
The gravity and buoyancy vector is given by \cite{fossen_handbook_2011}
\begin{equation}
    \mat{g}(\mat{R}) = -
    \begin{bmatrix}
        (W - B) \mat{R}\T \mat{e}_3 \\
        (W\mat{r}_g - B\mat{r}_b) \times \mat{R}\T \mat{e}_3
    \end{bmatrix},
\end{equation}
where $W \in \mathbb{R}_{> 0}$ is the gravitational force, $B \in \mathbb{R}_{> 0}$ is the buoyant force, $\mat{r}_g$ is the position of the center of gravity, $\mat{r}_b$ is the position of the center of buoyancy, and $\mat{e}_3 = \inlinevector{0, 0, 1}$.

\subsection{Control-Oriented Model of Underactuated \glspl{auv}}
\label{sec:model_control_oriented}
Before deriving the control-oriented model, we need to present the assumptions that allow us to simplify the matrix-vector model in \eqref{eq:background_nu_dot_relative}.

\begin{asm}
    \label{asm:symmetric}
    The vehicle is slender, torpedo-shaped, with port-starboard symmetry.
\end{asm}

\begin{asm}
    \label{asm:damping}
    The vehicle is maneuvering at low speeds.
    Consequently, the hydrodynamic damping can be considered linear.
\end{asm}

\noindent Under Assumption~\ref{asm:damping}, the hydrodynamic damping matrix is constant.
Under Assumption~\ref{asm:symmetric}, the inertia and damping matrices have the following structure
\begin{align}
    \mat{M}\! &= \!\!
    \begin{bmatrix}
        m_{11}\!\! & 0 & 0 & 0 & 0 & 0\\ 0 & \!m_{22}\!\! & 0 & 0 & 0 & m_{26}\\ 0 & 0 & \!m_{33}\!\! & 0 & \!m_{35}\!\! & 0\\ 0 & 0 & 0 & \!m_{44}\!\! & 0 & 0\\ 0 & 0 & \!m_{35}\!\! & 0 & \!m_{55}\!\! & 0\\ 0 & \!m_{26}\!\! & 0 & 0 & 0 & m_{66}
    \end{bmatrix}\!, &
    \mat{D}\! &= \!\!
    \begin{bmatrix}
        d_{11}\!\! & 0 & 0 & 0 & 0 & 0\\ 0 & \!d_{22}\!\! & 0 & 0 & 0 & d_{26}\\ 0 & 0 & \!d_{33}\!\! & 0 & \!d_{35}\!\! & 0\\ 0 & 0 & 0 & \!d_{44}\!\! & 0 & 0\\ 0 & 0 & \!d_{53}\!\! & 0 & \!d_{55}\!\! & 0\\ 0 & \!d_{62}\!\! & 0 & 0 & 0 & d_{66}
    \end{bmatrix}\!.
\end{align}

\begin{asm}
    \label{asm:actuators}
    The vehicle is equipped with a propeller and fins.
    Consequently, the vehicle is capable of generating a force in the surge direction and torques around all three axes.
\end{asm}
Under this assumption, the external forces acting on the vehicle are given by
\begin{align}
    \bs{\tau} &= \mat{B}\mat{f}, &
    \mat{B} &= 
    \begin{bmatrix}
        b_{11} & 0 & 0 & 0 \\ 0 & 0 & 0 & b_{24} \\ 0 & 0 & b_{33} & 0 \\ 0 & b_{42} & 0 & 0 \\ 0 & 0 & b_{53} & 0 \\ 0 & 0 & 0 & b_{64}
    \end{bmatrix},
\end{align}
where $\mat{f} = \inlinevector{T_u, T_p, T_q, T_r}$ is the control input consisting of the surge thrust and the forces produced by the fins.

If the vehicle model \eqref{eq:background_nu_dot_relative} satisfies Assumptions~\ref{asm:symmetric}--\ref{asm:actuators}, then we can perform a change of coordinates such that the actuators produce no sway or heave acceleration.
In other words, for all inputs $\mat{f}$, there exist $f_u, t_p, t_q, t_r \in \mathbb{R}$ such that
\begin{equation}
    \mat{M}^{-1}\mat{B}\mat{f} = \inlinevector{f_u, 0, 0, t_p, t_q, t_r}.
    \label{eq:neutral_point_force}
\end{equation}
This change of coordinates was demonstrated for 5\gls{dof} vehicles in \cite{borhaug_straight_2007}, and can be trivially extended to 6\glspl{dof}.
The transformed body-fixed velocities, $\vel_r^{\prime}$, are
\begin{equation}
    \vel_r^{\prime} = \inlinevector{u_r, v_r + \varepsilon_1 r, w_r + \varepsilon_2 q, p, q, r},
\end{equation}
where $\varepsilon_1, \varepsilon_2 \in \mathbb{R}$.
This transformation can also be written as
\begin{align}
    \vel_r &= \mat{H}\vel_r^{\prime}, &
    \mat{H} &=
    \begin{bmatrix}
        { \mat{I}_3} & {\begin{array}{ccc} 0 & 0 & 0 \\ 0 & 0 & -\varepsilon_1 \\ 0 & -\varepsilon_2 & 0 \end{array}} \\
        { \mat{O}_{3 \times 3}} & {\mat{I}_3}
    \end{bmatrix}.
    \label{eq:background_transformation}
\end{align}
The transformed dynamics are then given by
\begin{equation}
    \mat{M}^{\prime}\dot{\vel}_r^{\prime} + \mat{C}^{\prime}(\vel^{\prime}_r)\vel^{\prime}_r + \mat{D}^{\prime}\vel^{\prime}_r + \mat{H}\T\mat{g}(\mat{R}) = \mat{H}\T\bs{\tau},
\end{equation}
where $\mat{M}^{\prime} = \mat{H}\T \mat{M} \mat{H}$, $\mat{D}^{\prime} = \mat{H}\T \mat{D} \mat{H}$ are the transformed inertia and damping matrices, and $\mat{C}^{\prime}(\vel^{\prime}_r) = \mat{H}\T\mat{C}(\vel_r)\mat{H}$.

If we choose
\begin{align}
    \varepsilon_1 &= \frac{b_{24}\,m_{66}-b_{64}\,m_{26}}{b_{24}\,m_{26}-b_{64}\,m_{22}}, &
    \varepsilon_2 &= \frac{b_{33}\,m_{55}-b_{53}\,m_{35}}{b_{33}\,m_{35}-b_{53}\,m_{33}},
\end{align}
the effect of the actuators on the dynamics of $\vel^{\prime}$ is given by
\begin{equation}
    {\mat{M}^{\prime}}^{-1}\mat{H}\T\mat{B}\mat{u} =
    \begin{bmatrix}
        \frac{b_{11}}{m_{11}} T_{u} \\
        0 \\ 
        0 \\ 
        \frac{b_{42}}{m_{44}} T_{p} \\ 
        \frac{\left(b_{53}\,m_{33}-b_{33}\,m_{35}\right)}{m_{33}\,m_{55}-{m_{35}}^2} T_{q} \\ 
        \frac{\left(b_{64}\,m_{22}-b_{24}\,m_{26}\right)}{m_{22}\,m_{66}-{m_{26}}^2} T_{r}
    \end{bmatrix}.
\end{equation}
We have thus shown that \eqref{eq:neutral_point_force} is satisfied with
\begin{subequations}
    \begin{align}
        t_u &= \frac{b_{11}}{m_{11}} f_{u}, &
        t_p &= \frac{b_{42}}{m_{44}} t_{p}, \\
        t_q &= \frac{\left(b_{53}\,m_{33}-b_{33}\,m_{35}\right)}{m_{33}\,m_{55}-{m_{35}}^2} t_{q}, &
        t_r &= \frac{\left(b_{64}\,m_{22}-b_{24}\,m_{26}\right)}{m_{22}\,m_{66}-{m_{26}}^2} t_{r}.
    \end{align} 
    \label{eq:tau_upqr}
\end{subequations}

\noindent Moreover, if all the numerators in \eqref{eq:tau_upqr} are nonzero, then the converse holds as well, \emph{i.e.,} for all $t_u, t_p, t_q, t_r$, there exists an input $\mat{f}$ such that \eqref{eq:neutral_point_force} holds.
Consequently, we can treat $t_u, t_p, t_q, t_r$ as the new inputs to the system.

\begin{rmk*}
    ~If the vehicle is rotationally symmetric around the $x$-axis ~(\emph{i.e.,} if $m_{22} = m_{33}$, $m_{35} = -m_{26}$, $m_{55} = m_{66}$, $b_{33} = -b_{24}$, and $b_{53} = b_{64}$), then we have $\varepsilon_1 = -\varepsilon_2$, and the transformation \eqref{eq:background_transformation} corresponds to moving the body-fixed coordinate frame a distance $\varepsilon_1$ along the body-fixed $x$-axis.
\end{rmk*}

\begin{asm}
    \label{asm:buoyancy}
    The vehicle is neutrally buoyant, with the centers of gravity and buoyancy located on one vertical axis.
\end{asm}
\noindent Under this assumption, $\mat{g}(\mat{R})$ has the following shape
\begin{align}
    \mat{g}(\mat{R}) &= \begin{bmatrix}
        \mat{0}_3 \\ Wz_{gb} \mat{e}_3 \times \mat{R}\T \mat{e}_3
    \end{bmatrix},
    \label{eq:gravity}
\end{align}
where $z_{gb}$ is the distance between the centers of gravity and buoyancy.

\subsection{The Component Form}
\label{sec:model_component}
In this section, we express the matrix-vector model \eqref{eq:background_nu_dot_relative} in a \emph{component form} by writing out the equations of motion for the individual state variables.
The purpose of this model is to gain a better understanding of the \gls{auv} dynamics, which allows us to design low-level controllers and analyze their closed-loop behavior.

Consider an \gls{auv} model that satisfies Assumptions~\ref{asm:symmetric}--\ref{asm:buoyancy}.
Then, the model in the component form is given by
\begin{subequations}
    \begin{align}
        \dot{u}_r &= F_u(\vel_r) + f_u, \label{eq:background_component_form_u_dot} \\
        \dot{v}_r &= X_v(u_r)r + Y_v(u_r)v_r + Z_v(p)w_r, \\
        \dot{w}_r &= X_w(u_r)q + Y_w(u_r)w_r + Z_w(p)v_r + G(\mat{R}), \\
        \dot{p} &= F_p(\vel_r) + t_p, \\
        \dot{q} &= F_q(\vel_r) + t_q, \\
        \dot{r} &= F_r(\vel_r) + t_r.
    \end{align}
    \label{eq:background_component_form}
\end{subequations}

The expressions for $F_u$, $X_v$, $Y_v$, $Z_v$, $X_w$, $Y_w$, $Z_w$, $G$, $F_p$, $F_q$, and $F_r$ are shown in Appendix~\ref{app:component_form}.
Note that $\dot{u}_r$, $\dot{p}$, $\dot{q}$, and $\dot{r}$ depend on the control inputs.
Consequently, it is possible to stabilize these states using output feedback linearization.
The states $v$ and $w$ are commonly referred to as the \emph{underactuated} dynamics of the vehicle, as these states cannot be controlled directly.
The terms $Y_v$ and $Y_w$ represent the effects of hydrodynamic damping.
Because hydrodynamic damping is dissipative, the terms $Y_v$ and $Y_w$ are negative.
The term $G$ represents the effects of gravity and buoyancy on the heave velocity.
The remaining terms represent the Coriolis and centripetal forces.

In the remainder of this section, we derive a component form for absolute velocities.
From \eqref{eq:background_component_form}, we have
\begin{subequations}
    \begin{align}
        \dot{u} &= F_u(\vel - \vel_c) + \dot{u}_c + f_u, \\
        \dot{v} &= X_v(u - u_c)r + Y_v(u - u_c)(v - v_c) + Z_v(p)(w - w_c) + \dot{v}_c, \\
        \dot{w} &= X_w(u - u_c)q + Y_w(u - u_c)(w - w_c) + Z_w(p)(v - v_c) + G(\mat{R}) + \dot{w}_c, \\
        \dot{p} &= F_p(\vel - \vel_c) + t_p, \\
        \dot{q} &= F_q(\vel - \vel_c) + t_q, \\
        \dot{r} &= F_r(\vel - \vel_c) + t_r.
    \end{align}
    \label{eq:background_component_form_absolute_1}
\end{subequations}%

From the expressions in Appendix~\ref{app:component_form}, we conclude that all terms in \eqref{eq:background_component_form} that contain the relative velocities are either linear or quadratic.
Since the relative velocities are affine in the ocean current ($\bvel_r = \bvel - \mat{R}\T\ocean$), we can conclude that all terms in \eqref{eq:background_component_form_absolute_1} that contain the ocean current are also linear or quadratic. 
Consequently, if we denote the components of the ocean current velocity as $\ocean = \inlinevector{\oceanx, \oceany, \oceanz}$, the model \eqref{eq:background_component_form_absolute_1} can be written as
\begin{subequations}
    \begin{align}
        \dot{u} &= F_u(\vel) + f_u + \Phi_u(\vel, \mat{R})\T\mathbb{V}_c, \\
        \dot{v} &= X_v(u)r + Y_v(u)v + Z_v(p)w + \Phi_v(\vel, \mat{R})\T\mathbb{V}_c, \\
        \dot{w} &= X_w(u)q + Y_w(u)w + Z_w(p)v + G(\mat{R}) + \Phi_w(\vel, \mat{R})\T\mathbb{V}_c, \\
        \dot{p} &= F_p(\vel, \mat{R}) + t_p + \Phi_p(\vel, \mat{R})\T\mathbb{V}_c, \\
        \dot{q} &= F_q(\vel, \mat{R}) + t_q + \Phi_q(\vel, \mat{R})\T\mathbb{V}_c, \\
        \dot{r} &= F_r(\vel) + t_r + \Phi_r(\vel, \mat{R})\T\mathbb{V}_c,
    \end{align}
    \label{eq:background_component_form_absolute_2}
\end{subequations}

\noindent where $\mathbb{V}_c = \inlinevector{\oceanx, \oceany, \oceanz, \oceanx^2, \oceany^2, \oceanz^2, \oceanx\oceany, \oceanx\oceanz, \oceany\oceanz}$.
The expressions for $\Phi_u$, $\Phi_v$, $\Phi_w$, $\Phi_p$, $\Phi_q$, and $\Phi_r$ are omitted in this thesis.
Instead, let us present a method for finding them.

Let $\mat{r}_1$, $\mat{r}_2$, and $\mat{r}_3$ denote the columns of the rotation matrix $\mat{R}$.
The ocean current velocities in the body-fixed frame are given by
\begin{equation}
    \bvel_c = \mat{R}\T \ocean \quad \implies \quad
    u_c = \mat{r}_1\T \ocean, \,\,
    v_c = \mat{r}_2\T \ocean, \,\,
    w_c = \mat{r}_3\T \ocean.
\end{equation}
Suppose then that the right-hand side of \eqref{eq:background_component_form} contains a linear term, \emph{e.g.} $k u_r$, where $k \in \mathbb{R}$.
This term can be expressed as
\begin{equation}
    k u_r = k (u - u_c) = k u - k \mat{r}_1\T \ocean
    = k u + k \varphi_u\T \mathbb{V}_c,
\end{equation}
where $\varphi_u\T = \left[-\mat{r}_1\T, \mat{0}_6\T\right]$.
We can derive similar equations for linear terms containing $v_r$ and $w_r$.

Next, consider a quadratic term, \emph{e.g.,} $k u_r v_r$, where $k \in \mathbb{R}$.
This term can be expressed as
\begin{equation}
    k u_r v_r = k (uv - uv_c - u_cv + u_cv_c) = k u v + k \varphi_{uv}(u,v)\T \mathbb{V}_c,
\end{equation}
where
\begin{align}
    \varphi_{uv}(u,v)\T = \big[-u\mat{r}_2\T-v\mat{r}_1\T, r_{11}r_{21}, &\,r_{12}r_{22}, r_{13}r_{23}, r_{11}r_{22}+r_{12}r_{21}, \\
    &r_{11}r_{23}+r_{13}r_{21}, r_{12}r_{23}+r_{13}r_{22}\big], \nonumber
\end{align}
where
\begin{align}
    \mat{r}_1 &= \inlinevector{r_{11}, r_{12}, r_{13}}, &
    \mat{r}_2 &= \inlinevector{r_{21}, r_{22}, r_{23}}.
\end{align}
We can derive similar equations for all the other quadratic terms.

Finally, let us investigate the derivatives of the ocean current velocities.
The derivative of $\bvel_c$ is
\begin{equation}
    \dot{\bvel}_c = \left(\mat{R}\mat{S}(\bs{\omega})\right)\T \ocean \triangleq \inlinevector{\dot{\mat{r}}_1, \dot{\mat{r}}_2, \dot{\mat{r}}_3} \ocean,
\end{equation}
where $\dot{\mat{r}}_1$, $\dot{\mat{r}}_2$, and $\dot{\mat{r}}_3$ denote the columns of $\mat{R}\mat{S}(\bs{\omega})$.
The components of $\dot{\bvel}_c$ are thus given by
\begin{align}
    \dot{u}_c &= \dot{\mat{r}}_1\T \ocean, &
    \dot{v}_c &= \dot{\mat{r}}_2\T \ocean, &
    \dot{w}_c &= \dot{\mat{r}}_3\T \ocean.
\end{align}

We have thus shown that the \gls{auv} dynamics can be split into an ocean current-independent and an ocean current-dependent part.
%This fact can be used when designing a low-level controller.
%If the orientation and the absolute velocities of the \gls{auv} can be measured, but the ocean current is unknown, then we can design a series of observers, using $\Phi_u$, $\Phi_p$, $\Phi_q$, and $\Phi_r$ as regressors.

In the remainder of this section, we derive a component form for the 5\gls{dof} and 3\gls{dof} models.

\subsubsection{5\gls{dof} Component Form}
In the 5\gls{dof} model, the roll dynamics are disregarded.
Consequently, the inertia, damping, and Coriolis matrices of the 5\gls{dof} model are obtained by removing the fourth row and fourth column from the inertia, damping, and Coriolis matrices of the 6\gls{dof} model from Section~\ref{sec:model_control_oriented}.

The assumptions for deriving the 5\gls{dof} control-oriented model are analogous to the assumptions in Section~\ref{sec:model_control_oriented}, with one exception.
Due to fewer degrees of freedom, Assumption~\ref{asm:actuators} must be modified.

\begin{customasm}{\ref*{asm:actuators} (5DOF)}
    \label{asm:actuators_5DOF}
    The vehicle is equipped with a propeller and fins.
    Consequently, the vehicle is capable of generating a force in the surge direction and torques in pitch and yaw.
\end{customasm}
Under this assumption, the external forces acting on the vehicle are
\begin{align}
    \bs{\tau} &= \mat{B}\mat{f}, &
    \mat{B} &= 
    \begin{bmatrix}
        b_{11} & 0 & 0 \\ 0 & 0 & b_{23} \\ 0 & b_{32} & 0 \\ 0 & b_{42} & 0 \\ 0 & 0 & b_{53}
    \end{bmatrix},
\end{align}
where $\mat{f} = \inlinevector{T_u, T_q, T_r}$ is the control input.

Similarly to Section~\ref{sec:model_control_oriented}, we can perform a change of coordinates so that the actuators produce no sway or heave acceleration.
In other words, for all inputs $\mat{f}$, there exist $f_u, t_q, t_r \in \mathbb{R}$ such that
\begin{equation}
    \mat{M}^{-1}\mat{B}\mat{f} = \inlinevector{f_u, 0, 0, t_q, t_r}.
\end{equation}

The component form of the 5\gls{dof} model is then simply obtained by substituting $p = 0$ into \eqref{eq:background_component_form_absolute_2}
\begin{subequations}
    \begin{align}
        \dot{u} &= F_u(\vel) + f_u + \Phi_u(\vel, \theta, \psi)\T\mathbb{V}_c, \\
        \dot{v} &= X_v(u)r + Y_v(u)v + \Phi_v(\vel, \theta, \psi)\T\mathbb{V}_c, \\
        \dot{w} &= X_w(u)q + Y_w(u)w + G(\theta) + \Phi_w(\vel, \theta, \psi)\T\mathbb{V}_c, \\
        \dot{q} &= F_q(\vel) + t_q + \Phi_q(\vel, \theta, \psi)\T\mathbb{V}_c, \\
        \dot{r} &= F_r(\vel) + t_r + \Phi_r(\vel, \theta, \psi)\T\mathbb{V}_c.
    \end{align}
    \label{eq:background_component_form_5DOF}
\end{subequations}

\subsubsection{3\gls{dof} Component Form}
First, let us discuss the 3\gls{dof} control-oriented model.
Similarly to the previous section, the model is derived using assumptions that are analogous to the ones in Section~\ref{sec:model_control_oriented}, with some modifications.
Namely, due to fewer degrees of freedom, Assumption~\ref{asm:actuators} needs to be modified.

\begin{customasm}{\ref*{asm:actuators} (3DOF)}
    \label{asm:actuators_3DOF}
    The vehicle is equipped with a propeller and a rudder.
    Consequently, the vehicle is capable of generating a force in the surge direction and a torque in yaw.
\end{customasm}

Under this assumption, the external forces acting on the vehicle are
\begin{align}
    \bs{\tau} &= \mat{B}\mat{f}, &
    \mat{B} &= 
    \begin{bmatrix}
        b_{11} & 0 \\ 0 & b_{22} \\ 0 & b_{32}
    \end{bmatrix},
\end{align}
where $\mat{f} = \inlinevector{T_u, T_r}$ is the control input.

The inertia, damping, and Coriolis matrices of the 3\gls{dof} model are \cite{fredriksen_global_2006}
\begin{align}
    \mat{M} &= 
    \begin{bmatrix}
        m_{11} & 0 & 0 \\ 0 & m_{22} & m_{23} \\ 0 & m_{23} & m_{33}
    \end{bmatrix}, \quad
    \mat{D} =
    \begin{bmatrix}
        d_{11} & 0 & 0 \\ 0 & d_{22} & d_{23} \\ 0 & d_{32} & d_{33}
    \end{bmatrix}, \\
    \mat{C}(\vel_r) &=
    \begin{bmatrix}
        0 & 0 & -m_{22}v_r - m_{23}r \\ 0 & 0 & m_{11}u_r \\ m_{22}v_r + m_{23}r & -m_{11}u_r & 0
    \end{bmatrix}.
\end{align}

In \cite{fredriksen_global_2006}, it is shown that the origin of the body-fixed coordinate frame can be chosen such that the actuators produce no sway acceleration.
In other words, for all inputs $\mat{f}$, there exist $f_u, t_r \in \mathbb{R}$ such that
\begin{equation}
    \mat{M}^{-1}\mat{B}\mat{f} = \inlinevector{f_u, 0, t_r}.
\end{equation}
The change of coordinates is done by translating the origin of the body-fixed frame by $\varepsilon$ along the $x$-axis.
The transformed velocities are given by
\begin{align}
    \vel_r^{\prime} &= \mat{H}\vel_r, &
    \mat{H} &= 
    \begin{bmatrix}
        1 & 0 & 0 \\ 0 & 1 & \varepsilon \\ 0 & 0 & 1
    \end{bmatrix}.
\end{align}
Similarly to the procedure in Section~\ref{sec:model_control_oriented}, we define the transformed inertia matrix $\mat{M}^{\prime} = \mat{H}\T\mat{M}\mat{H}$.
If we choose
\begin{equation}
    \varepsilon = -\frac{b_{22}\,m_{33}-b_{32}\,m_{23}}{b_{22}\,m_{23}-b_{32}\,m_{22}},
\end{equation}
then the effect of actuators in the new coordinate frame is given by
\begin{equation}
    {\mat{M}^{\prime}}^{-1}\mat{H}\T\mat{B}\mat{f}
    =
    \begin{bmatrix} \frac{b_{11}}{m_{11}} T_{u} \\ 0 \\ \frac{b_{32}\,m_{22}-b_{22}\,m_{23}}{m_{22}\,m_{33}-{m_{23}}^2} T_{r} \end{bmatrix}.
\end{equation}
We can then perform a change of inputs
\begin{align}
    f_u &= \frac{b_{11}}{m_{11}} T_{u}, &
    t_r &= \frac{b_{32}\,m_{22}-b_{22}\,m_{23}}{m_{22}\,m_{33}-{m_{23}}^2} T_{r},
\end{align}
and express the 3\gls{dof} model in the following component form
\begin{subequations}
    \begin{align}
        \dot{u}_r &= F_u(\vel_r) + f_u, \\
        \dot{v}_r &= X_v(u_r)r + Y_v(u_r)v_r, \\
        \dot{r} &= F_r(\vel_r) + t_r.
    \end{align}
    \label{eq:background_component_form_3DOF}
\end{subequations}
The expressions for $F_u$, $X_v$, $Y_v$, and $F_r$ are shown in Appendix~\ref{app:component_form}.

