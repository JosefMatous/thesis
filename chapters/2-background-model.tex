\section{Mathematical Models of Marine Vehicles}
\label{sec:model}

This section presents control-oriented models of \glspl{asv} and \glspl{auv}.

\subsubsection*{State variables and degrees of freedom}
Marine vehicles are typically modeled as rigid bodies.
A rigid body moving in three-dimensional space has six \glspl{dof}, three for position and three for orientation.

The position of a marine vehicle is commonly expressed in a local \gls{ned} coordinate frame.
Although the \gls{ned} frame is not inertial, it is often used as an approximation of an inertial coordinate frame for short-term and short-distance missions, since the effect of Earth's rotation on the vehicles is negligible.
In general, we will denote the position of the vehicles as $\mat{p} = \inlinevector{x, y, z}$.

The orientation of a vehicle can be expressed using \emph{Euler angles}, $\bs{\Theta} = \inlinevector{\phi, \theta, \psi}$, where $\phi$ is the roll angle, $\theta$ is the pitch angle, and $\psi$ is the yaw angle.
The complete position and orientation vector of the vehicle is then given by $\bs{\eta}\T = \left[\mat{p}\T, \bs{\Theta}\T\right]$.

Euler angles can represent any orientation.
However, in some cases, this representation is not unique.
For example, the following two sets of Euler angles represent the same attitude
\begin{align}
    \bs{\Theta}_1 &= \inlinevector{\frac{\pi}{2}, \frac{\pi}{2}, 0}, &
    \bs{\Theta}_2 &= \inlinevector{0, \frac{\pi}{2}, -\frac{\pi}{2}}. &
\end{align}
At these attitudes, there exist mathematical singularities called \emph{gimbal locks} \cite{chaturvedi_rigid-body_2011}.
Furthermore, the use of Euler angles in control may lead to a phenomenon called \emph{unwinding} \cite{sanjay_topological_2000}, in which the vehicle performs an unnecessary rotation to reach the desired attitude.

The orientation of a vehicle can also be described using a \emph{rotation matrix}.
Rotation matrices are members of the special orthogonal group $SO(3)$.
Unlike Euler angles, rotation matrices do not suffer from singularities.
For a given set of Euler angles, the corresponding rotation matrix is given by \cite{fossen_handbook_2011}
\begin{equation}
    \mat{R}(\phi, \theta, \psi)
    =
    \begin{bmatrix} 
        {\rm c}_{\psi}\,{\rm c}_{\theta} & {\rm c}_{\psi}\,{\rm s}_{\phi}\,{\rm s}_{\theta}-{\rm c}_{\phi}\,{\rm s}_{\psi} & {\rm s}_{\phi}\,{\rm s}_{\psi}+{\rm c}_{\phi}\,{\rm c}_{\psi}\,{\rm s}_{\theta}\\ {\rm c}_{\theta}\,{\rm s}_{\psi} & {\rm c}_{\phi}\,{\rm c}_{\psi}+{\rm s}_{\phi}\,{\rm s}_{\psi}\,{\rm s}_{\theta} & {\rm c}_{\phi}\,{\rm s}_{\psi}\,{\rm s}_{\theta}-{\rm c}_{\psi}\,{\rm s}_{\phi}\\ -{\rm s}_{\theta} & {\rm c}_{\theta}\,{\rm s}_{\phi} & {\rm c}_{\phi}\,{\rm c}_{\theta} 
    \end{bmatrix},
    \label{eq:background_Rzyx}
\end{equation}
where ${\rm c}_x$ and ${\rm s}_x$ represent the cosine and sine of the corresponding angle.

Next, let us discuss the representation of velocities.
The velocities of the vehicle are expressed in the \emph{body-fixed} frame, a non-inertial coordinate frame attached to the vehicle, with the $x$-axis pointing towards the bow (front) side, the $y$-axis pointing to the starboard (right) side, and the $z$-axis pointing to the bottom side of the vehicle.
The \emph{linear velocities} of the vehicle $\mat{v} = \inlinevector{u, v, w}$ consist of the surge, sway, and heave velocities.
The \emph{angular velocities} of the vehicle $\bs{\omega} = \inlinevector{p, q, r}$ consist of the roll, pitch, and yaw rates.
The full velocity vector is then given by $\bs{\nu}\T = \left[\mat{v}\T, \bs{\omega}\T\right]$.

Finally, let us discuss simplified 3\gls{dof} and 5\gls{dof} models.
In the case of \glspl{asv} or \glspl{auv} moving in the horizontal plane, we often assume that the roll and pitch angles are zero, and the depth is constant.
Consequently, we can disregard the roll, pitch, and heave motion of the vehicle, and derive a simplified 3\gls{dof} model with $\bs{\eta} = \inlinevector{x, y, \psi}$ and $\bs{\nu} = \inlinevector{u, v, r}$.
In the case slender, torpedo shaped \glspl{auv}, the roll motion is assumed to be small and self-stabilizing by the design of the vehicle.
Consequently, we can disregard the roll motion and derive a simplified 5\gls{dof} model with $\bs{\eta} = \inlinevector{x, y, z, \theta, \psi}$ and $\bs{\nu} = \inlinevector{u, v, w, q, r}$.

\subsubsection*{Kinematics}
First, let us discuss the kinematics of the vehicles, starting with the 6\gls{dof} model.
The time-derivative of the position is
\begin{equation}
    \dot{\mat{p}} = \mat{R} \mat{v}.
    \label{eq:background_p_dot_6DOF}
\end{equation}
The time-derivative of Euler angles is given by \cite{fossen_handbook_2011}
\begin{align}
    \dot{\bs{\Theta}} &= \mat{T}(\bs{\Theta}) \bs{\omega}, &
    \mat{T}(\bs{\Theta}) &= 
    \begin{bmatrix}
        1 & {\rm s}_{\phi}{\rm t}_{\theta} & {\rm c}_{\phi}{\rm t}_{\theta} \\ 0 & {\rm c}_{\phi} & -{\rm s}_{\phi} \\ 0 & {\rm s}_{\phi}/{\rm c}_{\theta} & {\rm c}_{\phi}/{\rm c}_{\theta}
    \end{bmatrix},
    \label{eq:background_theta_dot_6DOF}
\end{align}
where ${\rm t}_{\theta} = \tan(\theta)$.
Due to the aforementioned singularities, $\dot{\bs{\Theta}}$ is not defined for $\theta = \pm \pi/2$.
The time-derivative of a rotation matrix is given by
\begin{align}
    \dot{\mat{R}} &= \mat{R} \mat{S}(\omega), &
    \mat{S}(\omega) &=
    \begin{bmatrix}
        0 & -r & q\\ r & 0 & -p\\ -q & p & 0
    \end{bmatrix}.
\end{align}

To derive the kinematics of the 5\gls{dof} model, we simply substitute $\phi = 0$ and $p = 0$ into \eqref{eq:background_p_dot_6DOF} and \eqref{eq:background_theta_dot_6DOF}
\begin{subequations}
    \begin{align}
        \dot{\mat{p}} &= \mat{R}(0, \theta, \psi) \mat{v}, \\
        \dot{\theta} &= q, \\
        \dot{\psi} &= \frac{r}{\cos(\theta)}.
    \end{align}
\end{subequations}


