\section{Mathematical Models of Marine Vehicles}
\label{sec:model}

This section presents control-oriented models of \glspl{asv} and \glspl{auv}.

\subsubsection*{State variables and degrees of freedom}
Marine vehicles are typically modeled as rigid bodies.
A rigid body moving in three-dimensional space has six \glspl{dof}, three for position and three for orientation.

The position of a marine vehicle is commonly expressed in a local \gls{ned} coordinate frame.
Although the \gls{ned} frame is not inertial, it is often used as an approximation of an inertial coordinate frame for short-term and short-distance missions, since the effect of Earth's rotation on the vehicles is negligible.
In general, we will denote the position of the vehicles as $\mat{p} = \inlinevector{x, y, z}$.

The orientation of a vehicle can be expressed using \emph{Euler angles}, $\bs{\Theta} = \inlinevector{\phi, \theta, \psi}$, where $\phi$ is the roll angle, $\theta$ is the pitch angle, and $\psi$ is the yaw angle.
The complete position and orientation vector of the vehicle is then given by $\bs{\eta}\T = \left[\mat{p}\T, \bs{\Theta}\T\right]$.

Euler angles can represent any orientation.
However, in some cases, this representation is not unique.
For example, the following two sets of Euler angles represent the same attitude
\begin{align}
    \bs{\Theta}_1 &= \inlinevector{\frac{\pi}{2}, \frac{\pi}{2}, 0}, &
    \bs{\Theta}_2 &= \inlinevector{0, \frac{\pi}{2}, -\frac{\pi}{2}}. &
\end{align}
At these attitudes, there exist mathematical singularities called \emph{gimbal locks} \cite{chaturvedi_rigid-body_2011}.
Furthermore, the use of Euler angles in control may lead to a phenomenon called \emph{unwinding} \cite{sanjay_topological_2000}, in which the vehicle performs an unnecessary rotation to reach the desired attitude.

The orientation of a vehicle can also be described using a \emph{rotation matrix}.
Rotation matrices are members of the special orthogonal group $SO(3)$.
Unlike Euler angles, rotation matrices do not suffer from singularities.
For a given set of Euler angles, the corresponding rotation matrix is given by \cite{fossen_handbook_2011}
\begin{equation}
    \mat{R}(\phi, \theta, \psi)
    =
    \begin{bmatrix} 
        {\rm c}_{\psi}\,{\rm c}_{\theta} & {\rm c}_{\psi}\,{\rm s}_{\phi}\,{\rm s}_{\theta}-{\rm c}_{\phi}\,{\rm s}_{\psi} & {\rm s}_{\phi}\,{\rm s}_{\psi}+{\rm c}_{\phi}\,{\rm c}_{\psi}\,{\rm s}_{\theta}\\ {\rm c}_{\theta}\,{\rm s}_{\psi} & {\rm c}_{\phi}\,{\rm c}_{\psi}+{\rm s}_{\phi}\,{\rm s}_{\psi}\,{\rm s}_{\theta} & {\rm c}_{\phi}\,{\rm s}_{\psi}\,{\rm s}_{\theta}-{\rm c}_{\psi}\,{\rm s}_{\phi}\\ -{\rm s}_{\theta} & {\rm c}_{\theta}\,{\rm s}_{\phi} & {\rm c}_{\phi}\,{\rm c}_{\theta} 
    \end{bmatrix},
    \label{eq:background_Rzyx}
\end{equation}
where ${\rm c}_x$ and ${\rm s}_x$ represent the cosine and sine of the corresponding angle.

Next, let us discuss the representation of velocities.
The velocities of the vehicle are expressed in the \emph{body-fixed} frame, a non-inertial coordinate frame attached to the vehicle, with the $x$-axis pointing towards the bow (front) side, the $y$-axis pointing to the starboard (right) side, and the $z$-axis pointing to the bottom side of the vehicle.
The \emph{linear velocities} of the vehicle $\mat{v} = \inlinevector{u, v, w}$ consist of the surge, sway, and heave velocities.
The \emph{angular velocities} of the vehicle $\bs{\omega} = \inlinevector{p, q, r}$ consist of the roll, pitch, and yaw rates.
The full velocity vector is then given by $\bs{\nu}\T = \left[\mat{v}\T, \bs{\omega}\T\right]$.

Finally, let us discuss simplified 3\gls{dof} and 5\gls{dof} models.
In the case of \glspl{asv} or \glspl{auv} moving in the horizontal plane, we often assume that the roll and pitch angles are zero, and the depth is constant.
Consequently, we can disregard the roll, pitch, and heave motion of the vehicle, and derive a simplified 3\gls{dof} model with $\bs{\eta} = \inlinevector{x, y, \psi}$ and $\bs{\nu} = \inlinevector{u, v, r}$.
In the case slender, torpedo shaped \glspl{auv}, the roll motion is assumed to be small and self-stabilizing by the design of the vehicle.
Consequently, we can disregard the roll motion and derive a simplified 5\gls{dof} model with $\bs{\eta} = \inlinevector{x, y, z, \theta, \psi}$ and $\bs{\nu} = \inlinevector{u, v, w, q, r}$.

\subsubsection*{Kinematics}
First, let us discuss the kinematics of the vehicles, starting with the 6\gls{dof} model.
The time-derivative of the position is
\begin{equation}
    \dot{\mat{p}} = \mat{R} \mat{v}.
    \label{eq:background_p_dot_6DOF}
\end{equation}
The time-derivative of Euler angles is given by \cite{fossen_handbook_2011}
\begin{align}
    \dot{\bs{\Theta}} &= \mat{T}(\bs{\Theta}) \bs{\omega}, &
    \mat{T}(\bs{\Theta}) &= 
    \begin{bmatrix}
        1 & {\rm s}_{\phi}{\rm t}_{\theta} & {\rm c}_{\phi}{\rm t}_{\theta} \\ 0 & {\rm c}_{\phi} & -{\rm s}_{\phi} \\ 0 & {\rm s}_{\phi}/{\rm c}_{\theta} & {\rm c}_{\phi}/{\rm c}_{\theta}
    \end{bmatrix},
    \label{eq:background_theta_dot_6DOF}
\end{align}
where ${\rm t}_{\theta} = \tan(\theta)$.
Due to the aforementioned singularities, $\dot{\bs{\Theta}}$ is not defined for $\theta = \pm \pi/2$.
The time-derivative of a rotation matrix is given by
\begin{align}
    \dot{\mat{R}} &= \mat{R} \mat{S}(\omega), &
    \mat{S}(\omega) &=
    \begin{bmatrix}
        0 & -r & q\\ r & 0 & -p\\ -q & p & 0
    \end{bmatrix}.
\end{align}

To derive the kinematics of the 5\gls{dof} model, we simply substitute $\phi = 0$ and $p = 0$ into \eqref{eq:background_p_dot_6DOF} and \eqref{eq:background_theta_dot_6DOF}
\begin{subequations}
    \begin{align}
        \dot{\mat{p}} &= \mat{R}(0, \theta, \psi) \mat{v}, \\
        \dot{\theta} &= q, \\
        \dot{\psi} &= \frac{r}{\cos(\theta)}.
    \end{align}
\end{subequations}

Similarly, we can derive the kinematics of the 3\gls{dof} model by substituting $z = \phi = \theta = w = p = q = 0$ into \eqref{eq:background_p_dot_6DOF} and \eqref{eq:background_theta_dot_6DOF}
\begin{align}
    \dot{\bs{\eta}} &= \mat{J}(\psi) \bs{\nu}, &
    \mat{J}(\psi) &=
    \begin{bmatrix}
        {\rm c}_{\psi} & -{\rm s}_{\psi} & 0 \\
        {\rm s}_{\psi} & {\rm c}_{\psi} & 0 \\
        0 & 0 & 1
    \end{bmatrix}.
\end{align}

\subsubsection*{Dynamics}
When modeling the dynamics of marine vehicles, we often need to consider the effect of \emph{sea loads} such as waves, wind, and ocean currents.
Let $\mat{V}_c \in \mathbb{R}^3$ be a vector that represents the velocity of the ocean current in the inertial coordinate frame.
Since the dynamics of ocean currents is typically much slower than the dynamics of the vehicle, the ocean current can be considered constant.
Let $\mat{v}_c = \mat{R}\T \mat{V}_c$ denote the velocity of the ocean current expressed in the vehicle's body-fixed frame.
Furthermore, let $\mat{v}_r = \mat{v} - \mat{v}_c \triangleq \inlinevector{u_r, v_r, w_r}$ denote the relative surge, sway, and heave velocity of the vehicle, and let $\bs{\nu}_r\T = \left[\mat{v}_r\T, \bs{\omega}\T\right]$ denote the full relative velocity vector.
The dynamics of the vehicle can then be expressed using the following matrix-vector model \cite{fossen_handbook_2011}
\begin{equation}
    \mat{M} \dot{\bs{\nu}}_r + \mat{C}(\bs{\nu}_r)\bs{\nu}_r + \mat{D}(\bs{\nu}_r)\bs{\nu}_r + \mat{g}(\mat{R}) = \bs{\tau},
    \label{eq:background_nu_dot_relative}
\end{equation}
where $\mat{M}$ is the mass and inertia matrix, including the added mass effects, $\mat{C}(\bs{\nu}_r)$ is the Coriolis and centripetal matrix, also including the added mass, $\mat{D}(\bs{\nu}_r)$ is the hydrodynamic damping matrix, $\mat{g}(\mat{R})$ represents the effects of gravity and buoyancy, and $\bs{\tau}$ represents additional forces and torques such as the effects of actuators and external disturbances.

The model in \eqref{eq:background_nu_dot_relative} can also be expressed in terms of absolute velocities
\begin{equation}
    \mat{M} \left(\dot{\bs{\nu}} - \dot{\bs{\nu}}_c\right) + \mat{C}(\bs{\nu} - \bs{\nu}_c)(\bs{\nu} - \bs{\nu}_c) + \mat{D}(\bs{\nu} - \bs{\nu}_c)(\bs{\nu} - \bs{\nu}_c) + \mat{g}(\mat{R}) = \bs{\tau},
\end{equation}
where $\bs{\nu}_c\T = \left[\mat{v}_c\T, \mat{0}\T\right]$.

The inertia matrix $\mat{M}$ is symmetric positive definite, the damping matrix $\mat{D}$ is positive definite, and the Coriolis matrix $\mat{C}$ is skew-symmetric.
There exist multiple expressions for the Coriolis matrix, \emph{e.g.,}
\begin{equation}
    \label{eq:background_coriolis}
    \mat{C}(\bs{\nu}_r)\! =\!
    \begin{bmatrix}
        \mat{O}_{3 \times 3} & \!\!-\mat{S}(\mat{M}_{11}\mat{v}_r + \mat{M}_{12}\bs{\omega}) \\
        -\mat{S}(\mat{M}_{11}\mat{v}_r + \mat{M}_{21}\bs{\omega}) & \!\!-\mat{S}(\mat{M}_{21}\mat{v}_r + \mat{M}_{22}\bs{\omega})
    \end{bmatrix}, \,
    \begin{bmatrix}
        \mat{M}_{11} & \!\!\!\mat{M}_{12} \\ \mat{M}_{21} & \!\!\!\mat{M}_{22}
    \end{bmatrix}\!
    = \!\mat{M}.
\end{equation}
The gravity and buoyancy vector is given by \cite{fossen_handbook_2011}
\begin{equation}
    \mat{g}(\mat{R}) = -
    \begin{bmatrix}
        (W - B) \mat{R}\T \mat{e}_3 \\
        (W\mat{r}_g - B\mat{r}_b) \times \mat{R}\T \mat{e}_3
    \end{bmatrix},
\end{equation}
where $W \in \mathbb{R}_{> 0}$ is the gravitational force, $B \in \mathbb{R}_{> 0}$ is the buoyant force, $\mat{r}_g$ is the position of the center of gravity, $\mat{r}_b$ is the position of the center of buoyancy, and $\mat{e}_3 = \inlinevector{0, 0, 1}$.

\subsubsection*{Control-oriented model of underactuated \glspl{auv}}
The goal of this section is to present a control-oriented model, a simplified mathematical model that is then used to design a controller and analyze its closed-loop properties.
The model in this section is similar to the one presented in \cite{borhaug_straight_2007}.
We begin by presenting the assumptions that allow us to simplify the matrix-vector model in \eqref{eq:background_nu_dot_relative}.

\begin{asm}
    \label{asm:symmetric}
    The vehicle is slender, torpedo-shaped with rotational symmetry around its $x$-axis.
\end{asm}

\begin{asm}
    \label{asm:damping}
    The vehicle is maneuvering at low speeds.
    Consequently, the hydrodynamic damping can be considered linear.
\end{asm}

\noindent Under Assumption~\ref{asm:damping}, the hydrodynamic damping matrix is constant.
Under Assumption~\ref{asm:symmetric}, the inertia and damping matrices have the following structure
\begin{align}
    \mat{M}\! &= \!\!
    \begin{bmatrix}
        m_{11}\!\! & 0 & 0 & 0 & 0 & 0\\ 0 & \!m_{22}\! & 0 & 0 & 0 & m_{26}\\ 0 & 0 & \!m_{33}\! & 0 & \!m_{35}\! & 0\\ 0 & 0 & 0 & \!m_{44}\! & 0 & 0\\ 0 & 0 & \!m_{35}\! & 0 & \!m_{55}\! & 0\\ 0 & \!m_{26}\! & 0 & 0 & 0 & m_{66}
    \end{bmatrix}\!, &
    \mat{D}\! &= \!\!
    \begin{bmatrix}
        d_{11}\!\! & 0 & 0 & 0 & 0 & 0\\ 0 & \!d_{22}\! & 0 & 0 & 0 & d_{26}\\ 0 & 0 & \!d_{33}\! & 0 & \!d_{35}\! & 0\\ 0 & 0 & 0 & \!d_{44}\! & 0 & 0\\ 0 & 0 & \!d_{53}\! & 0 & \!d_{55}\! & 0\\ 0 & \!d_{62}\! & 0 & 0 & 0 & d_{66}
    \end{bmatrix}\!,
\end{align}
where $m_{22} = m_{33}$, $m_{35} = -m_{26}$, and $m_{55} = m_{66}$.

\begin{asm}
    \label{asm:actuators}
    The vehicle is equipped with a propeller and fins.
    The fins are placed symmetrically around the $x$-axis of the vehicle.
    Consequently, the vehicle is capable of generating force in the surge direction and torque around all three axes.
\end{asm}
Under this assumption, the external forces acting on the vehicle are given by
\begin{align}
    \bs{\tau} &= \mat{B}\mat{u}, &
    \mat{B} &= 
    \begin{bmatrix}
        b_{11} & 0 & 0 & 0 \\ 0 & 0 & 0 & b_{24} \\ 0 & 0 & b_{33} & 0 \\ 0 & b_{42} & 0 & 0 \\ 0 & 0 & b_{53} & 0 \\ 0 & 0 & 0 & b_{64}
    \end{bmatrix},
\end{align}
where $\mat{u} = \inlinevector{f_u, t_p, t_q, t_r}$ is the control input consisting of surge thrust and the forces produced by the fins.
Furthermore, due to the symmetry assumption, it follows that $b_{33} = -b_{24}$ and $b_{53} = b_{64}$.

\begin{prop}
    \label{prop:neutral_point}
    If a vehicle model satisfies Assumptions~\ref{asm:symmetric}--\ref{asm:actuators}, then the origin of the body-fixed coordinate frame can be chosen such that the actuators produce no sway or heave acceleration.
    In other words, for all inputs $\mat{u}$, there exist $\tau_u, \tau_p, \tau_q, \tau_r \in \mathbb{R}$ such that
    \begin{equation}
        \mat{M}^{-1}\bs{\tau} = \mat{M}^{-1}\mat{B}\mat{u} = \inlinevector{\tau_u, 0, 0, \tau_p, \tau_q, \tau_r}.
        \label{eq:neutral_point_force}
    \end{equation}
\end{prop}
\begin{figure}[b]
    \centering
    \def\svgwidth{0.4\textwidth}
    \import{figures/background}{auv_frames.pdf_tex}
    \caption{Illustration of the coordinate frames. CN denotes the origin of the \gls{ned} frame, CO is the origin of the body-fixed frame, and ${\rm CO}^{\prime}$ is the transformed body-fixed coordinate frame.}
    \label{fig:background_auv_frames}
\end{figure}
\begin{proof}
    Let CO denote the original body-fixed coordinate frame, and let $\bs{\nu}$ denote the velocities of the vehicle expressed in CO.
    Consider a new coordinate frame, ${\rm CO}^{\prime}$, created by translating CO by a distance $x_{\varepsilon}$ along the $x$-axis (see Figure~\ref{fig:background_auv_frames}).
    Let $\bs{\nu}^{\prime}$ denote the velocities of the vehicle expressed in ${\rm CO}^{\prime}$.
    The relation between $\bs{\nu}$ and $\bs{\nu}^{\prime}$ is
    \begin{align}
        \bs{\nu} &= \mat{H}(\bs{\varepsilon}) \bs{\nu}^{\prime}, &
        \mat{H}(\bs{\varepsilon}) &=
        \begin{bmatrix}
            \mat{I}_3 & \mat{S}(\bs{\varepsilon}) \\
            \mat{O}_{3 \times 3} & \mat{I}_3
        \end{bmatrix},
    \end{align}
    where $\bs{\varepsilon} = \inlinevector{x_{\varepsilon}, 0, 0}$.
    The equations of motion expressed in ${\rm CO}^{\prime}$ are then given by \cite{fossen_handbook_2011}
    \begin{equation}
        \mat{M}^{\prime}\dot{\bs{\nu}}_r^{\prime} + \mat{C}^{\prime}(\bs{\nu}^{\prime}_r)\bs{\nu}^{\prime}_r + \mat{D}^{\prime}\bs{\nu}^{\prime}_r + \mat{H}(\bs{\varepsilon})\T\mat{g}(\mat{R}) = \mat{H}(\bs{\varepsilon})\T\bs{\tau},
    \end{equation}
    where $\mat{M}^{\prime} = \mat{H}(\bs{\varepsilon})\T \mat{M} \mat{H}(\bs{\varepsilon})$, $\mat{D}^{\prime} = \mat{H}(\bs{\varepsilon})\T \mat{D} \mat{H}(\bs{\varepsilon})$ are the transformed mass and damping matrices, and $\mat{C}^{\prime}(\bs{\nu}^{\prime}_r) = \mat{H}(\bs{\varepsilon})\T\mat{C}(\bs{\nu}_r)\mat{H}(\bs{\varepsilon})$.

    The effect of actuators on $\bs{\nu}^{\prime}$ is then given by
    \begin{equation}
        {\mat{M}^{\prime}}^{-1}\mat{H}(\bs{\varepsilon})\T\mat{B}\mat{u} =
        \begin{bmatrix}
            \frac{b_{11}}{m_{11}} f_{u}\\ \frac{\left(b_{24}\,m_{66}-b_{64}\,m_{26}-b_{24}\,m_{26}\,x_{\varepsilon}+b_{64}\,m_{22}\,x_{\varepsilon}\right)}{m_{22}\,m_{66}-{m_{26}}^2} t_{r} \\ \frac{\left(b_{33}\,m_{55}-b_{53}\,m_{35}+b_{33}\,m_{35}\,x_{\varepsilon}-b_{53}\,m_{33}\,x_{\varepsilon}\right)}{m_{33}\,m_{55}-{m_{35}}^2} t_{q} \\ \frac{b_{42}}{m_{44}} t_{p}\\ \frac{\left(b_{53}\,m_{33}-b_{33}\,m_{35}\right)}{m_{33}\,m_{55}-{m_{35}}^2} t_{q} \\ \frac{\left(b_{64}\,m_{22}-b_{24}\,m_{26}\right)}{m_{22}\,m_{66}-{m_{26}}^2} t_{r}
        \end{bmatrix}.
    \end{equation}
    Due to the symmetry assumptions, we can choose $x_{\varepsilon}$ such that
    \begin{equation}
        x_{\varepsilon} = 
        \frac{b_{24}\,m_{66}-b_{64}\,m_{26}}{b_{24}\,m_{26}-b_{64}\,m_{22}} = 
        -\frac{b_{33}\,m_{55}-b_{53}\,m_{35}}{b_{33}\,m_{35}-b_{53}\,m_{33}},
    \end{equation}
    and the effect of actuators on $\bs{\nu}^{\prime}$ thus becomes
    \begin{equation}
        {\mat{M}^{\prime}}^{-1}\mat{H}(\bs{\varepsilon})\T\mat{B}\mat{u} = \inlinevector{\tau_u, 0, 0, \tau_p, \tau_q, \tau_r},
    \end{equation} \vspace{-1.5em}
    \begin{subequations}
        \begin{align}
            \tau_u &= \frac{b_{11}}{m_{11}} f_{u}, &
            \tau_p &= \frac{b_{42}}{m_{44}} t_{p}, \\
            \tau_q &= \frac{\left(b_{53}\,m_{33}-b_{33}\,m_{35}\right)}{m_{33}\,m_{55}-{m_{35}}^2} t_{q}, &
            \tau_r &= \frac{\left(b_{64}\,m_{22}-b_{24}\,m_{26}\right)}{m_{22}\,m_{66}-{m_{26}}^2} t_{r}.
        \end{align} 
        \label{eq:tau_upqr}
    \end{subequations}
    We have thus shown that for all inputs $\mat{u}$, there exist $\tau_u, \tau_p, \tau_q, \tau_r$ such that \eqref{eq:neutral_point_force} holds.
    Moreover, if all the numerators in \eqref{eq:tau_upqr} are nonzero, then the converse holds as well, \emph{i.e.,} for all $\tau_u, \tau_p, \tau_q, \tau_r$, there exists an input $\mat{u}$ such that \eqref{eq:neutral_point_force} holds.
\end{proof}

Now, let us present a modified Assumption~\ref{asm:symmetric} and Proposition~\ref{prop:neutral_point} for 5\gls{dof} vehicles.

\begin{customasm}{1 (5DOF)}
    The vehicle is slender, torpedo-shaped with port-starboard symmetry.
\end{customasm}

\begin{asm}
    \label{asm:buoyancy}
    The vehicle is neutrally buoyant, with the centers of gravity and buoyancy located on one vertical axis.
\end{asm}
\noindent Under this assumption, $\mat{g}(\mat{R})$ has the following shape
\begin{align}
    \mat{g}(\mat{R}) &= \begin{bmatrix}
        \mat{0}_3 \\ Wz_{gb} \mat{e}_3 \times \mat{R}\T \mat{e}_3
    \end{bmatrix},
    \label{eq:gravity}
\end{align}
where $z_{gb}$ is the distance between the centers of gravity and buoyancy.
