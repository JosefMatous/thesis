\section{Null-Space-Behavioral Algorithm}
\label{sec:background_NSB}

This section describes the \acrfull{nsb} algorithm.
The \gls{nsb} algorithm is a method that allows us to combine several tasks in a hierarchic manner.
The algorithm was originally developed for first-order systems
\begin{equation}
    \dot{\mat{p}} = \ivel,
\end{equation}
where $\mat{p} \in \mathbb{R}^n$ are the generalized coordinates, and $\ivel \in \mathbb{R}^n$ are the input velocities.

In \gls{nsb} algorithms, the desired behavior of the system is divided into several tasks.
Let there be $M$ tasks arranged by priority in descending order (\emph{i.e.,} task 1 has the highest priority, task $M$ has the lowest priority).
Let $\bs{\sigma}_1, \ldots, \bs{\sigma}_M$ denote the so-called \emph{task variables}.
Each variable is a function of the system coordinates, \emph{i.e.,}
\begin{align}
    \bs{\sigma}_i &= f_i(\mat{p}), &
    f_i &: \mathbb{R}^n \mapsto \mathbb{R}^{m_i},
\end{align}
where $m_i \leq n$ is the dimensionality of task $i$.
Applying the chain rule, the time-derivative of $\bs{\sigma}_i$ is
\begin{equation}
    \dot{\bs{\sigma}}_i = \frac{\partial f_i(\mat{p})}{\partial \mat{p}} \ivel \triangleq \mat{J}_i(\mat{p}) \ivel.
\end{equation}
Let $\dot{\bs{\sigma}}_i^{*}$ be the desired closed-loop behavior of the task variable.
Then, the smallest input (in terms of Euclidean norm) that guarantees the desired behavior is
\begin{equation}
    \ivel_i = \mat{J}_i^{\dagger} \dot{\bs{\sigma}}_i^{*},
    \label{eq:background_NSB_velocity_i}
\end{equation}
where $\mat{J}_i^{\dagger}$ is the Moore-Penrose pseudoinverse of the task Jacobian.

\begin{rmk*}
    In many applications, the task variable should track some pre-defined desired value, $\bs{\sigma}_{d, i}$.
    In such cases, we typically choose
    \begin{equation}
        \dot{\bs{\sigma}}_i^{*} = \dot{\bs{\sigma}}_{d, i} - \bs{\Lambda}_i \left(\bs{\sigma}_i - \bs{\sigma}_{d,i}\right),  
        \label{eq:background_CLIK}  
    \end{equation}
    where $\bs{\Lambda}_i$ is a positive definite gain matrix.
    In the literature, \eqref{eq:background_CLIK} is commonly referred to as the \gls{clik} equation \cite{antonelli_kinematic_2006}.
\end{rmk*}

If the task is \emph{redundant}, \emph{i.e.}, if the inequality $m_i < n$ strictly holds, then there exists a subspace of control inputs that do not interfere with the task.
Let $\ivel_{\rm add}$ be an additional input.
Then, the following control input
\begin{equation}
    \ivel = \ivel_i + \mat{N}_i \ivel_{\rm add},
\end{equation}
where $\mat{N}_i = \mat{I}_{N} - \mat{J}_i^{\dagger}\mat{J}_i$ is the null-space projector of $\mat{J}_i$, guarantees the desired behavior of the task.
The additional control input is satisfied only if it does not interfere with the task.

In the \gls{nsb} algorithm, the control inputs from the individual tasks are composed by projecting the inputs from the lower-priority tasks onto the null space of the higher-priority tasks.
In the literature, there exist two variants of the algorithm.
The first variant calculates the control input $\ivel$ using the following equation
\begin{align}
    \ivel &= \ivel_{1} + \mat{N}_1 \biggl(\ivel_{2} + \mat{N}_2\Bigl(\ivel_3 \cdots + \mat{N}_{M-2}\bigl(\ivel_{M-1} + \mat{N}_{M-1}\ivel_M\bigr)\Bigr)\biggr),
    \label{eq:background_NSB_recursive}
\end{align}
with $\ivel_{i}$ given by \eqref{eq:background_NSB_velocity_i}.

The second variant uses the so-called \emph{augmented} Jacobians
\begin{align}
    \Bar{\mat{J}}_i &= \inlinevector{\mat{J}_1\T, \ldots, \mat{J}_i\T}.
    \label{eq:background_NSB_augmented}
\end{align}
Let $\Bar{\mat{N}}_i$ denote the null-space projector of $\Bar{\mat{J}}_i$.
Then, the control input $\ivel$ is given by
\begin{equation}
    \ivel = \ivel_1 + \sum_{i=2}^M \Bar{\mat{N}}_{i-1} \ivel_i.
\end{equation}
The advantages and disadvantages of both approaches are discussed in \cite{antonelli_stability_2008}.
In this thesis, we will mostly use the first variant.

\subsection{Independence and orthogonality}
\label{sec:background_NSB_independence}
The concepts of independence and orthogonality are important when analyzing the interactions between the tasks.
Specifically, these concepts determine whether the tasks can be executed simultaneously, and how the null-space projector affects the lower-priority tasks.

Two tasks are \emph{independent} if the pseudoinverses of their Jacobians are linearly independent.
Let $\mat{J}_i$ and $\mat{J}_j$ denote the Jacobians of task $i$ and $j$, respectively.
These tasks are independent if
\begin{equation}
    {\rank} \left(\mat{J}_i^{\dagger}\right) + {\rank} \left(\mat{J}_j^{\dagger}\right) = {\rm rank} \left(\left[ \mat{J}_i^{\dagger},\, \mat{J}_j^{\dagger} \right]\right),
    \label{eq:background_independence_pseudo}
\end{equation}
Antonelli \emph{et al.} \cite{antonelli_stability_2008} remark that the pseudoinverse and the transpose of the task Jacobian share the same span.
Consequently, the condition in \eqref{eq:background_independence_pseudo} is equivalent to
\begin{equation}
    {\rank} \left(\mat{J}_i\T\right) + {\rank} \left(\mat{J}_j\T\right) = {\rm rank} \left(\left[ \mat{J}_i\T,\, \mat{J}_j\T \right]\right).
    \label{eq:background_independence}
\end{equation}

Two tasks are \emph{orthogonal} if the subspaces spanned by these tasks are orthogonal, \emph{i.e.,} if
\begin{equation}
    \mat{J}_i \, \mat{J}_j^{\dagger} = \mat{O}_{n_i \times n_j}.
\end{equation}

Now, let us consider two consecutive tasks that are independent and orthogonal. 
Without loss of generality, let us denote these tasks $1$ and $2$.
The control input produced by combining these two tasks is
\begin{equation}
    \ivel = \mat{J}_1^{\dagger} \dot{\bs{\sigma}}_1^{*} + \mat{N}_1 \mat{J}_2^{\dagger} \dot{\bs{\sigma}}_2^{*}
    = \mat{J}_1^{\dagger} \dot{\bs{\sigma}}_1^{*} + \left(\mat{I} - \mat{J}_1^{\dagger}\mat{J}_1\right) \mat{J}_2^{\dagger} \dot{\bs{\sigma}}_2^{*}
    = \mat{J}_1^{\dagger} \dot{\bs{\sigma}}_1^{*} + \mat{J}_2^{\dagger} \dot{\bs{\sigma}}_2^{*}.
\end{equation}

We have thus shown that if two consecutive tasks are independent and orthogonal, they can be executed simultaneously.
Moreover, the null-space projector does not affect the lower-priority task.

\subsection{\gls{nsb} algorithm for the formation path-following problem}
\label{sec:background_nsb_formation_path_following}
In the remainder of this section, we demonstrate how the \gls{nsb} algorithm can be used to solve the formation path-following problem.
A similar scheme was proposed in \cite{arrichiello_formation_2006} for static formations, in \cite{antonelli_2006_kinematic,antonelli_stability_2008} for circular formations, and in \cite{eek_formation_2021} for dynamic formations.

Let $\mat{p}\T = \left[\mat{p}_1\T, \ldots, \mat{p}_N\T\right]$ be the concatenated position vector of $N$ vehicles.
To solve the problem, we define two tasks: formation-keeping and path-following.
The formation-keeping task has the highest priority.
The task variable, $\bs{\sigma}_1$, is given by
\begin{align}
    \bs{\sigma}_1 &= \inlinevector{\bs{\sigma}_{1, 1}\T, \bs{\sigma}_{1, 2}\T, \ldots, \bs{\sigma}_{1, N-1}\T}, &
    \bs{\sigma}_{1, i} &= \mat{p}_i - \mat{p}_b,
    \label{eq:background_formation_variable}
\end{align}
where $\mat{p}_b = \frac{1}{N}\sum\limits_{i=1}^N \mat{p}_i$ is the barycenter of the formation (see Section~\ref{sec:background_formation_path_following}).

\begin{rmk*}
    The formation-keeping task contains the relative positions of the first $N-1$ vehicles.
    The relative position of the last vehicle is omitted because it can be expressed as a linear combination of the remaining relative positions.
    Indeed, from \eqref{eq:background_formation_variable}, we get
    \begin{equation}
        \bs{\sigma}_{1, N} = \mat{p}_N - \mat{p}_b = - \sum_{i=1}^{N-1} \bs{\sigma}_{1, i}.
    \end{equation}
    By omitting the last relative position vector, the task Jacobian has full row rank.
    Indeed, the Jacobian of the formation-keeping task is
    \begin{equation}
        \begin{split}
            \mat{J}_1 = \frac{\partial \bs{\sigma}_1}{\partial \mat{p}} &= 
            \begin{bmatrix}
                \frac{N-1}{N} \mat{I}_3 & -\frac{1}{N} \mat{I}_3 & \cdots & -\frac{1}{N} \mat{I}_3 & -\frac{1}{N} \mat{I}_3 \\[3pt]
                -\frac{1}{N} \mat{I}_3 & \frac{N-1}{N} \mat{I}_3 & & -\frac{1}{N} \mat{I}_3 & -\frac{1}{N} \mat{I}_3 \\ 
                \vdots & & \ddots & & \vdots \\
                -\frac{1}{N} \mat{I}_3 & -\frac{1}{N} \mat{I}_3 & \cdots & \frac{N-1}{N} \mat{I}_3 & -\frac{1}{N} \mat{I}_3
            \end{bmatrix} \\[6pt]
            &= \left[\mat{I}_{3(N-1)},\, \mat{O}_{3(N-1) \times 3}\right] - \frac{1}{N} \mat{1}_{(N-1) \times N} \otimes \mat{I}_3.
        \end{split}
        \label{eq:background_NSB_formation_jacobian}
    \end{equation}
    One can verify that the rank of $\mat{J}_1$ is $3(N-1)$, and the Jacobian thus has full row rank.
\end{rmk*}

The desired value of the formation-keeping task variable is
\begin{align}
    \bs{\sigma}_{d, 1} &= \inlinevector{\mat{p}_{f, 1}\T, \mat{p}_{f, 2}\T, \ldots, \mat{p}_{f, N-1}\T},
    \label{eq:background_NSB_formation_reference}
\end{align}
where $\mat{p}_{f, i}$ is the desired position of vehicle $i$ within the formation (see Section~\ref{sec:background_formation_path_following}).

The formation-keeping control input $\ivel_1$ is then given by
\begin{equation}
    \ivel_1 = \mat{J}_1^{\dagger} \left(\dot{\bs{\sigma}}_{d, 1} - \bs{\Lambda}_1 (\bs{\sigma}_1 - \bs{\sigma}_{d, 1})\right),
    \label{eq:background_NSB_formation_velocity}
\end{equation}
where $\bs{\Lambda}_1$ is a positive definite gain matrix.

For the path-following task, the task variable is given by the position of the barycenter \emph{i.e.,}
\begin{equation}
    \bs{\sigma}_2 = \mat{p}_b = \frac{1}{N} \sum_{i=1}^{N} \mat{p}_i.
\end{equation}

\begin{prop}
    The formation-keeping and path-following tasks are independent and orthogonal.
\end{prop}
\begin{proof}
    The Jacobian of the path-following task is given by
    \begin{equation}
        \mat{J}_2 = \frac{\partial \bs{\sigma}_2}{\partial \mat{p}} = \frac{1}{N} \mat{1}_{1 \times N} \otimes \mat{I}_3.
    \end{equation}
    The matrices $\mat{J}_1$ and $\mat{J}_2$ satisfy
    \begin{equation}
        \rank \left(\left[\mat{J}_1\T,\, \mat{J}_2\T\right]\right) = 3N = \rank \left(\mat{J}_1\T\right) + \rank \left(\mat{J}_2\T\right),
    \end{equation}
    and the tasks are thus independent.
    Moreover, the pseudoinverse of $\mat{J}_2$ is given by
    \begin{equation}
        \mat{J}_2^{\dagger} = \mat{1}_{N \times 1} \otimes \mat{I}_3.
    \end{equation}
    One can then verify that the Jacobians satisfy
    \begin{equation}
        \mat{J}_1 \mat{J}_2^{\dagger} = \mat{O}_{3(N-1) \times 3},
    \end{equation}
    and the tasks are thus orthogonal.
\end{proof}

The desired value of the path-following task is given by
\begin{equation}
    \bs{\sigma}_{d, 2} = \mat{p}_p(s),
\end{equation}
where $s$ is the value of the path parameter.

We propose to solve the path-following problem using \acrlong{los} guidance.
The desired behavior of the path-following task is thus given by
\begin{equation}
    \bs{\sigma}_2^{*} = \ivel_{\rm LOS},
\end{equation}
where $\ivel_{\rm LOS}$ is either the coupled \eqref{eq:background_los_coupled} or decoupled \eqref{eq:background_los_decoupled} \gls{los} guidance law.
The path-following control input $\ivel_2$ is then given by
\begin{equation}
    \ivel_2 = \mat{J}_2^{\dagger} \bs{\sigma}_2^{*} = \mat{1}_{N \times 1} \otimes \ivel_{\rm LOS}.
\end{equation}

Thanks to the independence and orthogonality of the tasks, the combined control input $\ivel$ is given by
\begin{equation}
    \ivel = \ivel_1 + \mat{N}_1 \ivel_2 = \ivel_1 + \ivel_2.
\end{equation}

Finally, let us investigate the closed-loop behavior of the tasks.
First, we need to define the error variables.
The formation-keeping error is defined as
\begin{equation}
    \widetilde{\bs{\sigma}}_1 = \bs{\sigma}_1 - \bs{\sigma}_{1, d}.
    \label{eq:background_sigma_1_tilde}
\end{equation}
The path-following error is given by the position of the barycenter in the path-tangential coordinate frame, \emph{i.e.,}
\begin{equation}
    \widetilde{\bs{\sigma}}_2 = \mat{p}_b^p = \mat{R}_p(s)\T \left(\mat{p}_b - \mat{p}_p(s)\right).
    \label{eq:background_sigma_2_tilde}
\end{equation}

Now, let us analyze the closed-loop behavior of the formation-keeping error.
Differentiating \eqref{eq:background_sigma_1_tilde} with respect to time, we get
\begin{equation}
    \dot{\widetilde{\bs{\sigma}}}_1 = \mat{J}_1 \ivel - \dot{\bs{\sigma}}_{1, d} = - \Lambda_1 \widetilde{\bs{\sigma}}_1.
    \label{eq:background_sigma_1_tilde_dot}
\end{equation}
Since $\Lambda_1$ is positive definite by design, the closed-loop system \eqref{eq:background_sigma_1_tilde_dot} is \glspl{ges}.

From \eqref{eq:background_sigma_2_tilde}, the dynamics of the path-following error are given by
\begin{equation}
    \begin{split}
        \dot{\mat{p}}_b^p &= \mat{R}_p(s)\T \left(\mat{J}_2 \ivel - \dot{\mat{p}}_p(s)\right) - \mat{S}(\bs{\omega}_p(s))\mat{R}_p(s)\T\left(\mat{p}_b - \mat{p}_p(s)\right) \\
        &= \mat{R}_p(s)\T \left(\ivel_{\rm LOS} - \dot{\mat{p}}_p(s)\right) - \mat{S}(\bs{\omega}_p(s))\mat{p}_b^p.
    \end{split}
\end{equation}
The stability of the path-following task depends on the choice of the \gls{los} guidance law.
The stability of controllers that utilize decoupled and coupled \gls{los} guidance will be discussed in Chapters~\ref{chap:5dof_nsb} and \ref{chap:NSB_R}, respectively.
