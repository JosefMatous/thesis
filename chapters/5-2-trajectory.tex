\chapter{Trajectory Tracking and Path Following using the Hand Position Concept}
\label{chap:handpos_trajectory}

\newcommand{\citeexistingmethods}{\cite{rezazadegan_trajectory-tracking-backstepping_2015,alonge_trajectory-tracking-backstepping_2001,elmokadem_trajectory-tracking-SMC_2016,aguiar_trajectory-tracking-CLF_2007,abdelaal_trajectory-tracking-MPC_2015,caharija_path-following-ILOS_2016,xiang_path-following-robust_2017,miao_path-following-curvilinear_2017,borhaug_straight_2007}}

This chapter presents hand position-based trajectory tracking and path following controllers for underactuated underwater vehicles.
Using Lyapunov analysis, we show that the proposed controllers exponentially track the desired trajectory or path, while the angular velocities of the vehicle remain bounded.
The theoretical results are verified both in numerical simulations and experiments.
The contents of this chapter are based on \cite{matous_trajectory_2023}.

\section{Introduction}

The ability to track a given trajectory or path is vital to the function of \acrfullpl{auv}.
Many high-level mission planners rely on accurate trajectory tracking or path following \cite{mcgann_trex_2007}.
To solve the trajectory tracking problem, numerous methods based on backstepping \cite{rezazadegan_trajectory-tracking-backstepping_2015,alonge_trajectory-tracking-backstepping_2001}, sliding-mode control \cite{elmokadem_trajectory-tracking-SMC_2016}, control Lyapunov functions \cite{aguiar_trajectory-tracking-CLF_2007}, and model predictive control \cite{abdelaal_trajectory-tracking-MPC_2015} have been proposed.    
To solve the path following problem, most controllers utilize \acrfull{los} guidance.
In \cite{caharija_path-following-ILOS_2016}, an integral line of sight guidance scheme is used to counteract the sea loads, \cite{xiang_path-following-robust_2017} combines LOS guidance with an adaptively tuned PID controller, \cite{miao_path-following-curvilinear_2017} uses LOS with active disturbance rejection control, and \cite{borhaug_straight_2007} uses LOS guidance and consensus to control a formation of AUVs.

In the trajectory tracking and path following schemes presented in \citeexistingmethods, the goal is to control either the center of mass or the \emph{neutral point} (also referred to as the \emph{pivot point} \cite{paliotta_trajectory_2019}) of the vehicle.
The neutral point is a location on the $x$-axis (the stern-bow axis) of the vehicle such that, if chosen as the origin of the vehicle's body-fixed coordinate frame, the lateral motion of the vehicle is not affected by its control inputs.
In this work, we propose to instead use the \emph{hand position} concept to control the AUV.
The advantages of using the hand position as the output were discussed in the previous chapter.
This chapter extends the trajectory-tracking and path-following controllers proposed in \cite{paliotta_trajectory_2019} to three dimensions.

The remainder of the chapter is organized as follows.
Sections~\ref{sec:handpos_trajectory_trajectory_tracking} and \ref{sec:handpos_trajectory_path_following} present and analyze the trajectory-tracking and the path-following controller, respectively.
Sections~\ref{sec:handpos_trajectory_simulations} and \ref{sec:handpos_trajectory_experiments} show the results of numerical simulations and experiments, respectively.
Finally, Section~\ref{sec:handpos_trajectory_conclusions} presents some concluding remarks.

\section{Trajectory Tracking}
\label{sec:handpos_trajectory_trajectory_tracking}
In this section, we propose a control law for tracking a predefined trajectory.
Let $\xi_{1_d}$ represent the desired trajectory, and let $\xi_{2_d} = \dot{\xi}_{1_d}$.
We assume that there exist $\xi_{2_d, \max}$ and $\dot{\xi}_{2_d, \max}$ such that
\begin{align}
    \norm{V_c} &< \norm{\xi_{2_d}} \leq \xi_{2_d, \max}, &
    \norm{\dot{\xi}_{2_d}} \leq \dot{\xi}_{2_d, \max}.
\end{align}

The goal of trajectory tracking is to control the vehicle such that $x_1$ converges to $\xi_{1_d}$.
To achieve the goal, we define the following error states
\begin{subequations}
    \begin{align}
        \tilde{x}_1 &= x_1 - \xi_{1_d}, \\
        \tilde{x}_2 &= x_2 - \xi_{2_d}, \\
        \tilde{x}_I &= \int_{0}^t \tilde{x}_1(s)\, {\rm d}s,
    \end{align} \label{eq:handpos_trajectory_error_states}
\end{subequations}
and choose the following PID controller
\begin{equation}
    \mu = -k_p \tilde{x}_1 - k_d \tilde{x}_2 - k_I \tilde{x}_I + \dot{\xi}_{2_d}, \label{eq:handpos_trajectory_external_controller}
\end{equation}

\noindent where $k_p$, $k_d$, and $k_I$ are positive gains chosen such that the matrix
\begin{equation}
    H_{\xi}
    =
    \begin{bmatrix}
        O_{3 \times 3} & I_{3 \times 3} & O_{3 \times 3} \\ O_{3 \times 3} & O_{3 \times 3} & I_{3 \times 3} \\ -k_I I_{3 \times 3} & -k_p I_{3 \times 3} & -k_d I_{3 \times 3}
    \end{bmatrix} \label{eq:handpos_trajectory_H_xi}
\end{equation}
is Hurwitz.

\subsection{Closed-loop analysis}
In this section, we investigate the closed-loop dynamics of the system \eqref{eq:handpos_def_hand_position_dynamics} with the control law \eqref{eq:handpos_trajectory_external_controller}.
We begin by defining an additional change of coordinates:
\begin{subequations}
    \begin{align}
        \tilde{\xi}_1 &= \tilde{x}_1, \\
        \tilde{\xi}_2 &= \tilde{x}_2 + V_c, \\
        \tilde{\xi}_I &= \tilde{x}_I - \frac{k_d}{k_I}V_c.
    \end{align} \label{eq:handpos_trajectory_hand_transform_CL}
\end{subequations}
This change is necessary to transform the equilibrium of the closed-loop system to the origin, as the error states $\tilde{x}_2$ and $\tilde{x}_I$ defined in \eqref{eq:handpos_trajectory_external_controller} do not converge to zero.

For convenience, we will also define a concatenated state vector $\Xi\T = \left[{\tilde{\xi}}_I\T, {\tilde{\xi}}_1\T, {\tilde{\xi}}_2\T\right]$.
Differentiating \eqref{eq:handpos_trajectory_hand_transform_CL} with respect to time and substituting the external dynamics \eqref{eq:handpos_def_x_1_dot_transformed}--\eqref{eq:handpos_def_x_2_dot_transformed} and the control law \eqref{eq:handpos_trajectory_external_controller}, we get
\begin{equation}
    \dot{\Xi} = H_{\xi} \Xi. \label{eq:handpos_trajectory_external_dynamics}
\end{equation}

\begin{prop}
    \label{prop:handpos_trajectory_trajectory_tracking}
    The origin $\Xi = \mat{0}$ is a \acrfullpl{ges} equilibrium point of \eqref{eq:handpos_trajectory_external_dynamics}.
    Consequently, $x_1$, $x_2$, and $\tilde{x}_I$ exponentially converge to $\xi_{1_d}$, $\xi_{2_d} - V_c$, and $k_I/k_d\,V_c$, respectively.

    Moreover, let us define $a_x, \Bar{\alpha}_y, \Bar{\alpha}_z$ in accordance with \eqref{eq:handpos_def_a_bounds}.
    The internal dynamics are ultimately bounded if $a_x, \Bar{\alpha}_y, \Bar{\alpha}_z > 0$.
\end{prop}
\begin{proof}
    Since the matrix $H_{\xi}$, defined in \eqref{eq:handpos_trajectory_H_xi} is Hurwitz by design, we can conclude that the external dynamics are \glspl{ges}, and $\tilde{\xi}_1$, $\tilde{\xi}_2$, and $\tilde{\xi}_I$ exponentially converge to zero.

    From \eqref{eq:handpos_trajectory_error_states}, we can conclude that if $\tilde{\xi}_1$ exponentially converges to zero, then the hand position $x_1$ exponentially converges towards the desired trajectory $\xi_{1_d}$.
    Similarly, if $\tilde{\xi}_2$ converges to zero, then the relative hand velocity $x_2$ converges to $\xi_{2_d} - V_c$.
    Moreover, if $\tilde{\xi}_I$ converges to zero, then the integral state $\tilde{x}_I$ converges to $k_I/k_d\,V_c$.
    Consequently, the integral state provides an estimate of the ocean current.

    Moreover, because the external dynamics are stable, the error states $\tilde{x}_1$, $\tilde{x}_2$, and $\tilde{x}_I$ are bounded.
    Consequently, the control input $\mu$ is bounded.
    Therefore, if $a_x, \Bar{\alpha}_y, \Bar{\alpha}_z > 0$, then all assumptions of Lemma~\ref{lemma:handpos_def_ultimate_boundedness} are satisfied, and the internal dynamics are ultimately bounded.
\end{proof}

\subsection{Straight-line Trajectory Tracking}
\label{sec:handpos_trajectory_straight_line_trajectory}
In this section, we will focus on the special case when $\xi_{2_d}$ is constant, and the vehicle is thus tracking a straight line.
The purpose of this section is to demonstrate that in this special case, we can prove that both the external and internal dynamics are exponentially stable.
First, let us present the necessary definitions and assumptions.

\begin{dfn}
    Two vectors $a, b \in \mathbb{R}^3$ are aligned if there exists $k \in \mathbb{R}$ such that $a = kb$.
    Equivalently, $a$ and $b$ are aligned if $a \times b = \mat{0}_3$.
    \label{dfn:aligned}
\end{dfn}
\begin{asm}
    The distance between the centers of gravity and buoyancy, $z_{gb}$, is positive, and the vector $\xi_{2_{d, r}} = (\xi_{2_d} - V_c)$ is not aligned with $e_3$.
    \label{asm:straight_line_trajectory}
\end{asm}
\begin{rmk}
    From \eqref{eq:gravity}, one can verify that if $z_{gb}$ is positive, then the restoring forces stabilize the vehicle's roll angle around zero.
    Consequently, most commercial AUVs are designed so that $z_{gb} > 0$.
    The second part of Assumption~\ref{asm:straight_line_trajectory} can be satisfied by choosing an appropriate reference trajectory.
\end{rmk}

We begin by finding the equilibria of the closed-loop system.
Since the external dynamics is a linear system, $\Xi = 0_{9}$ is the only equilibrium.
From \eqref{eq:handpos_def_R_dot_CL}, we can conclude that $\dot{R} = O_{3 \times 3}$ if and only if $\omega = 0_3$.
Substituting $\Xi = 0_{9}$ and $\omega = 0_3$ into \eqref{eq:handpos_def_omega_dot_CL} yields
\begin{equation}
    %\begin{split}
        \dot{\omega} = \Bar{L} \times \left(\mathcal{D}_{\nu}(\zeta_r) + \mathcal{C}_{\nu}(\zeta_r)\right) 
        - \left(\Bar{L}L\T\right) \left(\mathcal{D}_{\omega}(\zeta_r) + M_{22}^{\prime}\left(Wz_{gb} e_3 \times R\T e_3\right)\right).
    %\end{split}
    \label{eq:handpos_trajectory_omega_dot_SS}
\end{equation}
Note that the right-hand-side of \eqref{eq:handpos_trajectory_omega_dot_SS} has the following form
\begin{equation}
    \dot{\omega} = \Bar{L} \times a + \left(\Bar{L}L\T\right) b,
\end{equation}
where $a = \inlinevector{a_1, a_2, a_3}$, $b = \inlinevector{b_1, b_2, b_3}$ are vectors in $\mathbb{R}^3$.
From the definition of $L$ and $\Bar{L}$, the following two equalities hold for any $a$ and $b$:
\begin{align}
    \Bar{L} \times a &= \frac{1}{h} \inlinevector{0, -a_3, a_2}, &
    \left(\Bar{L}L\T\right) b &= \inlinevector{b_1, 0, 0}.
\end{align}
Consequently, $\dot{\omega}$ is zero if and only if both terms of the right-hand-side of \eqref{eq:handpos_trajectory_omega_dot_SS} are zero.

The first term is zero only if $\left(\mathcal{D}_{\nu}(\zeta_r) + \mathcal{C}_{\nu}(\zeta_r)\right)$ is aligned with $e_1$.
In other words, there exists $k \in \mathbb{R}$ such that
\begin{equation}
    \mathcal{D}_{\nu}(\zeta_r) + \mathcal{C}_{\nu}(\zeta_r) = k e_1. \label{eq:handpos_trajectory_steady_state_condition_1}
\end{equation}
Substituting \eqref{eq:handpos_def_ode_components} into \eqref{eq:handpos_trajectory_steady_state_condition_1}, we get the following equation
\begin{equation}
    \begin{bmatrix}
        \frac{d_{11}}{m_{11}}\,u_r \\
        \frac{m_{66}d_{22} - d_{62}m_{26} + m_{26}\left(m_{11} - m_{22}\right)u_r}{m_{22}m_{66} - m_{26}^2}\,v_r \\
        \frac{m_{55}d_{33} - d_{53}m_{35} - m_{35}\left(m_{11} - m_{33}\right)u_r}{m_{33}m_{55} - m_{35}^2}\,w_r
    \end{bmatrix}
     = k e_1, \label{eq:handpos_trajectory_steady_state_condition_1_expanded}
\end{equation}
which has at least one solution given by
\begin{equation}
    \inlinevector{u_r, v_r, w_r} = \frac{m_{11}}{d_{11}} k e_1. \label{eq:handpos_trajectory_steady_state_solution_1}
\end{equation}
Note that if
\begin{equation}
    \frac{d_{62}m_{26} - m_{66}d_{22}}{m_{26}\left(m_{11} - m_{22}\right)}
    \neq
    \frac{m_{55}d_{33} - d_{53}m_{35}}{m_{35}\left(m_{11} - m_{33}\right)},
    \label{eq:handpos_trajectory_inequality}
\end{equation}
then \eqref{eq:handpos_trajectory_steady_state_solution_1} is the only solution of \eqref{eq:handpos_trajectory_steady_state_condition_1_expanded}.
From \eqref{eq:handpos_def_nu_r_transformed}, the steady-state linear velocities must also satisfy
\begin{align}
    \nu_r &= R\T \xi_{2_{d, r}}& &\implies& \norm{\nu_r} &= \norm{\xi_{2_{d, r}}}. \label{eq:handpos_trajectory_steady_state_condition_2}
\end{align}
Combining \eqref{eq:handpos_trajectory_steady_state_solution_1} and \eqref{eq:handpos_trajectory_steady_state_condition_2} gives us two possible steady-state linear velocities
\begin{equation}
    \nu_r = \pm \norm{\xi_{2_{d, r}}}\,e_1,
\end{equation}
which leads to the following condition on the steady-state orientation
\begin{equation}
    R e_1 = \pm \frac{\xi_{2_{d, r}}}{\norm{\xi_{2_{d, r}}}}. \label{eq:handpos_trajectory_steady_state_R_1}
\end{equation}
This condition does not uniquely define the steady-state orientation.
Indeed, in terms of Euler angles, \eqref{eq:handpos_trajectory_steady_state_R_1}, defines the steady-state yaw and pitch angles.
%Note that the steady-state pitch angle is $\pm \frac{\pi}{2}$ radians only if $\xi_{2_{d, r}}$ is aligned with $e_3$.
An additional condition on the steady-state orientation comes from the second term in \eqref{eq:handpos_trajectory_omega_dot_SS}.
This term is zero if
\begin{equation}
    e_1\T \left(Wz_{gb} e_3 \times R\T e_3\right) = 0. \label{eq:handpos_trajectory_steady_state_R_2}
\end{equation}
If Assumption~\ref{asm:straight_line_trajectory} holds, then \eqref{eq:handpos_trajectory_steady_state_R_2} is equivalent to
\begin{equation}
    \sin\phi = 0,
\end{equation}
where $\phi \in [0, 2\pi)$ is the steady-state roll angle.
Consequently, \eqref{eq:handpos_trajectory_steady_state_R_1} and \eqref{eq:handpos_trajectory_steady_state_R_2} result in four distinct steady-state orientations.
Intuitively, the vehicle can have positive or negative surge velocity, and a roll angle of zero or $\pi$ radians.
These four equilibria are illustrated in Figure~\ref{fig:equilibria}.

\begin{figure}[tb]
    \centering
    \def\svgwidth{0.48\textwidth}
    \import{figures/handpos_trajectory}{auv_equilibria.pdf_tex}
    \caption{Illustration of the four equilibria. (a) Positive surge velocity, zero roll angle. (b) Positive surge velocity, $\pi$ radians roll. (c) Negative surge velocity, zero roll angle. (d) Negative surge velocity, $\pi$ radians roll.}
    \label{fig:equilibria}
\end{figure}

In the remainder of this section, we will analyze the equilibrium in which the vehicle has a positive surge velocity, and the roll angle is zero, \emph{c.f.,} Figure~\ref{fig:equilibria}a.
Let $R_0$ denote the steady-state orientation.
We define the orientation error, $\delta$, as
\begin{equation}
    \delta = {\rm logm}\left(R_0\T\,R\right),
    \label{eq:handpos_trajectory_delta}
\end{equation}
where ${\rm logm}: SO(3) \mapsto \ball[3]{\pi}$ is the matrix logarithm \cite{iserles_lie_2000}, and introduce the following change of coordinates
\begin{align}
    \tilde{\xi}_I^{\prime} &= R\T\,\tilde{\xi}_I, &
    \tilde{\xi}_1^{\prime} &= R\T\,\tilde{\xi}_1, &
    \tilde{\xi}_2^{\prime} &= R\T\,\tilde{\xi}_2.
\end{align}
The motivation behind the change of coordinates is to simplify the relation between the external dynamics and $\nu_r$.
From, \eqref{eq:handpos_def_nu_r_transformed}, $\nu_r$ is given by
\begin{equation}
    \begin{split}
    \nu_r &= R\T\left(\tilde{\xi}_2 + \xi_{2_d} - V_c\right) - \omega \times L \\
        &= \tilde{\xi}_2^{\prime} + {\rm expm}(\delta)\T \norm{\xi_{2_{d,r}}}e_1 - \omega \times L,
    \end{split}
\end{equation}
where ${\rm expm}: \ball[3]{\pi} \mapsto SO(3)$ is the inverse of ${\rm logm}$.

The complete closed-loop system is then given by
\begin{subequations}
    \begin{align}
        \dot{\tilde{\xi}}_I^{\prime} &= \tilde{\xi}_1^{\prime} - \omega \times \tilde{\xi}_I^{\prime}, \\
        \dot{\tilde{\xi}}_1^{\prime} &= \tilde{\xi}_2^{\prime} - \omega \times \tilde{\xi}_1^{\prime}, \\
        \dot{\tilde{\xi}}_2^{\prime} &= -k_I \tilde{\xi}_I^{\prime} - k_p \tilde{\xi}_1^{\prime} - k_d \tilde{\xi}_2^{\prime} - \omega \times \tilde{\xi}_2^{\prime}, \\
        \dot{\delta} &= \omega, \\
        \dot{\omega} &= \Bar{L} \times \bigg(-k_I \tilde{\xi}_I^{\prime} - k_p \tilde{\xi}_1^{\prime} - k_d \tilde{\xi}_2^{\prime} + \mathcal{D}_{\nu}(\zeta_r) + \mathcal{C}_{\nu}(\zeta_r) \nonumber \\
        &\qquad \qquad - \omega \times \left(\tilde{\xi}_2^{\prime} + {\rm expm}(\delta)\T \norm{\xi_{2_{d,r}}}e_1\right)\bigg) \nonumber \\
        &\quad - \left(\Bar{L}L\T\right) \left(\mathcal{D}_{\omega}(\zeta_r) + M_{22}^{\prime}\left(Wz_{gb} e_3 \times R\T e_3\right)\right).
    \end{align}
\end{subequations}
Next, we define a vector $z\T = \left[{\tilde{\xi}_I^{\prime\,{\rm T}}}, {\tilde{\xi}_1^{\prime\,{\rm T}}}, {\tilde{\xi}_2^{\prime\,{\rm T}}}, \delta\T, \omega\T\right]$ and a function $f$ such that $\dot{z} = f(z)$.
Let $J$ denote the Jacobian of $f(z)$, evaluated at $z = 0_{15}$.
$J$ is given by

\begin{equation}
    J \!=\! \begin{bmatrix}
        O_{3 \times 3} & I_{3 \times 3} & O_{3 \times 3} & O_{3 \times 3} & O_{3 \times 3} \\
        O_{3 \times 3} & O_{3 \times 3} & I_{3 \times 3} & O_{3 \times 3} & O_{3 \times 3} \\
        -k_I I_{3 \times 3}\!\!\! & -k_p I_{3 \times 3}\!\!\! & -k_d I_{3 \times 3}\!\!\!\! & O_{3 \times 3} & O_{3 \times 3} \\
        O_{3 \times 3} & O_{3 \times 3} & O_{3 \times 3} & O_{3 \times 3} & I_{3 \times 3} \\
        J_{\xi_I} & J_{\xi_1} & J_{\xi_2} & J_{\delta} & J_{\omega}
    \end{bmatrix}, \label{eq:handpos_trajectory_jacobian}
\end{equation}
where
\begin{subequations}
    \begin{align}
        J_{\xi_I} &= -S\left(\frac{k_I}{h}e_1\right), &
        J_{\xi_1} &= -S\left(\frac{k_p}{h}e_1\right), \\
        J_{\xi_2} &= \begin{bmatrix}
            0 & 0 & 0 \\
            0 & 0 & \Xi_{23} \\
            0 & \Xi_{32} & 0
        \end{bmatrix}, &
        J_{\omega} &= - \begin{bmatrix}
            \Omega_1 & 0 & 0 \\
            0 & \Omega_2 & 0 \\
            0 & 0 & \Omega_3
        \end{bmatrix} \\            
        %{\rm diag}\left(\Omega_1, \Omega_2, \Omega_3\right), \\
        J_{\delta} &= \mathrlap{\begin{bmatrix}
            -\Delta_1\cos\theta & 0 & \Delta_1\sin\theta \\
            0 & -\Delta_2 & 0 \\
            0 & 0 & -\Delta_3
        \end{bmatrix}.}            
    \end{align}
    \label{eq:handpos_trajectory_jacobian_blocks}
\end{subequations}
where $\theta$ is the steady-state pitch angle.
The components of $J_{\xi_2}$, $J_{\omega}$, and $J_{\delta}$ are shown in Appendix~\ref{app:jacobian}.

\begin{prop}
    The point $z = 0_{15}$ is a (locally) exponentially stable equilibrium point of $\dot{z} = f(z)$ if Assumption~\ref{asm:straight_line_trajectory} holds, and all $\Delta_i$ and $\Omega_i$ for $i \in \{1,2,3\}$ are positive.
    \label{prop:straight_line_trajectory}
\end{prop}
\begin{proof}
    Using the indirect Lyapunov method, the system is locally exponentially stable if $J$ is Hurwitz.
    Let us define
    \begin{align}
        J_{21}\! &= \!\begin{bmatrix}
            O_{3 \times 3}\!\! & O_{3 \times 3}\!\! & O_{3 \times 3} \\
            J_{\xi_I} & J_{\xi_1} & J_{\xi_2}
        \end{bmatrix}, &
        J_{22}\! &= \!\begin{bmatrix}
            O_{3 \times 3}\!\! & I_{3 \times 3} \\
            J_{\delta} & J_{\omega}
        \end{bmatrix}\!.
    \end{align}
    The matrix $J$ can then be written in the following form
    \begin{equation}
        J = \begin{bmatrix}
            H_{\xi} & O_{9 \times 6} \\
            J_{21} & J_{22}
        \end{bmatrix}.
    \end{equation}
    Due to its block triangular structure, the eigenvalues of $J$ are equal to the union of eigenvalues of $H_{\xi}$ and $J_{22}$.
    Note that $H_{\xi}$ is Hurwitz by design.
    Furthermore, the eigenvalues of $J_{22}$, $\lambda_1, \ldots, \lambda_6$ are given by
    \begin{subequations}
        \begin{align}
            \lambda_1, \lambda_2 &= -\frac{\Omega_{1}}{2} \pm \frac{\sqrt{{\Omega_{1}}^2-4\,\Delta_{1}\cos\theta_0}}{2}, \\
            \lambda_3, \lambda_4 &= -\frac{\Omega_{2}}{2} \pm \frac{\sqrt{{\Omega_{2}}^2-4\,\Delta_{2}}}{2}, \\
            \lambda_5, \lambda_6 &= -\frac{\Omega_{3}}{2} \pm \frac{\sqrt{{\Omega_{3}}^2-4\,\Delta_{3}}}{2}.
        \end{align}
    \end{subequations}
    If Assumption~\ref{asm:straight_line_trajectory} holds, then the steady-state pitch angle satisfies $\abs{\theta} < \pi/2$.
    Therefore, the real part of $\lambda_1, \ldots, \lambda_6$ is strictly negative if Assumption~\ref{asm:straight_line_trajectory} holds and all $\Delta_i$ and $\Omega_i$ for $i \in \{1,2,3\}$ are positive.
\end{proof}

\begin{rmk}
    For surface vessels, it has been shown that a similar controller can achieve almost global asymptotic stability \cite{paliotta_trajectory_2019}.
    For underwater vehicles, proving almost global stability is complicated.
    It can be shown that vehicles with rotational symmetry around the $x$-axis violate inequality \eqref{eq:handpos_trajectory_inequality}.
    Since most commercial AUVs have a cylindrical shape, this type of symmetry is common among underwater vehicles.
    If a vehicle violates inequality \eqref{eq:handpos_trajectory_inequality}, then there exists a subspace of unstable equilibria in addition to the four previously described equilibrium points, making almost global results impossible.
\end{rmk}

\section{Path Following}
\label{sec:handpos_trajectory_path_following}
In this section, we present a path following controller based on the hand position concept.
Let $s$ be the path parameter, $\mat{p}_p: \mathbb{R} \mapsto \mathbb{R}^{3}$ the parametrization of the desired path, and $U_d > 0$ the desired path following speed.
We assume that $\mat{p}_p$ is $\mathcal{C}^2$ and parametrized by the arc length (see Section~\ref{sec:background_paths}).
Let us define the following three functions
\begin{align}
    \xi_{1_d} &= \mat{p}_p(s), &
    \xi_{2_d} &= \dot{\xi}_{1_d} = \dot{s}\frac{\partial \mat{p}_p\left(s\right)}{\partial s}, &
    \xi_{2_d}^{*} &= U_d \frac{\partial \mat{p}_p(s)}{\partial s},
\end{align}
The goal of the path following is to control the vehicle and continuously update the path parameter such that
\begin{align}
    \lim_{t \rightarrow \infty} x_1(t) - \xi_{1_d}(t) &= 0_3, &
    \lim_{t \rightarrow \infty} \dot{s}(t) &= U_d. \label{eq:handpos_trajectory_control_goal_path}
\end{align}

To solve the path following problem, we define the following error states
\begin{subequations}
    \begin{align}
        \tilde{x}_1 &= x_1 - \xi_{1_d}, \\
        \tilde{x}_2 &= x_2 - \xi_{2_d}^{*}, \\
        \tilde{x}_I &= \int_{0}^t \tilde{x}_1(\tau) {\rm d}\tau,
    \end{align}
\end{subequations}
and propose a PID controller analogous to the one in Section~\ref{sec:handpos_trajectory_trajectory_tracking}
\begin{equation}
    \mu = -k_p\tilde{x}_1 - k_d\tilde{x}_2 - k_I\tilde{x}_I + \dot{\xi}_{2_d}^{*}.
\end{equation}
Inspired by \cite{belleter_2019_observer}, we propose the following update law for the path parameter 
\begin{equation}
    \dot{s} = U_d\left(1 + \varepsilon\tanh\left(k_{\sigma}\sigma\right)\right), \label{eq:handpos_trajectory_s_dot}
\end{equation}
where $\varepsilon$ and $k_{\sigma}$ are positive gains, and $\sigma = \tilde{x}_1\T \frac{\partial \mat{p}_p(s)}{\partial s}$ is the projection of the path following error onto the path-tangential vector (see Figure~\ref{fig:path}).
The motivation behind this update scheme is to allow the path to parameter ``slow down'' or ``speed up'' if the vehicle is lagging or leading the desired path.

\begin{figure}[tb]
    \centering
    \def\svgwidth{0.6 \textwidth}
    \import{figures/handpos_trajectory}{path.pdf_tex}
    \vspace*{-2mm}
    \caption{Illustration of the path function, the path following error, and its projection.}
    \label{fig:path}
\end{figure}

\subsection{Closed-loop Analysis}
Here we investigate the properties of the closed-loop system.
Similarly to Section~\ref{sec:handpos_trajectory_trajectory_tracking}, we define the following change of coordinates
\begin{subequations}
    \begin{align}
        \tilde{\xi}_1 &= \tilde{x}_1, \\
        \tilde{\xi}_2 &= \tilde{x}_2 + V_c, \\
        \tilde{\xi}_I &= \tilde{x}_I - \frac{k_d}{k_I}V_c.
    \end{align} \label{eq:handpos_trajectory_hand_transform_CL_path}
\end{subequations}
The external dynamics of the vehicle are then given by
\begin{equation}
    \dot{\Xi} = H_{\xi}\Xi + \inlinevector{0_3\T, d\T, k_d d\T}, \label{eq:handpos_trajectory_external_dynamics_path}
\end{equation}
where $H_{\xi}$ is given by \eqref{eq:handpos_trajectory_H_xi}, and
\begin{equation}
    d = \varepsilon U_d \tanh\left(k_{\sigma}\sigma\right) \frac{\partial \mat{p}_p(s)}{\partial s}. \label{eq:handpos_trajectory_external_perturbation}
\end{equation}
Since $H_{\xi}$ is Hurwitz by design, for any positive definite matrix $Q$ there exists a positive definite matrix $P$ such that $H_{\xi}\T P + P H_{\xi} = -Q$.
Let $\varrho$ denote the ratio between the smallest eigenvalue of $Q$ and the largest eigenvalue of $P$, and let us choose $Q$ such that $\varrho$ is maximized.

\begin{prop}
    \label{prop:handpos_trajectory_path_following}
    The external dynamics \eqref{eq:handpos_trajectory_external_dynamics_path} are \glspl{ges} if
    \begin{equation}
        \varrho > 2\left(1 + k_d\right)\varepsilon k_{\sigma} U_d.
    \end{equation}
    Consequently, $x_1$, $x_2$, and $\tilde{x}_I$ exponentially converge to $\xi_{1_d}$, $\xi_{2_d}^{*} - V_c$, and $k_d/k_I\,V_c$, respectively, and $\dot{s}$ converges to $U_d$.

    Moreover, let us define 
    \begin{equation}
        \LBnu = \max_{s, \sigma \in \mathbb{R}} \norm{
            U_d\left(1 + \varepsilon\tanh(k_{\sigma}\sigma)\right) \frac{\partial \mat{p}_p(s)}{\partial s}
            - V_c}
            \label{eq:handpos_trajectory_LBnu_path}
    \end{equation}
    and $\Bar{\alpha}_y$ and $\Bar{\alpha}_z$ in accordance with \eqref{eq:handpos_def_a_bounds}.
    The internal dynamics are ultimately bounded if $a_x, \Bar{\alpha}_y, \Bar{\alpha}_z > 0$
\end{prop}
\begin{proof}
    Consider the following Lyapunov function candidate
    \begin{equation}
        V = \Xi\T P \Xi,
    \end{equation}
    The derivative of $V$ along the trajectories of the closed-loop system \eqref{eq:handpos_trajectory_external_dynamics_path} is given by
    \begin{equation}
    \begin{split}
        \dot{V} &= \Xi\T\left(H_{\xi}\T P + PH_{\xi}\right)\Xi + 2 \left[0_3\T, d\T, k_d d\T\right] P \Xi \\
            &\leq -\lambda_{Q,\min}\norm{\Xi}^2 + 2 \lambda_{P,\max}\left(1 + k_d\right)\norm{d}\norm{\Xi},
    \end{split}
    \end{equation}
    where $\lambda_{Q,\min}$ is the smallest eigenvalue of $Q$, and $\lambda_{P,\max}$ is the largest eigenvalue of $P$.
    From \eqref{eq:handpos_trajectory_external_perturbation}, we get the following upper bound on $\norm{d}$
    \begin{equation}
        \norm{d} \leq \varepsilon k_{\sigma} U_d \norm{\tilde{\xi}_1} \leq \varepsilon k_{\sigma} U_d \norm{\Xi},
    \end{equation}
    and arrive at the following expression
    \begin{equation}
        \dot{V} \leq - \left(\lambda_{Q,\min} - 2 \lambda_{P,\max}\left(1 + k_d\right)\varepsilon k_{\sigma} U_d\right) \norm{\Xi}^2.
    \end{equation}
    Therefore, if the following inequality holds
    \begin{equation}
        \frac{\lambda_{Q,\min}}{\lambda_{P,\max}} = \varrho > 2\left(1 + k_d\right)\varepsilon k_{\sigma} U_d,
    \end{equation}
    the origin of the closed-loop system is \glspl{ges}, and thus $\tilde{\xi}_1$, $\tilde{\xi}_2$, and $\tilde{\xi}_I$ exponentially converge to zero.
    Using the same arguments as in the proof of Proposition~\ref{prop:handpos_trajectory_trajectory_tracking}, we can conclude that $x_1$, $x_2$, and $\tilde{x}_I$ exponentially converge to $\xi_{1_d}$, $\xi_{2_d}^{*} - V_c$, and $k_d/k_I\,V_c$, respectively.
    In addition, substituting $\tilde{\xi}_1 = 0_3$ into the path parameter update law \eqref{eq:handpos_trajectory_s_dot} gives $\dot{s} = U_d$.

    Moreover, because the external dynamics are stable, the error states $\tilde{x}_1$, $\tilde{x}_2$, and $\tilde{x}_I$ are bounded.
    Consequently, the control input $\mu$ is bounded.
    Note that $\LBnu$ defined in \eqref{eq:handpos_trajectory_LBnu_path} represents the upper bound on $\norm{\xi_{2_d} - V_c}$.
    Therefore, if $a_x, \Bar{\alpha}_y, \Bar{\alpha}_z > 0$, then all assumptions of Lemma~\ref{lemma:handpos_def_ultimate_boundedness} are satisfied, and the internal dynamics are ultimately bounded.
\end{proof}

\subsection{Straight-line Path Following}
\label{sec:handpos_trajectory_straight_line_path}
In this section, we propose a path following controller for straight-line paths.
Similarly to Section~\ref{sec:handpos_trajectory_straight_line_trajectory}, we can prove the exponential stability of the whole closed-loop system.
Moreover, straight-line paths are natively supported by most guidance, navigation, and control systems, \emph{e.g.}, the \acrfull{dune} \cite{dune}.
Such paths can be parametrized by the following function
\begin{equation}
    \mat{p}_p(s) = \mat{p}_0 + R_p e_1 s, \label{eq:handpos_trajectory_straight_line_function}
\end{equation}
where $\mat{p}_0 \in \mathbb{R}^3$ is the origin of the path, and $R_{p} \in SO(3)$ defines the orientation of the path.

\begin{figure}[tb]
    \centering
    \def\svgwidth{0.6\textwidth}
    \import{figures/handpos_trajectory}{straight_line.pdf_tex}
    \caption{Illustration of straight-line path following.}
    \label{fig:straight_line}
\end{figure}

Instead of the path parameter update law \eqref{eq:handpos_trajectory_s_dot}, we propose to determine the path parameter by finding the closest point on the desired path to the vehicle's hand position.
This approach is commonly done in the literature when following straight lines or circles \cite{breivik_path_following_2004}.
From \eqref{eq:handpos_trajectory_straight_line_function}, the path parameter of the closest point to $x_1$ is given by
\begin{equation}
    s = \left(x_1 - \mat{p}_0\right)\T R_p e_1. \label{eq:handpos_trajectory_straight_line_parameter}
\end{equation}
In addition, let us define cross-track errors, $\widetilde{y}$ and $\widetilde{z}$, as
\begin{equation}
    \inlinevector{0, \widetilde{y}, \widetilde{z}} = R_p\T \left(x_1 - \mat{p}_p(s)\right). \label{eq:handpos_trajectory_cross_track_errors}
\end{equation}
The cross-track errors are illustrated in Figure~\ref{fig:straight_line}.
Substituting \eqref{eq:handpos_trajectory_straight_line_parameter} and \eqref{eq:handpos_trajectory_straight_line_function} into \eqref{eq:handpos_trajectory_cross_track_errors}, we get
\begin{align}
    \begin{bmatrix} \widetilde{y} \\ \widetilde{z} \end{bmatrix} &= \hat{I}\T R_p\T \left(x_1 - \mat{p}_0\right), &
    \hat{I}\T &= \begin{bmatrix} 0 & 1 & 0 \\ 0 & 0 & 1 \end{bmatrix}.
\end{align}

For a straight-line path, the control goal \eqref{eq:handpos_trajectory_control_goal_path} is equivalent to controlling the vehicle such that $\widetilde{y}$ and $\widetilde{z}$ converge to zero.
To achieve the goal, we define the following error states
%\begin{subequations}
    \begin{align}
        \widetilde{x}_1 &= \inlinevector{\widetilde{y}, \widetilde{z}}, &
        \widetilde{x}_2 &= R_p\T x_2 - U_d e_1, &
        \widetilde{x}_I &= \int_{0}^{t} \widetilde{x}_1(\tau) {\rm d}\tau,
    \end{align} 
%\end{subequations}
and a PID control law
\begin{equation}
    \mu = -k_p R_p \hat{I} \widetilde{x}_1 - k_d R_p \widetilde{x}_2 - k_I R_p \hat{I} \widetilde{x}_I,
    \label{eq:handpos_trajectory_straight_line_path_PID}
\end{equation}
where $k_p$, $k_d$, and $k_I$ are positive gains chosen such that the matrix
\begin{equation}
    \Bar{H}_{\xi}
    =
    \begin{bmatrix}
        O_{2 \times 2} & I_{2 \times 2} & O_{2 \times 3} \\
        O_{2 \times 2} & O_{2 \times 2} & \hat{I}\T \\
        -k_I \hat{I} & -k_p \hat{I} & -k_d I_{3 \times 3}
    \end{bmatrix},
    \label{eq:handpos_trajectory_H_xi_prime}
\end{equation}
is Hurwitz.
Similarly to the previous sections, we can perform the following change of coordinates
\begin{subequations}
    \begin{align}
        \widetilde{\xi}_1 &= \widetilde{x}_1, \\
        \widetilde{\xi}_2 &= \widetilde{x}_2 + \tilde{I} R_p\T V_c, \\
        \widetilde{\xi}_I &= \widetilde{x}_I - \frac{k_d}{k_I}\hat{I}\T R_p\T V_c,
    \end{align} \label{eq:handpos_trajectory_error_coordinates_straight_line_path}
\end{subequations}
where $\tilde{I} = \begin{bmatrix} 0 & 0 & 0 \\ 0 & 1 & 0 \\ 0 & 0 & 1 \end{bmatrix}$.
For convenience, let us define $\Xi^{\prime\T} = \left[\widetilde{\xi}_I^{\prime\T}, \widetilde{\xi}_1^{\prime\T}, \widetilde{\xi}_2^{\prime\T}\right]$.
\begin{prop}
    The external dynamics are \gls{ges}.
    Specifically, $\widetilde{y}$ and $\widetilde{z}$ converge to zero, $x_2$ converges to $U_d R_p e_1 - R_p \tilde{I} R_p\T V_c$, and $\widetilde{x}_I$ converges to $\frac{k_d}{k_I} \hat{I}\T R_p\T V_c$.

    Moreover, let us define $\xi_{2_d} = U_d R_p e_1$, $\LBnu = \norm{\xi_{2_d} - V_c}$, and $a_x$, $\Bar{\alpha}_y$, and $\Bar{\alpha}_z$ in accordance with \eqref{eq:handpos_def_a_bounds}.
    The internal dynamics are ultimately bounded if $a_x, \Bar{\alpha}_y, \Bar{\alpha}_z > 0$.
    \label{prop:straight_line_path}
\end{prop}
\begin{proof}
    Differentiating \eqref{eq:handpos_trajectory_error_coordinates_straight_line_path} with respect to time yields
    \begin{equation}
        \dot{\Xi} = \Bar{H}_{\xi} \Xi,
    \end{equation}
    with $\Bar{H}_{\xi}$ defined in \eqref{eq:handpos_trajectory_H_xi_prime}.
    Since $\Bar{H}_{\xi}$ is Hurwitz by design, $\Xi$ exponentially converges to zero.
    From \eqref{eq:handpos_trajectory_error_coordinates_straight_line_path}, if $\Xi$ converges to zero, then the cross-track errors $\widetilde{y}$ and $\widetilde{z}$ converge to zero, $x_2$ converges to $U_d R_p e_1 - R_p \tilde{I} R_p\T V_c$, and $\widetilde{x}_I$ converges to $\frac{k_d}{k_I} \hat{I}\T R_p\T V_c$.

    Moreover, because the external dynamics are stable, the error states $\tilde{x}_1$, $\tilde{x}_2$, and $\tilde{x}_I$ are bounded.
    Consequently, the control input $\mu$ is bounded.
    Therefore, if $a_x, \Bar{\alpha}_y, \Bar{\alpha}_z > 0$, then all assumptions of Lemma~\ref{lemma:handpos_def_ultimate_boundedness} are satisfied, and the internal dynamics are ultimately bounded.
\end{proof}

Finally, let us investigate the exponential stability of the whole closed-loop system.
We begin by finding the equilibria.
Let us define the desired relative velocity $\xi_{2_{d, r}} = U_d R_p e_1 - R_p \tilde{I} R_p\T V_c$.
Using the same procedure as in Section~\ref{sec:handpos_trajectory_straight_line_trajectory}, we can conclude that if Assumption~\ref{asm:straight_line_trajectory} holds, then the steady-state orientation must satisfy
\begin{align}
    Re_1 &= \pm \frac{\xi_{2_{d, r}}}{\norm{\xi_{2_{d, r}}}}, &
    \sin\phi &= 0,
\end{align}
where $\phi$ is the steady-state roll angle.

Using the orientation error $\delta$, as defined in \eqref{eq:handpos_trajectory_delta}, the complete closed-loop system is given by
\begin{subequations}
    \begin{align}
        \dot{\tilde{\xi}}_I &= \tilde{\xi}_1, \\
        \dot{\tilde{\xi}}_1 &= \tilde{\xi}_2, \\
        \dot{\tilde{\xi}}_2 &= -k_I \hat{I} \tilde{\xi}_I - k_p \hat{I} \tilde{\xi}_1 - k_d \tilde{\xi}_2, \\
        \dot{\delta} &= \omega, \\
        \dot{\omega} &= \Bar{L} \times \bigg(-k_I R\T R_p \hat{I} \tilde{\xi}_I - k_p R\T R_p \hat{I} \tilde{\xi}_1 \nonumber \\
        &\qquad \qquad - k_d R\T R_p \tilde{\xi}_2 + \mathcal{D}_{\nu}(\zeta) + \mathcal{C}_{\nu}(\zeta) \nonumber \\
        &\qquad \qquad - \omega \times \left(R\T\! R_p \tilde{\xi}_2 + {\rm expm}(\delta)\T \norm{\xi_{2_{d,r}}}e_1\right)\!\bigg) \nonumber \\
        &\quad - \left(\Bar{L}L\T\right) \left(\mathcal{D}_{\omega}(\zeta) + M_{22}\left(Wz_{gb} e_3 \times R\T e_3\right)\right),
    \end{align}
\end{subequations}
Next, we define a vector $\Bar{z}\T = \left[\Xi\T, \delta\T, \omega\T\right]$ and a function $\Bar{f}$ such that $\dot{\Bar{z}} = \Bar{f}(\Bar{z})$.
Let $\Bar{J}$ denote the Jacobian of $\Bar{f}(\Bar{z})$, evaluated at $\Bar{z} = 0_{13}$.
$\Bar{J}$ is given by

\begin{equation}
    \Bar{J} \!=\! \begin{bmatrix}
        O_{2 \times 2} & I_{2 \times 2} & O_{2 \times 3} & O_{2 \times 3}\!\!\! & O_{2 \times 3} \\
        O_{2 \times 2} & O_{2 \times 2} & \hat{I}\T & O_{2 \times 3}\!\!\! & O_{2 \times 3} \\
        -k_I \hat{I}\!\! & -k_p \hat{I}\!\!\! & -k_d I_{3 \times 3}\!\!\!\! & O_{3 \times 3}\!\!\! & O_{3 \times 3} \\
        O_{3 \times 2} & O_{3 \times 2} & O_{3 \times 3} & O_{3 \times 3}\!\!\! & I_{3 \times 3} \\
        R\T\! R_p \hat{I}J_{\xi_I}\!\!\!\! & R\T\! R_p \hat{I}J_{\xi_1}\!\!\!\! & R\T\! R_p J_{\xi_2}\!\!\!\! & J_{\delta} & J_{\omega}
    \end{bmatrix}\!. \label{eq:handpos_trajectory_path_jacobian}
\end{equation}
The blocks of $\Bar{J}$ are shown in \eqref{eq:handpos_trajectory_jacobian_blocks}.
Using the same reasoning as in the proof of Proposition~\ref{prop:straight_line_trajectory}, we can conclude that the closed-loop system is exponentially stable if Assumption~\ref{asm:straight_line_trajectory} holds and all $\Delta_i$ and $\Omega_i$ for $i \in \left\{1,2,3\right\}$ are positive. \qed

\section{Simulations}
\label{sec:handpos_trajectory_simulations}

\pgfplotsset{table/search path={figures/handpos_trajectory/data}}
\begin{figure}[htb]
    \centering
    \begin{minipage}{0.48\textwidth}
        \begin{subfigure}{\textwidth}
            \def\svgwidth{\textwidth}
            \import{figures/handpos_trajectory}{trajectory_trajectory.pdf_tex}
            \caption{Trajectory. The green line represents the hand position $\mat{x}_1$, the red line represents the desired trajectory $\bs{\xi}_{1, d}$, and the polygons represent the position and orientation of the AUV.}
            \label{fig:trajectory_trajectory}
        \end{subfigure}

        \vspace*{-3.5mm}

        \begin{subfigure}{\textwidth}
            \hspace*{\fill}
            % This file was created by matlab2tikz.
%
%The latest updates can be retrieved from
%  http://www.mathworks.com/matlabcentral/fileexchange/22022-matlab2tikz-matlab2tikz
%where you can also make suggestions and rate matlab2tikz.
%
\definecolor{mycolor1}{rgb}{0.10588,0.61961,0.46667}%
\definecolor{mycolor2}{rgb}{0.91373,0.37255,0.00784}%
\definecolor{mycolor3}{rgb}{0.45882,0.43529,0.70196}%
%
\begin{tikzpicture}

\begin{axis}[%
width=0.77\textwidth,
height=20mm,
at={(0,0)},
scale only axis,
xmin=0,
xmax=400,
xlabel style={font=\color{white!15!black}, yshift=1mm},
xlabel={$t$ [s]},
ymin=-0.05,
ymax=0.1,
ytick={0,0.1},
ylabel style={font=\color{white!15!black},yshift=-3.5mm},
ylabel={$\omega$ [rad/s]},
xmajorgrids,
ymajorgrids,
axis background/.style={fill=white},
title style={font=\bfseries,yshift=-2.5mm},
title={Angular velocities},
legend style={legend cell align=left, align=left, draw=white!15!black},
legend columns=3
]
\addplot [color=mycolor1, line width=1.0pt]
  table[]{trajectory_omega-1.tsv};
\addlegendentry{$p$}

\addplot [color=mycolor2, line width=1.0pt]
  table[]{trajectory_omega-2.tsv};
\addlegendentry{$q$}

\addplot [color=mycolor3, line width=1.0pt]
  table[]{trajectory_omega-3.tsv};
\addlegendentry{$r$}

\end{axis}
\end{tikzpicture}%
            \caption{The angular velocities.}
            \label{fig:trajectory_omega}
        \end{subfigure}
        \vspace*{-2mm}
    \end{minipage}
    \hspace*{\fill}
    \begin{minipage}{0.48\textwidth}
        \begin{subfigure}{\textwidth}
            % This file was created by matlab2tikz.
%
%The latest updates can be retrieved from
%  http://www.mathworks.com/matlabcentral/fileexchange/22022-matlab2tikz-matlab2tikz
%where you can also make suggestions and rate matlab2tikz.
%
\definecolor{mycolor1}{rgb}{0.10588,0.61961,0.46667}%
\definecolor{mycolor2}{rgb}{0.91373,0.37255,0.00784}%
\definecolor{mycolor3}{rgb}{0.45882,0.43529,0.70196}%
%
\begin{tikzpicture}

\begin{axis}[%
width=0.775\textwidth,
height=13mm,
at={(0,0)},
scale only axis,
xmin=0,
xmax=400,
xlabel style={font=\color{white!15!black},yshift=1mm},
xlabel={$t$ [s]},
ymin=-50,
ymax=20,
ylabel style={font=\color{white!15!black},yshift=-1mm},
ylabel={$\tilde{\xi}_1$ [m]},
xmajorgrids,
ymajorgrids,
axis background/.style={fill=white},
title style={font=\bfseries,yshift=-2.5mm},
title={Position tracking error},
legend style={legend cell align=left, align=left, draw=white!15!black, at={(0.98,0.09)}, anchor=south east},
legend columns=3
]
\addplot [color=mycolor1, line width=1.0pt]
  table[]{trajectory_position-1.tsv};
\addlegendentry{$x$}

\addplot [color=mycolor2, line width=1.0pt]
  table[]{trajectory_position-2.tsv};
\addlegendentry{$y$}

\addplot [color=mycolor3, line width=1.0pt]
  table[]{trajectory_position-3.tsv};
\addlegendentry{$z$}

\end{axis}

\end{tikzpicture}%
            %\vspace*{-7.5mm}
            \caption{The $x$- $y$- and $z$-components of $\tilde{\bs{\xi}}_1$.}
            \label{fig:trajectory_position}
        \end{subfigure}

        \vspace*{-3.5mm}

        \begin{subfigure}{\textwidth}
            % This file was created by matlab2tikz.
%
%The latest updates can be retrieved from
%  http://www.mathworks.com/matlabcentral/fileexchange/22022-matlab2tikz-matlab2tikz
%where you can also make suggestions and rate matlab2tikz.
%
\definecolor{mycolor1}{rgb}{0.10588,0.61961,0.46667}%
\definecolor{mycolor2}{rgb}{0.91373,0.37255,0.00784}%
\definecolor{mycolor3}{rgb}{0.45882,0.43529,0.70196}%
%
\begin{tikzpicture}

\begin{axis}[%
width=0.82\textwidth,
height=13mm,
at={(0,0)},
scale only axis,
xmin=0,
xmax=400,
xlabel style={font=\color{white!15!black},yshift=1mm},
xlabel={$t$ [s]},
ymin=-1,
ymax=3,
ylabel style={font=\color{white!15!black},yshift=-2.5mm},
ylabel={$\tilde{\xi}_2$ [m/s]},
xmajorgrids,
ymajorgrids,
axis background/.style={fill=white},
title style={font=\bfseries,yshift=-2.5mm},
title={Velocity tracking error},
legend style={legend cell align=left, align=left, draw=white!15!black, at={(0.98,0.92)}, anchor=north east},
legend columns=3
]
\addplot [color=mycolor1, line width=1.0pt]
  table[]{trajectory_velocity-1.tsv};
\addlegendentry{$x$}

\addplot [color=mycolor2, line width=1.0pt]
  table[]{trajectory_velocity-2.tsv};
\addlegendentry{$y$}

\addplot [color=mycolor3, line width=1.0pt]
  table[]{trajectory_velocity-3.tsv};
\addlegendentry{$z$}

\end{axis}

\end{tikzpicture}%
            %\vspace*{-7.5mm}
            \caption{The $x$- $y$- and $z$-components of $\tilde{\bs{\xi}}_2$.}
            \label{fig:trajectory_velocity}
        \end{subfigure}

        \vspace*{-3.5mm}

        \begin{subfigure}{\textwidth}
            % This file was created by matlab2tikz.
%
%The latest updates can be retrieved from
%  http://www.mathworks.com/matlabcentral/fileexchange/22022-matlab2tikz-matlab2tikz
%where you can also make suggestions and rate matlab2tikz.
%
\definecolor{mycolor1}{rgb}{0.10588,0.61961,0.46667}%
\definecolor{mycolor2}{rgb}{0.91373,0.37255,0.00784}%
\definecolor{mycolor3}{rgb}{0.45882,0.43529,0.70196}%
%
\begin{tikzpicture}

\begin{axis}[%
width=0.74\textwidth,
height=13mm,
at={(0,0)},
scale only axis,
xmin=0,
xmax=400,
xlabel style={font=\color{white!15!black},yshift=1mm},
xlabel={$t$ [s]},
ymin=-500,
ymax=100,
ylabel style={font=\color{white!15!black}},
ylabel={$\tilde{\xi}_I$ [m$\cdot$s]},
xmajorgrids,
ymajorgrids,
axis background/.style={fill=white},
title style={font=\bfseries,yshift=-2.5mm},
title={Integral error},
legend style={legend cell align=left, align=left, draw=white!15!black, at={(0.98,0.09)}, anchor=south east},
legend columns=3
]
\addplot [color=mycolor1, line width=1.0pt]
  table[]{trajectory_integral-1.tsv};
\addlegendentry{$x$}

\addplot [color=mycolor2, line width=1.0pt]
  table[]{trajectory_integral-2.tsv};
\addlegendentry{$y$}

\addplot [color=mycolor3, line width=1.0pt]
  table[]{trajectory_integral-3.tsv};
\addlegendentry{$z$}

\end{axis}

\end{tikzpicture}%
            %\vspace*{-7.5mm}
            \caption{The $x$- $y$- and $z$-components of $\tilde{\bs{\xi}}_I$.}
            \label{fig:trajectory_integral}
        \end{subfigure}
        \vspace*{-2mm}
    \end{minipage}
    \caption{Simulation results of the trajectory-tracking algorithm proposed in Section~\ref{sec:trajectory_tracking}.
    %(a) Trajectory. The green line represents the hand position $x_1$, the red line represents the desired trajectory $\xi_{1_d}$, and the polygons represent the position and orientation of the AUV.
    %(b) Angular velocities of the vehicle.
    %(c) The $x$- $y$- and $z$-components of $\tilde{\xi}_1$.
    %(d) The $x$- $y$- and $z$-components of $\tilde{\xi}_2$.
    %(e) The $x$- $y$- and $z$-components of $\tilde{\xi}_I$.
    }
    \label{fig:trajectory_tracking}
\end{figure}


\begin{figure}[htb]
    \centering
    \begin{minipage}{0.48\textwidth}
        \begin{subfigure}{\textwidth}
            \def\svgwidth{\textwidth}
            \import{figures/handpos_trajectory}{path_trajectory.pdf_tex}
            %\vspace*{-5.5mm}
            \caption{Trajectory. The green line represents the hand position $\mat{x}_1$, the red line represents the desired trajectory $\bs{\xi}_{1, d}$, and the polygons represent the position and orientation of the AUV.}
            \label{fig:path_trajectory}
        \end{subfigure}

        %\vspace*{-1.5mm}

        \begin{subfigure}{\textwidth}
            \hspace*{\fill}
            % This file was created by matlab2tikz.
%
%The latest updates can be retrieved from
%  http://www.mathworks.com/matlabcentral/fileexchange/22022-matlab2tikz-matlab2tikz
%where you can also make suggestions and rate matlab2tikz.
%
\definecolor{mycolor1}{rgb}{0.10588,0.61961,0.46667}%
\definecolor{mycolor2}{rgb}{0.91373,0.37255,0.00784}%
\definecolor{mycolor3}{rgb}{0.45882,0.43529,0.70196}%
%
\begin{tikzpicture}

\begin{axis}[%
width=0.77\textwidth,
height=15mm,
at={(0,0)},
scale only axis,
xmin=0,
xmax=350,
xlabel style={font=\color{white!15!black},yshift=1mm},
xlabel={$t$ [s]},
% ymin=-0.15,
% ymax=0.15,
ylabel style={font=\color{white!15!black},yshift=-3.5mm,xshift=3mm},
ylabel={$\omega$ [rad/s]},
xmajorgrids,
ymajorgrids,
axis background/.style={fill=white},
title style={font=\bfseries,yshift=-2.5mm},
title={Angular velocities},
legend style={legend cell align=left, align=left, draw=white!15!black, at={(0.98,0.0)}, anchor=south east},
legend columns=3
]
\addplot [color=mycolor1, line width=1.0pt]
  table[]{path_omega-1.tsv};
\addlegendentry{$p$}

\addplot [color=mycolor2, line width=1.0pt]
  table[]{path_omega-2.tsv};
\addlegendentry{$q$}

\addplot [color=mycolor3, line width=1.0pt]
  table[]{path_omega-3.tsv};
\addlegendentry{$r$}

\end{axis}
\end{tikzpicture}%
            \vspace*{-8mm}
            \caption{The angular velocities.}
            \label{fig:path_omega}
        \end{subfigure}

        %\vspace*{-1.5mm}

        \begin{subfigure}{\textwidth}
            \hspace*{-1.5mm}
            % This file was created by matlab2tikz.
%
%The latest updates can be retrieved from
%  http://www.mathworks.com/matlabcentral/fileexchange/22022-matlab2tikz-matlab2tikz
%where you can also make suggestions and rate matlab2tikz.
%
\definecolor{mycolor1}{rgb}{0.10588,0.61961,0.46667}%
\definecolor{mycolor2}{rgb}{0.91373,0.37255,0.00784}%
%
\begin{tikzpicture}

\begin{axis}[%
width=0.8\textwidth,
height=15mm,
at={(0mm,0mm)},
scale only axis,
xmin=0,
xmax=350,
xlabel style={font=\color{white!15!black},yshift=1mm},
xlabel={$t$ [s]},
ymin=0,
ymax=1.2,
ylabel style={font=\color{white!15!black},yshift=-1.5mm},
ylabel={$U$ [m/s]},
xmajorgrids,
ymajorgrids,
axis background/.style={fill=white},
title style={font=\bfseries,yshift=-2.5mm},
title={Path parameter derivative},
legend style={legend cell align=left, align=left, draw=white!15!black, at={(0.98,0.09)}, anchor=south east},
legend columns=2
]
\addplot [color=mycolor1, line width=1.0pt]
  table[]{path_parameter-1.tsv};
\addlegendentry{$\dot{s}$}

\addplot [color=mycolor2, dashed, line width=1.5pt]
  table[]{path_parameter-2.tsv};
\addlegendentry{$U_d$}

\end{axis}

\end{tikzpicture}%
            \vspace*{-8mm}
            \caption{The time-derivative of the path parameter.}
            \label{fig:path_parameter}
        \end{subfigure}
        \vspace*{-7mm}
    \end{minipage}
    \hspace*{\fill}
    \begin{minipage}{0.48\textwidth}
        \begin{subfigure}{\textwidth}
            \hspace*{-2.5mm}
            % This file was created by matlab2tikz.
%
%The latest updates can be retrieved from
%  http://www.mathworks.com/matlabcentral/fileexchange/22022-matlab2tikz-matlab2tikz
%where you can also make suggestions and rate matlab2tikz.
%
\definecolor{mycolor1}{rgb}{0.10588,0.61961,0.46667}%
\definecolor{mycolor2}{rgb}{0.91373,0.37255,0.00784}%
\definecolor{mycolor3}{rgb}{0.45882,0.43529,0.70196}%
%
\begin{tikzpicture}

\begin{axis}[%
width=0.80\textwidth,
height=18mm,
at={(0,0)},
scale only axis,
xmin=0,
xmax=350,
xlabel style={font=\color{white!15!black},yshift=1mm},
xlabel={$t$ [s]},
ymin=-50,
ymax=20,
%ytick={-60,-30,0,30},
ylabel style={font=\color{white!15!black},yshift=-1mm},
ylabel={$\tilde{\xi}_1$ [m]},
xmajorgrids,
ymajorgrids,
axis background/.style={fill=white},
title style={font=\bfseries,yshift=-2.5mm},
title={Position tracking error},
legend style={legend cell align=left, align=left, draw=white!15!black, at={(0.98,0.09)}, anchor=south east},
legend columns=3
]
\addplot [color=mycolor1, line width=1.0pt]
  table[]{path_position-1.tsv};
\addlegendentry{$x$}

\addplot [color=mycolor2, line width=1.0pt]
  table[]{path_position-2.tsv};
\addlegendentry{$y$}

\addplot [color=mycolor3, line width=1.0pt]
  table[]{path_position-3.tsv};
\addlegendentry{$z$}

\end{axis}

\end{tikzpicture}%
            \vspace*{-8mm}
            \caption{The $x$- $y$- and $z$-components of $\tilde{\bs{\xi}}_1$.}
            \label{fig:path_position}
        \end{subfigure}

        %\vspace*{-1.5mm}

        \begin{subfigure}{\textwidth}
            \hspace*{1mm}
            % This file was created by matlab2tikz.
%
%The latest updates can be retrieved from
%  http://www.mathworks.com/matlabcentral/fileexchange/22022-matlab2tikz-matlab2tikz
%where you can also make suggestions and rate matlab2tikz.
%
\definecolor{mycolor1}{rgb}{0.10588,0.61961,0.46667}%
\definecolor{mycolor2}{rgb}{0.91373,0.37255,0.00784}%
\definecolor{mycolor3}{rgb}{0.45882,0.43529,0.70196}%
%
\begin{tikzpicture}

\begin{axis}[%
width=0.80\textwidth,
height=18mm,
at={(0,0)},
scale only axis,
xmin=0,
xmax=350,
xlabel style={font=\color{white!15!black},yshift=1mm},
xlabel={$t$ [s]},
xtick={0,100,200,300},
ymin=-1,
ymax=3.5,
ylabel style={font=\color{white!15!black},yshift=-1.5mm},
ylabel={$\tilde{\xi}_2$ [m/s]},
xmajorgrids,
ymajorgrids,
axis background/.style={fill=white},
title style={font=\bfseries,yshift=-2.5mm},
title={Velocity tracking error},
legend style={legend cell align=left, align=left, draw=white!15!black},
legend columns=3
]
\addplot [color=mycolor1, line width=1.0pt]
  table[]{path_velocity-1.tsv};
\addlegendentry{$x$}

\addplot [color=mycolor2, line width=1.0pt]
  table[]{path_velocity-2.tsv};
\addlegendentry{$y$}

\addplot [color=mycolor3, line width=1.0pt]
  table[]{path_velocity-3.tsv};
\addlegendentry{$z$}

\end{axis}

\end{tikzpicture}%
            \vspace*{-8mm}
            \caption{The $x$- $y$- and $z$-components of $\tilde{\bs{\xi}}_2$.}
            \label{fig:path_velocity}
        \end{subfigure}

        %\vspace*{-1.5mm}

        \begin{subfigure}{\textwidth}
            \hspace*{-1.7mm}
            % This file was created by matlab2tikz.
%
%The latest updates can be retrieved from
%  http://www.mathworks.com/matlabcentral/fileexchange/22022-matlab2tikz-matlab2tikz
%where you can also make suggestions and rate matlab2tikz.
%
\definecolor{mycolor1}{rgb}{0.10588,0.61961,0.46667}%
\definecolor{mycolor2}{rgb}{0.91373,0.37255,0.00784}%
\definecolor{mycolor3}{rgb}{0.45882,0.43529,0.70196}%
%
\begin{tikzpicture}

\begin{axis}[%
width=0.76\textwidth,
height=18mm,
at={(0,0)},
scale only axis,
xmin=0,
xmax=350,
xlabel style={font=\color{white!15!black},yshift=1mm},
xlabel={$t$ [s]},
ymin=-500,
ymax=100,
ylabel style={font=\color{white!15!black}},
ylabel={$\tilde{\xi}_I$ [m$\cdot$s]},
xmajorgrids,
ymajorgrids,
axis background/.style={fill=white},
title style={font=\bfseries,yshift=-2.5mm},
title={Integral error},
legend style={legend cell align=left, align=left, draw=white!15!black, at={(0.98,0.09)}, anchor=south east},
legend columns=3
]
\addplot [color=mycolor1, line width=1.0pt]
  table[]{path_integral-1.tsv};
\addlegendentry{$x$}

\addplot [color=mycolor2, line width=1.0pt]
  table[]{path_integral-2.tsv};
\addlegendentry{$y$}

\addplot [color=mycolor3, line width=1.0pt]
  table[]{path_integral-3.tsv};
\addlegendentry{$z$}

\end{axis}

\end{tikzpicture}%
            \vspace*{-8mm}
            \caption{The $x$- $y$- and $z$-components of $\tilde{\bs{\xi}}_I$.}
            \label{fig:path_integral}
        \end{subfigure}
        \vspace*{-7mm}
    \end{minipage}
    \caption{Simulation results of the path-following algorithm proposed in Section~\ref{sec:handpos_trajectory_path_following}.
    %(a) Trajectory. The green line represents the hand position $x_1$, the red line represents the desired trajectory $\bs{\xi}_{1_d}$, and the polygons represent the position and orientation of the AUV.
    %(b) Angular velocities of the vehicle.
    %(c) Time-derivative of the path parameter, compared to the desired path-following speed $U_d$.
    %(d) The $x$- $y$- and $z$-components of $\tilde{\bs{\xi}}_1$.
    %(e) The $x$- $y$- and $z$-components of $\tilde{\bs{\xi}}_2$.
    %(f) The $x$- $y$- and $z$-components of $\tilde{\bs{\xi}}_I$.
    }
    \label{fig:path_following}
\end{figure}


In this section, we present the results of numerical simulations.
The simulations were carried out in MATLAB using a model of the \acrfull{lauv}.

We tested the trajectory tracking algorithm proposed in Section~\ref{sec:handpos_trajectory_trajectory_tracking}, the curved path following algorithm proposed in Section~\ref{sec:handpos_trajectory_path_following}, and the straight-line path following algorithm in Section~\ref{sec:handpos_trajectory_straight_line_path}.
The following parameters are common for the first two tests:
The initial state of the vehicle is $\eta(0) = 0_3$, $R(0) = I_{3 \times 3}$, $\nu_r(0) = e_1$, $\omega(0) = 0_3$.
The hand length is $h = \SI{5}{\meter}$, the velocity of the ocean current is $V = \inlinevector{0.15, -0.1, 0.05}$, and the PID gains are $k_p = 0.03, k_d = 0.4, k_I = 8.5 \cdot 10^{-4}$.

\subsection{Trajectory Tracking}
\begin{figure}[t]
    \centering
    \begin{minipage}{0.48\textwidth}
        \begin{subfigure}{\textwidth}
            \def\svgwidth{\textwidth}
            \import{figures/handpos_trajectory}{waypoints_trajectory.pdf_tex}
            %\vspace*{-5.5mm}
            \caption{Trajectory. The green line represents the hand position $\mat{x}_1$, the red line represents the desired trajectory $\bs{\xi}_{1, d}$, and the polygons represent the position and orientation of the AUV.}
            \label{fig:waypoint_trajectory}
        \end{subfigure}

        %\vspace*{-4.5mm}

        \begin{subfigure}{\textwidth}
            \hspace*{-2mm}
            % This file was created by matlab2tikz.
%
%The latest updates can be retrieved from
%  http://www.mathworks.com/matlabcentral/fileexchange/22022-matlab2tikz-matlab2tikz
%where you can also make suggestions and rate matlab2tikz.
%
\definecolor{mycolor1}{rgb}{0.10588,0.61961,0.46667}%
\definecolor{mycolor2}{rgb}{0.91373,0.37255,0.00784}%
\definecolor{mycolor3}{rgb}{0.45882,0.43529,0.70196}%
%
\begin{tikzpicture}

\begin{axis}[%
width=0.79\textwidth,
height=20mm,
at={(0,0)},
scale only axis,
xmin=0,
xmax=130,
xlabel style={font=\color{white!15!black}, yshift=1mm},
xlabel={$t$ [s]},
ymin=-0.4,
ymax=0.3,
ymajorgrids,
ylabel style={font=\color{white!15!black}, yshift=-1mm},
ylabel={$\omega$ [rad/s]},
axis background/.style={fill=white},
title style={font=\bfseries, yshift=-2.5mm},
title={Angular velocities},
legend style={legend cell align=left, align=left, draw=white!15!black},
legend columns=3
]
\addplot [color=mycolor1, line width=1.0pt]
  table[]{waypoints_omega-1.tsv};
\addlegendentry{$\omega_1$}

\addplot [color=mycolor2, line width=1.0pt]
  table[]{waypoints_omega-2.tsv};
\addlegendentry{$\omega_2$}

\addplot [color=mycolor3, line width=1.0pt]
  table[]{waypoints_omega-3.tsv};
\addlegendentry{$\omega_3$}

\addplot [color=black, dotted, line width=1.0pt, forget plot]
  table[]{waypoints_omega-4.tsv};
\addplot [color=black, dotted, line width=1.0pt, forget plot]
  table[]{waypoints_omega-5.tsv};
\addplot [color=black, dotted, line width=1.0pt, forget plot]
  table[]{waypoints_omega-6.tsv};
\addplot [color=black, dotted, line width=1.0pt, forget plot]
  table[]{waypoints_omega-7.tsv};
\addplot [color=black, dotted, line width=1.0pt, forget plot]
  table[]{waypoints_omega-8.tsv};
\addplot [color=black, dotted, line width=1.0pt, forget plot]
  table[]{waypoints_omega-9.tsv};
\end{axis}
\end{tikzpicture}%
            \vspace*{-5mm}
            \caption{The angular velocities.}
            \label{fig:waypoint_omega}
        \end{subfigure}

        \vspace*{-4mm}
    \end{minipage}
    \hspace*{\fill}
    \begin{minipage}{0.48\textwidth}
        \begin{subfigure}{\textwidth}
            % This file was created by matlab2tikz.
%
%The latest updates can be retrieved from
%  http://www.mathworks.com/matlabcentral/fileexchange/22022-matlab2tikz-matlab2tikz
%where you can also make suggestions and rate matlab2tikz.
%
\definecolor{mycolor1}{rgb}{0.10588,0.61961,0.46667}%
\definecolor{mycolor2}{rgb}{0.91373,0.37255,0.00784}%
\definecolor{mycolor3}{rgb}{0.45882,0.43529,0.70196}%
%
\begin{tikzpicture}

\begin{axis}[%
width=0.775\textwidth,
height=17mm,
at={(0,0)},
scale only axis,
xmin=0,
xmax=130,
xlabel style={font=\color{white!15!black}, yshift=1mm},
xlabel={$t$ [s]},
ymin=-4,
ymax=5,
ymajorgrids,
ylabel style={font=\color{white!15!black}, yshift=-1mm},
ylabel={$\tilde{\xi}_1$ [m]},
axis background/.style={fill=white},
title style={font=\bfseries, yshift=-2.5mm},
title={Position tracking error},
legend style={legend cell align=left, align=left, draw=white!15!black, at={(0.98,0.98)}, anchor=north east},
legend columns=2
]
\addplot [color=mycolor2, line width=1.0pt]
  table[]{waypoints_position-1.tsv};
\addlegendentry{$\widetilde{y}$}

\addplot [color=mycolor3, line width=1.0pt]
  table[]{waypoints_position-2.tsv};
\addlegendentry{$\widetilde{z}$}

\addplot [color=black, dotted, line width=1.0pt, forget plot]
  table[]{waypoints_position-3.tsv};
\addplot [color=black, dotted, line width=1.0pt, forget plot]
  table[]{waypoints_position-4.tsv};
\addplot [color=black, dotted, line width=1.0pt, forget plot]
  table[]{waypoints_position-5.tsv};
\addplot [color=black, dotted, line width=1.0pt, forget plot]
  table[]{waypoints_position-6.tsv};
\addplot [color=black, dotted, line width=1.0pt, forget plot]
  table[]{waypoints_position-7.tsv};
\addplot [color=black, dotted, line width=1.0pt, forget plot]
  table[]{waypoints_position-8.tsv};
\end{axis}
\end{tikzpicture}%
            \vspace*{-8mm}
            \caption{The $x$- $y$- and $z$-components of $\tilde{\bs{\xi}}_1$.}
            \label{fig:waypoint_position}
        \end{subfigure}

        %\vspace*{-3.5mm}

        \begin{subfigure}{\textwidth}
            % This file was created by matlab2tikz.
%
%The latest updates can be retrieved from
%  http://www.mathworks.com/matlabcentral/fileexchange/22022-matlab2tikz-matlab2tikz
%where you can also make suggestions and rate matlab2tikz.
%
\definecolor{mycolor1}{rgb}{0.10588,0.61961,0.46667}%
\definecolor{mycolor2}{rgb}{0.91373,0.37255,0.00784}%
\definecolor{mycolor3}{rgb}{0.45882,0.43529,0.70196}%
%
\begin{tikzpicture}

\begin{axis}[%
width=0.775\textwidth,
height=17mm,
at={(0,0)},
scale only axis,
xmin=0,
xmax=130,
xlabel style={font=\color{white!15!black}, yshift=1mm},
xlabel={$t$ [s]},
ymin=-2.5,
ymax=2,
ymajorgrids,
ylabel style={font=\color{white!15!black}, yshift=-1mm},
ylabel={$\tilde{\xi}_2$ [m/s]},
axis background/.style={fill=white},
title style={font=\bfseries, yshift=-2.5mm},
title={Velocity tracking error},
legend style={legend cell align=left, align=left, draw=white!15!black, at={(0.98,0.05)}, anchor=south east},
legend columns=3
]
\addplot [color=mycolor1, line width=1.0pt]
  table[]{waypoints_velocity-1.tsv};
\addlegendentry{$x$}

\addplot [color=mycolor2, line width=1.0pt]
  table[]{waypoints_velocity-2.tsv};
\addlegendentry{$y$}

\addplot [color=mycolor3, line width=1.0pt]
  table[]{waypoints_velocity-3.tsv};
\addlegendentry{$z$}

\addplot [color=black, dotted, line width=1.0pt, forget plot]
  table[]{waypoints_velocity-4.tsv};
\addplot [color=black, dotted, line width=1.0pt, forget plot]
  table[]{waypoints_velocity-5.tsv};
\addplot [color=black, dotted, line width=1.0pt, forget plot]
  table[]{waypoints_velocity-6.tsv};
\addplot [color=black, dotted, line width=1.0pt, forget plot]
  table[]{waypoints_velocity-7.tsv};
\addplot [color=black, dotted, line width=1.0pt, forget plot]
  table[]{waypoints_velocity-8.tsv};
\addplot [color=black, dotted, line width=1.0pt, forget plot]
  table[]{waypoints_velocity-9.tsv};
\end{axis}
\end{tikzpicture}%
            \vspace*{-8mm}
            \caption{The $x$- $y$- and $z$-components of $\tilde{\bs{\xi}}_2$.}
            \label{fig:waypoint_velocity}
        \end{subfigure}

        %\vspace*{-3.5mm}

        \begin{subfigure}{\textwidth}
            % This file was created by matlab2tikz.
%
%The latest updates can be retrieved from
%  http://www.mathworks.com/matlabcentral/fileexchange/22022-matlab2tikz-matlab2tikz
%where you can also make suggestions and rate matlab2tikz.
%
\definecolor{mycolor1}{rgb}{0.10588,0.61961,0.46667}%
\definecolor{mycolor2}{rgb}{0.91373,0.37255,0.00784}%
\definecolor{mycolor3}{rgb}{0.45882,0.43529,0.70196}%
%
\begin{tikzpicture}

\begin{axis}[%
width=0.775\textwidth,
height=17mm,
at={(0,0)},
scale only axis,
xmin=0,
xmax=130,
xlabel style={font=\color{white!15!black}, yshift=1mm},
xlabel={$t$ [s]},
ymin=-6,
ymax=6,
ymajorgrids,
ylabel style={font=\color{white!15!black}, yshift=-1mm},
ylabel={$\tilde{\xi}_I$ [m$\cdot$s]},
axis background/.style={fill=white},
title style={font=\bfseries, yshift=-2.5mm},
title={Integral error},
legend style={legend cell align=left, align=left, draw=white!15!black, at={(0.98,0.98)}, anchor=north east},
legend columns=2
]
\addplot [color=mycolor2, line width=1.0pt]
  table[]{waypoints_integral-1.tsv};
\addlegendentry{$y$}

\addplot [color=mycolor3, line width=1.0pt]
  table[]{waypoints_integral-2.tsv};
\addlegendentry{$z$}

\addplot [color=black, dotted, line width=1.0pt, forget plot]
  table[]{waypoints_integral-3.tsv};
\addplot [color=black, dotted, line width=1.0pt, forget plot]
  table[]{waypoints_integral-4.tsv};
\addplot [color=black, dotted, line width=1.0pt, forget plot]
  table[]{waypoints_integral-5.tsv};
\addplot [color=black, dotted, line width=1.0pt, forget plot]
  table[]{waypoints_integral-6.tsv};
\addplot [color=black, dotted, line width=1.0pt, forget plot]
  table[]{waypoints_integral-7.tsv};
\addplot [color=black, dotted, line width=1.0pt, forget plot]
  table[]{waypoints_integral-8.tsv};
\end{axis}
\end{tikzpicture}%
            \vspace*{-8mm}
            \caption{The $x$- $y$- and $z$-components of $\tilde{\bs{\xi}}_I$.}
            \label{fig:waypoint_integral}
        \end{subfigure}
        \vspace*{-4mm}
    \end{minipage}
    \caption{Simulation results of the path-following algorithm proposed in Section~\ref{sec:handpos_trajectory_path_following}.
    %(a) Trajectory. The green line represents the hand position $x_1$, the red line represents the desired trajectory $\xi_{1_d}$, and the polygons represent the position and orientation of the AUV.
    %(b) Angular velocities of the vehicle.
    %(c) The cross-track errors.
    %(d) The $x$- $y$- and $z$-components of $\tilde{\xi}_2$.
    %(e) The integral errors.
    }
    \label{fig:waypoint_following}
\end{figure}

In this test, the vehicle should track a figure eight trajectory
\begin{equation}
    \xi_{1_d}\!(t) \!=\!\! \inlinevector{50\!\cos\!\left(\frac{\pi}{200}t\right)\!\!, 25\!\sin\!\left(\frac{2\pi}{200}t\right)\!\!, 15\cos\!\left(\frac{2\pi}{200}t\right)}\!\!\!\!\!.
\end{equation}
We use the trajectory tracking controller proposed in Section~\ref{sec:handpos_trajectory_trajectory_tracking}.
The PID gains are chosen such that $H_{\xi}$ is Hurwitz, guaranteeing the stability of the external dynamics.
The value of $\Bar{\alpha}_y = \Bar{\alpha}_z$ for the chosen parameters is $0.05$, guaranteeing the boundedness of the internal dynamics by Proposition~\ref{prop:handpos_trajectory_trajectory_tracking}.

The results are shown in Figure~\ref{fig:trajectory_tracking}.
Figure~\ref{fig:trajectory_trajectory} shows the 3D trajectory of the vehicle, and Figures~\ref{fig:trajectory_position}, \ref{fig:trajectory_velocity}, and \ref{fig:trajectory_integral} show the external dynamics.
The vehicle converges to the desired trajectory after approximately $120$ seconds.
Figure~\ref{fig:trajectory_omega} shows the angular velocities of the vehicle.
Initially, the angular velocities grow.
However, after the external dynamics have converged, the angular velocities remain bounded.

\subsection{Path Following}
\begin{figure}[t]
    \begin{subfigure}{0.48\textwidth}
        % This file was created by matlab2tikz.
%
%The latest updates can be retrieved from
%  http://www.mathworks.com/matlabcentral/fileexchange/22022-matlab2tikz-matlab2tikz
%where you can also make suggestions and rate matlab2tikz.
%
\definecolor{mycolor1}{rgb}{0.10588,0.61961,0.46667}%
\definecolor{mycolor2}{rgb}{0.91373,0.37255,0.00784}%
\definecolor{mycolor3}{rgb}{0.45882,0.43529,0.70196}%
%
\begin{tikzpicture}

\begin{axis}[%
width=0.9\textwidth,
height=60mm,
at={(0\textwidth,0mm)},
scale only axis,
plot box ratio=1.007 1 1.042,
xmin=-20,
xmax=40.4020387912503,
tick align=outside,
xlabel style={font=\color{white!15!black},yshift=2mm},
xlabel={$y$ [m]},
ymin=-60,
ymax=0,
ylabel style={font=\color{white!15!black},yshift=2mm},
ylabel={$x$ [m]},
z dir=reverse,
zmin=-0.5,
zmax=2,
zlabel style={font=\color{white!15!black}},
zlabel={$z$ [m]},
view={69.3}{38.7},
axis background/.style={fill=white},
title style={font=\bfseries, yshift=-2.5mm},
title={Trajectory},
axis x line*=bottom,
axis y line*=left,
axis z line*=left,
xmajorgrids,
ymajorgrids,
zmajorgrids,
legend style={at={(0.01,0.98)}, anchor=north west, legend cell align=left, align=left, draw=white!15!black, font=\footnotesize}
]
\addplot3 [color=mycolor1, line width=1.0pt]
 table[] {experiment_trajectory-1.tsv};
 \addlegendentry{Hand position}

\addplot3 [color=mycolor2, dashed, mark=x, mark options={scale=1.5, solid}, line width=1.5pt]
 table[] {experiment_trajectory-2.tsv};
 \addlegendentry{Reference}

\addplot3[-Straight Barb, color=mycolor3, line width=1.2pt, quiver={u=\thisrow{u}, v=\thisrow{v}, w=\thisrow{w}, every arrow/.append style={-{Straight Barb[angle'=35, scale={1.2}]}}}]
 table[] {experiment_trajectory-3.tsv};
\addlegendentry{Vehicle orientation}

\end{axis}
\end{tikzpicture}%
        \caption{The trajectory of the vehicle estimated by its onboard navigation system. \emph{Note:} the $z$-axis is upscaled 20 times.}
        \label{fig:experiment_trajectory}
    \end{subfigure}        
    \begin{subfigure}{0.48\textwidth}
        % This file was created by matlab2tikz.
%
%The latest updates can be retrieved from
%  http://www.mathworks.com/matlabcentral/fileexchange/22022-matlab2tikz-matlab2tikz
%where you can also make suggestions and rate matlab2tikz.
%
\definecolor{mycolor1}{rgb}{0.10588,0.61961,0.46667}%
%
\begin{tikzpicture}

\begin{axis}[%
width=0.8\textwidth,
height=25.116mm,
at={(0\textwidth,34.884mm)},
scale only axis,
xmin=0,
xmax=105,
xlabel style={font=\color{white!15!black}, yshift=1.5mm},
xlabel={$t$ [s]},
ymin=-5.15143273101685,
ymax=5.88349529916549,
ylabel style={font=\color{white!15!black}, yshift=-1mm},
ylabel={$\widetilde{y}$ [m]},
axis background/.style={fill=white},
title style={font=\bfseries, yshift=-2mm},
title={Cross-track errors},
ymajorgrids,
legend style={legend cell align=left, align=left, draw=white!15!black}
]
\addplot [color=mycolor1, line width=1.0pt]
  table[]{experiment_errors-1.tsv};
\addlegendentry{Error}

\addplot [color=black, dotted, line width=1.0pt]
  table[]{experiment_errors-2.tsv};
\addlegendentry{Waypoint change}

\addplot [color=black, dotted, line width=1.0pt, forget plot]
  table[]{experiment_errors-3.tsv};
\addplot [color=black, dotted, line width=1.0pt, forget plot]
  table[]{experiment_errors-4.tsv};
\addplot [color=black, dotted, line width=1.0pt, forget plot]
  table[]{experiment_errors-5.tsv};
\addplot [color=black, dotted, line width=1.0pt, forget plot]
  table[]{experiment_errors-6.tsv};
\end{axis}

\begin{axis}[%
width=0.8\textwidth,
height=25.116mm,
at={(0\textwidth,0mm)},
scale only axis,
xmin=0,
xmax=105,
xlabel style={font=\color{white!15!black}, yshift=1.5mm},
xlabel={$t$ [s]},
ymin=-1.11,
ymax=0.5,
ylabel style={font=\color{white!15!black}, yshift=-2.5mm, xshift=3mm},
ylabel={$\widetilde{z}$ [m]},
axis background/.style={fill=white},
ymajorgrids
]
\addplot [color=mycolor1, line width=1.0pt, forget plot]
  table[]{experiment_errors-7.tsv};
\addplot [color=black, dotted, line width=1.0pt, forget plot]
  table[]{experiment_errors-8.tsv};
\addplot [color=black, dotted, line width=1.0pt, forget plot]
  table[]{experiment_errors-9.tsv};
\addplot [color=black, dotted, line width=1.0pt, forget plot]
  table[]{experiment_errors-10.tsv};
\addplot [color=black, dotted, line width=1.0pt, forget plot]
  table[]{experiment_errors-11.tsv};
\addplot [color=black, dotted, line width=1.0pt, forget plot]
  table[]{experiment_errors-12.tsv};
\end{axis}
\end{tikzpicture}%

        \caption{Cross-track errors calculated from the trajectory estimates.}
        \label{fig:experiment_errors}
    \end{subfigure}
    \vspace*{-4mm}
    \caption{Experimental results.
    %(a) The trajectory of the vehicle, as estimated by its onboard navigation system.
    %\emph{Note:} the $z$-axis is upscaled 20 times.
    %(b) Cross-track errors calculated from the trajectory estimates.
    }
    \label{fig:experiment}
\end{figure}


In this test, we have chosen a path with the same shape as in the trajectory tracking test.
The path parametrization is given by
\begin{equation}
    \mat{p}_p(s) = \inlinevector{50\cos\left(\gamma(s)\right), 25\sin\left(2\gamma(s)\right), 15\cos\left(2\gamma(s)\right)}\!,
\end{equation}
where $\gamma: \mathbb{R} \mapsto \mathbb{R}$ is a function chosen such that $\mat{p}_p(s)$ is a parametrization by arc length (see Section~\ref{sec:background_paths}).

We use the path following controller proposed in Section~\ref{sec:handpos_trajectory_path_following}.
The gains of the path parameter update law are chosen as $\varepsilon = 0.5$, and $k_{\sigma} = 0.1$.
The value of $\Bar{\alpha}_y = \Bar{\alpha}_z$ for the chosen parameters is $0.02$, guaranteeing the boundedness of the internal dynamics by Proposition~\ref{prop:handpos_trajectory_path_following}.

The results are shown in Figure~\ref{fig:trajectory_tracking}.
Figure~\ref{fig:trajectory_trajectory} shows the 3D trajectory of the vehicle, and Figures~\ref{fig:trajectory_position}, \ref{fig:trajectory_velocity}, and \ref{fig:trajectory_integral} show the external dynamics.
Compared to the trajectory tracking simulation, the vehicle converges to the desired path faster, after approximately $90$ seconds.
Figure~\ref{fig:trajectory_omega} shows the angular velocities of the vehicle.
The behavior is very similar to the trajectory tracking simulation.
Figure~\ref{fig:path_parameter} shows the derivative of the path parameter.
We can see that $\dot{s}$ decreases or increases depending on whether the vehicle is ``behind'' or ``in front of'' the desired path.
After the transient period, $\dot{s}$ converges to $U_d$.

Since the vehicle model and parameters of the controller are identical for both simulations, the results of these simulations are very similar.
The main difference between the proposed controllers is that in path following, one has an additional ``degree of freedom'' when choosing the path parameter, while in trajectory tracking, the trajectory is parametrized in time and thus fixed.
In \cite{aguiar_trajectory_tracking_2007}, it is argued that the control signals of path following controllers are smoother and have a lower peak value than in the case of trajectory tracking controllers.
This fact is confirmed by the simulations, as the peak value of the control input $\mu$ is $\SI{0.1}{\meter\per\second\squared}$ lower, and the surge velocity $u_r$ is $\SI{0.2}{\meter\per\second}$ lower in the path following simulation.

\subsection{Straight-line Path Following}
\label{sec:handpos_trajectory_simulation_waypoints}

In this test, the vehicle should follow a path consisting of a series of waypoints connected by line segments.
The vehicle switches to the next line segment when the distance to the current waypoint is less than five meters.
The desired path is shown in Figure~\ref{fig:waypoints}.
The parameters of the simulation are chosen identically to the experiment described in the next section.
The initial position of the vehicle is $\eta(0) = \inlinevector{-1.8, -9.3, 0}$, the initial yaw angle is $185$ degrees, and the remaining angles and velocities are zero.
The desired path-following speed is $U_d = \SI{1.3}{\meter\per\second}$, and the gains of the PID controller are $k_p = 0.2$, $k_d = 0.9$, $k_I = 0.01$.
The gains are chosen so that $\Bar{H}_{\xi}$ is Hurwitz, guaranteeing the exponential stability of the external dynamics by Proposition~\ref{prop:straight_line_path}.

The results are shown in Figure~\ref{fig:waypoint_following}.
Figure~\ref{fig:waypoint_trajectory} shows the trajectory of the vehicle.
The vehicle starts converging to the desired line segment.
When it reaches the circle of acceptance, \emph{i.e.,} five meters within the current waypoint, it switches to the next segment.
Figure~\ref{fig:waypoint_omega} shows the angular velocities.
The dotted vertical lines indicate when the waypoints change.
We can see that after each change, there is a transient period where the angular velocities increase before converging back to zero.
Figures~\ref{fig:waypoint_position}--\ref{fig:waypoint_integral} show the position, velocity, and integral errors.
When the waypoints change, the steady-state value of the external states changes as well, causing an abrupt increase in the error states.
The error states then exponentially converge to zero.

\section{Experiments}
\label{sec:handpos_trajectory_experiments}

In this section, we present the results of an experiment performed on the \gls{lauv}.
In the experiment, we verify the effectiveness of the straight-line path following controller proposed in Section~\ref{sec:handpos_trajectory_straight_line_path}.
The reason for choosing this specific controller for experimental validation is that straight-line paths are natively supported by \gls{dune} \cite{dune}, the onboard software running on the \gls{lauv}.

% When implementing the path following controller in DUNE, we have decided to utilize the existing low-level surge velocity and angular rate controllers for the LAUV.
% The reason for this choice is that the existing controllers are fine-tuned for the specific vehicle, making them more robust towards model uncertainties.
% Consider the hand position $x_1$, as defined in \eqref{eq:handpos_trajectory_hand_position}.
% The derivative of $x_1$ is given by
% \begin{equation}
%     \dot{x}_1 = R\nu_r + R \left(\omega \times L\right) + V_c. \label{eq:handpos_trajectory_x_1_dot}
% \end{equation}
% By treating $u_r$, $q$, and $r$ as the input to our system, we can use the following change of inputs
% \begin{equation}
%     \begin{bmatrix} u_r \\ q \\ r \end{bmatrix}
%     =
%     \begin{bmatrix}
%         1 & 0 & 0 \\ 0 & 0 & -\frac{1}{h} \\ 0 & \frac{1}{h} & 0
%     \end{bmatrix}
%     \left( R\T\hat{\mu} - \begin{bmatrix} 0 \\ v_r \\ w_r \end{bmatrix} \right),
% \end{equation}
% where $\hat{\mu} \in \mathbb{R}^3$ is the new input, to transform \eqref{eq:handpos_trajectory_x_1_dot} into
% \begin{equation}
%     \dot{x}_1 = \hat{\mu} + V_c.
% \end{equation}
% Consider then the error states $\widetilde{x}_1^{\prime}$ and $\widetilde{x}_I^{\prime}$, as defined in \eqref{eq:handpos_trajectory_straight_line_path_PID}, and the following PI control law
% \begin{equation}
%     \hat{\mu} = -k_p\hat{I}\widetilde{x}_1^{\prime} - k_I\hat{I}\widetilde{x}_I^{\prime} + U_d R_p e_1,
% \end{equation}
% where $k_p$ and $k_I$ are positive gains.
% Similarly to Section~\ref{sec:handpos_trajectory_straight_line_path}, we introduce a change of coordinates
% \begin{subequations}
%     \begin{align}
%         \Bar{\xi}_1^{\prime} &= \widetilde{x}_1^{\prime}, \\
%         \Bar{\xi}_I^{\prime} &= \widetilde{x}_I^{\prime} - \frac{k_p}{k_I}\tilde{I}V_c,
%     \end{align}
% \end{subequations}
% and define a vector $\Bar{\Xi}^{\prime\T} = \left[\Bar{\xi}_I^{\prime\T}, \Bar{\xi}_1^{\prime\T}\right]$.
% It is straightforward to show that $\dot{\Bar{\Xi}}^{\prime} = \Bar{H}\Bar{\Xi}^{\prime}$, where
% \begin{equation}
%     \Bar{H} 
%     = 
%     \begin{bmatrix}
%         O_{2 \times 2} & I_{2 \times 2} \\
%         -k_I I_{2 \times 2} & -k_p I_{2 \times 2}
%     \end{bmatrix}.
% \end{equation}
% The matrix $\Bar{H}$ is Hurwitz for any positive $k_p$ and $k_I$, and the external dynamics are thus GES.

The experiment was performed at the harbor of Porto, Portugal.
Due to shallow water, the depth of the vehicle had to be restricted to two meters.
To fully utilize the available space, the desired path was given by a series of waypoints with varying depths arranged in a hexagon (see Figure~\ref{fig:waypoints}).
The vehicle follows straight-line segments given by the waypoints, and switches to the next waypoint when the distance to the current waypoint is less than five meters.
The parameters of the controller are identical to the simulation in Section~\ref{sec:handpos_trajectory_simulation_waypoints}.

The results of the experiment are shown in Figure~\ref{fig:experiment}.
Figure~\ref{fig:experiment_trajectory} shows the trajectory of the AUV.
The green line represents the hand position $x_1$, the red line represents the desired path, and the arrows represent the orientation of the vehicle, with the base of the arrow located at $\eta$, and the tip of the arrow pointing towards $x_1$.
Figure~\ref{fig:experiment_errors} shows the cross-track errors.
The dotted lines show when the waypoints change.
The sudden increase in cross-track errors is caused by the switching logic explained in the previous paragraph.
The errors then exponentially converge to within 0.2 meters of zero, which is approximately the measurement noise of the navigation system.
In addition to measurement noise, the control system is also subject to disturbances caused by the sea loads, and perturbations caused by modeling errors.
However, the experiments confirm that the integral state and the overall exponential stability of the controller provide some robustness to these effects.

\section{Conclusions and Future Work}
\label{sec:handpos_trajectory_conclusions}
In this chapter, we extended the hand position concept to 6DOF underactuated underwater vehicles.
By choosing the hand position as the output of our system, we could apply output feedback linearization to simplify the underactuated 6DOF vehicle dynamics to a double integrator without introducing any singularities. %with a constant disturbance caused by the ocean current.
We then showed that a simple PID-based controller can be used to solve both the trajectory tracking and path following control problems.
Using Lyapunov analysis, we proved the exponential stability of the external dynamics and ultimate boundedness of the internal dynamics.
Moreover, in the special case of straight-line trajectories and paths, we could modify the controllers and prove the exponential stability of the total system.
The proposed controllers were tested both in numerical simulations and experiments.

As mentioned in the Introduction, the hand position concept and its ability to transform a nonlinear underactuated model to a double integrator without singularities present an opportunity to utilize numerous control strategies that could otherwise not be used on nonholonomic or underactuated vehicles.
The work presented in this chapter may thus be used as a basis for applying other control algorithms, \emph{e.g.,} multiagent consensus control, to the underwater domain, similarly to how the stability proofs for marine vehicles moving in the horizontal plane in \cite{paliotta_trajectory_2019} served as a basis for the consensus algorithm proposed in \cite{restrepo_tracking-formation_2022}.
