\section{Line-of-sight Guidance}
\label{sec:background_LOS}

This section describes the \gls{los} guidance algorithm.
\gls{los} is an intuitive method for steering vehicles towards the desired path.
A review of \gls{los} guidance methods for marine vehicles is presented in \cite{gu_LOS_2023}.

\begin{figure}[b]
    \centering
    \begin{subfigure}{0.48\textwidth}
        \centering
        \def\svgwidth{0.9\textwidth}
        \import{figures/background}{los_2d.pdf_tex}
        \caption{Two-dimensional line-of-sight guidance.}
        \label{fig:background_los_2d}
    \end{subfigure}
    \begin{subfigure}{0.48\textwidth}
        \def\svgwidth{\textwidth}
        \import{figures/background}{los_3d.pdf_tex}
        \caption{Three-dimensional coupled line-of-sight guidance.}
        \label{fig:background_los_3d}
    \end{subfigure}
    \caption{Illustrations of line-of-sight guidance methods.}
\end{figure}

First, let us discuss \gls{los} guidance for vehicles moving in the horizontal plane.
Let $\mat{p}_b^p = \inlinevector{x_b^p, y_b^p}$ denote the path-following error of the barycenter.
Let $\ivel_{\rm LOS}$ denote the desired (inertial) line-of-sight velocity that steers the barycenter towards the desired path.
This velocity is given by \cite{fossen_handbook_2011,pettersen_waypoint_2001}
\begin{align}
    \ivel_{\rm LOS} &= U_{\rm LOS} \inlinevector{\cos(\chi_{\rm LOS}), \sin(\chi_{\rm LOS})}\!\!, &
    \chi_{\rm LOS} &= \psi_{p\!} - \arctan\left(\frac{y_b^p}{\Delta}\right)\!,
    \label{eq:background_los_2d}
\end{align}
where $U_{\rm LOS} > 0$ is the desired path-following speed, $\psi_p$ is the path-tangential angle, as defined in \eqref{eq:background_path_tangential_2D}, and $\Delta > 0$ is the so-called \emph{lookahead} distance.
An illustration of \gls{los} guidance in the horizontal plane is shown in \figref{fig:background_los_2d}.

For vehicles moving in three dimensions, there exist two types of \gls{los} guidance algorithms: \emph{decoupled} \cite{caharija_path_2012,abdurahman_switching_2019} and \emph{coupled} \cite{breivik_principles_2005,yu_nonlinear_2017,yu_LOS_2020}.
A decoupled \gls{los} algorithm consists of two separate guidance schemes that steer the vehicle in the horizontal and vertical plane.
Let $\mat{p}_b^p = \inlinevector{x_b^p, y_b^p, z_b^p}$ denote the path-following error of the barycenter.
Let us assume that the path-tangential coordinate frame is chosen according to \eqref{eq:background_R_p_zero_roll}, so that the rotation matrix $\mat{R}_p(s)$ has a zero roll angle.
The desired line-of-sight velocity, $\ivel_{\rm LOS}$, is then given by
\begin{align}
    \ivel_{\rm LOS} &= U_{\rm LOS} \!
    \begin{bmatrix}
        \cos(\gamma_{\rm LOS}) \cos(\chi_{\rm LOS}) \\
        \cos(\gamma_{\rm LOS}) \sin(\chi_{\rm LOS}) \\
        -\sin(\gamma_{\rm LOS})
    \end{bmatrix}\!, &
    \begin{split}
        \gamma_{\rm LOS} &= \theta_{p\!} + \arctan\left(\frac{z_b^p}{\Delta_z}\right)\!, \\
        \chi_{\rm LOS} &= \psi_{p\!} - \arctan\left(\frac{y_b^p}{\Delta_y}\right)\!, \\
    \end{split}
    \label{eq:background_los_decoupled}
\end{align}
where $U_{\rm LOS} > 0$ is the desired path-following speed, $\theta_p$ and $\psi_p$ are the path-tangential angles, as defined in \eqref{eq:background_path_tangential_3D}, and $\Delta_y, \Delta_z > 0$ are the lookahead distances of the horizontal and vertical guidance scheme, respectively.
Comparing the decoupled guidance scheme in \eqref{eq:background_los_decoupled} to the two-dimensional \gls{los} algorithm in \eqref{eq:background_los_2d}, we can see that the decoupled guidance scheme effectively consists of two two-dimensional \gls{los} guidance algorithms.

In a \emph{coupled} \gls{los} guidance scheme, the desired velocity is given by
\begin{equation}
    \ivel_{\rm LOS} = \frac{U_{\rm LOS}}{\sqrt{\Delta^2 + {y_b^p}^2 + {z_b^p}^2}} \mat{R}_p \inlinevector{\Delta, -y_b^p, -z_b^p},
    \label{eq:background_los_coupled}
\end{equation}
where $U_{\rm LOS} > 0$ is the desired path-following speed, $\mat{R}_p$ is the rotation matrix between the path-tangential and the inertial coordinate frame, and $\Delta > 0$ is the lookahead distance.
An illustration of this scheme is shown in \figref{fig:background_los_3d}.
The coupled scheme can be seen as an extension of the horizontal \gls{los} guidance scheme to three dimensions.
Indeed, the two-dimensional guidance scheme \eqref{eq:background_los_2d} can also be written as
\begin{equation}
    \ivel_{\rm LOS} = 
    \frac{U_{\rm LOS}}{\sqrt{\Delta^2 + {y_b^p}^2}} \begin{bmatrix} \Delta\cos(\psi_p) + y_b^p\sin(\psi_p) \\ \Delta\sin(\psi_p) - y_b^p\cos(\psi_p) \end{bmatrix} = 
    \frac{U_{\rm LOS}}{\sqrt{\Delta^2 + {y_b^p}^2}} \mat{R}_p \begin{bmatrix} \Delta \\ -y_b^p \end{bmatrix}. 
    \label{eq:background_los_2d_R}
\end{equation}
Comparing \eqref{eq:background_los_2d_R} to \eqref{eq:background_los_coupled}, we can see that both equations have a similar form.
