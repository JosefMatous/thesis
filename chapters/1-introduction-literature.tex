\section{Previous Research}

\subsection{Collision Avoidance for Autonomous Vehicles}
Autonomous vehicles are being increasingly used in cluttered and unpredictable environments where considerations to other vehicles and obstacles need to be made. 
Therefore, the control system of autonomous vehicles should include some form of collision avoidance (COLAV).

Reviews of various COLAV concepts are presented in \cite{statheros_autonomous_2008,tam_review_2009,hoy_algorithms_2015}.
In general, algorithms for COLAV can be split into two categories: motion planning and reactive algorithms.

Motion planning algorithms include, among others, various types of path planning algorithms \cite{wang_ship_2017,kuwata_safe_2014,lazarowska_ships_2015}, the dynamic window algorithm \cite{fox_dynamic_1997}, and model predictive control (MPC).
MPC can be used both for a single vehicle \cite{hagen_mpc-based_2018,sun_collision_2018} and for multi-agent systems in a distributed form \cite{kuriki_formation_2015,dai_distributed_2017}.
%Some motion planing algorithms also include consideration of relevant traffic protocols that apply in the given domain.

Reactive algorithms for COLAV include, among others, virtual potential fields \cite{roussos_3d_2008}, geometric guidance \cite{mujumdar_reactive_2011}, and control barrier functions (CBFs) \cite{squires_constructive_2018,igarashi_collision_2018,romdlony_stabilization_2016,basso_safety-critical_2020,ames_control_2014}.
Reactive algorithms are often used together with motion planning algorithms in a hybrid controller.
In such a controller, the reactive algorithm ensures the safety of the vehicle in unexpected situations.
Such an algorithm is proposed in \cite{hedjar_automatic_2019}, where a collision-free velocity reference is obtained through numerical optimization.
The proposed algorithm is designed specifically for autonomous surface vehicles (ASVs).

CBFs offer a COLAV method that is applicable for a wide range of systems \cite{ames_control_2019}.
In the literature, there are typically two ways in which CBFs are applied for COLAV.
They are either applied to a simplified model of the vehicle (\emph{e.g.,} a unicycle model \cite{squires_constructive_2018,igarashi_collision_2018}) to provide safe velocity references, or they are used together with control Lyapunov functions (CLFs) \cite{romdlony_stabilization_2016,basso_safety-critical_2020,ames_control_2014} on the complete model.

Reactive COLAV methods that work with a simplified model do not take into account the physical limitations of the vehicle, such as acceleration or actuator constraints.
Consequently, these methods may output reference signals that the underlying controllers cannot track.
To mitigate this, reactive COLAV methods should be included into the lowest-possible control level.