\chapter{Concluding Remarks and Future Work}
\label{part:conclusions}

\setlength{\epigraphwidth}{0.46\textwidth}
\epigraph{ \it
    We know everything, that is, \\
    that we know nothing.
}{
    L. Smoljak, ``Vyšetřování ztráty třídní knihy,'' 1967 (translated from Czech).
}

In this thesis, we presented and analyzed multiple control algorithms.
The majority of these algorithms solve the formation path-following problem of underactuated \acrfullpl{auv}.
We conclude the thesis by presenting some remarks for each chapter and suggestions for future work.

\subsubsection{Chapter~\ref{chap:collision_avoidance}: \nameref{chap:collision_avoidance}}
In Chapter~\ref{chap:collision_avoidance}, we proposed a method for integrating a \acrfull{colav} scheme into control allocation through the use of \acrfullpl{cbf}.
We demonstrated its effectiveness on two models of \acrfullpl{asv}, where it significantly improved the safety.
The proposed method can be readily implemented on vehicles that already use optimization-based control allocation by simply including the constraints given by the \acrfullpl{cbf} in the optimization.

We note that the performance of the proposed scheme depends on the choice of the parameters, such as the class-$\mathcal{K}_{\infty}$ function $\gamma$, and the weight matrices $\mat{Q}$, $\mat{R}_{\rm abs}$, and $\mat{R}_{\rm rel}$.
Finding a systematic method for choosing the parameter values that guarantee safety for a given vehicle model is a topic for future work.

\subsection*{Part~\ref{part:NSB}: \nameref{part:NSB}}

In this part, we proposed three \gls{nsb} algorithms for solving the formation path-following problem.

\subsubsection{Chapter~\ref{chap:5dof_nsb}: \nameref{chap:5dof_nsb}}

In Chapter~\ref{chap:5dof_nsb}, we proposed a formation path-following method for an arbitrary number of \glspl{auv}, proved the stability of the path following part, and verified its effectiveness in simulations.

Because the proposed algorithm is centralized, our method can only be used in scenarios where all the vehicles can communicate with each other.
A distributed version of the algorithm is introduced later in Chapter~\ref{chap:distr_NSB}.

In the simulations, the formation-keeping error shows exponential convergence to zero.
However, the stability of the formation-keeping task has not been theoretically proven.
A modified version of the algorithm with provable stability of the formation-keeping task is presented in Chapter~\ref{chap:NSB_R}.

\subsubsection{Chapter~\ref{chap:NSB_R}: \nameref{chap:NSB_R}}

This chapter extends the formation path-following \gls{nsb} algorithm to underactuated 6\gls{dof} vehicles while adding obstacle avoidance and depth-limiting capabilities. 
Both the path-following and formation-keeping parts are proven to be stable.     
In the proofs, we assume that the avoidance and depth-limiting tasks are not active.
An analysis of the closed-loop system with active avoidance and depth-limiting tasks is left for future work.

\subsubsection{Chapter~\ref{chap:distr_NSB}: \nameref{chap:distr_NSB}}

In Chapter~\ref{chap:distr_NSB}, we discussed to combine \acrlong{nsb} control with consensus, and in this way solve the formation path-following problem in a fully distributed fashion. We also found that it is possible to implement this concept in two different ways, using a continuous-time consensus algorithm, or a discrete-time one (the latter being more suitable for real-life implementations).
Using Lyapunov analysis, we have shown that in the special case of straight-line paths, the continuous-time version achieves uniform semiglobal exponential stability.
The discrete-time version is based on event-triggered paradigms, so to account for practical limitations in the way agents exchange information.

Both versions are then verified in simulations.
Comparing the discrete-time version to a similar cooperative path following algorithm presented in \cite{praveen_cooperative_2018}, we found 
that the proposed algorithm requires fewer transmissions between the vehicles, while having similar steady-state error.
Finally, we demonstrated the real-life effectiveness of the the discrete-time algorithm in field experiments with a fleet of light autonomous underwater vehicles.

Future work includes extending the stability proofs to the more general case of curved paths and more complex vehicle dynamics, as well as investigating the effects of the event-triggered scheme on the performance of the algorithm.
In addition, we plan to perform additional experiments with more vehicles and underwater communications.

\subsection*{Part~\ref{part:hand_position}: \nameref{part:hand_position}}

In this part, we extended the hand position concept to underactuated \glspl{auv} and presented multiple applications of this concept.

\subsubsection{Chapter~\ref{chap:handpos_definition}: \nameref{chap:handpos_definition}}

In this section, we extended the hand position concept to 6\gls{dof} underactuated underwater vehicles.
By choosing the hand position as the output of our system, we could apply output feedback linearization to simplify the underactuated 6\gls{dof} vehicle dynamics to a double integrator without introducing any singularities.
Then, we introduced the sufficient conditions under which the internal states are ultimately bounded.

As mentioned in the Introduction, the hand position concept and its ability to transform a nonlinear underactuated model to a double integrator without singularities present an opportunity to utilize numerous control strategies that could otherwise not be used on nonholonomic or underactuated vehicles.
Examples of such controllers were presented in Chapters~\ref{chap:handpos_trajectory}--\ref{chap:handpos_NSB}.

\subsubsection{Chapter~\ref{chap:handpos_trajectory}: \nameref{chap:handpos_trajectory}}

In this chapter, we showed how the hand position transformation combined with a simple PID-based controller can be used to solve both the trajectory-tracking and path-following control problems.
Using Lyapunov analysis, we proved the exponential stability of the external dynamics.
Moreover, in the special case of straight-line trajectories and paths, we could modify the controllers and prove the exponential stability of the total system.
The proposed controllers were tested both in numerical simulations and experiments.

\subsubsection{Chapter~\ref{chap:handpos_MPC}: \nameref{chap:handpos_MPC}}

In this chapter, we have proposed a distributed spline-based \gls{mpc} scheme for the formation path-following problem.
We have shown that using splines makes the distributed control problem computationally tractable.
Compared to collocation, the spline parametrization allows us to represent a longer prediction horizon using fewer variables.    
This is also beneficial for the communication, and thus makes it easier to do distributed control in environments where the communication bandwidth is limited (\emph{e.g.,} underwater).

One might argue that restricting the output to splines limits the subspace of feasible trajectories.
However, simulation results show that cubic splines provide a good approximation of many curves.
Another limiting factor is the need for differential flatness.
However, it is often possible to simplify the structure of the model to guarantee differential flatness.        
The proposed spline-based \gls{mpc} scheme can thus be seen as a trade-off between lower computational requirements and more restrictive assumptions on the model.

\subsubsection{Chapter~\ref{chap:handpos_tracking}: \nameref{chap:handpos_tracking}}

In this chapter, we addressed the tracking-in-formation control problem of cooperative autonomous underwater vehicles interacting over directed graphs and under \emph{hard} inter-agent constraints (proximity and collision avoidance) and \emph{soft} constraints (positive surge velocity). 
We proposed a distributed control law that solves this problem and that guarantees, simultaneously, connectivity preservation and inter-agent collision avoidance.
With respect to the stability analysis, it is important to emphasize that, beyond mere convergence properties as usually established in the literature of multi-agent systems, we establish almost-everywhere uniform asymptotic stability with exponential convergence of the tracking errors. 
Current research focuses on validating the results experimentally.

\subsubsection{Chapter~\ref{chap:handpos_NSB}: \nameref{chap:handpos_NSB}}

In this chapter, we proposed an extended \gls{nsb} method for double-integrator systems. The method was proved to provide \acrlongpl{ges} task error dynamics. The method was demonstrated in a case study of formation path-following with underactuated \glspl{auv}. We defined the second-order kinematic tasks for collision avoidance, formation-keeping, and path-following. To force a bounded velocity, we introduced a saturation term to the formation-keeping acceleration. The closed-loop formation-error system with the reformulated formation-keeping acceleration was proved to be \acrlongpl{ugas}, and the closed-loop path-following system was proved to be \acrlongpl{usges}. Simulation results demonstrate the effectiveness of our approach. Possible future work includes verifying the presented results through experiments.
