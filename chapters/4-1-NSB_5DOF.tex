\chapter{Formation Path-Following Control of 5DOF Underactuated AUVs}
\label{chap:5dof_nsb}

\setlength{\epigraphwidth}{0.55\textwidth}
\epigraph{\it
    An ant is very stupid \dots \\ and yet, many ants together are smart.
}{
    Kurzgesagt --- In a Nutshell, ``Emergence,'' \url{youtu.be/16W7c0mb-rE}.
}

This chapter presents a novel method for formation path following of multiple underactuated autonomous underwater vehicles.
The method combines \acrlong{los} guidance with \acrlong{nsb} control, allowing the vehicles to follow curved paths while maintaining the desired formation.
We investigate the dynamics of the path-following error using cascaded systems theory, and show that the closed-loop system is \acrlongpl{usges}.
We validate the theoretical results through numerical simulations.
The contents of this chapter are based on \cite{matouvs_formation_2022}.

The chapter is organized as follows.
Section \ref{sec:nsb_5dof_model} defines the formation path-following problem that is addressed in this chapter.
In Section \ref{sec:nsb_5dof_control}, we describe the control system.
The stability of the control system is proven in Section \ref{sec:nsb_5dof_path_stability}.
Finally, Section \ref{sec:nsb_5dof_simulations} contains the results of a numerical simulation.

\section{Problem Definition}
\label{sec:nsb_5dof_model}
In this section, we briefly present the \gls{auv} model and the formation path-following problem.

\subsection{Vehicle Model}
We consider a fleet of $N$ underactuated \glspl{auv}.
The dynamics are described using the 5\gls{dof} control-oriented model from Section~\ref{sec:model_control_oriented}.
The pose ($\pose$) and velocities ($\vel$) of the \glspl{auv} are defined as
\begin{align}
    \pose &= \left[x, y, z, \theta, \psi\right]\T, &
    \vel &= \left[u, v, w, q, r\right]\T.
\end{align}
The roll dynamics are disregarded as the roll motion is assumed to be small and self-stabilizing by the vehicle design.
Let $\ocean \in \mathbb{R}^3$ be the velocity of an unknown, constant and irrotational ocean current.

Recalling \eqref{eq:background_component_form_5DOF}, the dynamics of the \glspl{auv} are
\begin{subequations}
    \begin{align}
        \dot{x} &= u\cos\left(\psi \right)\cos\left(\theta \right)-v\sin\left(\psi \right)+w\cos\left(\psi \right)\sin\left(\theta \right), \\
        \dot{y} &= u\cos\left(\theta \right)\sin\left(\psi \right)+v\cos\left(\psi \right)+w\sin\left(\psi \right)\sin\left(\theta \right), \\
        \dot{z} &= -u\sin\left(\theta \right)+w\cos\left(\theta \right), \\
        \dot{\theta} &= q, \label{eq:nsb_5dof_theta_dot} \\
        \dot{\psi} &= \frac{1}{\cos\left(\theta \right)}r, \label{eq:nsb_5dof_psi_dot} \\
        \dot{u} &= f_{u} + F_u(u, v, w, q, r) + \bs{\phi}_u(u, v, w, q, r, \theta, \psi)\T\ocean, \label{eq:nsb_5dof_u_dot} \\
        \dot{v} &= X_v(u, u_{c})r + Y_v(u, u_{c})v_{r}, \\
        \dot{w} &= X_w(u, u_{c})q + Y_w(u, u_{c})w_{r} + G(\theta), \\
        \dot{q} &= t_{q} + F_q(u, w, q, \theta) + \bs{\phi}_q(u, w, q, \theta, \psi)\T\mathbb{V}_c, \label{eq:nsb_5dof_q_dot} \\
        \dot{r} &= t_{r} + F_r(u, v, r) + \bs{\phi}_r(u, v, r, \theta, \psi)\T\mathbb{V}_c. \label{eq:nsb_5dof_r_dot} 
    \end{align} \label{eq:nsb_5dof_components}
\end{subequations}

\subsection{Control Objectives}
\label{sec:nsb_5dof_objectives}
The goal is to control the \glspl{auv} so that they move in a prescribed formation while avoiding collisions, and their barycenter follows a given path.

The prescribed path is parametrized by a smooth function $\mat{p}_p: \mathbb{R} \rightarrow \mathbb{R}^3$.
We assume that the parametrization is $\mathcal{C}^2$ and regular.
Therefore, for every point $\mat{p}_p(s)$ on the path, there exist path-tangential angles, $\theta_p(s)$ and $\psi_p(s)$, and a corresponding path-tangential coordinate frame (see Section~\ref{sec:background_paths} for more details).

The path-following error $\mat{p}_b^p$ is given by the position of the barycenter expressed in the path-tangential coordinate frame
\begin{equation}
    \mat{p}_b^p = \mat{R}_p(s)\T \, \big(\mat{p}_b - \mat{p}_p(s)\big), \label{eq:nsb_5dof_barycenter}
\end{equation}
where
\begin{align}
    \mat{p}_b &= \frac{1}{N} \sum_{i=1}^N \mat{p}_i, & 
    \mat{p}_i &= \left[x_i, y_i, z_i\right]\T.
\end{align}

The vehicles should converge to a dynamic formation that rotates with the desired path (see Section~\ref{sec:background_formation_keeping} for details).
Let $\mat{p}_{f,1}^f, \ldots, \mat{p}_{f,n}^f$ be the position vectors that represent the desired formation.
The objective is to control the vehicles so that
\begin{align}
    \mat{p}_i - \mat{p}_b &\rightarrow \mat{R}_p(s) \mat{p}_{f,i}^f, &
    \forall i &\in \left\{1, \ldots, N \right\}.
\end{align}

We propose to solve this problem using an \gls{nsb} algorithm.
We note that the proposed algorithm is \emph{centralized}.
Consequently, to implement this algorithm in a real-life situation, there must be a central node that can communicate and coordinate with all the \glspl{auv}.
Alternatively, every \gls{auv} must have access to the complete state of all other \glspl{auv}.

\section{Control System}
\label{sec:nsb_5dof_control}
To solve the formation path following problem, we propose a method that combines \acrfull{colav}, formation keeping, and path following in a hierarchic manner using an \gls{nsb} algorithm.
Since the \gls{nsb} algorithm outputs inertial velocity references, we also need a low-level attitude control system to track these references.

In this section, we first present the attitude control system.
Then, in Section~\ref{sec:nsb_5dof_NSB}, we present the \gls{nsb} algorithm and the associated tasks.
Finally, in Section~\ref{sec:nsb_5dof_path_parameter}, we demonstrate how to use the update law of the path variable to cancel unwanted terms in the path-following error dynamics.

\subsection{Attitude Control System}
\label{sec:nsb_5dof_ACS}
This system controls the surge velocity, pitch, and yaw via the corresponding accelerations.
The system is based on the autopilots in \cite{moe_LOS_2016}, but extended to five \acrlongpl{dof}.

Let $u_d$ be the desired surge velocity and $\dot{u}_d$ its derivative.
Let $\oceanhat$ be the estimate of the ocean current.
Furthermore, let us define $\tilde{u} = u - u_d$ and $\oceantilde = \oceanhat - \ocean$.
The surge controller consists of an output-linearizing sliding-mode P-controller and an ocean current observer \vspace{-1mm}
\begin{align}
    f_u &= \dot{u}_d - F_u(\cdot) - \bs{\phi}_u(\cdot)\T\,\oceanhat - k_u\,\tilde{u} - k_c\,{\rm sign}\left(\tilde{u}\right), \label{eq:nsb_5dof_t_u} \\
    \dot{\hat{\ivel}}_c &= c_u\,\bs{\phi}_u(\cdot)\,\tilde{u}, \label{eq:nsb_5dof_V_hat_u} \vspace{-2mm}
\end{align}
where $k_u$, $k_c$ and $c_u$ are positive gains.

Let $\theta_d$ be the desired pitch angle and $\dot{\theta}_d, \ddot{\theta}_d$ its derivatives.
Let $\hat{\mathbb{V}}_q$ be the estimate of $\mathbb{V}_c$.
Furthermore, let us define $\tilde{\theta} = \theta - \theta_d$, $\tilde{q} = q - \dot{\theta}_d$ and $\tilde{\mathbb{V}}_q = \hat{\mathbb{V}}_q - \mathbb{V}_c$.
Inspired by \cite{moe_set-based_2017}, we introduce the following transformation \vspace{-1.5mm}
\begin{equation}
    s_q = \tilde{q} + \lambda_q\,\tilde{\theta}, \vspace{-2mm}
\end{equation}
where $\lambda_q$ is a positive constant.
The pitch controller consists of an output-linearizing sliding-mode PD-controller and an ocean current observer \vspace{-1mm}
\begin{align}
    \begin{split}
        t_q &= \ddot{\theta}_d - F_q(\cdot) - \bs{\phi}_q(\cdot)\T\,\hat{\mathbb{V}}_q - \lambda_q\,\tilde{q} \\
        & \quad - k_{\theta}\,\tilde{\theta} - k_q\,s_q - k_d\,{\rm sign}(s_q), 
    \end{split} \label{eq:nsb_5dof_t_q} \\
    \dot{\hat{\mathbb{V}}}_q &= c_q\,\bs{\phi}_q(\cdot)\,s_q, \label{eq:nsb_5dof_V_hat_q}
\end{align}
\vspace{-5.5mm}

\noindent where $k_{\theta}$, $k_q$, $k_d$ and $c_q$ are positive gains.

Let $\psi_d$ be the desired yaw angle and $\dot{\psi}_d, \ddot{\psi}_d$ its derivatives.
Let $\hat{\mathbb{V}}_r$ be the estimate of $\mathbb{V}_c$.
Furthermore, let us define $\tilde{\psi} = \psi - \psi_d$ and $\tilde{\mathbb{V}}_r = \hat{\mathbb{V}}_r - \mathbb{V}_c$.
Similarly to the pitch controller, we introduce the following transformation
\begin{equation}
    s_r = \dot{\tilde{\psi}} + \lambda_r\,\tilde{\psi} = \frac{r}{\cos\theta} - \dot{\psi}_d + \lambda_r\,\tilde{\psi},
\end{equation}
where $\lambda_r$ is a positive constant.
The yaw controller is analogous to the pitch controller introduced previously
\begin{align}        
    \begin{split}
        \scale[0.95]{t_r} &\scale[0.95]{= - F_r(\cdot) - \bs{\phi}_r(\cdot)\T\,\hat{\mathbb{V}}_r - r\,\tan(\theta)\dot{\theta}} \\
        & \scale[0.95]{ + \cos(\theta)\left(\ddot{\psi}_d - \lambda_r\,\dot{\tilde{\lambda}} - k_{\psi}\,\tilde{\psi} - k_r\,s_r - k_d\,{\rm sign}(s_r)\right), }
    \end{split} \label{eq:nsb_5dof_t_r} \\
    \scale[0.95]{\dot{\hat{\mathbb{V}}}_r} & \scale[0.95]{= c_r\,\bs{\phi}_r(\cdot)\,s_r,} \label{eq:nsb_5dof_V_hat_r}
\end{align}
where $k_{\psi}$, $k_r$, $k_d$ and $c_r$ are positive gains.

\subsection{NSB Tasks}
\label{sec:nsb_5dof_NSB}
Let us denote the variables associated with the \gls{colav}, formation keeping, and path following tasks by lower indices $1$, $2$, and $3$, respectively.
Each task produces a vector of desired velocities, $\ivel_{1}, \ivel_{2}, \ivel_{3} \in \mathbb{R}^{3N}$.

First, let us consider the \gls{colav} task.
Let {$d_{\rm COLAV}$} be the \emph{activation distance}, \emph{i.e.,} the distance at which the vehicles need to start performing the evasive maneuvers.
The task variable is then given by a vector of relative distances between the vehicles smaller than $d_{\rm COLAV}$, \emph{i.e.,}
\begin{align}
        \bs{\sigma}_1 &= \big[\norm{\mat{p}_i - \mat{p}_j}\big]\T, &
        \begin{split} 
            \forall &i,j\in\{1,\ldots,N\}, j > i, \\
            &\norm{\mat{p}_i - \mat{p}_j} < d_{\rm COLAV}.
        \end{split}
\end{align}
The desired value of the task variable is
\begin{equation}
    \bs{\sigma}_{d,1} = d_{\rm COLAV} \, \mat{1},
\end{equation}
where $\mat{1}$ is a vector of ones of the corresponding size.
The velocity associated with the \gls{colav} task is given by
\begin{equation}
    \ivel_{1} = - \mat{J}_1^{\dagger}\, \bs{\Lambda}_1\,\widetilde{\bs{\sigma}}_1,
\end{equation}
where $\bs{\Lambda}_1$ is a positive definite gain matrix, and $\widetilde{\bs{\sigma}}_1 = \bs{\sigma}_1 - \bs{\sigma}_{d,1}$.
Note that this task does not guarantee robust collision avoidance.
During the transients, the relative distance may become smaller than $d_{\rm COLAV}$.
Therefore, to ensure collision avoidance, $d_{\rm COLAV}$ shuld be chosen as $d_{\rm min} + d_{\rm sec}$, where $d_{\rm min}$ is the minimum safe distance between the vehicles, and $d_{\rm sec}$ is an additional security distance.

The formation-keeping and path-following tasks are defined identically as in Section~\ref{sec:background_nsb_formation_path_following}.
The task variable of the formation-keeping task is
\begin{align}
    \bs{\sigma}_2 &= \left[\bs{\sigma}_{2,1}\T, \ldots, \bs{\sigma}_{2,N-1}\T\right]\T, \label{eq:nsb_5dof_sigma_2} &
    \bs{\sigma}_{2,i} &= \mat{p}_i - \mat{p}_b,
\end{align}
and its desired values are
\begin{equation}
    \bs{\sigma}_{d,2} = \begin{bmatrix}
        \mat{R}\left(\theta_p(s), \psi_p(s)\right)\,\mat{p}_{f,1}^f \\
        \vdots \\
        \mat{R}\left(\theta_p(s), \psi_p(s)\right)\,\mat{p}_{f,N-1}^f
    \end{bmatrix}. \label{eq:nsb_5dof_sigma_d_2}
\end{equation}
The desired velocity of the formation keeping task is given by
\begin{align}
    \ivel_{2} &= \mat{J}_2^{\dagger}\,\left(\dot{\bs{\sigma}}_{d,2} - \bs{\Lambda}_2\,\tilde{\bs{\sigma}}_2\right), \label{eq:nsb_5dof_CLIK}
\end{align}
where $\widetilde{\bs{\sigma}}_2 = \bs{\sigma}_2 - \bs{\sigma}_{d,2}$ is the error, and $\mat{\Lambda}_2$ is a positive definite gain matrix.

The task variable and the desired value of the path-following task is given by $\mat{p}_b$ and $\mat{p}_p(s)$, respectively.
The desired velocity of the path-following task is obtained using the decoupled \gls{los} guidance algorithm \eqref{eq:background_los_decoupled}.
We choose the same lookahead distance for the horizontal and vertical guidance schemes, \emph{i.e.,} $\Delta_y = \Delta_z = \Delta$.
Inspired by \cite{belleter_2019_observer}, we employ a time-varying error-dependent lookahead distance
\begin{equation}
    \Delta\left(\mat{p}_b^p\right) = \sqrt{\Delta_0^2 + \left(x_b^p\right)^2 + \left(y_b^p\right)^2 + \left(z_b^p\right)^2}, \label{eq:nsb_5dof_delta}
\end{equation}
where $\Delta_0 > 0$ is a constant.
The desired velocity of the path-following task is then given by
\begin{equation}
    \ivel_{ 3} = \mat{1}_N \otimes \ivel_{\rm LOS},
\end{equation}
where
\begin{align}
    \ivel_{\rm LOS} &= U_{\rm LOS} \!
    \begin{bmatrix}
        \cos(\gamma_{\rm LOS}) \cos(\chi_{\rm LOS}) \\
        \cos(\gamma_{\rm LOS}) \sin(\chi_{\rm LOS}) \\
        -\sin(\gamma_{\rm LOS})
    \end{bmatrix}\!, &
    \begin{split}
        \gamma_{\rm LOS} &= \theta_{p\!} + \arctan\left(\frac{z_b^p}{\Delta(\mat{p}_b^p)}\right)\!, \\
        \chi_{\rm LOS} &= \psi_{p\!} - \arctan\left(\frac{y_b^p}{\Delta(\mat{p}_b^p)}\right)\!, \\
    \end{split}
    \label{eq:nsb_5dof_v_LOS}
\end{align}
where $U_{\rm LOS} > 0$ is the desired path-following speed.

The three tasks are then combined using the recursive \gls{nsb} algorithm \eqref{eq:background_NSB_recursive}.
If the \gls{colav} task is active, the \gls{nsb} velocity is given by
\begin{equation}
    \ivel_{\rm NSB} = \ivel_{ 1} + \mat{N}_1 \left(\ivel_{ 2} + \mat{N}_2 \ivel_{ 3}\right).
    \label{eq:nsb_5dof_v_NSB_COLAV}
\end{equation}
If the \gls{colav} task is inactive, \eqref{eq:nsb_5dof_v_NSB_COLAV} is simplified to
\begin{equation}
    \ivel_{\rm NSB} = \ivel_{ 2} + \ivel_{ 3},
\end{equation}
thanks to the independence and orthogonality of the formation-keeping and path-following task.

Let $\ivel_{{\rm NSB}, i}$ be the desired \gls{nsb} velocity associated with vehicle $i$, \emph{i.e.,}
\begin{equation}
    \left[\ivel_{{\rm NSB}, 1}\T, \ldots, \ivel_{{\rm NSB}, N}\T\right] = \ivel_{\rm NSB}\T.
\end{equation}
These velocities must be decomposed into surge, pitch, and yaw references that can be tracked by the attitude control system presented in Section~\ref{sec:nsb_5dof_ACS}.
Similarly to \cite{arrichiello_formation_2006}, we propose a method with angle of attack and sideslip compensation
\begin{align}
    \scale[0.96]{u_{d, i}} & \scale[0.96]{= U_{{\rm NSB}, i}\,\frac{1 + \cos\left(\gamma_{{\rm NSB}, i} - \gamma_i\right)\cos\left(\chi_{{\rm NSB}, i} - \chi_i\right)}{2},} \label{eq:nsb_5dof_u_d}\\
    %\scale[0.96]{u_{d, i}} & \scale[0.96]{= \sqrt{U_{{\rm NSB}, i}^2 - v_i^2 - w_i^2}} \label{eq:nsb_5dof_u_d}\\
    \scale[0.96]{\theta_{d, i}} & \scale[0.96]{= \gamma_{{\rm NSB}, i} + \alpha_{d, i}, \quad \alpha_{d, i} = \arctan\left(\frac{w_i}{u_{d, i}}\right),} \label{eq:nsb_5dof_theta_d} \\
    \scale[0.96]{\psi_{d, i}} & \scale[0.96]{= \chi_{{\rm NSB}, i} - \beta_{d, i}, \quad \beta_{d, i} = \arcsin\left(\frac{v_i}{\sqrt{u_{d, i}^2 + v_i^2 + w_i^2}}\right), } \label{eq:nsb_5dof_psi_d}
\end{align}
%where $U_{{\rm NSB}, i}$, $\gamma_{{\rm NSB}, i}$, and $\chi_{{\rm NSB}, i}$ are the norm, pitch angle, and yaw angle of the $i^{\rm th}$ desired NSB velocity, respectively, and $v_i$, $w_i$, $\gamma_i$, and $\chi_i$ are the sway and heave velocities, and the flight-path and course angles of the $i^{\rm th}$ vehicle, respectively.
where $v_i$ and $w_i$ are the sway and heave velocities, and $\gamma_i$ and $\chi_i$ are the flight-path and course angles of the $i^{\rm th}$ vehicle, respectively, and $U_{{\rm NSB},i}$, $\gamma_{{\rm NSB},i}$ and $\chi_{{\rm NSB},i}$ are given by
\begin{subequations}
    \begin{align}
        U_{{\rm NSB}, i} &= \norm{\ivel_{{\rm NSB}, i}}, \quad
        \ivel_{{\rm NSB}, i} = \begin{bmatrix} \dot{x}_{{\rm NSB}, i} \\ \dot{y}_{{\rm NSB}, i} \\ \dot{z}_{{\rm NSB}, i}\end{bmatrix}, \\*
        \gamma_{{\rm NSB}, i} &= - \arcsin\left(\frac{\dot{y}_{{\rm NSB}, i}}{U_{{\rm NSB}, i}}\right), \\
        \chi_{{\rm NSB}, i} &= \mathrm{arctan}_2 \left(\dot{y}_{{\rm NSB}, i}, \dot{x}_{{\rm NSB}, i}\right).
    \end{align}
\end{subequations}

\subsection{Path Parameter Update Law}
\label{sec:nsb_5dof_path_parameter}
Inspired by \cite{belleter_2019_observer}, we use the update law of the path variable $s$ to get desirable behavior of the along-track error ($x_b^p$).

Note that the kinematics of the $i^{\rm th}$ vehicle can be alternatively expressed using the total speed ($U_i$) and the flight-path ($\gamma_i$) and course ($\chi_i$) angles of the vehicle as
% \begin{align}
%     \dot{\mat{p}}_i &= \begin{bmatrix}
%         \cos\left(\chi_i\right)\,\cos\left(\gamma_i\right)\\ 
%         \cos\left(\gamma_i\right)\,\sin\left(\chi_i\right)\\ 
%         -\sin\left(\gamma_i\right)
%     \end{bmatrix} \, U_i, &
%     U_i &= \sqrt{u_i^2 + v_i^2 + w_i^2}. \label{eq:nsb_5dof_kinematics}
% \end{align}
\begin{equation}
    \scale[0.91]{\dot{\mat{p}}_i = \left[\cos\left(\chi_i\right)\,\cos\left(\gamma_i\right) ,\, \cos\left(\gamma_i\right)\,\sin\left(\chi_i\right) ,\, -\sin\left(\gamma_i\right)\right]\T \, U_i.} \label{eq:nsb_5dof_kinematics}
\end{equation}
Now, let us investigate the kinematics of the barycenter.
Differentiating \eqref{eq:nsb_5dof_barycenter} with respect to time and substituting \eqref{eq:nsb_5dof_kinematics} yields the following equations
\begin{subequations}
    \begin{align}
        \begin{split}
            \dot{x}_b^p &= \frac{1}{N}\sum_{i=1}^N U_i\,\Omega_x\left(\gamma_i, \theta_p, \chi_i, \psi_p\right) \\
            &\quad - \scale{\norm{\frac{\partial \mat{p}_p(s)}{\partial s}}}\dot{s} + \omega_zy_b^p - \omega_yz_b^p,
        \end{split} \label{eq:nsb_5dof_x_pb} \\
        \dot{y}_b^p &= \frac{1}{N}\sum_{i=1}^N U_i\,\Omega_y\left(\gamma_i, \theta_p, \chi_i, \psi_p\right) + \omega_xz_b^p - \omega_zx_b^p, \label{eq:nsb_5dof_y_pb} \\
        \dot{z}_b^p &= \frac{1}{N}\sum_{i=1}^N U_i\,\Omega_z\left(\gamma_i, \theta_p, \chi_i, \psi_p\right) + \omega_yx_b^p - \omega_xy_b^p, \label{eq:nsb_5dof_z_pb}
    \end{align} \label{eq:nsb_5dof_barycenter_kinematics}
\end{subequations}
where
\begin{subequations}
    \begin{align}
        \scale[1]{\Omega_x(\cdot)} & \mathrlap{\scale[1]{= \sin\left(\theta_p\right)\sin\left(\gamma_i\right) + \cos\left(\theta_p\right)\cos\left(\gamma_i\right)\cos\left(\psi_p-\chi_i\right),}} \\
        \scale[1]{\Omega_y(\cdot)} & \mathrlap{\scale[1]{= -\cos\left(\gamma_i\right)\sin\left(\psi_p - \chi_i\right),}} \\
        \scale[1]{\Omega_z(\cdot)} & \mathrlap{\scale[1]{=-\cos\left(\theta_p\right)\sin\left(\gamma_i\right) + \cos\left(\gamma_i\right)\sin(\theta_p)\cos\left(\psi_p-\chi_i\right)}} \\
        \scale[1]{\omega_x} & \scale[1]{= -\iota\dot{s}\sin(\theta_p),} &
        \scale[1]{\omega_y} & \scale[1]{= \kappa\dot{s},} &
        \scale[1]{\omega_z} & \scale[1]{= \iota\dot{s}\cos(\theta_p),} \\
        \scale[1]{\kappa(s)} & \scale[1]{= \frac{\partial \theta_p(s)}{\partial s},} &
        \scale[1]{\iota(s)} & \scale[1]{= \frac{\partial \psi_p(s)}{\partial s}.}
    \end{align}
\end{subequations}
To stabilize the along-track error dynamics, we choose the following path variable update 
\begin{equation}
    \dot{s} = \norm{\frac{\partial \mat{p}_p(s)}{\partial s}}^{-1} \left( \frac{1}{N}\sum_{i=1}^N U_i\,\Omega_x\left(\gamma_i, \theta_p, \chi_i, \psi_p\right) + k_{s}\,\frac{x_b^p}{\sqrt{1+\left(x_b^p\right)^2}}\right),
    \label{eq:nsb_5dof_path_update}
\end{equation}
where $k_{s} > 0$ is a constant.

\section{Closed-Loop Analysis}
\label{sec:nsb_5dof_path_stability}
In this section, we investigate the closed-loop stability of the path following task.
We define two error states, $\tilde{\mat{X}}_1$ and $\tilde{\mat{X}}_2$, as
\begin{align}
    \tilde{\mat{X}}_1 &= \left[x_b^p, y_b^p, z_b^p\right]\T, && \\
    \tilde{\mat{X}}_2 &= \left[\tilde{\mat{X}}_{2,1}\T, \ldots, \tilde{\mat{X}}_{2,N}\T\right]\T, &
    \tilde{\mat{X}}_{2,i} &= \left[\tilde{u}_i, s_{q,i}, \tilde{\theta}_i, s_{r,i}, \tilde{\psi}_i\right]\T,
\end{align}

Now, we can take the barycenter kinematics from \eqref{eq:nsb_5dof_barycenter_kinematics} and express it in terms of $\tilde{\mat{X}}_1$ and $\tilde{\mat{X}}_2$ as
\begin{subequations}
    \begin{align}
        &\dot{x}_b^p = -k_{s}\frac{x_b^p}{\sqrt{1+\left(x_b^p\right)^2}} + \omega_zy_b^p - \omega_yz_b^p, \label{eq:nsb_5dof_x_pb_CL} \\
        &\begin{aligned}
            \dot{y}_b^p &= - \frac{1}{N}\sum_{i=1}^N U_{d,i}{\frac{\cos\left(\gamma_{\rm LOS}\right)y_b^p}{\sqrt{\Delta\left(\mat{p}_b^p\right)^2 + \left(y_b^p\right)^2}}} + \omega_xz_b^p - \omega_zx_b^p \\
            & \quad + G_y\big(\tilde{u}_1, \ldots, \tilde{u}_N, \tilde{\psi}_1, \ldots, \tilde{\psi}_N, \gamma_1, \ldots, \gamma_N, \\
            & \qquad\qquad u_{d,1}, \ldots, u_{d,N}, v_1, \ldots, v_N, w_1, \ldots, w_N, \mat{p}_b^p, \psi_p\big),
        \end{aligned} \\
        &\begin{aligned}
            \dot{z}_b^p &= \frac{1}{N} \sum_{i=1}^N U_{d,i}\frac{z_b^p}{\sqrt{\Delta\left(\mat{p}_b^p\right)^2 + \left(z_b^p\right)^2}} + \omega_yx_b^p - \omega_xy_b^p \\
            & \quad + G_z\big(\tilde{u}_1, \ldots, \tilde{u}_N, \tilde{\theta}_1, \ldots, \tilde{\theta}_N, \gamma_1, \ldots, \gamma_N, \chi_1, \ldots, \chi_N, \\
            & \qquad \qquad\, u_{d,1}, \ldots, u_{d,N}, v_1, \ldots, v_N, w_1, \ldots, w_N, \mat{p}_b^p, \psi_p, \theta_p\big).
        \end{aligned} \label{eq:nsb_5dof_z_pb_CL}
    \end{align} \label{eq:nsb_5dof_nominal}
\end{subequations}

\noindent The equations for $G_y(\cdot)$ and $G_z(\cdot)$ are given in Appendix~\ref{app:5dof_nsb_barycenter}.
Substituting the attitude control system \eqref{eq:nsb_5dof_t_u}--\eqref{eq:nsb_5dof_V_hat_r} into vehicle dynamics \eqref{eq:nsb_5dof_components} yields the following closed-loop behavior of $\tilde{\mat{X}}_2$
\begin{subequations}
    \begin{align}
        \scale[0.97]{\dot{\tilde{u}}_i} &= \scale[0.97]{- k_u\,\tilde{u}_i - k_c\,{\rm sign}\left(\tilde{u}_i\right) - \bs{\phi}_u(\cdot)\T\tilde{\ivel}_{c,i},} \label{eq:nsb_5dof_u_tilde} \\
        \scale[0.97]{\dot{s}_{q,i}} &= \scale[0.97]{-k_{\theta}\,\tilde{\theta}_i - k_q\,s_{q,i} - k_d\,{\rm sign}(s_{q,i}) - \bs{\phi}_q(\cdot)\T\,\tilde{\mathbb{V}}_{q,i}, }\\
        \scale[0.97]{\dot{\tilde{\theta}}_i} &= \scale[0.97]{s_{q,i} - \lambda_q\,\tilde{\theta}_i,} \\
        \scale[0.97]{\dot{s}_{r,i}} &= \scale[0.97]{-k_{\theta}\,\tilde{\theta}_i - k_r\,s_{r,i} - k_d\,{\rm sign}(s_{r,i}) - \bs{\phi}_r(\cdot)\T\,\tilde{\mathbb{V}}_{r,i},} \\
        \scale[0.97]{\dot{\tilde{\psi}}_i} &= \scale[0.97]{s_{r,i} - \lambda_r\,\tilde{\psi}_i,} \label{eq:nsb_5dof_psi_tilde}
    \end{align} \label{eq:nsb_5dof_perturbing}
\end{subequations}
the ocean current estimate errors
\begin{subequations}
    \begin{align}
        \dot{\tilde{\ivel}}_{c,i} & = c_u\,\bs{\phi}_u(\cdot)\,\tilde{u}_i, \label{eq:nsb_5dof_V_c_tilde} \\
        \dot{\tilde{\mathbb{V}}}_{q,i} &= c_q\,\bs{\phi}_q(\cdot)\,s_{q,i}, \\
        \dot{\tilde{\mathbb{V}}}_{r,i} &= c_r\,\bs{\phi}_r(\cdot)\,s_{r,i}, \label{eq:nsb_5dof_theta_r_tilde}
    \end{align} \label{eq:nsb_5dof_estimates}
\end{subequations}
and the underactuated sway and heave dynamics
\begin{align}
    \dot{v}_i &= X_v(u_i, u_c)\,r_i + Y_v(u_i, u_c)\,(v_i - v_c), \label{eq:nsb_5dof_v_dot} \\
    \dot{w}_i &= X_w(u_i, u_c)\,q_i + Y_w(u_i, u_c)\,(w_i - w_c) + G(\theta_i). \label{eq:nsb_5dof_w_dot}
\end{align}

To prove the stability of the closed-loop system, we need the results of the three following lemmas.
The lemmas follow the same structure as the 2D case for two ASVs in \cite{eek_formation_2021}, and are extended to handle an arbitrary number of \glspl{auv} moving in 3D.
\begin{lemma}
    \label{lemma_1}
    The trajectories of the closed-loop system \eqref{eq:nsb_5dof_nominal}--\eqref{eq:nsb_5dof_w_dot} are forward complete.
\end{lemma}

\begin{proof}
    The complete proof is given in Appendix~\ref{app:5dof_nsb_lemma_1}.
    Here, we only present a sketch of the proof.

    The proof is split into three parts: proving the forward-completeness of the attitude control system \eqref{eq:nsb_5dof_perturbing}, \eqref{eq:nsb_5dof_estimates}, the underactuated dynamics \eqref{eq:nsb_5dof_v_dot}, \eqref{eq:nsb_5dof_w_dot}, and the path-following errors \eqref{eq:nsb_5dof_nominal}.

    Using the same arguments as for the horizontal case in \cite{moe_LOS_2016}, we can prove that the system \eqref{eq:nsb_5dof_perturbing} is \glspl{ges} and the ocean current estimates \eqref{eq:nsb_5dof_estimates} are bounded.
    Exponential stability and boundedness imply forward completeness. Therefore, \eqref{eq:nsb_5dof_perturbing} and \eqref{eq:nsb_5dof_estimates} are forward complete.

    For the underactuated dynamics, we define Lyapunov function candidates
    \begin{align} 
        V_v(v_i) &= \frac{1}{2} v_i^2, &
        V_w(w_i) &= \frac{1}{2} w_i^2, \label{eq:nsb_5dof_underactuated_LFC}
    \end{align}
    and show that there exist positive constants $\alpha_v, \alpha_w, \beta_v, \beta_w$ such that
    \begin{align} 
        \dot{V}_v(v_i) & \leq \alpha_v V_v(v_i) + \beta_v, &
        \dot{V}_w(w_i) & \leq \alpha_w V_w(w_i) + \beta_w.
    \end{align}
    Using the comparison lemma, we conclude that $v_i$ and $w_i$ are forward-complete.

    For the path-following errors, we define a Lyapunov function candidate
    \begin{equation}
        V_b\left(\mat{p}_b^p\right) = \frac{1}{2} \left(\left(x_b^p\right)^2 + \left(y_b^p\right)^2 + \left(z_b^p\right)^2\right),
    \end{equation}
    and show that there exists a class-$\mathcal{K}_{\infty}$ function $\zeta_p$ such that
    \begin{equation}
        \dot{V}_p\left(\mat{p}_b^p\right) \leq V_p\left(\mat{p}_b^p\right) + \zeta_p\left(v_i, w_i, \tilde{\mat{X}}_2\right).
    \end{equation}
    Since all the arguments of $\zeta_p(\cdot)$ are forward complete, Corollary 2.11 of \cite{angeli_forward_1999} is satisfied, and the barycenter dynamics is forward complete, concluding the proof of Lemma~\ref{lemma_1}.
\end{proof}

\begin{lemma}
    \label{lemma_2}
    The underactuated sway and heave dynamics are bounded near the manifold $\left[\tilde{\mat{X}}_1\T, \tilde{\mat{X}}_2\T\right] = \mat{0}\T$ if $Y_v(u, u_c) < 0$, $Y_w(u, u_c) < 0$ and the curvature of the path satisfies
    \begin{align}
        \abs{\kappa(s)} &< \frac{N}{2}\abs{\frac{Y_w(u, u_c)}{X_w(u, u_c)}}, &
        \abs{\iota(s)} &< \frac{N}{2}\abs{\frac{Y_v(u, u_c)}{X_v(u, u_c)}}, \label{eq:nsb_5dof_curvature_limit}
    \end{align}
    for all $u > 0$ and $u_c \in [-\norm{\ocean}, \norm{\ocean}]$.
\end{lemma}

\begin{proof}
    The complete proof is given in Appendix~\ref{app:5dof_nsb_lemma_2}.
    Here, we only present a sketch of the proof.

    Consider the derivatives of the Lyapunov function candidates $V_v, V_w$ from \eqref{eq:nsb_5dof_underactuated_LFC}.
    Substituting $\tilde{\mat{X}}_1 = \mat{0}, \tilde{\mat{X}}_2 = \mat{0}$, we get the following inequalities
    \begin{align} 
            \dot{V}_v(v_i) &\leq \left(X_v\left(u_{d,i}, u_c\right)\frac{2}{N}\abs{\iota(\xi)} + Y_v\left(u_{d,i}, u_c\right)\right)v_i^2 + F_v(v_i), \\
            \dot{V}_w(w_i) &\leq \left(X_w\left(u_{d,i}, u_c\right)\frac{2}{N}\abs{\kappa(\xi)} + Y_w\left(u_{d,i}, u_c\right)\right)w_i^2 + F_w(w_i),
    \end{align}

    \noindent where $F_v$ and $F_w$ grow at most linearly with $v_i$ and $w_i$, respectively.
    Then, we conclude that for a sufficiently large $v_i, w_i$, the quadratic terms will dominate the linear terms.
    Therefore, the underactuated dynamics are bounded if the quadratic terms are negative, which is equivalent to condition \eqref{eq:nsb_5dof_curvature_limit}.
\end{proof}

\begin{lemma}
    \label{lemma_3}
    The underactuated sway and heave dynamics are bounded near the manifold $\tilde{\mat{X}}_2 = \mat{0}$, independently of $\tilde{\mat{X}}_1$ if the assumptions in Lemma~\ref{lemma_2} are satisfied and the constant term $\Delta_0$ in the lookahead distance \eqref{eq:nsb_5dof_delta} is chosen so that
    \begin{equation}
        \Delta_0 > \max\left\{\frac{3}{N\abs{\frac{Y_v\left(u, u_c\right)}{X_v\left(u, u_c\right)}} - 2\abs{\iota(s)}}, \frac{3}{N\abs{\frac{Y_w\left(u, u_c\right)}{X_w\left(u, u_c\right)}} - 2\abs{\kappa(s)}}\right\},
        \label{eq:nsb_5dof_lookahead_limit}
    \end{equation}
    for all $u > 0$ and $u_c \in [-\norm{\ocean}, \norm{\ocean}]$.
\end{lemma}

\begin{proof}
    The complete proof is given in Appendix~\ref{app:5dof_nsb_lemma_3}.
    Here, we only present a sketch of the proof.

    Once again, we consider the derivatives of the Lyapunov function candidates {$V_v, V_w$} from \eqref{eq:nsb_5dof_underactuated_LFC}.
    Substituting $\tilde{\mat{X}}_2 = \mat{0}$, we get the following inequalities
    \begin{align} 
            \dot{V}_v(v_i) &\leq \!\Bigg(\!X_{v\!}\left(u_{d,i}, u_c\right)\left(\frac{2}{N}\abs{\iota(\xi)} + \frac{3}{N\,\Delta(\mat{p}_b^p)}\right) + Y_v\left(u_{d,i}, u_c\right)\!\Bigg)v_i^2 + F_v(v_i), \\
            \dot{V}_w(w_i) &\leq \!\Bigg(\!X_{w\!}\left(u_{d,i}, u_c\right)\left(\frac{2}{N}\abs{\kappa(\xi)} + \frac{3}{N\,\Delta(\mat{p}_b^p)}\right) + Y_w\left(u_{d,i}, u_c\right)\!\Bigg)w_i^2 + F_w(w_i),
    \end{align}\vspace{-3mm}

    \noindent where $F_v$ and $F_w$ grow at most linearly with $v_i$ and $w_i$, respectively.
    Using the same arguments as in the proof of Lemma~\ref{lemma_2}, we conclude that the underactuated dynamics are bounded if both \eqref{eq:nsb_5dof_curvature_limit} and \eqref{eq:nsb_5dof_lookahead_limit} hold.
\end{proof}

\begin{theorem}
    The origin $\left[\tilde{\mat{X}}_1\T, \tilde{\mat{X}}_2\T\right] = \mat{0}\T$ of the system described by \eqref{eq:nsb_5dof_nominal}, \eqref{eq:nsb_5dof_perturbing} is a USGES equilibrium point if the conditions of Lemmas~\ref{lemma_2} and \ref{lemma_3} hold and the maximum pitch angle of the path satisfies
    \begin{equation}
        \theta_{p, {\rm max}} = \max_{s \in \mathbb{R}} \abs{\theta_p(s)} < \frac{\pi}{4}.
        \label{eq:nsb_5dof_theta_max}
    \end{equation}
    Moreover, the ocean current estimate errors \eqref{eq:nsb_5dof_estimates} and the underactuated sway and heave dynamics \eqref{eq:nsb_5dof_v_dot}, \eqref{eq:nsb_5dof_w_dot} are bounded.
\end{theorem}

\begin{rmk*}
Condition \eqref{eq:nsb_5dof_theta_max} is needed to ensure that $\abs{\gamma_{\rm LOS}} < \pi/2$.
Indeed, from \eqref{eq:nsb_5dof_v_LOS}, the largest possible LOS reference angle is
\begin{equation}
    \begin{split}
        \gamma_{\rm LOS, max} &= \theta_{p, {\rm max}} + \lim_{z_b^p \rightarrow \infty} \arctan\left(\scale[1]{\frac{z_b^p}{\sqrt{\Delta_0^2 + \left(z_b^p\right)^2}}}\right) \\
        &= \theta_{p, {\rm max}} + \frac{\pi}{4}.
    \end{split}
\end{equation}
With \eqref{eq:nsb_5dof_theta_max} satisfied, the cosine of $\gamma_{\rm LOS}$ is always positive. We will use this fact in the proof.
\end{rmk*}

\begin{proof}
The proof follows along the lines of \cite{eek_formation_2021}, but is extended to an arbitrary number of 5DOF vehicles.
We will also use the results of \cite{pettersen_lyapunov_2017} to prove that the system is USGES.

In Lemmas~\ref{lemma_1}--\ref{lemma_3}, we have shown that the closed-loop system is forward complete and the underactuated sway and heave dynamics are bounded near the manifold $\tilde{\mat{X}}_2 = \mat{0}$.
Since \eqref{eq:nsb_5dof_perturbing} is UGES \cite{moe_LOS_2016}, we can conclude that there exists a finite time $T > t_0$ such that the solutions of \eqref{eq:nsb_5dof_perturbing} will be sufficiently close to $\tilde{\mat{X}}_2 = \mat{0}$ to guarantee boundedness of $v_i$ and $w_i$.
Having established that the underactuated dynamics are bounded, we will now utilize cascaded theory to analyze the cascade \eqref{eq:nsb_5dof_nominal}, \eqref{eq:nsb_5dof_perturbing}, where \eqref{eq:nsb_5dof_perturbing} perturbs the nominal dynamics \eqref{eq:nsb_5dof_nominal} through the terms $G_y(\cdot)$ and $G_z(\cdot)$.

Now, consider the nominal dynamics of $\tilde{\mat{X}}_1$ (\emph{i.e.,} \eqref{eq:nsb_5dof_nominal} without the perturbing terms $G_y$ and $G_z$), and a Lyapunov function candidate
\begin{equation}
    V(\tilde{\mat{X}}_1) = \frac{1}{2} \tilde{\mat{X}}_1\T\,\tilde{\mat{X}}_1 = \frac{1}{2} \left((x_b^p)^2 + (y_b^p)^2 + (z_b^p)^2\right), \label{eq:nsb_5dof_LCF}
\end{equation}
whose derivative along the trajectories of \eqref{eq:nsb_5dof_nominal} is
\begin{subequations}
    \begin{align}
        \dot{V}(\tilde{\mat{X}}_1) &= - \tilde{\mat{X}}_1\T\,\mat{Q}\,\tilde{\mat{X}}_1, &
        \mat{Q} &= {\rm diag}(q_1, q_2, q_3), \\
        q_1 &= {\frac{k_{s}}{\sqrt{1+\left(x_b^p\right)^2}}}, &
        q_2 &= {{\frac{\frac{1}{N}\sum_{i=1}^N U_{d,i}\cos\left(\gamma_{\rm LOS}\right)}{\sqrt{\Delta\left(\mat{p}_b^p\right)^2 + \left(y_b^p\right)^2}}}}, \\
        &&q_3 &= {\frac{\frac{1}{N} \sum_{i=1}^N U_{d,i}}{\sqrt{\Delta\left(\mat{p}_b^p\right)^2 + \left(z_b^p\right)^2}}}. 
    \end{align}
\end{subequations}
Note that $\mat{Q}$ is positive definite, and the nominal system is thus UGAS.
Furthermore, note that the following inequality
\begin{subequations}
    \begin{align}
        \dot{V}(\tilde{\mat{X}}_1) &\leq - q_{\rm min} \norm{\tilde{\mat{X}}_1}^2, \\
        q_{\rm min} &= {\min \left\{ \frac{k_{s}}{\sqrt{1+r^2}}, \frac{\frac{1}{N} \sum_{i=1}^N U_{d,i}\cos\left(\gamma_{\rm LOS}\right)}{\sqrt{\Delta_0^2 + 4r^2}} \right\}},
    \end{align}
\end{subequations}
holds $\forall \tilde{\mat{X}}_1 \in \mathcal{B}_r$.
Thus, the conditions of \cite[Theorem 5]{pettersen_lyapunov_2017} are fulfilled with $k_1 = k_2 = 1/2$, $a = 2$, and $k_3 = q_{\rm min}$, and the nominal system is USGES.

As discussed in the proof of Lemma~\ref{lemma_1}, the perturbing system \eqref{eq:nsb_5dof_perturbing} is UGES, implying both UGAS and USGES.
Furthermore, it is straightforward to show that the following holds for the Lyapunov function \eqref{eq:nsb_5dof_LCF}
\begin{align}
    \left\| \frac{\partial V}{\partial \tilde{\mat{X}}_1} \right\| \, \left\|\tilde{\mat{X}}_1\right\| &= \left\|\tilde{\mat{X}}_1\right\|^2 = 2\,V\left(\tilde{\mat{X}}_1\right), \quad \forall \tilde{\mat{X}}_1, \\
    \left\| \frac{\partial V}{\partial \tilde{\mat{X}}_1} \right\| &= \left\|\tilde{\mat{X}}_1\right\| \leq \mu, \quad \quad \forall \left\|\tilde{\mat{X}}_1\right\| \leq \mu.
\end{align}
Therefore, \cite[Assumption 1]{pettersen_lyapunov_2017} is satisfied with $c_1 = 2$ and $c_2 = \mu$ for any $\mu > 0$.

Finally, \cite[Assumption 2]{pettersen_lyapunov_2017} must be investigated.
From \eqref{eq:nsb_5dof_G_y}, \eqref{eq:nsb_5dof_G_z}, it can be shown that for both perturbing terms there exist positive functions $\zeta_{y,1}(\cdot)$, $\zeta_{y,2}(\cdot)$, $\zeta_{z,1}(\cdot)$, $\zeta_{z,2}(\cdot)$, such that
\begin{align}
    \left|G_y(\cdot)\right| &\leq \zeta_{y,1}\left(\norm{\tilde{\mat{X}}_2}\right) + \zeta_{y,2}\left(\norm{\tilde{\mat{X}}_2}\right)\norm{\tilde{\mat{X}}_1}, \\
    \left|G_z(\cdot)\right| &\leq \zeta_{z,1}\left(\norm{\tilde{\mat{X}}_2}\right) + \zeta_{z,2}\left(\norm{\tilde{\mat{X}}_2}\right)\norm{\tilde{\mat{X}}_1}.
\end{align}
Therefore, all conditions of \cite[Proposition 9]{pettersen_lyapunov_2017} are satisfied, and the closed-loop system is USGES.    
\end{proof}

\section{Simulation Results}
\label{sec:nsb_5dof_simulations}

\pgfplotsset{table/search path={figures/nsb_5dof/data}}
\begin{figure}[p]
    \centering
    %\resizebox{0.99\textwidth}{!}{% This file was created by matlab2tikz.
%
%The latest updates can be retrieved from
%  http://www.mathworks.com/matlabcentral/fileexchange/22022-matlab2tikz-matlab2tikz
%where you can also make suggestions and rate matlab2tikz.
%
\definecolor{mycolor1}{rgb}{0.00000,0.44700,0.74100}%
\definecolor{mycolor2}{rgb}{0.85000,0.32500,0.09800}%
\definecolor{mycolor3}{rgb}{0.92900,0.69400,0.12500}%
%
\begin{tikzpicture}

\pgfplotsset{every tick label/.append style={font=\small}}

\begin{axis}[%
width=0.375\textwidth,
height=25.399mm,
at={(0mm,39mm)},
scale only axis,
xmin=0,
xmax=150,
xtick={0,50,100,150},
ymin=-5.5,
ymax=6.9,
ylabel style={font=\color{white!15!black},yshift=-3.5mm},
ylabel={Error [m]},
xlabel style={font=\color{white!15!black},yshift=2mm},
xlabel={$t$ [s]},
axis background/.style={fill=white},
axis x line*=bottom,
axis y line*=left,
xmajorgrids,
ymajorgrids,
title style={font=\bfseries,yshift=-0.5mm},
title={Path-following error},
legend columns=3
]
\addplot [color=mycolor1, line width=1.0pt]
  table[]{simout-1.tsv};
  \addlegendentry{$x_b^p$}

\addplot [color=mycolor2, line width=1.0pt]
  table[]{simout-2.tsv};
  \addlegendentry{$y_b^p$}

\addplot [color=mycolor3, line width=1.0pt]
  table[]{simout-3.tsv};
  \addlegendentry{$z_b^p$}  
\end{axis}

\begin{axis}[%
  width=0.375\textwidth,
  height=25.699mm,
  at={(0mm,0mm)},
  scale only axis,
  xmin=0,
  xmax=150,
  xlabel style={font=\color{white!15!black}, yshift=2mm},
  xlabel={$t$ [s]},
  ymin=0,
  ymax=20.62,
  ylabel style={font=\color{white!15!black}, yshift=-1.5mm},
  ylabel={Distance [m]},
  axis background/.style={fill=white},
  title style={font=\bfseries, yshift=-1.5mm},
  title={Inter-vehicle distance},
  axis x line*=bottom,
  axis y line*=left,
  xmajorgrids,
  ymajorgrids,
  legend style={at={(0.97,0.03)}, anchor=south east, legend cell align=left, align=left, draw=white!15!black},
  legend columns=2
  ]
  \addplot [color=mycolor1, line width=1.0pt]
    table[]{simout-16.tsv};
  \addlegendentry{$d_{1,2}$~}
  
  \addplot [color=mycolor2, line width=1.0pt]
    table[]{simout-17.tsv};
  \addlegendentry{$d_{1,3}$}
  
  \addplot [color=mycolor3, line width=1.0pt]
    table[]{simout-18.tsv};
  \addlegendentry{$d_{2,3}$~}
  
  \addplot [color=black, dashed, line width=1.0pt]
    table[]{simout-19.tsv};
  \addlegendentry{$d_{\rm COLAV}$}
  
\end{axis}

\begin{axis}[%
width=0.375\textwidth,
height=19.539mm,
at={(0.5\textwidth,45.439mm)},
scale only axis,
xmin=0,
xmax=150,
xtick={0,50,100,150},
xticklabels={\empty},
ymin=-5.32018044501408,
ymax=2.72629468835089,
ylabel style={font=\color{white!15!black}, yshift=-2mm},
ylabel={$\tilde{\sigma}_x$ [m]},
axis background/.style={fill=white},
title style={font=\bfseries,yshift=-0.5mm},
title={Formation-keeping error},
axis x line*=bottom,
axis y line*=left,
xmajorgrids,
ymajorgrids,
]
\addplot[area legend, dashed, draw=black, fill=gray, fill opacity=0.5, forget plot]
table[] {simout-7.tsv}--cycle;

\addplot [color=mycolor1, line width=1.0pt, forget plot]
  table[]{simout-4.tsv};

\addplot [color=mycolor2, line width=1.0pt, forget plot]
  table[]{simout-5.tsv};

\addplot [color=mycolor3, line width=1.0pt, forget plot]
  table[]{simout-6.tsv};
\end{axis}

\begin{axis}[%
width=0.375\textwidth,
height=19.539mm,
at={(0.5\textwidth,22.72mm)},
scale only axis,
xmin=0,
xmax=150,
xtick={0,50,100,150},
xticklabels={\empty},
ymin=-20,
ymax=20,
ylabel style={font=\color{white!15!black}, xshift=2mm, yshift=-3mm},
ylabel={$\tilde{\sigma}_y$ [m]},
axis background/.style={fill=white},
axis x line*=bottom,
axis y line*=left,
xmajorgrids,
ymajorgrids,
legend style={at={(0.97,1.35)}, anchor=north east, legend cell align=left, align=left, draw=white!15!black}
]
\addplot[area legend, dashed, draw=black, fill=gray, fill opacity=0.5, forget plot]
table[] {simout-11.tsv}--cycle;

\addplot [color=mycolor1, line width=1.0pt]
  table[]{simout-8.tsv};
  \addlegendentry{Vehicle 1}

\addplot [color=mycolor2, line width=1.0pt]
  table[]{simout-9.tsv};
  \addlegendentry{Vehicle 2}

\addplot [color=mycolor3, line width=1.0pt]
  table[]{simout-10.tsv};
  \addlegendentry{Vehicle 3}

\end{axis}

\begin{axis}[%
width=0.375\textwidth,
height=19.539mm,
at={(0.5\textwidth,0mm)},
scale only axis,
xmin=0,
xmax=150,
xlabel style={font=\color{white!15!black},yshift=2mm},
xlabel={$t$ [s]},
ymin=-10,
ymax=20,
ylabel style={font=\color{white!15!black}, yshift=-3mm},
ylabel={$\tilde{\sigma}_z$ [m]},
axis background/.style={fill=white},
axis x line*=bottom,
axis y line*=left,
xmajorgrids,
ymajorgrids,
]
\addplot[area legend, dashed, draw=black, fill=gray, fill opacity=0.5, forget plot]
table[] {simout-15.tsv}--cycle;

\addplot [color=mycolor1, line width=1.0pt, forget plot]
  table[]{simout-12.tsv};
\addplot [color=mycolor2, line width=1.0pt, forget plot]
  table[]{simout-13.tsv};
\addplot [color=mycolor3, line width=1.0pt, forget plot]
  table[]{simout-14.tsv};

\end{axis}


\end{tikzpicture}%}
    % This file was created by matlab2tikz.
%
%The latest updates can be retrieved from
%  http://www.mathworks.com/matlabcentral/fileexchange/22022-matlab2tikz-matlab2tikz
%where you can also make suggestions and rate matlab2tikz.
%
\definecolor{mycolor1}{rgb}{0.00000,0.44700,0.74100}%
\definecolor{mycolor2}{rgb}{0.85000,0.32500,0.09800}%
\definecolor{mycolor3}{rgb}{0.92900,0.69400,0.12500}%
%
\begin{tikzpicture}

\pgfplotsset{every tick label/.append style={font=\small}}

\begin{axis}[%
width=0.375\textwidth,
height=25.399mm,
at={(0mm,39mm)},
scale only axis,
xmin=0,
xmax=150,
xtick={0,50,100,150},
ymin=-5.5,
ymax=6.9,
ylabel style={font=\color{white!15!black},yshift=-3.5mm},
ylabel={Error [m]},
xlabel style={font=\color{white!15!black},yshift=2mm},
xlabel={$t$ [s]},
axis background/.style={fill=white},
axis x line*=bottom,
axis y line*=left,
xmajorgrids,
ymajorgrids,
title style={font=\bfseries,yshift=-0.5mm},
title={Path-following error},
legend columns=3
]
\addplot [color=mycolor1, line width=1.0pt]
  table[]{simout-1.tsv};
  \addlegendentry{$x_b^p$}

\addplot [color=mycolor2, line width=1.0pt]
  table[]{simout-2.tsv};
  \addlegendentry{$y_b^p$}

\addplot [color=mycolor3, line width=1.0pt]
  table[]{simout-3.tsv};
  \addlegendentry{$z_b^p$}  
\end{axis}

\begin{axis}[%
  width=0.375\textwidth,
  height=25.699mm,
  at={(0mm,0mm)},
  scale only axis,
  xmin=0,
  xmax=150,
  xlabel style={font=\color{white!15!black}, yshift=2mm},
  xlabel={$t$ [s]},
  ymin=0,
  ymax=20.62,
  ylabel style={font=\color{white!15!black}, yshift=-1.5mm},
  ylabel={Distance [m]},
  axis background/.style={fill=white},
  title style={font=\bfseries, yshift=-1.5mm},
  title={Inter-vehicle distance},
  axis x line*=bottom,
  axis y line*=left,
  xmajorgrids,
  ymajorgrids,
  legend style={at={(0.97,0.03)}, anchor=south east, legend cell align=left, align=left, draw=white!15!black},
  legend columns=2
  ]
  \addplot [color=mycolor1, line width=1.0pt]
    table[]{simout-16.tsv};
  \addlegendentry{$d_{1,2}$~}
  
  \addplot [color=mycolor2, line width=1.0pt]
    table[]{simout-17.tsv};
  \addlegendentry{$d_{1,3}$}
  
  \addplot [color=mycolor3, line width=1.0pt]
    table[]{simout-18.tsv};
  \addlegendentry{$d_{2,3}$~}
  
  \addplot [color=black, dashed, line width=1.0pt]
    table[]{simout-19.tsv};
  \addlegendentry{$d_{\rm COLAV}$}
  
\end{axis}

\begin{axis}[%
width=0.375\textwidth,
height=19.539mm,
at={(0.5\textwidth,45.439mm)},
scale only axis,
xmin=0,
xmax=150,
xtick={0,50,100,150},
xticklabels={\empty},
ymin=-5.32018044501408,
ymax=2.72629468835089,
ylabel style={font=\color{white!15!black}, yshift=-2mm},
ylabel={$\tilde{\sigma}_x$ [m]},
axis background/.style={fill=white},
title style={font=\bfseries,yshift=-0.5mm},
title={Formation-keeping error},
axis x line*=bottom,
axis y line*=left,
xmajorgrids,
ymajorgrids,
]
\addplot[area legend, dashed, draw=black, fill=gray, fill opacity=0.5, forget plot]
table[] {simout-7.tsv}--cycle;

\addplot [color=mycolor1, line width=1.0pt, forget plot]
  table[]{simout-4.tsv};

\addplot [color=mycolor2, line width=1.0pt, forget plot]
  table[]{simout-5.tsv};

\addplot [color=mycolor3, line width=1.0pt, forget plot]
  table[]{simout-6.tsv};
\end{axis}

\begin{axis}[%
width=0.375\textwidth,
height=19.539mm,
at={(0.5\textwidth,22.72mm)},
scale only axis,
xmin=0,
xmax=150,
xtick={0,50,100,150},
xticklabels={\empty},
ymin=-20,
ymax=20,
ylabel style={font=\color{white!15!black}, xshift=2mm, yshift=-3mm},
ylabel={$\tilde{\sigma}_y$ [m]},
axis background/.style={fill=white},
axis x line*=bottom,
axis y line*=left,
xmajorgrids,
ymajorgrids,
legend style={at={(0.97,1.35)}, anchor=north east, legend cell align=left, align=left, draw=white!15!black}
]
\addplot[area legend, dashed, draw=black, fill=gray, fill opacity=0.5, forget plot]
table[] {simout-11.tsv}--cycle;

\addplot [color=mycolor1, line width=1.0pt]
  table[]{simout-8.tsv};
  \addlegendentry{Vehicle 1}

\addplot [color=mycolor2, line width=1.0pt]
  table[]{simout-9.tsv};
  \addlegendentry{Vehicle 2}

\addplot [color=mycolor3, line width=1.0pt]
  table[]{simout-10.tsv};
  \addlegendentry{Vehicle 3}

\end{axis}

\begin{axis}[%
width=0.375\textwidth,
height=19.539mm,
at={(0.5\textwidth,0mm)},
scale only axis,
xmin=0,
xmax=150,
xlabel style={font=\color{white!15!black},yshift=2mm},
xlabel={$t$ [s]},
ymin=-10,
ymax=20,
ylabel style={font=\color{white!15!black}, yshift=-3mm},
ylabel={$\tilde{\sigma}_z$ [m]},
axis background/.style={fill=white},
axis x line*=bottom,
axis y line*=left,
xmajorgrids,
ymajorgrids,
]
\addplot[area legend, dashed, draw=black, fill=gray, fill opacity=0.5, forget plot]
table[] {simout-15.tsv}--cycle;

\addplot [color=mycolor1, line width=1.0pt, forget plot]
  table[]{simout-12.tsv};
\addplot [color=mycolor2, line width=1.0pt, forget plot]
  table[]{simout-13.tsv};
\addplot [color=mycolor3, line width=1.0pt, forget plot]
  table[]{simout-14.tsv};

\end{axis}


\end{tikzpicture}%
    \vspace{-1.5mm}
    \caption{Simulation results. The top-left plot shows the $x$-, $y$- and $z$-components of the path-following error $\mathbf{p}_b^p$, as defined in \eqref{eq:nsb_5dof_barycenter}. The bottom-left plot shows the distance between the vehicles ($d_{i,j} = \|\mathbf{p}_i - \mathbf{p}_j\|$). The plots on the right show the $x$-, $y$- and $z$-components of the formation-keeping error $\tilde{\bs{\sigma}} = \bs{\sigma}_2 - \bs{\sigma}_{d,2}$ with $\bs{\sigma}_2$ given by \eqref{eq:nsb_5dof_sigma_2} and $\bs{\sigma}_{d,2}$ given by \eqref{eq:nsb_5dof_sigma_d_2}. The grey rectangles mark the intervals when the COLAV task is active.}
    \label{fig:nsb_5dof_results}
    \vspace{-5mm}
\end{figure}
\begin{figure}[p]
    \centering
    \def\svgwidth{.8\textwidth}
    \import{figures/nsb_5dof}{trajectory_3d.pdf_tex}
    \caption{3D trajectory of the vehicles. The markers represent the position of the vehicles at times $t = 0, 25, 50, \ldots, 150$ seconds. Markers with corresponding times are connected by dotted lines to better illustrate the resulting formation.}
    \label{fig:nsb_5dof_trajectory}
\end{figure}

In this section, we present the results of a numerical simulation of three \glspl{lauv} \cite{sousa_LAUV_2012}.
The parameters of the simulation are summarized in Table \ref{tab:nsb_5dof_params}.
The barycenter should follow a spiral path given by
\begin{equation}
    \mat{p}_p(s) = \left[s, a\,\cos(\omega\,s), b\,\sin(\omega\,s)\right]\T.
\end{equation}
The maximum curvature of this path is
\begin{align}
    \max_{s \in \mathbb{R}} \abs{\kappa(s)} &= \frac{b\,\omega ^2}{\sqrt{a^2\,\omega ^2+1}}, &
    \max_{s \in \mathbb{R}} \abs{\iota(s)} &= a\,\omega ^2,
\end{align}
while the smallest absolute values of $Y_v / X_v$ and $Y_w / X_w$ for the \gls{lauv} model are approximately $0.26$.
Consequently, the path is chosen such that the maximum curvature is
\begin{align}
    \max_{s \in \mathbb{R}} \abs{\kappa(s)} &= 0.013, &
    \max_{s \in \mathbb{R}} \abs{\iota(s)} &= 0.040,
\end{align}
and \eqref{eq:nsb_5dof_curvature_limit} is satisfied.
From \eqref{eq:nsb_5dof_lookahead_limit}, the lookahead distance must then satisfy $\Delta_0 > 4.29$.
We choose $\Delta_0 = 5$, since smaller distances guarantee faster convergence.

The very minimum relative distance to avoid collision is the length of the \gls{lauv}, \emph{i.e.} $2.4$ m.
For additional safety, we design the COLAV task with $d_{\rm min} = 5$ m.
To add a security zone during transients, $d_{\rm COLAV}$ is chosen to be $10$ m.

The desired formation is an isosceles triangle parallel to the $yz$ plane.
Specifically, the desired positions of the three vehicles are
\begin{align}
    \mathbf{p}_{f,1}^f &= \begin{bmatrix} 0 \\ 10 \\ 5\end{bmatrix}, &
    \mathbf{p}_{f,2}^f &= \begin{bmatrix} 0 \\ -10 \\ 5\end{bmatrix}, &
    \mathbf{p}_{f,3}^f &= \begin{bmatrix} 0 \\ 0 \\ -10\end{bmatrix}.
\end{align}

The gains of the low-level control systems \eqref{eq:nsb_5dof_t_u},\eqref{eq:nsb_5dof_t_q},\eqref{eq:nsb_5dof_t_r} are chosen such that the settling time is approximately 10 seconds.
The gains of the pitch and yaw PD controllers are chosen such that the closed-loop system is critically damped.

The results of the numerical simulation are shown in Figures \ref{fig:nsb_5dof_results} and \ref{fig:nsb_5dof_trajectory}.
The vehicles start in an inverted triangular formation.
The COLAV task is briefly activated, and the distance between the vehicles drops to approximately 8 meters during the transient.
Eventually, the vehicles resolve the situation and continue to converge to the desired path and formation.

Note that while the COLAV task is active, the formation-keeping error is diverging.
After resolving the situation, the formation-keeping error converges to zero exponentially.
The rate of convergence is given by the formation-keeping gain $\bs{\Lambda}_2$.

The path-following error seems to converge linearly at first, and then exponentially as the error gets smaller.
This phenomenon is caused by the LOS guidance law \eqref{eq:nsb_5dof_v_LOS}, \emph{cf.} \cite{fossen_uniform_2014}, and the path parameter update law \eqref{eq:nsb_5dof_path_update}.
The inverse tan in \eqref{eq:nsb_5dof_v_LOS} and the last term in \eqref{eq:nsb_5dof_path_update} act as a saturation, slowing the convergence for large errors.
The rate of convergence of the along-track error ($x_b^p$) is given by the path parameter update gain $k_{s}$, while the rate of convergence of the cross-track errors ($y_b^p, z_b^p$) is given by the lookahead distance $\Delta_0$.
The path-following error seems to increase at $t = \SI{150}{\second}$.
This increase is probably caused by low-level tracking errors.
To avoid chattering, the sign functions in the low-level sliding-mode controllers \eqref{eq:nsb_5dof_t_u}--\eqref{eq:nsb_5dof_t_r} are approximated using hyperbolic tan.
These approximations result in a non-zero steady-state error.

\begin{table}[t]
    \centering
    \begin{tabular}[t]{r|l}
        {\bf Parameter} & {\bf Value} \\ \hline
        $k_u$ & $0.05$ \\
        $k_c$ & $0.1$ \\
        $k_{\theta}, k_{\psi}$ & $0.0625$ \\
        $k_q, k_r$ & $0.25$ \\
        $k_d$ & $0.1$ \\
        $\lambda_q, \lambda_r$ & $0.75$ \\
        $c_u$ & $5$ \\
        $c_q, c_r$ & $1$ \\
        $\bs{\Lambda}_1$ & $\mat{I}$ \\
        $\bs{\Lambda}_2$ & $0.05\,\mat{I}$
    \end{tabular}
    \begin{tabular}[t]{r|l}
        {\bf Parameter} & {\bf Value} \\ \hline
        $\ocean$ & $\left[0, 0.25, 0.05\right]\T$ \\
        $\Delta_0$ & $5$ \\
        $d_{\rm COLAV}$ & $10$ \\
        $U_{\rm LOS}$ & $1$ \\
        $k_{s}$ & $1$ \\
        $\mat{p}_0$ & $\mat{0}_3$ \\
        $a$ & $40$ \\
        $b$ & $20$ \\
        $\omega$ & $\pi / 100$
    \end{tabular}
    \caption{Simulation parameters}
    \label{tab:nsb_5dof_params}
\end{table}

%\addtolength{\textheight}{-5cm}
