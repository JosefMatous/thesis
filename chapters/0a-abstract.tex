\chapter*{Summary}

This thesis investigates various control algorithms for marine vehicles.
Most of the algorithms proposed in the thesis address the formation path-following problem for a fleet of underactuated \acrlongpl{auv}, although other types of vehicles, such as \acrlongpl{asv} and differential drive robots, and other types of control problems, such as collision avoidance, trajectory tracking, and path following, are also considered.
The thesis is divided into three parts.

In the first part, we develop a collision avoidance algorithm for overactuated vehicles.
The vehicles must reach a desired position while maintaining some minimum safety distance from each other.
To solve this problem, we propose an optimization-based control allocation scheme augmented with \acrlongpl{cbf}.
Control allocation is a collection of methods for finding an actuator configuration that satisfies a given goal (\emph{e.g.}, reaching a desired position).
\Acrlongpl{cbf} allows us to enforce constraints on dynamical systems (\emph{e.g.}, keeping a minimum safety distance).
By combining control allocation with \acrlongpl{cbf}, we can create a controller that satisfies a given goal while avoiding collisions.
The proposed controller is tested in numerical simulations on two types of \acrlongpl{asv}: the milliAmpere ferry, and the Inocean Cat I drillship.

The second part deals with the formation path-following problem.
We solve the problem using the \acrlong{nsb} method.
This method allows us to decompose the problem into several tasks.
Then, by combining these tasks in a hierarchic manner, we can achieve the desired behavior.
To solve the formation path-following problem, we define three tasks: collision avoidance, formation keeping, and path following.
In this thesis, we propose three \acrlong{nsb} algorithms for the formation path-following problem.

The first algorithm uses a model of an \acrlong{auv} with five \acrlongpl{dof}.
Using Lyapunov analysis, we show that the path-following task is \acrlongpl{usges}.
Numerical simulations then validate this result.

The second algorithm uses a six degree-of-freedom model.
Compared to the previous method, this algorithm does not suffer from numerical singularities.
This algorithm also contains additional tasks, namely obstacle avoidance and depth limiting.
Moreover, we prove that both the path-following and the formation-keeping tasks are \acrlongpl{usges}.
These theoretical results are then validated in numerical simulations.

One common issue of both \acrlong{nsb} algorithms is that they are centralized.
In many applications, centralized algorithms are difficult to implement, as they require one central node or agent that can communicate and coordinate with other agents in real time.
To solve this issue, the third algorithm combines a \acrlong{nsb} algorithm with consensus, resulting in a fully distributed controller.
We propose two types of consensus algorithm.
First, we propose a continuous-time consensus algorithm and prove its stability using Lyapunov analysis.
Then, we present a modified discrete-time version of the algorithm based on event-triggered control.
The effectiveness of both the continuous- and discrete-time algorithms is demonstrated in numerical simulations.
Furthermore, the discrete-time version is also tested in field experiments.

The third part of the thesis extends the hand position approach to underactuated underwater vehicles moving in three dimensions.
Hand position is a point located a given distance in front of the neutral point along the $x$-axis of the vehicle.
By treating this point as the new output of the system, we can use input-output feedback linearization to transform the underactuated highly nonlinear vehicle model into a system with linear external dynamics and nonlinear internal dynamics.
We analyze the closed-loop behavior of a generic hand position-based controller and present four applications of the hand position approach.

First, we use this approach to solve the trajectory-tracking and path-following problems.
We propose simple PID-based controllers to solve these problems and show that using these controllers renders the external states \acrlongpl{ges}, while the internal states remain bounded.
The theoretical results are validated in numerical simulations as well as field experiments.

Next, we present a spline-based \acrlong{mpc} method for solving the formation path-following problem.
The proposed method is not restricted to the hand position approach only.
In fact, the method is applicable to any vehicle with a differentially flat model.
To demonstrate this, we present two case studies: underwater vehicles with the hand position controller, and differential drive robots.

Next, we use the hand position concept to solve the tracking-in-formation problem for a fleet of \acrlongpl{auv}.
The proposed method combines consensus with \acrlongpl{blf}, allowing the fleet to reach the desired formation while avoiding collisions and maintaining connectivity.
We show that the closed-loop system is almost-everywhere uniformly asymptotically stable and that the output error dynamics converge to the origin exponentially fast, while satisfying the constraints.
The theoretical results are verified in numerical simulations.

Finally, we combine the hand position approach with \acrlong{nsb} control.
Specifically, we extend the \acrlong{nsb} algorithm, which was originally developed for first-order kinematic systems, to second-order systems.
Similarly to our previous work, we then design the path-following, formation-keeping, and collision-avoidance tasks, so that the fleet can follow a given path in a formation while avoiding collisions.
We prove the stability of the control scheme using Lyapunov analysis and verify its effectiveness in simulations. 
