\chapter{Background}
\label{chap:background}

\setlength{\epigraphwidth}{0.7\textwidth}
\epigraph{ \it
    The latter consisted simply of six hydrocoptic marzlevanes, so fitted to the ambifacient lunar waneshaft that side fumbling was effectively prevented.
}{--- John Hellins Quick, ``The turbo-encabulator in industry,'' Students' Quarterly Journal, 1944.}

This chapter presents some of the background theory that is used in the thesis.
The theory presented here is relevant to multiple chapters throughout the thesis.
In Section~\ref{sec:model}, we present a control-oriented model of marine vehicles.
Section~\ref{sec:background_paths} presents the theory behind paths, that is then used in Section~\ref{sec:background_formation_path_following} to define the formation path-following problem.
In Section~\ref{sec:background_LOS}, we then present the line-of-sight guidance algorithm as a method for solving the path-following problem.

\section{Mathematical Models of Marine Vehicles}
\label{sec:model}

This section presents control-oriented models of \glspl{asv} and \glspl{auv}.

\subsubsection*{State variables and degrees of freedom}
Marine vehicles are typically modeled as rigid bodies.
A rigid body moving in three-dimensional space has six \glspl{dof}, three for position and three for orientation.

The position of a marine vehicle is commonly expressed in a local \gls{ned} coordinate frame.
Although the \gls{ned} frame is not inertial, it is often used as an approximation of an inertial coordinate frame for short-term and short-distance missions, since the effect of Earth's rotation on the vehicles is negligible.
In general, we will denote the position of the vehicles as $\mat{p} = \inlinevector{x, y, z}$.

The orientation of a vehicle can be expressed using \emph{Euler angles}, $\bs{\Theta} = \inlinevector{\phi, \theta, \psi}$, where $\phi$ is the roll angle, $\theta$ is the pitch angle, and $\psi$ is the yaw angle.
The complete position and orientation vector of the vehicle is then given by $\bs{\eta}\T = \left[\mat{p}\T, \bs{\Theta}\T\right]$.

Euler angles can represent any orientation.
However, in some cases, this representation is not unique.
For example, the following two sets of Euler angles represent the same attitude
\begin{align}
    \bs{\Theta}_1 &= \inlinevector{\frac{\pi}{2}, \frac{\pi}{2}, 0}, &
    \bs{\Theta}_2 &= \inlinevector{0, \frac{\pi}{2}, -\frac{\pi}{2}}. &
\end{align}
At these attitudes, there exist mathematical singularities called \emph{gimbal locks} \cite{chaturvedi_rigid-body_2011}.
Furthermore, the use of Euler angles in control may lead to a phenomenon called \emph{unwinding} \cite{sanjay_topological_2000}, in which the vehicle performs an unnecessary rotation to reach the desired attitude.

The orientation of a vehicle can also be described using a \emph{rotation matrix}.
Rotation matrices are members of the special orthogonal group $SO(3)$.
Unlike Euler angles, rotation matrices do not suffer from singularities.
For a given set of Euler angles, the corresponding rotation matrix is given by \cite{fossen_handbook_2011}
\begin{equation}
    \mat{R}(\phi, \theta, \psi)
    =
    \begin{bmatrix} 
        {\rm c}_{\psi}\,{\rm c}_{\theta} & {\rm c}_{\psi}\,{\rm s}_{\phi}\,{\rm s}_{\theta}-{\rm c}_{\phi}\,{\rm s}_{\psi} & {\rm s}_{\phi}\,{\rm s}_{\psi}+{\rm c}_{\phi}\,{\rm c}_{\psi}\,{\rm s}_{\theta}\\ {\rm c}_{\theta}\,{\rm s}_{\psi} & {\rm c}_{\phi}\,{\rm c}_{\psi}+{\rm s}_{\phi}\,{\rm s}_{\psi}\,{\rm s}_{\theta} & {\rm c}_{\phi}\,{\rm s}_{\psi}\,{\rm s}_{\theta}-{\rm c}_{\psi}\,{\rm s}_{\phi}\\ -{\rm s}_{\theta} & {\rm c}_{\theta}\,{\rm s}_{\phi} & {\rm c}_{\phi}\,{\rm c}_{\theta} 
    \end{bmatrix},
    \label{eq:background_Rzyx}
\end{equation}
where ${\rm c}_x$ and ${\rm s}_x$ represent the cosine and sine of the corresponding angle.

Next, let us discuss the representation of velocities.
The velocities of the vehicle are expressed in the \emph{body-fixed} frame, a non-inertial coordinate frame attached to the vehicle, with the $x$-axis pointing towards the bow (front) side, the $y$-axis pointing to the starboard (right) side, and the $z$-axis pointing to the bottom side of the vehicle.
The \emph{linear velocities} of the vehicle $\mat{v} = \inlinevector{u, v, w}$ consist of the surge, sway, and heave velocities.
The \emph{angular velocities} of the vehicle $\bs{\omega} = \inlinevector{p, q, r}$ consist of the roll, pitch, and yaw rates.
The full velocity vector is then given by $\bs{\nu}\T = \left[\mat{v}\T, \bs{\omega}\T\right]$.

Finally, let us discuss simplified 3\gls{dof} and 5\gls{dof} models.
In the case of \glspl{asv} or \glspl{auv} moving in the horizontal plane, we often assume that the roll and pitch angles are zero, and the depth is constant.
Consequently, we can disregard the roll, pitch, and heave motion of the vehicle, and derive a simplified 3\gls{dof} model with $\bs{\eta} = \inlinevector{x, y, \psi}$ and $\bs{\nu} = \inlinevector{u, v, r}$.
In the case slender, torpedo shaped \glspl{auv}, the roll motion is assumed to be small and self-stabilizing by the design of the vehicle.
Consequently, we can disregard the roll motion and derive a simplified 5\gls{dof} model with $\bs{\eta} = \inlinevector{x, y, z, \theta, \psi}$ and $\bs{\nu} = \inlinevector{u, v, w, q, r}$.

\subsubsection*{Kinematics}
First, let us discuss the kinematics of the vehicles, starting with the 6\gls{dof} model.
The time-derivative of the position is
\begin{equation}
    \dot{\mat{p}} = \mat{R} \mat{v}.
    \label{eq:background_p_dot_6DOF}
\end{equation}
The time-derivative of Euler angles is given by \cite{fossen_handbook_2011}
\begin{align}
    \dot{\bs{\Theta}} &= \mat{T}(\bs{\Theta}) \bs{\omega}, &
    \mat{T}(\bs{\Theta}) &= 
    \begin{bmatrix}
        1 & {\rm s}_{\phi}{\rm t}_{\theta} & {\rm c}_{\phi}{\rm t}_{\theta} \\ 0 & {\rm c}_{\phi} & -{\rm s}_{\phi} \\ 0 & {\rm s}_{\phi}/{\rm c}_{\theta} & {\rm c}_{\phi}/{\rm c}_{\theta}
    \end{bmatrix},
    \label{eq:background_theta_dot_6DOF}
\end{align}
where ${\rm t}_{\theta} = \tan(\theta)$.
Due to the aforementioned singularities, $\dot{\bs{\Theta}}$ is not defined for $\theta = \pm \pi/2$.
The time-derivative of a rotation matrix is given by
\begin{align}
    \dot{\mat{R}} &= \mat{R} \mat{S}(\omega), &
    \mat{S}(\omega) &=
    \begin{bmatrix}
        0 & -r & q\\ r & 0 & -p\\ -q & p & 0
    \end{bmatrix}.
\end{align}

To derive the kinematics of the 5\gls{dof} model, we simply substitute $\phi = 0$ and $p = 0$ into \eqref{eq:background_p_dot_6DOF} and \eqref{eq:background_theta_dot_6DOF}
\begin{subequations}
    \begin{align}
        \dot{\mat{p}} &= \mat{R}(0, \theta, \psi) \mat{v}, \\
        \dot{\theta} &= q, \\
        \dot{\psi} &= \frac{r}{\cos(\theta)}.
    \end{align}
\end{subequations}

Similarly, we can derive the kinematics of the 3\gls{dof} model by substituting $z = \phi = \theta = w = p = q = 0$ into \eqref{eq:background_p_dot_6DOF} and \eqref{eq:background_theta_dot_6DOF}
\begin{align}
    \dot{\bs{\eta}} &= \mat{J}(\psi) \bs{\nu}, &
    \mat{J}(\psi) &=
    \begin{bmatrix}
        {\rm c}_{\psi} & -{\rm s}_{\psi} & 0 \\
        {\rm s}_{\psi} & {\rm c}_{\psi} & 0 \\
        0 & 0 & 1
    \end{bmatrix}.
\end{align}

\subsubsection*{Dynamics}
When modeling the dynamics of marine vehicles, we often need to consider the effect of \emph{sea loads} such as waves, wind, and ocean currents.
Let $\mat{V}_c \in \mathbb{R}^3$ be a vector that represents the velocity of the ocean current in the inertial coordinate frame.
Since the dynamics of ocean currents is typically much slower than the dynamics of the vehicle, the ocean current can be considered constant.
Let $\mat{v}_c = \mat{R}\T \mat{V}_c$ denote the velocity of the ocean current expressed in the vehicle's body-fixed frame.
Furthermore, let $\mat{v}_r = \mat{v} - \mat{v}_c \triangleq \inlinevector{u_r, v_r, w_r}$ denote the relative surge, sway, and heave velocity of the vehicle, and let $\bs{\nu}_r\T = \left[\mat{v}_r\T, \bs{\omega}\T\right]$ denote the full relative velocity vector.
The dynamics of the vehicle can then be expressed using the following matrix-vector model \cite{fossen_handbook_2011}
\begin{equation}
    \mat{M} \dot{\bs{\nu}}_r + \mat{C}(\bs{\nu}_r)\bs{\nu}_r + \mat{D}(\bs{\nu}_r)\bs{\nu}_r + \mat{g}(\mat{R}) = \bs{\tau},
    \label{eq:background_nu_dot_relative}
\end{equation}
where $\mat{M}$ is the mass and inertia matrix, including the added mass effects, $\mat{C}(\bs{\nu}_r)$ is the Coriolis and centripetal matrix, also including the added mass, $\mat{D}(\bs{\nu}_r)$ is the hydrodynamic damping matrix, $\mat{g}(\mat{R})$ represents the effects of gravity and buoyancy, and $\bs{\tau}$ represents additional forces and torques such as the effects of actuators and external disturbances.

The model in \eqref{eq:background_nu_dot_relative} can also be expressed in terms of absolute velocities
\begin{equation}
    \mat{M} \left(\dot{\bs{\nu}} - \dot{\bs{\nu}}_c\right) + \mat{C}(\bs{\nu} - \bs{\nu}_c)(\bs{\nu} - \bs{\nu}_c) + \mat{D}(\bs{\nu} - \bs{\nu}_c)(\bs{\nu} - \bs{\nu}_c) + \mat{g}(\mat{R}) = \bs{\tau},
\end{equation}
where $\bs{\nu}_c\T = \left[\mat{v}_c\T, \mat{0}\T\right]$.

The inertia matrix $\mat{M}$ is symmetric positive definite, the damping matrix $\mat{D}$ is positive definite, and the Coriolis matrix $\mat{C}$ is skew-symmetric.
There exist multiple expressions for the Coriolis matrix, \emph{e.g.,}
\begin{equation}
    \label{eq:background_coriolis}
    \mat{C}(\bs{\nu}_r)\! =\!
    \begin{bmatrix}
        \mat{O}_{3 \times 3} & \!\!-\mat{S}(\mat{M}_{11}\mat{v}_r + \mat{M}_{12}\bs{\omega}) \\
        -\mat{S}(\mat{M}_{11}\mat{v}_r + \mat{M}_{21}\bs{\omega}) & \!\!-\mat{S}(\mat{M}_{21}\mat{v}_r + \mat{M}_{22}\bs{\omega})
    \end{bmatrix}, \,
    \begin{bmatrix}
        \mat{M}_{11} & \!\!\!\mat{M}_{12} \\ \mat{M}_{21} & \!\!\!\mat{M}_{22}
    \end{bmatrix}\!
    = \!\mat{M}.
\end{equation}
The gravity and buoyancy vector is given by \cite{fossen_handbook_2011}
\begin{equation}
    \mat{g}(\mat{R}) = -
    \begin{bmatrix}
        (W - B) \mat{R}\T \mat{e}_3 \\
        (W\mat{r}_g - B\mat{r}_b) \times \mat{R}\T \mat{e}_3
    \end{bmatrix},
\end{equation}
where $W \in \mathbb{R}_{> 0}$ is the gravitational force, $B \in \mathbb{R}_{> 0}$ is the buoyant force, $\mat{r}_g$ is the position of the center of gravity, $\mat{r}_b$ is the position of the center of buoyancy, and $\mat{e}_3 = \inlinevector{0, 0, 1}$.

\subsubsection*{Control-oriented model of underactuated \glspl{auv}}
The goal of this section is to present a control-oriented model, a simplified mathematical model that is then used to design a controller and analyze its closed-loop properties.
We begin by presenting the assumptions that allow us to simplify the matrix-vector model in \eqref{eq:background_nu_dot_relative}.

\begin{asm}
    \label{asm:symmetric}
    The vehicle is slender, torpedo-shaped with rotational symmetry around its $x$-axis.
\end{asm}

\begin{asm}
    \label{asm:damping}
    The vehicle is maneuvering at low speeds.
    Consequently, the hydrodynamic damping can be considered linear.
\end{asm}

\noindent Under Assumption~\ref{asm:damping}, the hydrodynamic damping matrix is constant.
Under Assumption~\ref{asm:symmetric}, the inertia and damping matrices have the following structure
\begin{align}
    \mat{M}\! &= \!\!
    \begin{bmatrix}
        m_{11}\!\! & 0 & 0 & 0 & 0 & 0\\ 0 & \!m_{22}\! & 0 & 0 & 0 & m_{26}\\ 0 & 0 & \!m_{33}\! & 0 & \!m_{35}\! & 0\\ 0 & 0 & 0 & \!m_{44}\! & 0 & 0\\ 0 & 0 & \!m_{35}\! & 0 & \!m_{55}\! & 0\\ 0 & \!m_{26}\! & 0 & 0 & 0 & m_{66}
    \end{bmatrix}\!, &
    \mat{D}\! &= \!\!
    \begin{bmatrix}
        d_{11}\!\! & 0 & 0 & 0 & 0 & 0\\ 0 & \!d_{22}\! & 0 & 0 & 0 & d_{26}\\ 0 & 0 & \!d_{33}\! & 0 & \!d_{35}\! & 0\\ 0 & 0 & 0 & \!d_{44}\! & 0 & 0\\ 0 & 0 & \!d_{53}\! & 0 & \!d_{55}\! & 0\\ 0 & \!d_{62}\! & 0 & 0 & 0 & d_{66}
    \end{bmatrix}\!,
\end{align}
where $m_{22} = m_{33}$, $m_{35} = m_{26}$, and $m_{55} = m_{66}$.

\begin{asm}
    \label{asm:actuators}
    The vehicle is equipped with a propeller and fins.
    Consequently, the vehicle is capable of generating force in the surge direction and torque around all three axes.
\end{asm}
Under this assumption, the external forces acting on the vehicle are given by
\begin{align}
    \bs{\tau} &= \mat{B}\mat{u}, &
    \mat{B} &= 
    \begin{bmatrix}
        b_{11} & 0 & 0 & 0 \\ 0 & 0 & 0 & b_{24} \\ 0 & 0 & b_{33} & 0 \\ 0 & b_{42} & 0 & 0 \\ 0 & 0 & b_{53} & 0 \\ 0 & 0 & 0 & b_{64}
    \end{bmatrix},
\end{align}
where $\mat{u} = \inlinevector{f_u, t_p, t_q, t_r}$ is the control input consisting of surge thrust and the forces produced by the fins.

\begin{prop}
    \label{prop:neutral_point}
    If a vehicle model satisfies Assumptions~\ref{asm:symmetric}--\ref{asm:actuators}, then the origin of the body-fixed coordinate frame can be chosen such that the actuators produce no sway or heave acceleration.
    In other words, the following equality
    \begin{equation}
        \mat{M}^{-1}\bs{\tau} = \mat{M}^{-1}\mat{B}\mat{u} = \inlinevector{\tau_u, 0, 0, \tau_p, \tau_q, \tau_r},
    \end{equation}
    where $\tau_u, \tau_p, \tau_p, \tau_p \in \mathbb{R}$, holds.
\end{prop}
\begin{proof}
    Let CO denote the original body-fixed coordinate frame, and let $\bs{\nu}$ denote the velocities of the vehicle expressed in CO.
    Consider a new coordinate frame, ${\rm CO}^{\prime}$, that is parallel to CO.
    Let $\bs{\varepsilon} \in \mathbb{R}^3$ be the displacement between ${\rm CO}^{\prime}$ and CO, and let $\bs{\nu}^{\prime}$ denote the velocities of the vehicle expressed in ${\rm CO}^{\prime}$.
    The relation between $\bs{\nu}$ and $\bs{\nu}^{\prime}$ is
    \begin{align}
        \bs{\nu} &= \mat{H}(\bs{\varepsilon}) \bs{\nu}^{\prime}, &
        \mat{H}(\bs{\varepsilon}) &=
        \begin{bmatrix}
            \mat{I}_3 & -\mat{S}(\bs{\varepsilon}) \\
            \mat{O}_{3 \times 3} & \mat{I}_3
        \end{bmatrix}.
    \end{align}
    The equations of motion expressed in ${\rm CO}^{\prime}$ are thus given by
    \begin{equation}
        \mat{M}^{\prime}\dot{\bs{\nu}}_r^{\prime} + \mat{C}^{\prime}(\bs{\nu}^{\prime}_r)\bs{\nu}^{\prime}_r + \mat{D}^{\prime}\bs{\nu}^{\prime}_r + \mat{H}(\bs{\varepsilon})\T\mat{g}(\mat{R}) = \mat{H}(\bs{\varepsilon})\T\bs{\tau},
    \end{equation}
    where $\mat{M}^{\prime} = \mat{H}(\bs{\varepsilon})\T \mat{M} \mat{H}(\bs{\varepsilon})$, $\mat{D}^{\prime} = \mat{H}(\bs{\varepsilon})\T \mat{D} \mat{H}(\bs{\varepsilon})$, and $\mat{C}^{\prime}$ is the Coriolis matrix calculated from \eqref{eq:background_coriolis} using $\mat{M}^{\prime}$ as the inertia matrix.


\end{proof}

\begin{asm}
    \label{asm:buoyancy}
    The vehicle is neutrally buoyant, with the centers of gravity and buoyancy located on one vertical axis.
\end{asm}
\noindent Under this assumption, $\mat{g}(\mat{R})$ has the following shape
\begin{align}
    \mat{g}(\mat{R}) &= \begin{bmatrix}
        \mat{0}_3 \\ Wz_{gb} \mat{e}_3 \times \mat{R}\T \mat{e}_3
    \end{bmatrix},
    \label{eq:gravity}
\end{align}
where $z_{gb}$ is the distance between the centers of gravity and buoyancy.

\section{Geometric Paths}
\label{sec:background_paths}

This section presents the definitions of paths in the context of guidance.
The theory presented in this section applies to two- and three-dimensional Euclidean spaces.
Let $n_d \in \{2,3\}$ denote the number of dimensions.

\subsection{Paths and their parametrizations}
A \emph{path} is a curve in $n_d$-dimensional space.
A path can be expressed as a set $P \subset \mathbb{R}^{n_d}$.
A \emph{parametrization} of a path is a function $\mat{p}_p : \mathbb{R} \mapsto \mathbb{R}^{n_d}$ whose image space represents the given path, \emph{i.e.,} $\{\mat{p}_p(s) | s \in \mathbb{R}\} = P$.
Note that for a given path, there exist infinitely many parametrizations.
For example, the following two functions
\begin{align}
    \mat{p}_{p, 1}(s) &= \inlinevector{s, 0, 0}, &
    \mat{p}_{p, 1}(s) &= \inlinevector{s^3, 0, 0},
    \label{eq:background_path_example}
\end{align}
represent the same path --- a straight line going through the origin, parallel to the $x$-axis.
Furthermore, if we multiply these parametrizations by a positive scalar, we also get a valid parametrization.
In general, if $\mat{p}_p(s)$ is a path parametrization that is defined for all $s \in \mathbb{R}$, and $\rho(s)$ is a monotonically increasing function that is also defined for all $s \in \mathbb{R}$, then $\mat{p}_p(s)$ and $\mat{p}_p(\rho(s))$ parametrize the same path.
We will refer to $\mat{p}_p(\rho(s))$ as a \emph{reparametrization} of $\mat{p}_p(s)$.

\subsection{Continuity and regularity}
\emph{Continuity}, also referred to as smoothness, is an important property, as some vehicles are not able to follow a path that has discontinuities or sharp turns.
There are two types of continuity --- \emph{parametric} and \emph{geometric}.
Parametric continuity is related to a specific parametrization of a path, while geometric continuity is related to the curve itself.
Here, we will only present the definition of parametric continuity, as it will be used furhter in the thesis.
For details on geometric continuity, the reader is referred to \cite{barsky_geometric_1984}.
Parametric continuity is denoted $\mathcal{C}_n$, where $n \in \mathbb{Z}_{\geq 0}$ is the order.
A parametrization $\mat{p}_p(s)$ is $\mathcal{C}_n$ if it is $n$-times continuously differentiable.

A parametrization if \emph{regular} if
\begin{equation}
    \norm{\frac{\partial \mat{p}_p}{\partial s}} \neq 0.
\end{equation}
A regular parametrization means that there are no ``stops'' along the path.
Recalling the two examples in \eqref{eq:background_path_example}, both $\mat{p}_{p, 1}$ and $\mat{p}_{p, 2}$ are $C_{\infty}$, but only $\mat{p}_{p, 1}$ is regular since the derivative of $\mat{p}_{p, 2}(s)$ at $s = 0$ is zero.
Regularity is an important property when defining the path-tangential vector and the path-tangential coordinate frame, as we discuss next.

\subsection{Path-tangential vector and coordinate frame}
\label{sec:background_path_tangential}
If a parametrization is $\mathcal{C}_1$ and regular, then the \emph{path-tangential vector} is simply the first partial derivative of $\mat{p}_p(s)$ with respect to $s$.

A \emph{path-tangential coordinate frame} has its origin at $\mat{p}_p(s)$, and is oriented such that the path-tangential vector is aligned with the $x$-axis.
In the case of two-dimensional paths, this frame is uniquely defined.
Let $\mat{R}_p(s) \in SO(2)$ be the rotation matrix between the path-tangential and the inertial frame.
This matrix is given by
\begin{equation}
    \mat{R}_p(s) = 
    \begin{bmatrix}
        \cos\left(\psi_{p}(s)\right) & -\sin\left(\psi_{p}(s)\right) \\ \sin\left(\psi_{p}(s)\right) & \cos\left(\psi_{p}(s)\right)
    \end{bmatrix}, \,
    \psi_{p}(s) = \mathrm{arctan}_{2}\left(\frac{\partial  y_{p}(s)}{\partial s}, \frac{\partial  x_{p}(s)}{\partial s}\right),
    \label{eq:background_path_tangential_2D}
\end{equation}
where $x_p(s)$ and $y_p(s)$ are the components of $\mat{p}_p(s)$, and $\mathrm{arctan}_2$ is the four quadrant inverse tan.

In the case of three-dimensional paths, the path-tangential frame is not unique.
To make the $x$-axis of the coordinate frame aligned with the path-tangential vector, the rotation matrix $\mat{R}_p(s) \in SO(3)$ must satisfy
\begin{equation}
    \mat{R}_p(s) \inlinevector{1, 0, 0} = \norm{\frac{\partial \mat{p}_p(s)}{\partial s}}^{-1} \frac{\partial \mat{p}_p(s)}{\partial s}.
    \label{eq:background_R_p_equation}
\end{equation}
There exists a subspace of rotation matrices $\mat{R}_p(s)$ that satisfy \eqref{eq:background_R_p_equation}.
For the purposes of this thesis, the choice of $\mat{R}_p(s)$ is not important as long as it is smooth (\emph{i.e.,} the partial derivative of $\mat{R}_p(s)$ with respect to $s$ exists and is continuous).

One potential method for choosing $\mat{R}_p(s)$ is to use Euler angles and enforce a zero roll angle.
The rotation matrix is then given by
\begin{equation}
    \mat{R}_p(s) =
    \begin{bmatrix}
         \cos\left(\psi_{p}(s)\right)\,\cos\left(\theta_{p}(s)\right) & -\sin\left(\psi_{p}(s)\right) & \cos\left(\psi_{p}(s)\right)\,\sin\left(\theta_{p}(s)\right) \\ \cos\left(\theta_{p}(s)\right)\,\sin\left(\psi_{p}(s)\right) & \cos\left(\psi_{p}(s)\right) & \sin\left(\psi_{p}(s)\right)\,\sin\left(\theta_{p}(s)\right) \\ -\sin\left(\theta_{p}(s)\right) & 0 & \cos\left(\theta_{p}(s)\right)
    \end{bmatrix}\mathrlap{,}
    \label{eq:background_R_p_zero_roll}
\end{equation}
where
\begin{align}
    \theta_{p}(s) &= - \arcsin\left(\frac{\partial z_p(s) / \partial s}{\norm{\partial \mat{p}_p(s) / \partial s}}\right), &
    \psi_{p}(s) = \mathrm{arctan}_{2}\left(\frac{\partial  y_{p}(s)}{\partial s}, \frac{\partial  x_{p}(s)}{\partial s}\right).
    \label{eq:background_path_tangential_3D}
\end{align}
An illustration is shown in \figref{fig:background_path}.
Note that the yaw angle $\psi_{p}(s)$ is not defined if the path-tangential vector is aligned with the $z$-axis (\emph{i.e.,} if both $\partial x_p(s) / \partial s$ and $\partial y_p(s) / \partial s$ are zero).
However, we also note that most commercial \glspl{auv} can only reach a limited range of pitch angles, making them unable to move vertically.
Consequently, we should avoid designing vertical paths, where the singularities of Euler angles become an issue.

\begin{figure}[t]
    \centering
    \def\svgwidth{0.6\textwidth}
    \import{figures/background}{path.pdf_tex}
    \vspace*{-1em}
    \caption{Illustration of the path-tangential coordinate frame. $\mat{O}$ denotes the origin of the inertial coordinate frame, $\mat{O}^p$ denotes the origin of the path-tangential coordinate frame. The grey line represents the projection of the path-tangential vector onto the $xy$ plane of the inertial coordinate frame.}
    \label{fig:background_path}
\end{figure}

\subsection{Curvature}
As previously mentioned, some vehicles are unable to follow paths with ``sharp turns''.
For the purposes of this thesis, we define \emph{curvature} as a measure of change of the path-tangential coordinate frame.

In the two-dimensional case, the curvature, $\kappa(s)$, is defined as
\begin{equation}
    \kappa(s) = \frac{\partial \psi_{p}(s)}{\partial s}.
\end{equation}

In the three-dimensional case, the curvature is not defined as a scalar, but rather as a vector $\bs{\omega}_p(s) \in \mathbb{R}^3$ such that
\begin{equation}
    \frac{\partial \mat{R}_p(s)}{\partial s} = \mat{R}_p(s) \mat{S}\left(\bs{\omega}_p(s)\right).
\end{equation}
If the rotation matrix $\mat{R}_p(s)$ was chosen according to \eqref{eq:background_R_p_zero_roll}, then we can also define curvature in pitch and yaw, $\kappa(s)$ and $\iota(s)$, as
\begin{align}
    \kappa(s) &= \frac{\partial \theta_{p}(s)}{\partial s}, &
    \iota(s) &= \frac{\partial \psi_{p}(s)}{\partial s}.
\end{align}
The vector $\bs{\omega}_p(s)$ is then given by
\begin{equation}
    \bs{\omega}_p(s) = \inlinevector{-\iota(s)\sin\left(\theta_{p}(s)\right), \kappa(s), \iota(s)\cos\left(\theta_{p}(s)\right)}.
\end{equation}

\subsection{Parametrization by arc length}
A path parametrization $\mat{p}_p(s)$ is said to be a \emph{parametrization by arc length} if for all $s_1, s_2 \in \mathbb{R}, s_2 > s_1$, we have
\begin{equation}
    \int_{s_1}^{s_2} \norm{\frac{\partial \mat{p}_p(s)}{\partial s}} {\rm d}s = s_2 - s_1.
\end{equation}
This condition is equivalent to
\begin{equation}
    \norm{\frac{\partial \mat{p}_p(s)}{\partial s}} = 1.
\end{equation}

A convenient property of parametrizations by arc length is that the path parameter $s$ can be interpreted as distance.
Consequently, parametrizations by arc length are useful when we want the vehicles to follow the path at a constant speed.
For example, choosing the path parameter $s(t)$ such that $\dot{s}(t) = 1$ means that the vehicles should follow the path at a speed of $1$ meter per second.

Now, let us discuss how to find a parametrization by arc length.
Let $\mat{p}_p(s)$ be an arbitrary parametrization that is $\mathcal{C}_1$ and regular.
Then, we can find a parametrization by arc length by reparametrizing $\mat{p}_p(s)$, \emph{i.e.,} by finding a monotonically increasing function $\rho(s) : \mathbb{R} \mapsto \mathbb{R}$ such that
\begin{equation}
    \norm{\frac{\partial \mat{p}_p(\rho(s))}{\partial s}} = 1.
\end{equation}
The function $\rho(s)$ can be found by solving the following differential equation
\begin{align}
    \frac{\partial \rho(s)}{\partial s} &= \norm{\frac{\partial \mat{p}_p(\rho)}{\partial \rho}}^{-1}, &
    \rho(0) &= \rho_0,
\end{align}
where $\rho_0 \in \mathbb{R}$ is the initial condition.
Although the initial condition is arbitrary, it is convenient to choose $\rho_0 = 0$ so that the parametrization starts at the same point.

\section{Formation Path Following}

This section formally defines the formation path following problem.
Throughout the section, we consider a fleet of $N$ vehicles.
Let $\mat{p}_1, \ldots \mat{p}_N$ denote the positions of the vehicles.

\subsection{The path-following problem}
To formulate the path-following problem, we first need to define the \emph{barycenter} of the fleet.
The barycenter, $\mat{p}_b$, is given by the mean position of the vehicles, \emph{i.e.,}
\begin{equation}
    \mat{p}_b = \frac{1}{N} \sum_{i=1}^N \mat{p}_i.
\end{equation}

To solve the path-following problem, we need to control the vehicles such that the barycenter coincides with the desired path.
Let $\mat{p}_p(s)$ be the parametrization of the desired path.
Then, the goal of path-following is to control the vehicles such that
\begin{equation}
    \mat{p}_b \rightarrow \mat{p}_p(s).
\end{equation}

Let $\mat{p}_b^p$ denote the position of the barycenter in the path-tangential coordinate frame (see Section~\ref{sec:background_path_tangential}).
It is given by
\begin{equation}
    \mat{p}_b^p = \mat{R}_p\T \left(\mat{p}_b - \mat{p}_p(s)\right).
\end{equation}
Note that $\mat{p}_b^p$ can be interpreted as the path-following error.
Indeed, $\mat{p}_b$ is equal to $\mat{p}_p(s)$ if and only if $\mat{p}_b^p$ is zero.

Let $x_b^p$, $y_b^p$, and $z_b^p$ denote the components of $\mat{p}_b^p$.
The component $x_b^p$ is commonly referred to as the \emph{along-track error}, since the value of $x_b^p$ indicates whether the barycenter is ``in front of'' or ``behind'' the desired path.
The components $y_b^p$ and $z_b^p$ are referred to as the \emph{cross-track errors}, since they indicate the lateral deviation from the desired path.

\subsubsection{Path-following versus trajectory-tracking}


\section{Line-of-sight Guidance}
\label{sec:background_LOS}

This section describes the \gls{los} guidance algorithm.
\gls{los} is an intuitive method for steering vehicles towards the desired path.

\begin{figure}[t]
    \centering
    \begin{subfigure}{0.48\textwidth}
        \centering
        \def\svgwidth{0.9\textwidth}
        \import{figures/background}{los_2d.pdf_tex}
        \caption{Two-dimensional line-of-sight guidance.}
        \label{fig:background_los_2d}
    \end{subfigure}
    \begin{subfigure}{0.48\textwidth}
        \def\svgwidth{\textwidth}
        \import{figures/background}{los_3d.pdf_tex}
        \caption{Three-dimensional coupled line-of-sight guidance.}
        \label{fig:background_los_3d}
    \end{subfigure}
    \caption{Illustrations of line-of-sight guidance methods.}
\end{figure}

First, let us discuss \gls{los} guidance for vehicles moving in the horizontal plane.
Let $\mat{p}_b^p = \inlinevector{x_b^p, y_b^p}$ denote the path-following error of the barycenter.
Let $\mat{V}_{\rm LOS}$ denote the desired (inertial) line-of-sight velocity that steers the barycenter towards the desired path.
$\mat{V}_{\rm LOS}$ is given by
\begin{align}
    \mat{V}_{\rm LOS} &= U_{\rm LOS} \inlinevector{\cos(\chi_{\rm LOS}), \sin(\chi_{\rm LOS})}\!\!, &
    \chi_{\rm LOS} &= \psi_{p\!} - \arctan\left(\frac{y_b^p}{\Delta}\right)\!,
    \label{eq:background_los_2d}
\end{align}
where $U_{\rm LOS} > 0$ is the desired path-following speed, $\psi_p$ is the path-tangential angle, as defined in \eqref{eq:background_path_tangential_2D}, and $\Delta > 0$ is the so-called \emph{lookahead} distance.
An illustration of \gls{los} guidance in the horizontal plane is shown in \figref{fig:background_los_2d}.

For vehicles moving in three dimensions, there exist two types of \gls{los} guidance algorithms: \emph{decoupled} and \emph{coupled}.
A decoupled \gls{los} algorithm consists of two separate guidance schemes that steer the vehicle in the horizontal and vertical plane.
Let $\mat{p}_b^p = \inlinevector{x_b^p, y_b^p, z_b^p}$ denote the path-following error of the barycenter.
Let us assume that the path-tangential coordinate frame is chosen according to \eqref{eq:background_R_p_zero_roll}, so that the rotation matrix $\mat{R}_p(s)$ has a zero roll angle.
The desired line-of-sight velocity, $\mat{V}_{\rm LOS}$, is then given by
\begin{align}
    \mat{V}_{\rm LOS} &= U_{\rm LOS} \!
    \begin{bmatrix}
        \cos(\gamma_{\rm LOS}) \cos(\chi_{\rm LOS}) \\
        \cos(\gamma_{\rm LOS}) \sin(\chi_{\rm LOS}) \\
        -\sin(\gamma_{\rm LOS})
    \end{bmatrix}\!, &
    \begin{split}
        \gamma_{\rm LOS} &= \theta_{p\!} + \arctan\left(\frac{z_b^p}{\Delta_z}\right)\!, \\
        \chi_{\rm LOS} &= \psi_{p\!} - \arctan\left(\frac{y_b^p}{\Delta_y}\right)\!, \\
    \end{split}
    \label{eq:background_los_decoupled}
\end{align}
where $U_{\rm LOS} > 0$ is the desired path-following speed, $\theta_p$ and $\psi_p$ are the path-tangential angles, as defined in \eqref{eq:background_path_tangential_3D}, and $\Delta_y, \Delta_z > 0$ are the lookahead distances of the horizontal and vertical guidance scheme.
Comparing the decoupled guidance scheme in \eqref{eq:background_los_decoupled} to the two-dimensional \gls{los} algorithm in \eqref{eq:background_los_2d}, we can see that the decoupled guidance scheme effectively consists of two two-dimensional \gls{los} guidance algorithms.

In a \emph{coupled} \gls{los} guidance scheme, the desired velocity is given by
\begin{equation}
    \mat{V}_{\rm LOS} = \frac{U_{\rm LOS}}{\sqrt{\Delta^2 + {y_b^p}^2 + {z_b^p}^2}} \mat{R}_p \inlinevector{\Delta, -y_b^p, -z_b^p},
\end{equation}
where $U_{\rm LOS} > 0$ is the desired path-following speed, $\mat{R}_p$ is the rotation matrix between the path-tangential and the inertial coordinate frame, and $\Delta > 0$ is the lookahead distance.
An illustration of this scheme is shown in \figref{fig:background_los_3d}.
The coupled scheme can be seen as an extension of the horizontal \gls{los} guidance scheme to three dimensions.
Indeed, the scheme in \eqref{eq:background_los_2d} can also be written as
\begin{equation}
    \mat{V}_{\rm LOS} = 
    \frac{U_{\rm LOS}}{\sqrt{\Delta^2 + {y_b^p}^2}} \begin{bmatrix} \Delta\cos(\psi_p) + y_b^p\sin(\psi_p) \\ \Delta\sin(\psi_p) - y_b^p\cos(\psi_p) \end{bmatrix} = 
    \frac{U_{\rm LOS}}{\sqrt{\Delta^2 + {y_b^p}^2}} \mat{R}_p \begin{bmatrix} \Delta \\ -y_b^p \end{bmatrix}. 
    \label{eq:background_los_2d_R}
\end{equation}
Comparing \eqref{eq:background_los_2d_R} to \eqref{eq:background_los_coupled}, we can see that both equations have a similar form.

\section{Null Space Behavioral Algorithm}
\label{sec:background_NSB}

This section describes the \gls{nsb} algorithm.
The \gls{nsb} algorithm is a method that allows us to combine several tasks in a hierarchic manner.
The algorithm was originally developed for first-order systems
\begin{equation}
    \dot{\mat{p}} = \mat{v},
\end{equation}
where $\mat{p} \in \mathbb{R}^N$ are the generalized coordinates, and $\mat{v} \in \mathbb{R}^N$ are the control inputs.

In \gls{nsb} algorithms, the desired behavior of the system is divided into several tasks.
Let there be $m$ tasks, arranged by priority in a descending order (\emph{i.e.,} task 1 has the highest priority, task $m$ has the lowest priority).
Let $\bs{\sigma}_1, \ldots, \bs{\sigma}_m$ denote the so-called \emph{task variables}.
Each variable is a function of the system coordinates, \emph{i.e.,}
\begin{align}
    \bs{\sigma}_i &= f_i(\mat{p}), &
    f_i &: \mathbb{R}^N \mapsto \mathbb{R}^{n_i},
\end{align}
where $n_i \leq N$ is the dimensionality of task $i$.
Applying the chain rule, the time-derivative of $\bs{\sigma}_i$ is
\begin{equation}
    \dot{\bs{\sigma}}_i = \frac{\partial f_i(\mat{p})}{\partial \mat{p}} \mat{v} \triangleq \mat{J}_i(\mat{p}) \mat{v}.
\end{equation}
Let $\dot{\bs{\sigma}}_i^{*}$ be the desired closed-loop behavior of the task variable\footnote{
In many applications, the task variable should track some pre-defined desired value, $\bs{\sigma}_{d, i}$.
In such cases, we typically choose the following desired closed-loop behavior
$
    \dot{\bs{\sigma}}_i^{*} = \dot{\bs{\sigma}}_{d, i} - \bs{\Lambda}_i \left(\bs{\sigma}_i - \bs{\sigma}_{d,i}\right),    
$
where $\bs{\Lambda}_i$ is a positive definite gain matrix.
}.
Then, the smallest input (in terms of Euclidean norm) that guarantees the desired behavior is
\begin{equation}
    \mat{v}_i = \mat{J}_i^{\dagger} \dot{\bs{\sigma}}_i^{*},
    \label{eq:background_NSB_velocity_i}
\end{equation}
where $\mat{J}_i^{\dagger}$ is the Moore-Penrose pseudoinverse of the task Jacobian.

If the task is \emph{redundant}, \emph{i.e.}, if $n_i < N$, then there exists a subspace of control inputs that do not interfere with the task.
Let $\mat{v}_{\rm add}$ be an additional input.
Then, the following control input
\begin{equation}
    \mat{v} = \mat{v}_i + \mat{N}_i \mat{v}_{\rm add},
\end{equation}
where $\mat{N}_i = \mat{I}_{N} - \mat{J}_i^{\dagger}\mat{J}_i$ is the null space projector of $\mat{J}_i$, guarantees the desired behavior of the task.
The additional control input is satisfied only if it does not interfere with the task.

In the \gls{nsb} algorithm, the control inputs from the individual tasks are composed by projecting the inputs from the lower-priority tasks onto the null space of the higher-priority tasks.
In the literature, there exist two variants of the algorithm.
The first variant calculates the control input $\mat{v}$ using the following equation
\begin{align}
    \mat{v} &= \mat{v}_{1} + \mat{N}_1 \biggl(\mat{v}_{2} + \mat{N}_2\Bigl(\mat{v}_3 \cdots + \mat{N}_{m-2}\bigl(\mat{v}_{m-1} + \mat{N}_{m-1}\mat{v}_m\bigr)\Bigr)\biggr),
\end{align}
with $\mat{v}_i$ given by \eqref{eq:background_NSB_velocity_i}.

The second variant uses the so-called \emph{augmented} Jacobians
\begin{align}
    \Bar{\mat{J}}_i &= \inlinevector{\mat{J}_1\T, \ldots, \mat{J}_i\T}.
\end{align}
Let $\Bar{\mat{N}}_i$ denote the null space projector of $\Bar{\mat{J}}_i$.
Then, the control input $\mat{v}$ is given by
\begin{equation}
    \mat{v} = \mat{v}_1 + \sum_{i=2}^m \Bar{\mat{N}}_{i-1} \mat{v}_i.
\end{equation}
The advantages and disadvantages of both approaches are discussed in \cite{antonelli_stability_2008}.
In the thesis, we will mostly use the first variant.

\subsection{Independence and orthogonality}
The concepts of independence and orthogonality are important when analyzing the interactions between the tasks.
Specifically, these concepts determine whether the tasks can be executed simultaneously, and how the null space projector affects the lower-priority tasks.

Two tasks are \emph{independent} if the pseudoinverses of their Jacobians are linearly independent.
Let $\mat{J}_i$ and $\mat{J}_j$ denote the Jacobians of task $i$ and $j$, respectively.
These tasks are independent if
\begin{equation}
    {\rank} \left(\mat{J}_i^{\dagger}\right) + {\rank} \left(\mat{J}_j^{\dagger}\right) = {\rm rank} \left(\left[ \mat{J}_i^{\dagger},\, \mat{J}_j^{\dagger} \right]\right),
    \label{eq:background_independence_pseudo}
\end{equation}
Antonelli \emph{et al.} \cite{antonelli_stability_2008} remark that the pseudoinverse and the transpose of the task Jacobian share the same span.
Consequently, the condition in \eqref{eq:background_independence_pseudo} is equivalent to
\begin{equation}
    {\rank} \left(\mat{J}_i\T\right) + {\rank} \left(\mat{J}_j\T\right) = {\rm rank} \left(\left[ \mat{J}_i\T,\, \mat{J}_j\T \right]\right).
    \label{eq:background_independence}
\end{equation}

Two tasks are \emph{orthogonal} if the subspaces spanned by these tasks are orthogonal, \emph{i.e.,} if
\begin{equation}
    \mat{J}_i \, \mat{J}_j^{\dagger} = \mat{O}_{n_i \times n_j},
\end{equation}
where $\mat{O}_{n_i \times n_j}$ is an $n_i$-by-$n_j$ null matrix.

Now, let us consider two consecutive tasks that are independent and orthogonal. 
Without loss of generality, let us denote these tasks $1$ and $2$.
The control input produced by combining these two tasks is
\begin{equation}
    \mat{v} = \mat{J}_1^{\dagger} \dot{\bs{\sigma}}_1^{*} + \mat{N}_1 \mat{J}_2^{\dagger} \dot{\bs{\sigma}}_2^{*}
    = \mat{J}_1^{\dagger} \dot{\bs{\sigma}}_1^{*} + \left(\mat{I} - \mat{J}_1^{\dagger}\mat{J}_1\right) \mat{J}_2^{\dagger} \dot{\bs{\sigma}}_2^{*}
    = \mat{J}_1^{\dagger} \dot{\bs{\sigma}}_1^{*} + \mat{J}_2^{\dagger} \dot{\bs{\sigma}}_2^{*}.
\end{equation}

We have thus shown that if two consecutive tasks are independent and orthogonal, they can be executed simultaneously.
Moreover, the null-space projector does not affect the lower-priority task.

\subsection{\gls{nsb} algorithm for the formation path-following problem}
In the remainder of the section, we demonstrate how the \gls{nsb} algorithm can be used to solve the formation path-following problem.

Let $\mat{p}\T = \left[\mat{p}_1\T, \ldots, \mat{p}_n\T\right]$ be the concatenated position vector of $n$ vehicles.
To solve the problem, we define two tasks: formation-keeping and path-following.
The formation-keeping task has the highest priority.
The task variable, $\bs{\sigma}_1$, is given by
\begin{align}
    \bs{\sigma}_1 &= \inlinevector{\bs{\sigma}_{1, 1}\T, \bs{\sigma}_{1, 2}\T, \ldots, \bs{\sigma}_{1, n-1}\T}, &
    \bs{\sigma}_{1, i} &= \mat{p}_i - \mat{p}_b,
    \label{eq:background_formation_variable}
\end{align}
where $\mat{p}_b = \frac{1}{n}\sum\limits_{i=1}^n \mat{p}_i$ is the barycenter of the formation (see Section~\ref{sec:background_formation_path_following}).

\begin{rmk}
    The formation-keeping task contains the relative positions of the first $n-1$ vehicles.
    The relative position of the last vehicle is omitted, because it can be expressed as a linear combination of the remaining relative positions.
    Indeed, from \eqref{eq:background_formation_variable}, we get
    \begin{equation}
        \bs{\sigma}_{1, n} = \mat{p}_n - \mat{p}_b = - \sum_{i=1}^{n-1} \bs{\sigma}_{1, i}.
    \end{equation}
    By omitting the last relative position vector, the task Jacobian is full column rank.
    Indeed, the Jacobian of the formation-keeping task is
    \begin{equation}
        \begin{split}
            \mat{J}_1 &= 
            \begin{bmatrix}
                \frac{N-1}{N} \mat{I}_3 & -\frac{1}{N} \mat{I}_3 & \cdots & -\frac{1}{N} \mat{I}_3 & -\frac{1}{N} \mat{I}_3 \\[3pt]
                -\frac{1}{N} \mat{I}_3 & \frac{N-1}{N} \mat{I}_3 & & -\frac{1}{N} \mat{I}_3 & -\frac{1}{N} \mat{I}_3 \\ 
                \vdots & & \ddots & & \vdots \\
                -\frac{1}{N} \mat{I}_3 & -\frac{1}{N} \mat{I}_3 & \cdots & \frac{N-1}{N} \mat{I}_3 & -\frac{1}{N} \mat{I}_3
            \end{bmatrix} \\[6pt]
            &= \left[\mat{I}_{3(n-1)},\, \mat{O}_{3(n-1) \times 3}\right] - \frac{1}{N} \mat{1}_{(n-1) \times n} \otimes \mat{I}_3.
        \end{split}
    \end{equation}
    One can verify that the rank of $\mat{J}_1$ is $3(n-1)$, and the Jacobian is thus full column rank.
\end{rmk}

The desired value of the formation-keeping task variable is
\begin{align}
    \bs{\sigma}_{d, 1} &= \inlinevector{\mat{p}_{f, 1}\T, \mat{p}_{f, 2}\T, \ldots, \mat{p}_{f, n-1}\T},
\end{align}
where $\mat{p}_{f, i}$ is the desired position of vehicle $i$ within the formation (see Section~\ref{sec:background_formation_path_following}).

The formation-keeping control input $\mat{v}_1$ is then given by
\begin{equation}
    \mat{v}_1 = \mat{J}_1^{\dagger} \left(\dot{\bs{\sigma}}_{d, 1} - \bs{\Lambda}_1 (\bs{\sigma}_1 - \bs{\sigma}_{d, 1})\right),
\end{equation}
where $\bs{\Lambda}_1$ is a positive definite gain matrix.



\section{Uniform Semiglobal Exponential Stability}
\label{sec:background_USGES}

This section discusses the concept of \gls{usges}.
In some cases, dynamical systems cannot attain global stability due to, for instance, high-order nonlinearities, the choice of the control law, or actuator saturations.
An example of such a system is marine vehicles controlled by \acrlong{los} guidance laws.
In \cite{fossen_uniform_2014}, it has been shown that the structure of proportional \gls{los} guidance laws prevents the system from having global exponential convergence.

\Acrlong{usges} has been studied, \emph{e.g.,} in \cite{loria_cascaded_2005,pettersen_lyapunov_2017}.
In these works, \gls{usges} is defined as follows.

\begin{dfn}[USGES]
    Consider a nonlinear system given by the following set of \glspl{ode}
    \begin{align}
        \dot{\mat{x}} &= f(t, \mat{x}), &
        \mat{x}(0) &= \mat{x}_0,
        \label{eq:background_USGES_ODE}
    \end{align}
    with the origin $\mat{x} = \mat{0}$ being the equilibrium point of the system.

    Let $\mat{x}(t | \mat{x}_0)$ be a solution to \eqref{eq:background_USGES_ODE} that is defined for all $t \geq 0$.
    The origin $\mat{x} = \mat{0}$ is a \glspl{usges} equilibrium point of \eqref{eq:background_USGES_ODE} if for all $\Delta > 0$, there exist positive constants $k_{\Delta}$ and $\lambda_{\Delta}$ such that $\forall \mat{x}_0 \in \ball{\Delta}$
    \begin{align}
        &\norm{\mat{x}(t | \mat{x}_0)} \leq k_{\Delta} \norm{\mat{x}_0} {\rm e}^{-\lambda_{\Delta}t},&
        &\forall t \geq 0.
    \end{align}
\end{dfn}

\begin{rmk*}
    The work in \cite{pettersen_lyapunov_2017} studies parametric systems, i.e., systems with \glspl{ode} in the following form
    \begin{equation}
        \dot{\mat{x}} = f(\mat{x}, t, \theta),
    \end{equation}
    where $\theta \in \Theta \subset \mathbb{R}^m$ is a constant parameter.
    However, since this thesis does not consider parametric systems, and since the parameter $\theta$ is assumed constant, we can omit the parametric dependence for the sake of simplicity.
\end{rmk*}

\subsection{Lyapunov sufficient conditions for uniform semiglobal exponential stability}
In this section, we restate Theorem 5 and Proposition 9 from \cite{pettersen_lyapunov_2017}.

Theorem 5 introduces sufficient conditions for \acrfull{usges} of nonlinear systems.
\begin{theorem}[Theorem 5. \cite{pettersen_lyapunov_2017}]
    \label{thm:background_USGES}
    Consider the system given in \eqref{eq:background_USGES_ODE}.
    If for any $\Delta > 0$, there exist a continuously differentiable Lyapunov function $V_{\Delta} : \mathbb{R}_{\geq 0} \times \ball{\Delta} \mapsto \mathbb{R}_{\geq 0}$ and positive constants $k_{1_\Delta}$, $k_{2_\Delta}$, $k_{3_\Delta}$, and $a$, such that $\forall \mat{x} \in \ball{\Delta}$, $\forall t \geq 0$
    \begin{subequations}
        \begin{align}
            k_{1_\Delta} \norm{\mat{x}}^a \leq V_{\Delta}(t, \mat{x}) &\leq k_{2_\Delta} \norm{\mat{x}}^a, \\
            \lim_{\Delta \rightarrow \infty} \left(\frac{k_{1_\Delta}}{k_{2_\Delta}}\right)^{1/a} \Delta &= \infty, \\
            \frac{\partial V_{\Delta}}{\partial t} + \frac{\partial V_{\Delta}}{\partial \mat{x}} f(t, \mat{x}) &\leq -k_{3_\Delta} \norm{\mat{x}}^a,
        \end{align}
    \end{subequations}
    then the origin of \eqref{eq:background_USGES_ODE} is \glspl{usges}.
\end{theorem}

Proposition 9 then introduces sufficient conditions for \acrfull{usges} of nonlinear cascaded systems.
\begin{prop}[Proposition 9. \cite{pettersen_lyapunov_2017}]
    \label{prop:background_cascade}
    Consider the following cascaded nonlinear time-varying system
    \begin{subequations}
        \begin{align}
            \dot{\mat{x}}_1 &= f_1(t, \mat{x}_1) + g(t, \mat{x}_1)\mat{x}_2, \\
            \dot{\mat{x}}_2 &= f_2(t, \mat{x}_2),
        \end{align}
        \label{eq:background_USGES_cascade}
    \end{subequations}

    \noindent where $t \in \mathbb{R}_{\geq 0}$, $\mat{x}_1 \in \mathbb{R}^{n_1}$, $\mat{x}_2 \in \mathbb{R}^{n_2}$.
    The functions $f_1$, $f_2$, and $g$ are continuous in $t$ and locally Lipschitz in $\mat{x}_1$ and $\mat{x}_2$.
    Furthermore, $f_1$ is assumed $\mathcal{C}^1$ in $t$ and $\mat{x}_1$, and the origin $\left[\mat{x}_1\T, \mat{x}_2\T\right] = \mat{0}\T$ is an equilibrium point of \eqref{eq:background_USGES_cascade}.

    Let each of the systems
    \begin{align}
        \dot{\mat{x}}_1 &= f_1(t, \mat{x}_1), \label{eq:background_nominal} \\
        \dot{\mat{x}}_2 &= f_2(t, \mat{x}_2), 
    \end{align}
    be \glspl{ugas} and satisfy the conditions of Theorem~\ref{thm:background_USGES}.

    Then, the origin of the cascaded system \eqref{eq:background_USGES_cascade} is \glspl{usges} and \glspl{ugas} if the following two assumptions hold
    \begin{enumerate}
        \item There exist constants $c_1, c_2, \eta > 0$ and a positive definite, radially unbounded Lyapunov function $V : \mathbb{R}_{\geq 0} \times \mathbb{R}^{n_1}$ of \eqref{eq:background_nominal} such that $\dot{V}(t, \mat{x}_1) \leq 0$ and
        \begin{subequations}
            \begin{align}
                \norm{\frac{\partial V}{\partial \mat{x}_1}} \norm{\mat{x}_1} &\leq c_1 V,& \forall&\norm{\mat{x}_1} \geq \eta, \\
                \norm{\frac{\partial V}{\partial \mat{x}_1}} &\leq c_2,& \forall&\norm{\mat{x}_1} \leq \eta.
            \end{align}
        \end{subequations}
        \item There exist two continuous functions $\alpha_1, \alpha_2: \mathbb{R}_{\geq 0} \mapsto \mathbb{R}_{\geq 0}$ such that
        \begin{equation}
            \norm{g(t, \mat{x}_1, \mat{x}_2)} \leq \alpha_1 \left(\norm{\mat{x}_2}\right) + \alpha_2 \left(\norm{\mat{x}_2}\right) \norm{\mat{x}_1}.
            \label{eq:background_USGES_cascade_assumption_2}
        \end{equation}
    \end{enumerate}
\end{prop}

