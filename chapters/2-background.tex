\chapter{Background}
\label{chap:background}

Information on background theory, methods, etc.

\section{Geometric Paths}
\label{sec:background_paths}

This section presents the definitions of paths in the context of guidance.
The theory presented in this section applies to two- and three-dimensional Euclidean spaces.
Let $n_d \in \{2,3\}$ denote the number of dimensions.

\subsubsection*{Paths and their parametrizations}
A \emph{path} is a curve in $n_d$-dimensional space.
A \emph{parametrization} of a path is a function $\mat{p}_p : \mathbb{R} \mapsto \mathbb{R}^{n_d}$ that describes the curve.
Note that there exist infinitely many parametrizations of the same path.
For example, the following two functions
\begin{align}
    \mat{p}_{p, 1}(s) &= \inlinevector{s, 0, 0}, &
    \mat{p}_{p, 1}(s) &= \inlinevector{s^3, 0, 0}, &
\end{align}
represent the same path --- a straight line going through the origin, parallel to the $x$-axis.

\subsubsection*{Continuity and regularity}
There are two types of continuity --- \emph{parametric} and \emph{geometric}.
Parametric continuity is related to a specific parametrization of a path, while geometric continuity is related to the curve itself.
Here, we will only present the definition of parametric continuity, as it will be used furhter in the thesis.
For details on geometric continuity, the reader is referred to \cite{barsky_geometric_1984}.
Parametric continuity is denoted $\mathcal{C}_n$, where $n \in \mathbb{Z}_{\geq 0}$ is the order.
A parametrization $\mat{p}_p(s)$ is $\mathcal{C}_n$ if it is $n$-times continuously differentiable.

A parametrization if \emph{regular} if 

