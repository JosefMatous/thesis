\chapter{Background}
\label{chap:background}

\setlength{\epigraphwidth}{0.7\textwidth}
\epigraph{ \it
    The latter consisted simply of six hydrocoptic marzlevanes, so fitted to the ambifacient lunar waneshaft that side fumbling was effectively prevented.
}{--- John Hellins Quick, ``The turbo-encabulator in industry,'' Students' Quarterly Journal, 1944.}

Information on background theory, methods, etc.

\section{Geometric Paths}
\label{sec:background_paths}

This section presents the definitions of paths in the context of guidance.
The theory presented in this section applies to two- and three-dimensional Euclidean spaces.
Let $n_d \in \{2,3\}$ denote the number of dimensions.

\subsubsection*{Paths and their parametrizations}
A \emph{path} is a curve in $n_d$-dimensional space.
A \emph{parametrization} of a path is a function $\mat{p}_p : \mathbb{R} \mapsto \mathbb{R}^{n_d}$ that describes the curve.
Note that there exist infinitely many parametrizations of the same path.
For example, the following two functions
\begin{align}
    \mat{p}_{p, 1}(s) &= \inlinevector{s, 0, 0}, &
    \mat{p}_{p, 1}(s) &= \inlinevector{s^3, 0, 0}, &
\end{align}
represent the same path --- a straight line going through the origin, parallel to the $x$-axis.

\subsubsection*{Continuity and regularity}
There are two types of continuity --- \emph{parametric} and \emph{geometric}.
Parametric continuity is related to a specific parametrization of a path, while geometric continuity is related to the curve itself.
Here, we will only present the definition of parametric continuity, as it will be used furhter in the thesis.
For details on geometric continuity, the reader is referred to \cite{barsky_geometric_1984}.
Parametric continuity is denoted $\mathcal{C}_n$, where $n \in \mathbb{Z}_{\geq 0}$ is the order.
A parametrization $\mat{p}_p(s)$ is $\mathcal{C}_n$ if it is $n$-times continuously differentiable.

A parametrization if \emph{regular} if
\begin{equation}
    \norm{\frac{\partial \mat{p}_p}{\partial s}} \neq 0.
\end{equation}
A regular parametric means that there are no ``stops'' along the path.
Regularity is an important property when defining the path-tangential vector and the path-tangential coordinate frame, as we discuss next.

\subsubsection*{Path-tangential vector and coordinate frame}
If a parametrization is $\mathcal{C}_1$ and regular, then the \emph{path-tangential vector} is simply the first partial derivative of $\mat{p}_p(s)$ with respect to $s$.

A \emph{path-tangential coordinate frame} has its origin at $\mat{p}_p(s)$, and is oriented such that the path-tangential vector is aligned with the $x$-axis.
In the case of two-dimensional paths, this frame is uniquely defined.
Let $\mat{R}_p \in SO(2)$ be the rotation matrix between the path-tangential and the inertial frame.
This matrix is given by
\begin{align}
    \mat{R}_p &= 
    \begin{bmatrix}
        \cos\left(\psi_p(s)\right) & -\sin\left(\psi_p(s)\right) \\ \sin\left(\psi_p(s)\right) & \cos\left(\psi_p(s)\right)
    \end{bmatrix}, &
    \psi_p(s) &= \mathrm{arctan}_2\left(\frac{\partial y_p(s)}{\partial s}, \frac{\partial x_p(s)}{\partial s}\right),
\end{align}
where $x_p(s)$ and $y_p(s)$ are the components of $\mat{p}_p(s)$, and $\mathrm{arctan}_2$ is the four quadrant inverse tan.

In the case of three-dimensional paths, the path-tangential frame is not unique.
To make the $x$-axis of the coordinate frame aligned with the path-tangential vector, the rotation matrix $\mat{R}_p \in SO(3)$ must satisfy
\begin{equation}
    \mat{R}_p \inlinevector{1, 0, 0} = \norm{\frac{\partial \mat{p}_p(s)}{\partial s}}^{-1} \frac{\partial \mat{p}_p(s)}{\partial s}.
\end{equation}

