\section{Geometric Paths}
\label{sec:background_paths}

This section presents the definitions of paths in the context of guidance.
The theory presented in this section applies to two- and three-dimensional Euclidean spaces.
Let $n_d \in \{2,3\}$ denote the number of dimensions.

\subsection{Paths and their parametrizations}
A \emph{path} is a curve in $n_d$-dimensional space.
A path can be expressed as a set $P \subset \mathbb{R}^{n_d}$.
A \emph{parametrization} of a path is a function $\mat{p}_p : \mathbb{R} \mapsto \mathbb{R}^{n_d}$ whose image space represents the given path, \emph{i.e.,} $\{\mat{p}_p(s) | s \in \mathbb{R}\} = P$.
Note that for a given path, there exist infinitely many parametrizations.
For example, the following two functions
\begin{align}
    \mat{p}_{p, 1}(s) &= \inlinevector{s, 0, 0}, &
    \mat{p}_{p, 1}(s) &= \inlinevector{s^3, 0, 0},
    \label{eq:background_path_example}
\end{align}
represent the same path --- a straight line going through the origin, parallel to the $x$-axis.
Furthermore, if we multiply these parametrizations by a positive scalar, we also get a valid parametrization.
In general, if $\mat{p}_p(s)$ is a path parametrization that is defined for all $s \in \mathbb{R}$, and $\rho(s)$ is a monotonically increasing function that is also defined for all $s \in \mathbb{R}$, then $\mat{p}_p(s)$ and $\mat{p}_p(\rho(s))$ parametrize the same path.
We will refer to $\mat{p}_p(\rho(s))$ as a \emph{reparametrization} of $\mat{p}_p(s)$.

\subsection{Continuity and regularity}
\emph{Continuity}, also referred to as smoothness, is an important property, as some vehicles are not able to follow a path that has discontinuities or sharp turns.
There are two types of continuity --- \emph{parametric} and \emph{geometric}.
Parametric continuity is related to a specific parametrization of a path, while geometric continuity is related to the curve itself.
Here, we will only present the definition of parametric continuity, as it will be used furhter in the thesis.
For details on geometric continuity, the reader is referred to \cite{barsky_geometric_1984}.
Parametric continuity is denoted $\mathcal{C}_n$, where $n \in \mathbb{Z}_{\geq 0}$ is the order.
A parametrization $\mat{p}_p(s)$ is $\mathcal{C}_n$ if it is $n$-times continuously differentiable.

A parametrization if \emph{regular} if
\begin{equation}
    \norm{\frac{\partial \mat{p}_p}{\partial s}} \neq 0.
\end{equation}
A regular parametrization means that there are no ``stops'' along the path.
Recalling the two examples in \eqref{eq:background_path_example}, both $\mat{p}_{p, 1}$ and $\mat{p}_{p, 2}$ are $C_{\infty}$, but only $\mat{p}_{p, 1}$ is regular since the derivative of $\mat{p}_{p, 2}(s)$ at $s = 0$ is zero.
Regularity is an important property when defining the path-tangential vector and the path-tangential coordinate frame, as we discuss next.

\subsection{Path-tangential vector and coordinate frame}
\label{sec:background_path_tangential}
If a parametrization is $\mathcal{C}_1$ and regular, then the \emph{path-tangential vector} is simply the first partial derivative of $\mat{p}_p(s)$ with respect to $s$.

A \emph{path-tangential coordinate frame} has its origin at $\mat{p}_p(s)$, and is oriented such that the path-tangential vector is aligned with the $x$-axis.
In the case of two-dimensional paths, this frame is uniquely defined.
Let $\mat{R}_p(s) \in SO(2)$ be the rotation matrix between the path-tangential and the inertial frame.
This matrix is given by
\begin{equation}
    \mat{R}_p(s) = 
    \begin{bmatrix}
        \cos\left(\psi_{p}(s)\right) & -\sin\left(\psi_{p}(s)\right) \\ \sin\left(\psi_{p}(s)\right) & \cos\left(\psi_{p}(s)\right)
    \end{bmatrix}, \,
    \psi_{p}(s) = \mathrm{arctan}_{2}\left(\frac{\partial  y_{p}(s)}{\partial s}, \frac{\partial  x_{p}(s)}{\partial s}\right),
    \label{eq:background_path_tangential_2D}
\end{equation}
where $x_p(s)$ and $y_p(s)$ are the components of $\mat{p}_p(s)$, and $\mathrm{arctan}_2$ is the four quadrant inverse tan.

In the case of three-dimensional paths, the path-tangential frame is not unique.
To make the $x$-axis of the coordinate frame aligned with the path-tangential vector, the rotation matrix $\mat{R}_p(s) \in SO(3)$ must satisfy
\begin{equation}
    \mat{R}_p(s) \inlinevector{1, 0, 0} = \norm{\frac{\partial \mat{p}_p(s)}{\partial s}}^{-1} \frac{\partial \mat{p}_p(s)}{\partial s}.
    \label{eq:background_R_p_equation}
\end{equation}
There exists a subspace of rotation matrices $\mat{R}_p(s)$ that satisfy \eqref{eq:background_R_p_equation}.
For the purposes of this thesis, the choice of $\mat{R}_p(s)$ is not important as long as it is smooth (\emph{i.e.,} the partial derivative of $\mat{R}_p(s)$ with respect to $s$ exists and is continuous).

One potential method for choosing $\mat{R}_p(s)$ is to use Euler angles and enforce a zero roll angle.
The rotation matrix is then given by
\begin{equation}
    \mat{R}_p(s) =
    \begin{bmatrix}
         \cos\left(\psi_{p}(s)\right)\,\cos\left(\theta_{p}(s)\right) & -\sin\left(\psi_{p}(s)\right) & \cos\left(\psi_{p}(s)\right)\,\sin\left(\theta_{p}(s)\right) \\ \cos\left(\theta_{p}(s)\right)\,\sin\left(\psi_{p}(s)\right) & \cos\left(\psi_{p}(s)\right) & \sin\left(\psi_{p}(s)\right)\,\sin\left(\theta_{p}(s)\right) \\ -\sin\left(\theta_{p}(s)\right) & 0 & \cos\left(\theta_{p}(s)\right)
    \end{bmatrix}\mathrlap{,}
    \label{eq:background_R_p_zero_roll}
\end{equation}
where
\begin{align}
    \theta_{p}(s) &= - \arcsin\left(\frac{\partial z_p(s) / \partial s}{\norm{\partial \mat{p}_p(s) / \partial s}}\right), &
    \psi_{p}(s) = \mathrm{arctan}_{2}\left(\frac{\partial  y_{p}(s)}{\partial s}, \frac{\partial  x_{p}(s)}{\partial s}\right).
    \label{eq:background_path_tangential_3D}
\end{align}
An illustration is shown in \figref{fig:background_path}.
Note that the yaw angle $\psi_{p}(s)$ is not defined if the path-tangential vector is aligned with the $z$-axis (\emph{i.e.,} if both $\partial x_p(s) / \partial s$ and $\partial y_p(s) / \partial s$ are zero).
However, we also note that most commercial \glspl{auv} can only reach a limited range of pitch angles, making them unable to move vertically.
Consequently, we should avoid designing vertical paths, where the singularities of Euler angles become an issue.

\begin{figure}[t]
    \centering
    \def\svgwidth{0.6\textwidth}
    \import{figures/background}{path.pdf_tex}
    \vspace*{-1em}
    \caption{Illustration of the path-tangential coordinate frame. $\mat{O}$ denotes the origin of the inertial coordinate frame, $\mat{O}^p$ denotes the origin of the path-tangential coordinate frame. The grey line represents the projection of the path-tangential vector onto the $xy$ plane of the inertial coordinate frame.}
    \label{fig:background_path}
\end{figure}

\subsection{Curvature}
As previously mentioned, some vehicles are unable to follow paths with ``sharp turns''.
For the purposes of this thesis, we define \emph{curvature} as a measure of change of the path-tangential coordinate frame.

In the two-dimensional case, the curvature, $\kappa(s)$, is defined as
\begin{equation}
    \kappa(s) = \frac{\partial \psi_{p}(s)}{\partial s}.
\end{equation}

In the three-dimensional case, the curvature is not defined as a scalar, but rather as a vector $\bs{\omega}_p(s) \in \mathbb{R}^3$ such that
\begin{equation}
    \frac{\partial \mat{R}_p(s)}{\partial s} = \mat{R}_p(s) \mat{S}\left(\bs{\omega}_p(s)\right).
\end{equation}
If the rotation matrix $\mat{R}_p(s)$ was chosen according to \eqref{eq:background_R_p_zero_roll}, then we can also define curvature in pitch and yaw, $\kappa(s)$ and $\iota(s)$, as
\begin{align}
    \kappa(s) &= \frac{\partial \theta_{p}(s)}{\partial s}, &
    \iota(s) &= \frac{\partial \psi_{p}(s)}{\partial s}.
\end{align}
The vector $\bs{\omega}_p(s)$ is then given by
\begin{equation}
    \bs{\omega}_p(s) = \inlinevector{-\iota(s)\sin\left(\theta_{p}(s)\right), \kappa(s), \iota(s)\cos\left(\theta_{p}(s)\right)}.
\end{equation}

\subsection{Parametrization by arc length}
A path parametrization $\mat{p}_p(s)$ is said to be a \emph{parametrization by arc length} if for all $s_1, s_2 \in \mathbb{R}, s_2 > s_1$, we have
\begin{equation}
    \int_{s_1}^{s_2} \norm{\frac{\partial \mat{p}_p(s)}{\partial s}} {\rm d}s = s_2 - s_1.
\end{equation}
This condition is equivalent to
\begin{equation}
    \norm{\frac{\partial \mat{p}_p(s)}{\partial s}} = 1.
\end{equation}

A convenient property of parametrizations by arc length is that the path parameter $s$ can be interpreted as distance.
Consequently, parametrizations by arc length are useful when we want the vehicles to follow the path at a constant speed.
For example, choosing the path parameter $s(t)$ such that $\dot{s}(t) = 1$ means that the vehicles should follow the path at a speed of $1$ meter per second.

Now, let us discuss how to find a parametrization by arc length.
Let $\mat{p}_p(s)$ be an arbitrary parametrization that is $\mathcal{C}_1$ and regular.
Then, we can find a parametrization by arc length by reparametrizing $\mat{p}_p(s)$, \emph{i.e.,} by finding a monotonically increasing function $\rho(s) : \mathbb{R} \mapsto \mathbb{R}$ such that
\begin{equation}
    \norm{\frac{\partial \mat{p}_p(\rho(s))}{\partial s}} = 1.
\end{equation}
The function $\rho(s)$ can be found by solving the following differential equation
\begin{align}
    \frac{\partial \rho(s)}{\partial s} &= \norm{\frac{\partial \mat{p}_p(\rho)}{\partial \rho}}^{-1}, &
    \rho(0) &= \rho_0,
\end{align}
where $\rho_0 \in \mathbb{R}$ is the initial condition.
Although the initial condition is arbitrary, it is convenient to choose $\rho_0 = 0$ so that the parametrization starts at the same point.
