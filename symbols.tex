\chapter*{List of Symbols}

\begin{tabularx}{\textwidth}{lX}
    \ensuremath{\mathbb{R}}              & The set of real numbers. \\
    \ensuremath{\mathbb{Z}_{\geq 0}}     & The set of nonnegative integers. \\
    \ensuremath{\mathbb{R}^n}            & The set of $n$-dimensional real vectors. \\
    \ensuremath{\mathbb{R}^{n \times m}} & The set of $n$-by-$m$ real matrices. \\
    \ensuremath{SO(3)}                   & The special orthogonal group of dimension 3. \\
    \ensuremath{\mathfrak{so}(3)}        & The space of 3-by-3 skew-symmetric matrices, and the Lie algebra of $SO(3)$. \\
    \ball[n]{r}                          & The closed ball of radius $r$ ($\ball[n]{r} = \left\{\mat{x} \in \mathbb{R}^n | \norm{\mat{x}} \leq r\right\}$). \\
    \ensuremath{\partial \mathcal{M}}    & The boundary of an open set $\mathcal{M} \subset \mathbb{R}^n$. \\
    \ensuremath{\mat{I}_n}               & The $n$-by-$n$ identity matrix. \\
    \ensuremath{\mat{O}_{n \times m}}    & The $n$-by-$m$ null matrix. \\
    \ensuremath{\mat{0}_{n}}             & The $n$-dimensional null vector. \\
    \ensuremath{\mat{1}_{n}}             & The $n$-dimensional vector of ones. \\
    \ensuremath{\mat{1}_{n \times m}}    & The $n$-by-$m$ matrix of ones. \\
    \ensuremath{\norm{\mat{x}}}          & The Euclidean norm of a vector $\mat{x} \in \mathbb{R}^n$. \\
    \dst{\mat{x}}{\mathcal{M}}           & The Euclidean distance between a point $\mat{x}$ and a set $\mathcal{M}$ ($\dst{\mat{x}}{\mathcal{M}} = \inf_{\mat{y} \in \mathcal{M}} \norm{\mat{x} - \mat{y}}$). \\
    \ensuremath{\mat{e}_1, \mat{e}_2, \mat{e}_3} & The standard basis of $\mathbb{R}^3$ ($\mat{e}_1 = \inlinevector{1, 0, 0}$, $\mat{e}_2 = \inlinevector{0, 1, 0}$, $\mat{e}_3 = \inlinevector{0, 0, 1}$). \\
    \ensuremath{\mat{A}^{\dagger}}       & The Moore-Penrose pseudoinverse of a matrix $\mat{A} \in \mathbb{R}^{n \times m}$. \\
    \ensuremath{\mat{A} \otimes \mat{B}} & The Kronecker tensor product of matrices $\mat{A} \in \mathbb{R}^{n_a \times m_a}$ and $\mat{B} \in \mathbb{R}^{n_b \times m_b}$. \\
    \ensuremath{\arctan_2}               & Four-quadrant inverse tan; $\arctan_2(y, x)$ is equivalent to the phase of the complex number $x + iy$.
\end{tabularx}
